\documentclass[11pt]{article}
\usepackage{amsmath, amsthm, amssymb,lscape, natbib}
\usepackage{mathtools}
\usepackage{subfigure}
\usepackage[font=footnotesize,labelfont=bf]{caption}
\usepackage{graphicx}
\usepackage{colortbl}
\usepackage{hhline}
\usepackage{multirow}
\usepackage{multicol}
\usepackage{setspace}
\usepackage[final]{pdfpages}
\usepackage[left=2.5cm,top=2.5cm,right=2.5cm, bottom=2.5cm]{geometry}
\usepackage{natbib} 
\usepackage{bibentry} 
\newcommand{\bibverse}[1]{\begin{verse} \bibentry{#1} \end{verse}}
\newcommand{\vs}{\vspace{.3in}}
\renewcommand{\ni}{\noindent}
\usepackage{xr-hyper}
\usepackage[]{hyperref}
\hypersetup{
    colorlinks=true,
    linkcolor=blue,
    filecolor=magenta,      
    urlcolor=cyan,
}
 
\urlstyle{same}
\usepackage[capposition=top]{floatrow}
\usepackage{amssymb}
\usepackage{relsize}
\usepackage[dvipsnames]{xcolor}
\usepackage{fancyhdr}
\usepackage{tikz}
 
\pagestyle{fancy} % customize header and footer
\fancyhf{} % clear initial header and footer
%\rhead{Overleaf}
\lhead{\centering \rightmark} % this adds subsection number and name
\lfoot{\centering \rightmark} 
\rfoot{\thepage} % put page number (the centering command puts it in the middle, don't matter if you put it in right or left footer)

\def \myFigPath {../figures/} 
% BE CAREFUL WITH FIGNAMES, IN LATEX THEY'RE NOT CASE SENSITIVE!!
\def \myTablePath {../tables/} 

%\definecolor{mygreen}{RGB}{0, 100, 0}
\definecolor{mygreen}{RGB}{0, 128, 0}

\definecolor{citec}{rgb}{0,0,.5}
\definecolor{linkc}{rgb}{0,0,.6}
\definecolor{bcolor}{rgb}{1,1,1}
\hypersetup{
%hidelinks = true
  colorlinks = true,
  urlcolor=linkc,
  linkcolor=linkc,
  citecolor = citec,
  filecolor = linkc,
  pdfauthor={Laura G\'ati},
}


\geometry{left=.83in,right=.89in,top=1in,
bottom=1in}
\linespread{1.5}
\renewcommand{\[}{\begin{equation}}
\renewcommand{\]}{\end{equation}}

% New Options
\newtheorem{prop}{Proposition}
\newtheorem{definition}{Definition}[section]
\newtheorem*{remark}{Remark}
\newtheorem{lemma}{Lemma}
\newtheorem{corollary}{Corollary}
\newtheorem{conjecture}{Conjecture}

%\newtheorem{theorem}{Theorem}[section] % the third argument specifies that their number will be adopted to the section
%\newtheorem{corollary}{Corollary}[theorem]
%\newtheorem{lemma}[theorem]{Lemma}
%\declaretheorem{proposition}
%\linespread{1.3}
%\raggedbottom
%\font\reali=msbm10 at 12pt

% New Commands
\newcommand{\real}{\hbox{\reali R}}
\newcommand{\realp}{\hbox{\reali R}_{\scriptscriptstyle +}}
\newcommand{\realpp}{\hbox{\reali R}_{\scriptscriptstyle ++}}
\newcommand{\R}{\mathbb{R}}
\DeclareMathOperator{\E}{\mathbb{E}}
\DeclareMathOperator{\argmin}{arg\,min}
\newcommand\w{3.0in}
\newcommand\wnum{3.0}
\def\myFigWidth{5.3in}
\def\mySmallerFigWidth{2.1in}
\def\myEvenBiggerFigScale{0.8}
\def\myPointSixFigScale{0.6}
\def\myBiggerFigScale{0.4}
\def\myFigScale{0.3}
\def\myMediumFigScale{0.25}
\def\mySmallFigScale{0.22}
\def\mySmallerFigScale{0.18}
\def\myTinyFigScale{0.16}
\def\myPointFourteenFigScale{0.14}
\def\myTinierFigScale{0.12}
\def\myAdjustableFigScale{0.14}
\newcommand\numberthis{\addtocounter{equation}{1}\tag{\theequation}} % this defines a command to make align only number this line
\newcommand{\code}[1]{\texttt{#1}} %code %

\renewcommand*\contentsname{Overview}
\setcounter{tocdepth}{2}

% define a command to make a huge question mark (it works in math mode)
\newcommand{\bigqm}[1][1]{\text{\larger[#1]{\textbf{?}}}}

\begin{document}

\linespread{1.0}

\title{Materials 20 - Optimal Taylor rule coefficients}
\author{Laura G\'ati} 
\date{\today}
\maketitle

%%%%%%%%%%%%%%%%%%%%             DOCUMENT           %%%%%%%%%%%%%%%%%% 

\tableofcontents

%\listoffigures

\section{Procedure}
\begin{enumerate}
\item obtain the optimal noninertial plan for the endogenous variables,
\item  perform coefficient-comparison on the Taylor rule.
\end{enumerate}

Details:

\noindent Given optimal noninertial paths of the form $z_t = \bar{z} + f_zu_t + g_z r_t^n, \; z = {\pi,x,i}$, the Taylor rule
\begin{equation}
i_t = \psi_{\pi}(\pi_t -\bar{\pi}) + \psi_{x} (x_t -\bar{x}) + \bar{i}_t  \label{TR}
\end{equation}
can be written as
\begin{align}
i_t & = \bar{i} + \psi_{\pi}(f_{\pi}u_t + g_{\pi}r_t^n) + \psi_x(f_{x}u_t + g_{x}r_t^n)
\end{align}
which has to satisfy
\begin{align}
i_t & = \bar{i} + f_i u_t + g_i r^n_t
\end{align}
allowing one to solve for $(\psi_{\pi}^*, \psi_x^*)$ as the solution to
\begin{align}
f_i & = \psi_{\pi}f_{\pi} + \psi_xf_{x} \\
g_i & = \psi_{\pi}g_{\pi} + \psi_xg_{x}
\end{align}


\newpage
\noindent Details on obtaining the coefficients $f_z, g_z$ of the optimal noninertial plan $z_t = \bar{z} + f_z u_t + g_z r_t^n$ for the anchoring model 
\begin{enumerate}
\item Conjecture $z_t = \bar{z} + f_z u_t + g_z r_t^n$ where $z = \{\pi,x,i, f_a, f_b, \bar{\pi}, k^{-1}\}$
\item Plug conjecture into model equations (\ref{midsimple_first}) - (\ref{midsimple_last}) (the simplified version of the baseline model):
 \begin{align}
 &  \pi_t - \kappa x_t -(1-\alpha)\beta f_a(t) -\kappa\alpha\beta b_2 (I_3 - \alpha\beta h_x)^{-1}s_t - e_3(I_3 - \alpha\beta h_x)^{-1}s_t = 0 \label{midsimple_first}\\
 & x_t + \sigma i_t -\sigma f_b(t)  -  (1-\beta)b_2 (I_3 - \beta h_x)^{-1}s_t + \sigma\beta b_3 (I_3 - \beta h_x)^{-1}s_t -\sigma e_1(I_3 - \beta h_x)^{-1}s_t  \big)=0 \\
 &  f_a(t) - \frac{1}{1-\alpha\beta}\bar{\pi}_{t-1}  - b_1(I_3 - \alpha\beta h_x)^{-1}s_t  =0\\
 &  f_b(t) - \frac{1}{1-\beta}\bar{\pi}_{t-1}  - b_1(I_3 - \beta h_x)^{-1}s_t =0  \\
  &  \bar{\pi}_{t} - \bar{\pi}_{t-1} - k_t^{-1}\big(\pi_{t} -(\bar{\pi}_{t-1}+b_1 s_{t-1}) \big)  =0 \\
  &   k_t^{-1} - f(\pi_t - \bar{\pi}_{t-1}-b_1 s_{t-1})  =0 \label{midsimple_last}
\end{align}
\item Note that there are $n_y$ deterministic components $\bar{z}$, $n_y$ $f_z$-terms and $n_y$ $g_z$-terms. Imposing that the conjecture fulfills the model equations (\ref{midsimple_first}) - (\ref{midsimple_last}) yields $n_y -1$ constraints \emph{separately} for $\bar{z}$, $f_z$ and $g_z$. That is, I have three non-interacting equation systems, each consisting of $n_y -1$ equations in $n_y$ variables.
\item Solve 3 minimizations:
\begin{align}
\bar{z} & = \argmin L^{det} \quad s.t \quad \text{the 1st set of } \; n_y -1 \; \text{constraints from step 3}. \\
f_z & = \argmin L^{stab,u} \quad s.t \quad \text{the 2nd set of } \; n_y -1 \; \text{constraints from step 3}. \\
g_z & = \argmin L^{stab,r} \quad s.t \quad \text{the 3rd set of } \; n_y -1 \; \text{constraints from step 3}.
\end{align}
where I'm using the following decomposition of the central bank's loss:
The central bank's loss function
\begin{equation}
L^{CB} =\E_t \sum_{T=t}^{\infty}\{\pi_T^2 +\lambda_x(x_T - x^*)^2 +\lambda_i(i_T - i^*)\} \label{CBloss}
\end{equation}
can be decomposed into a component coming from the long-run means, and a component from fluctuations  of the endogenous variables:
\begin{align}
L^{det} & = \sum_{T=t}^{\infty}\beta^{T-t}\{\E_t{\pi_T^2} +\lambda_x(\E_tx_T -x^*)^2 + \lambda_i(\E_ti_T -i^*)^2 \}\\
L^{stab} & = \sum_{T=t}^{\infty}\beta^{T-t}\{\text{var}_t(\pi_T)+\lambda_x\text{var}_t(x_T) + \lambda_i\text{var}_t(i_T) \} \\
L^{stab,u} & \propto f_{\pi}^2 + \lambda_{x}f_x^2 + \lambda_i f_i^2\\
L^{stab,r} & \propto g_{\pi}^2 + \lambda_{x}g_x^2 + \lambda_i g_i^2
\end{align}
\item Solve the three decoupled $n_y \times n_y$ equation systems for $\bar{z}$, $f_z$ and $g_z$. \end{enumerate}
(I didn't solve for $\bar{z}$ because the TR-coefficients only depend on $f_z$ and $g_z$.)

\section{Remarks on the noninertial plan for the anchoring model}
\begin{enumerate}
\item Treat long-run expectations $f_a, f_b$, the gain $k^{-1}$ and expected mean inflation $\bar{\pi}$ as endogenous variables (part of the vector $z$).
\item Postulate a gain function 
\begin{equation}
k_t^{-1} -k_{t-1}^{-1} = c + d(\pi_t -\bar{\pi}_{t-1} - b_{11}r_t^n - b_{13}u_t)
\end{equation}
\item Step 2 yields interaction terms between $f_k,f_{\bar{\pi}}$ and $g_k, g_{\bar{\pi}}$, as well as between the shocks (of the form $u_tr_t^n, u_t^2, (r_t^n)^2$) and these coefficients loading on lagged shocks $u_{t-1}, r_{t-1}^n$ (which also show up in interaction terms). 
\begin{itemize}
\item To keep the solution linear, I therefore impose $f_k = g_k = f_{\bar{\pi}} = g_{\bar{\pi}} = 0$. 
\item[] Interpretation: in the optimal plan, the planner wants the gain and expected mean inflation not to fluctuate in response to shocks.
\item A direct consequence of this is:
\begin{align}
f_{f_a} = \frac{b_{13}}{1-\alpha  \beta  \rho _u}\quad \quad  f_{f_b}=  \frac{b_{13}}{1-\beta  \rho _u} \quad \quad g_{f_a}  = \frac{b_{11}}{1-\alpha  \beta  \rho _r} \quad \quad g_{f_b} = \frac{b_{11}}{1-\beta  \rho _r}
\end{align}
i.e. long-run expectations are just the discounted sums of the rational expectation of disturbances.
\end{itemize}
\end{enumerate}

\section{Optimal Taylor-rule coefficients}
\begin{align}
\psi_{\pi}^{anchor} & = \frac{\kappa  \sigma }{\lambda_i} \label{opt_psipi_anchor}
\\
\psi_{x}^{anchor} & =  \frac{\lambda_x\sigma }{\lambda_i } \label{opt_psix_anchor}
\end{align}
For the rational expectations version of the model with the assumption $\rho \equiv \rho_u = \rho_r$, the coefficients are
\begin{align}
\psi_{\pi}^{RE} & = \frac{\kappa  \sigma }{\lambda_i(\rho -1) (\beta  \rho -1)-\kappa  \lambda_i \rho  \sigma } \label{opt_psipi_RE}
\\
\psi_{x}^{RE} & =  \frac{\lambda_x\sigma  (1-\beta  \rho )}{\lambda_i (\rho -1) (\beta  \rho -1)-\kappa  \lambda_i \rho  \sigma } \label{opt_psix_RE}
\end{align}


\end{document}





