\documentclass[11pt]{article}
\usepackage{amsmath, amsthm, amssymb,lscape, natbib}
\usepackage{mathtools}
\usepackage{subfigure}
\usepackage[font=footnotesize,labelfont=bf]{caption}
\usepackage{graphicx}
\usepackage{colortbl}
\usepackage{hhline}
\usepackage{multirow}
\usepackage{multicol}
\usepackage{setspace}
\usepackage[final]{pdfpages}
\usepackage[left=2.5cm,top=2.5cm,right=2.5cm, bottom=2.5cm]{geometry}
\usepackage{natbib} 
\usepackage{bibentry} 
\newcommand{\bibverse}[1]{\begin{verse} \bibentry{#1} \end{verse}}
\newcommand{\vs}{\vspace{.3in}}
\renewcommand{\ni}{\noindent}
\usepackage{xr-hyper}
\usepackage[]{hyperref}
\hypersetup{
    colorlinks=true,
    linkcolor=blue,
    filecolor=magenta,      
    urlcolor=cyan,
}
 
\urlstyle{same}
\usepackage[capposition=top]{floatrow}
\usepackage{amssymb}
\usepackage{relsize}
\usepackage[dvipsnames]{xcolor}
\usepackage{fancyhdr}
\usepackage{tikz}
 
\pagestyle{fancy} % customize header and footer
\fancyhf{} % clear initial header and footer
%\rhead{Overleaf}
\lhead{\centering \rightmark} % this adds subsection number and name
\lfoot{\centering \rightmark} 
\rfoot{\thepage} % put page number (the centering command puts it in the middle, don't matter if you put it in right or left footer)

\def \myFigPath {../../figures/} 
% BE CAREFUL WITH FIGNAMES, IN LATEX THEY'RE NOT CASE SENSITIVE!!
\def \myTablePath {../../tables/} 

%\definecolor{mygreen}{RGB}{0, 100, 0}
\definecolor{mygreen}{RGB}{0, 128, 0}

\definecolor{citec}{rgb}{0,0,.5}
\definecolor{linkc}{rgb}{0,0,.6}
\definecolor{bcolor}{rgb}{1,1,1}
\hypersetup{
%hidelinks = true
  colorlinks = true,
  urlcolor=linkc,
  linkcolor=linkc,
  citecolor = citec,
  filecolor = linkc,
  pdfauthor={Laura G\'ati},
}


\geometry{left=.83in,right=.89in,top=1in,
bottom=1in}
\linespread{1.5}
\renewcommand{\[}{\begin{equation}}
\renewcommand{\]}{\end{equation}}

% New Options
\newtheorem{prop}{Proposition}
\newtheorem{definition}{Definition}[section]
\newtheorem*{remark}{Remark}
\newtheorem{lemma}{Lemma}
\newtheorem{corollary}{Corollary}
\newtheorem{conjecture}{Conjecture}

%\newtheorem{theorem}{Theorem}[section] % the third argument specifies that their number will be adopted to the section
%\newtheorem{corollary}{Corollary}[theorem]
%\newtheorem{lemma}[theorem]{Lemma}
%\declaretheorem{proposition}
%\linespread{1.3}
%\raggedbottom
%\font\reali=msbm10 at 12pt

% New Commands
\newcommand{\real}{\hbox{\reali R}}
\newcommand{\realp}{\hbox{\reali R}_{\scriptscriptstyle +}}
\newcommand{\realpp}{\hbox{\reali R}_{\scriptscriptstyle ++}}
\newcommand{\R}{\mathbb{R}}
\DeclareMathOperator{\E}{\mathbb{E}}
\DeclareMathOperator{\argmin}{arg\,min}
\newcommand\w{3.0in}
\newcommand\wnum{3.0}
\def\myFigWidth{5.3in}
\def\mySmallerFigWidth{2.1in}
\def\myEvenBiggerFigScale{0.8}
\def\myPointSixFigScale{0.6}
\def\myBiggerFigScale{0.4}
\def\myFigScale{0.3}
\def\myMediumFigScale{0.25}
\def\mySmallFigScale{0.22}
\def\mySmallerFigScale{0.18}
\def\myTinyFigScale{0.16}
\def\myPointFourteenFigScale{0.14}
\def\myTinierFigScale{0.12}
\def\myAdjustableFigScale{0.17}
\newcommand\numberthis{\addtocounter{equation}{1}\tag{\theequation}} % this defines a command to make align only number this line
\newcommand{\code}[1]{\texttt{#1}} %code %

\renewcommand*\contentsname{Overview}
\setcounter{tocdepth}{2}

% define a command to make a huge question mark (it works in math mode)
\newcommand{\bigqm}[1][1]{\text{\larger[#1]{\textbf{?}}}}

\begin{document}

\linespread{1.0}

\title{Collection of figures for draft - for purposes of having a unified visual identity}
\author{Laura G\'ati} 
\date{\today}
\maketitle

%%%%%%%%%%%%%%%%%%%%             DOCUMENT           %%%%%%%%%%%%%%%%%% 

%\tableofcontents

\listoffigures

\begin{figure}[h!]
\includegraphics[scale = \mySmallFigScale]{\myFigPath epi10_2020_06_04}
\caption{Market-based inflation expectations, 10 year, average, \%}
\label{epi}
\end{figure}

\begin{figure}[h!]
\subfigure[$i(k^{-1})$]{\includegraphics[scale=0.17]{\myFigPath analyze_opt_policy_ik}}
\subfigure[$i(\bar{\pi})$]{\includegraphics[scale=0.17]{\myFigPath analyze_opt_policy_ip}}
\caption{Comparative statics: policy in function of the endogenous states}
\label{di}
\end{figure}

\begin{figure}[h!]
\subfigure[Simulation using a Taylor rule]{\includegraphics[scale=0.24]{\myFigPath implement_anchTC_obs_TR}}
\hfill % this is great to intro dpace between subfigures
\subfigure[Simulation using optimal policy]{\includegraphics[scale=0.24]{\myFigPath implement_anchTC_obs}}
\caption{}
\label{pea_TCvsTR}
\end{figure}

\clearpage
\begin{figure}[h!]
\subfigure[RE]{\includegraphics[scale = \mySmallerFigScale]{\myFigPath plot_sim_loss_loss_RE_params_psi_pi_1_5_psi_x_0_gbar_0_145_thetbar_4_thettilde_2_5_kap_0_8_lamx_0_lami_0_2020_02_09}}
\subfigure[Anchoring]{\includegraphics[scale = \mySmallerFigScale]{\myFigPath plot_sim_loss_loss_again_critCUSUM_constant_only_params_psi_pi_1_5_psi_x_0_gbar_0_145_thetbar_4_thettilde_2_5_kap_0_8_lamx_0_lami_0_2020_02_09}}
\caption{Central bank loss function as a function of $\psi_{\pi}$}
\floatfoot{}
\label{fig_loss}
\end{figure}

% relabel the fig number so we keep the numbering of the draft
\renewcommand{\thefigure}{4.2}

\begin{figure}[h!]
\subfigure[RE]{\includegraphics[scale = 0.14]{\myFigPath plot_sim_loss_pretty_loss_RE_again_critCUSUM_constant_only_params_psi_pi_1_5_psi_x_0_gbar_0_145_thetbar_16_thettilde_2_5_kap_0_8_lamx_0_lami_0_2020_06_05}}
\subfigure[Anchoring]{\includegraphics[scale = 0.14]{\myFigPath plot_sim_loss_pretty_loss_again_critCUSUM_constant_only_T_400_N_100_burnin_0_params_psi_pi_1_5_psi_x_0_gbar_0_145_thetbar_16_thettilde_2_5_kap_0_8_lamx_0_lami_0_date_2020_06_05}}
\caption{Central bank loss function as a function of $\psi_{\pi}$}
\floatfoot{Sample length is $T=400$ with a cross-section of $N=100$.}
\label{fig_loss}
\end{figure}

Note: Loss is computed for various values of $\psi_{\pi}$ when $\psi_x = 0, \lambda_x=\lambda_i=0$ and for the CUSUM-criterion with parameters $\tilde{\kappa}=0.8, \tilde{\theta} = 4$. $T=400, N=100, n_{burn-in}=0$

\clearpage
% set the figure counter back so we resume counting where we left off
\setcounter{figure}{4}
\renewcommand{\thefigure}{\arabic{figure}}

\begin{figure}[h!]
\subfigure[RE against anchoring, expectations anchored]{\includegraphics[scale = \mySmallerFigScale]{\myFigPath command_IRFs_anchoring_RIR_LH_anchmonpol_again_critCUSUM_constant_only_2020_02_10}}
\subfigure[RE against anchoring, expectations unanchored]{\includegraphics[scale = \mySmallerFigScale]{\myFigPath command_IRFs_anchoring_RIR_LH_unanchmonpol_again_critCUSUM_constant_only_2020_02_10}}
\caption{Impulse responses after a contractionary monetary policy shock}
\floatfoot{Shock imposed at $t=25$ of a sample length of $T=400$ (with 100 initial burn-in periods), cross-sectional average with a cross-section size of $N=100$. For the rest of the section, I keep these simulation values unless otherwise stated. For the anchoring model, the remark refers to whether expectations are anchored at the time the shock hits.}
\label{IRF}
\end{figure}

% relabel the fig number so we keep the numbering of the draft
\renewcommand{\thefigure}{5.2}

\begin{figure}[h!]
\subfigure[RE against anchoring, expectations anchored]{\includegraphics[scale = \mySmallerFigScale]{\myFigPath command_IFS_anchoring_pretty_RIR_LH_anch_monpol_again_critCUSUM_constant_only_T_400_N_1000_burnin_5_params_psi_pi_1_5_psi_x_0_gbar_0_145_thetbar_16_thettilde_2_5_kap_0_8_lamx_0_lami_0_date_2020_06_05}}
\subfigure[RE against anchoring, expectations unanchored]{\includegraphics[scale = \mySmallerFigScale]{\myFigPath command_IFS_anchoring_pretty_RIR_LH_unanch_monpol_again_critCUSUM_constant_only_T_400_N_1000_burnin_5_params_psi_pi_1_5_psi_x_0_gbar_0_145_thetbar_16_thettilde_2_5_kap_0_8_lamx_0_lami_0_date_2020_06_05}}
\caption{Impulse responses after a contractionary monetary policy shock}
\floatfoot{Shock imposed at $t=25$ of a sample length of $T=400$ (with 5 initial burn-in periods), cross-sectional average with a cross-section size of $N=1000$. For the anchoring model, the remark refers to whether expectations are anchored at the time the shock hits.}
\label{IRF}
\end{figure}

% set the figure counter back so we resume counting where we left off
\setcounter{figure}{5}
\renewcommand{\thefigure}{\arabic{figure}}

\begin{figure}[h!]
\subfigure[$\psi_{\pi} = 1.01$]{\includegraphics[scale = \mySmallerFigScale]{\myFigPath command_IRFs_anchoring_loss_again_critCUSUM_constant_only_params_psi_pi_1_01_psi_x_0_gbar_0_145_thetbar_4_thettilde_2_5_kap_0_8_lamx_0_lami_0_2020_02_09}}
\subfigure[$\psi_{\pi} = 1.5$]{\includegraphics[scale = \mySmallerFigScale]{\myFigPath command_IRFs_anchoring_loss_again_critCUSUM_constant_only_params_psi_pi_1_5_psi_x_0_gbar_0_145_thetbar_4_thettilde_2_5_kap_0_8_lamx_0_lami_0_2020_02_09}}
\subfigure[$\psi_{\pi} = 2$]{\includegraphics[scale = \mySmallerFigScale]{\myFigPath command_IRFs_anchoring_loss_again_critCUSUM_constant_only_params_psi_pi_2_psi_x_0_gbar_0_145_thetbar_4_thettilde_2_5_kap_0_8_lamx_0_lami_0_2020_02_09}}
\caption{Cross-sectional average gains for various values of $\psi_{\pi}$}
\floatfoot{}
\label{anchor_psi}
\end{figure}

% relabel the fig number so we keep the numbering of the draft
\renewcommand{\thefigure}{6.2}

\begin{figure}[h!]
\subfigure[$\psi_{\pi} = 1.01$]{\includegraphics[scale = \mySmallerFigScale]{\myFigPath command_IFS_anchoring_pretty_invgain_again_critCUSUM_constant_only_params_psi_pi_1_01_psi_x_0_gbar_0_145_thetbar_16_thettilde_2_5_kap_0_8_lamx_0_lami_0_2020_06_05}}
\subfigure[$\psi_{\pi} = 1.5$]{\includegraphics[scale = \mySmallerFigScale]{\myFigPath command_IFS_anchoring_pretty_invgain_again_critCUSUM_constant_only_params_psi_pi_1_5_psi_x_0_gbar_0_145_thetbar_16_thettilde_2_5_kap_0_8_lamx_0_lami_0_2020_06_05}}
\subfigure[$\psi_{\pi} = 2$]{\includegraphics[scale = \mySmallerFigScale]{\myFigPath command_IFS_anchoring_pretty_invgain_again_critCUSUM_constant_only_params_psi_pi_2_psi_x_0_gbar_0_145_thetbar_16_thettilde_2_5_kap_0_8_lamx_0_lami_0_2020_06_05}}
\caption{Cross-sectional average gains for various values of $\psi_{\pi}$}
\floatfoot{Sample length is $T=400$ (with 5 initial burn-in periods), cross-section size is $N=1000$.}
\label{anchor_psi}
\end{figure}

% set the figure counter back so we resume counting where we left off
\setcounter{figure}{6}
\renewcommand{\thefigure}{\arabic{figure}}

\begin{figure}[h!]
\subfigure[$\psi_{\pi} = 1.01$]{\includegraphics[scale = \mySmallerFigScale]{\myFigPath command_IRFs_anchoring_RIR_LH_unanch_monpol_again_critCUSUM_constant_only_psi_pi_1_01_2020_02_10}}
\subfigure[$\psi_{\pi} = 1.5$]{\includegraphics[scale = \mySmallerFigScale]{\myFigPath command_IRFs_anchoring_RIR_LH_unanch_monpol_again_critCUSUM_constant_only_psi_pi_1_5_2020_02_10}}
\subfigure[$\psi_{\pi} = 2$]{\includegraphics[scale = \mySmallerFigScale]{\myFigPath command_IRFs_anchoring_RIR_LH_unanch_monpol_again_critCUSUM_constant_only_psi_pi_2_2020_02_10}}
\caption{Impulse responses for unanchored expectations for various values of $\psi_{\pi}$}
\floatfoot{}
\label{IRF_unanchored_psi}
\end{figure}


% relabel the fig number so we keep the numbering of the draft
\renewcommand{\thefigure}{7.2}

\begin{figure}[h!]
\subfigure[$\psi_{\pi} = 1.01$]{\includegraphics[scale = \mySmallerFigScale]{\myFigPath command_IFS_anchoring_pretty_RIR_LH_unanch_monpol_again_critCUSUM_constant_only_T_400_N_1000_burnin_5_params_psi_pi_1_01_psi_x_0_gbar_0_145_thetbar_16_thettilde_2_5_kap_0_8_lamx_0_lami_0_date_2020_06_05}}
\subfigure[$\psi_{\pi} = 1.5$]{\includegraphics[scale = \mySmallerFigScale]{\myFigPath command_IFS_anchoring_pretty_RIR_LH_unanch_monpol_again_critCUSUM_constant_only_T_400_N_1000_burnin_5_params_psi_pi_1_5_psi_x_0_gbar_0_145_thetbar_16_thettilde_2_5_kap_0_8_lamx_0_lami_0_date_2020_06_05}}
\subfigure[$\psi_{\pi} = 2$]{\includegraphics[scale = \mySmallerFigScale]{\myFigPath command_IFS_anchoring_pretty_RIR_LH_unanch_monpol_again_critCUSUM_constant_only_T_400_N_1000_burnin_5_params_psi_pi_2_psi_x_0_gbar_0_145_thetbar_16_thettilde_2_5_kap_0_8_lamx_0_lami_0_date_2020_06_05}}
\caption{Impulse responses for unanchored expectations for various values of $\psi_{\pi}$}
\floatfoot{Shock imposed at $t=25$ of a sample length of $T=400$ (with 5 initial burn-in periods), cross-sectional average with a cross-section size of $N=1000$.}
\label{IRF_unanchored_psi}
\end{figure}

% set the figure counter back so we resume counting where we left off
\setcounter{figure}{7}
\renewcommand{\thefigure}{\arabic{figure}}

\begin{figure}[h!]
\subfigure[$i(X_1)$]{\includegraphics[scale=\myAdjustableFigScale]{\myFigPath compare_value_pea_results_value_outputs_server_inputs_pretty}} % original was: materials32_policy
\subfigure[$i(X_2)$]{\includegraphics[scale=\myAdjustableFigScale]{\myFigPath compare_value_pea_results_value_outputs_server02_Jun_2020_14_58_12_pea_outputs_30_May_2020_10_18_28_pretty}}
\caption{Policy function for two particular histories of states, $X^{PEA}$}
\end{figure}



\end{document}

%%%%%%%%%%%%%    SUBFIGURE  %%%%%%%%%%%
%\begin{figure}[h!]
%\subfigure[Hodrick-Prescott, $\lambda=1600$]{\includegraphics[scale=\myAdjustableFigScale]{\myFigPath materials22_gain_dhat_HP}}
%\hfill % this is great to intro dpace between subfigures
%\subfigure[Hamilton, 4 lags, $h=8$]{\includegraphics[scale=\myAdjustableFigScale]{\myFigPath materials22_gain_dhat_Hamilton}}
%\subfigure[Baxter-King, $(6,32)$ quarters, truncation at 12 lags]{\includegraphics[scale=\myAdjustableFigScale]{\myFigPath materials22_gain_dhat_BK}}
%\caption{Inverse gain for $\hat{d}$ for the different filters}
%\end{figure}

%%%%%%%%%%%%%    TABLE  %%%%%%%%%%%
%\begin{center}
%\begin{table}[h!]
%\caption{$\hat{d}$}
%\begin{tabular}{ c |c |c }
%  & $W = I$ & $W = \text{diag}(\hat{\sigma}_{ac(0)}, \dots, \hat{\sigma}_{ac(K)})$ \\ 
%  \hline
% HP & 77.7899 & 10 \\  
% \hline
% Hamilton & 32.1649 & 10 \\  
% \hline
% BK & 90.3929 & 10    
%\end{tabular}
%\end{table}
%\end{center}





