\documentclass[11pt]{article}
\usepackage{amsmath, amsthm, amssymb,lscape, natbib}
\usepackage{mathtools}
\usepackage{subfigure}
\usepackage[font=footnotesize,labelfont=bf]{caption}
\usepackage{graphicx}
\usepackage{colortbl}
\usepackage{hhline}
\usepackage{multirow}
\usepackage{multicol}
\usepackage{setspace}
\usepackage[final]{pdfpages}
\usepackage[left=2.5cm,top=2.5cm,right=2.5cm, bottom=2.5cm]{geometry}
\usepackage{natbib} 
\usepackage{bibentry} 
\newcommand{\bibverse}[1]{\begin{verse} \bibentry{#1} \end{verse}}
\newcommand{\vs}{\vspace{.3in}}
\renewcommand{\ni}{\noindent}
\usepackage{xr-hyper}
\usepackage[]{hyperref}
\usepackage[capposition=top]{floatrow}
\usepackage{amssymb}

\def \myFigPath {../figures/} 
% BE CAREFUL WITH FIGNAMES, IN LATEX THEY'RE NOT CASE SENSITIVE!!
\def \myTablePath {../tables/} 

\definecolor{citec}{rgb}{0,0,.5}
\definecolor{linkc}{rgb}{0,0,.6}
\definecolor{bcolor}{rgb}{1,1,1}
\hypersetup{
%hidelinks = true
  colorlinks = true,
  urlcolor=linkc,
  linkcolor=linkc,
  citecolor = citec,
  filecolor = linkc,
  pdfauthor={Laura G\'ati},
}


\geometry{left=.83in,right=.89in,top=1in,
bottom=1in}
\linespread{1.5}
\renewcommand{\[}{\begin{equation}}
\renewcommand{\]}{\end{equation}}

% New Options
\newtheorem{prop}{Proposition}
\newtheorem{definition}{Definition}[section]
\newtheorem*{remark}{Remark}
\newtheorem{lemma}{Lemma}
\newtheorem{corollary}{Corollary}
%\newtheorem{theorem}{Theorem}[section] % the third argument specifies that their number will be adopted to the section
%\newtheorem{corollary}{Corollary}[theorem]
%\newtheorem{lemma}[theorem]{Lemma}
%\declaretheorem{proposition}
%\linespread{1.3}
%\raggedbottom
%\font\reali=msbm10 at 12pt

% New Commands
\newcommand{\real}{\hbox{\reali R}}
\newcommand{\realp}{\hbox{\reali R}_{\scriptscriptstyle +}}
\newcommand{\realpp}{\hbox{\reali R}_{\scriptscriptstyle ++}}
\newcommand{\R}{\mathbb{R}}
\DeclareMathOperator{\E}{\mathbb{E}}
\DeclareMathOperator{\argmin}{arg\,min}
\newcommand\w{3.0in}
\newcommand\wnum{3.0}
\def\myFigWidth{5.3in}
\def\mySmallerFigWidth{2.1in}
\def\myEvenBiggerFigScale{0.8}
\def\myPointSixFigScale{0.6}
\def\myBiggerFigScale{0.4}
\def\myFigScale{0.3}
\def\mySmallFigScale{0.22}
\def\mySmallerFigScale{0.18}
\def\myTinyFigScale{0.16}
\def\myPointFourteenFigScale{0.14}
\def\myTinierFigScale{0.12}
\newcommand\numberthis{\addtocounter{equation}{1}\tag{\theequation}} % this defines a command to make align only number this line
\newcommand{\code}[1]{\texttt{#1}} %code %

\renewcommand*\contentsname{Overview}
\setcounter{tocdepth}{2}

\begin{document}

\linespread{1.0}

\title{Materials 5b - Revisiting the timing}
\author{Laura G\'ati} 
\date{\today}
\maketitle

%%%%%%%%%%%%%%%%%%%%             DOCUMENT           %%%%%%%%%%%%%%%%%% 

%\tableofcontents

%\listoffigures

%\newpage
Define some objects:
\begin{align}
f_t &= \hat{\E}_t(z_{t+1}) \quad \text{one-period-ahead forecast formed at time } t \\
fe_t & = z_{t+1}-f_t \quad \text{one-period-ahead forecast realized at time } t+1 \\
& = ALM(t+1) - PLM(t+1) \\
\theta_t &=  \hat{\E}_{t-1}(z_{t}) - \E_{t-1}(z_{t}) \quad \text{CEMP's criterion} \\
& = PLM(t) - \E_{t-1}ALM(t)
\end{align}


\begin{multicols}{2}

\begin{enumerate}
\item[] $PLM_1: \hat{\E}_t z_{t+1} = \bar{z}_{\color{red}{t-1}} + bs_t$
\item[] Then at time $t$ available: $\bar{z}_{\color{red}{t-1}}, s_t, k_{t-1}, f_{t-2}$
\item Form all future expectations using $PLM_1 \rightarrow z_t$ realized 
\item Form $\theta_t \rightarrow k_t$ realized
\item Update $\bar{z}_t = \bar{z}_{t-1} + k_t^{-1}(fe_{t-1})$
\item[] where $fe_{t-1} = z_{t}-f_{t-1} = $ 
\item[] $fe_{t-1} = z_{t}-(\bar{z}_{\color{red}{t-2}} + bs_{t-1})$, so:
\item[] $\bar{z}_t = \bar{z}_{t-1} + k_t^{-1}(z_{t}-(\bar{z}_{\color{red}{t-2}} + bs_{t-1}))$
\end{enumerate}

\columnbreak
\begin{enumerate}
\item[] $PLM_2: \hat{\E}_t z_{t+1} = \bar{z}_{\color{red}{t}} + bs_t$
\item[] Then at time $t$ available: $\bar{z}_{\color{red}{t}}, s_t, k_{t-1}, f_{t-2}$
\item Form all future expectations using $PLM_2 \rightarrow z_t$ realized
\item Form $\theta_t \rightarrow k_t$ realized
\item Update $\bar{z}_{t+1} = \bar{z}_{t} + k_t^{-1}(fe_{t-1})$
\item[] where $fe_{t-1} = z_{t}-f_{t-1} = $ 
\item[] $fe_{t-1} = z_{t}-(\bar{z}_{\color{red}{t-1}} + bs_{t-1})$, so:
\item[] $\bar{z}_t = \bar{z}_{t-1} + k_t^{-1}(z_{t}-(\bar{z}_{\color{red}{t-1}} + bs_{t-1}))$
\end{enumerate}
\end{multicols}
\textbf{Issue \# 1}: Updating of $\bar{z}$ is in any case a function of last period's $\bar{z}$ (i.e. not the one available to use this morning) (unless you use an ``assessment forecast": the $\bar{z}$ you used this morning with $s_{t-1}$ you used yesterday).

The second issue will be about the criterion. Recall:
\begin{align*}
\theta_t &=  \hat{\E}_{t-1}(z_{t}) - \E_{t-1}(z_{t})  \\
& = PLM(t) - \E_{t-1}ALM(t)
\end{align*}
\begin{multicols}{2}
 Recall: $PLM_1: \hat{\E}_t z_{t+1} = \bar{z}_{\color{red}{t-1}} + bs_t$
\begin{align*}
ALM_t & = \text{stuff} \times \bar{z}_{t-1} + \text{stuff} \times s_t \\
\theta_t &= \bar{z}_{\color{red}{t-2}} + bs_{t-1} - \E_{t-1}( \text{stuff} \times \bar{z}_{t-1} + \text{stuff} \times s_t)
\end{align*}

\columnbreak
Recall: $PLM_2: \hat{\E}_t z_{t+1} = \bar{z}_{\color{red}{t}} + bs_t$
\begin{align*}
ALM_t & = \text{stuff} \times \bar{z}_{t} + \text{stuff} \times s_t \\
\theta_t &= \bar{z}_{\color{red}{t-1}} + bs_{t-1} - \E_{t-1}( \text{stuff} \times \bar{z}_{t} + \text{stuff} \times s_t)
\end{align*}
\end{multicols}
\textbf{Issue \#2}: I had this issue before, but it's not clear what the RE of $\bar{z}$ is. In particular, I don't know what the index of $\E_{t-1}$ refers to: the morning of $t-1$ or the evening?
\begin{multicols}{2}
\begin{itemize}
\item If it's the morning, then $ \E_{t-1}(\bar{z}_{t-1}) =  \bar{z}_{t-2}$
\item[] $\rightarrow \theta_t = f(\bar{z}_{t-2}, s_{t-1})$
\item If it's the evening, then $ \E_{t-1}(\bar{z}_{t-1}) =  \bar{z}_{t-1}$
\item[] $\rightarrow \theta_t = f(\bar{z}_{t-2}, \bar{z}_{t-1}, s_{t-1})$
\end{itemize}

\columnbreak
\begin{itemize}
\item If it's the morning, then $ \E_{t-1}(\bar{z}_{t}) =  \bar{z}_{t-1}$
\item[] $\rightarrow \theta_t = f(\bar{z}_{t-1}, s_{t-1})$
\item If it's the evening, then $ \E_{t-1}(\bar{z}_{t}) =  \bar{z}_{t}$
\item[] $\rightarrow \theta_t = f(\bar{z}_{t-1}, \bar{z}_{t}, s_{t-1})$
\end{itemize}
\end{multicols}
The ``evening" assumption isn't cool because the criterion depends on the intercept at several time periods, the ``morning" assumption isn't cool because just like in Issue \#1, we need access to yesterday morning's estimate of the intercept.

But one can use Ryan's PLM ($PLM_1$) in any case, because the issues are exactly the same, the notation is just different. 


\end{document}





