\documentclass[11pt]{article}
\usepackage{amsmath, amsthm, amssymb,lscape, natbib}
\usepackage{mathtools}
\usepackage{subfigure}
\usepackage[font=footnotesize,labelfont=bf]{caption}
\usepackage{graphicx}
\usepackage{colortbl}
\usepackage{hhline}
\usepackage{multirow}
\usepackage{multicol}
\usepackage{setspace}
\usepackage[final]{pdfpages}
\usepackage[left=2.5cm,top=2.5cm,right=2.5cm, bottom=2.5cm]{geometry}
\usepackage{natbib} 
\usepackage{bibentry} 
\newcommand{\bibverse}[1]{\begin{verse} \bibentry{#1} \end{verse}}
\newcommand{\vs}{\vspace{.3in}}
\renewcommand{\ni}{\noindent}
\usepackage{xr-hyper}
\usepackage[]{hyperref}
\hypersetup{
    colorlinks=true,
    linkcolor=blue,
    filecolor=magenta,      
    urlcolor=cyan,
}
 
\urlstyle{same}
\usepackage[capposition=top]{floatrow}
\usepackage{amssymb}
\usepackage{relsize}
\usepackage[dvipsnames]{xcolor}
\usepackage{fancyhdr}
\usepackage{tikz}
 
\pagestyle{fancy} % customize header and footer
\fancyhf{} % clear initial header and footer
%\rhead{Overleaf}
\lhead{\centering \rightmark} % this adds subsection number and name
\lfoot{\centering \rightmark} 
\rfoot{\thepage} % put page number (the centering command puts it in the middle, don't matter if you put it in right or left footer)

\def \myFigPath {../figures/} 
% BE CAREFUL WITH FIGNAMES, IN LATEX THEY'RE NOT CASE SENSITIVE!!
\def \myTablePath {../tables/} 

%\definecolor{mygreen}{RGB}{0, 100, 0}
\definecolor{mygreen}{RGB}{0, 128, 0}

\definecolor{citec}{rgb}{0,0,.5}
\definecolor{linkc}{rgb}{0,0,.6}
\definecolor{bcolor}{rgb}{1,1,1}
\hypersetup{
%hidelinks = true
  colorlinks = true,
  urlcolor=linkc,
  linkcolor=linkc,
  citecolor = citec,
  filecolor = linkc,
  pdfauthor={Laura G\'ati},
}


\geometry{left=.83in,right=.89in,top=1in,
bottom=1in}
\linespread{1.5}
\renewcommand{\[}{\begin{equation}}
\renewcommand{\]}{\end{equation}}

% New Options
\newtheorem{prop}{Proposition}
\newtheorem{definition}{Definition}[section]
\newtheorem*{remark}{Remark}
\newtheorem{lemma}{Lemma}
\newtheorem{corollary}{Corollary}
\newtheorem{conjecture}{Conjecture}

%\newtheorem{theorem}{Theorem}[section] % the third argument specifies that their number will be adopted to the section
%\newtheorem{corollary}{Corollary}[theorem]
%\newtheorem{lemma}[theorem]{Lemma}
%\declaretheorem{proposition}
%\linespread{1.3}
%\raggedbottom
%\font\reali=msbm10 at 12pt

% New Commands
\newcommand{\real}{\hbox{\reali R}}
\newcommand{\realp}{\hbox{\reali R}_{\scriptscriptstyle +}}
\newcommand{\realpp}{\hbox{\reali R}_{\scriptscriptstyle ++}}
\newcommand{\R}{\mathbb{R}}
\DeclareMathOperator{\E}{\mathbb{E}}
\DeclareMathOperator{\argmin}{arg\,min}
\newcommand\w{3.0in}
\newcommand\wnum{3.0}
\def\myFigWidth{5.3in}
\def\mySmallerFigWidth{2.1in}
\def\myEvenBiggerFigScale{0.8}
\def\myPointSixFigScale{0.6}
\def\myBiggerFigScale{0.4}
\def\myFigScale{0.3}
\def\myMediumFigScale{0.25}
\def\mySmallFigScale{0.22}
\def\mySmallerFigScale{0.18}
\def\myTinyFigScale{0.16}
\def\myPointFourteenFigScale{0.14}
\def\myTinierFigScale{0.12}
\def\myAdjustableFigScale{0.18}
\newcommand\numberthis{\addtocounter{equation}{1}\tag{\theequation}} % this defines a command to make align only number this line
\newcommand{\code}[1]{\texttt{#1}} %code %

\renewcommand*\contentsname{Overview}
\setcounter{tocdepth}{2}

% define a command to make a huge question mark (it works in math mode)
\newcommand{\bigqm}[1][1]{\text{\larger[#1]{\textbf{?}}}}

\begin{document}

\linespread{1.0}

\title{Materials for Susanto - IRFs for learning}
\author{Laura G\'ati} 
\date{\today}
\maketitle

%%%%%%%%%%%%%%%%%%%%             DOCUMENT           %%%%%%%%%%%%%%%%%% 

\tableofcontents

%\listoffigures

%\newpage


\newpage


\newpage
\section{Observables for 3 shocks}	
\begin{figure}[h!]
\subfigure[Decreasing gain, $t=25$]{\includegraphics[scale=\myAdjustableFigScale]{\myFigPath materials10_RIR_fafb_d_monpolslope_and_constant_rho0_psi_pi_1_5_sig_1_dt_25}}
\subfigure[Constant gain, $t=25$]{\includegraphics[scale=\myAdjustableFigScale]{\myFigPath materials10_RIR_fafb_c_monpolslope_and_constant_rho0_psi_pi_1_5_sig_1_dt_25}}
\caption{Natural rate shock}
\end{figure}

\begin{figure}[h!]
\subfigure[Decreasing gain, $t=25$]{\includegraphics[scale=\myAdjustableFigScale]{\myFigPath materials10_RIR_fafb_d_monpolslope_and_constant_rho0_psi_pi_1_5_sig_1_dt_25}}
\subfigure[Constant gain, $t=25$]{\includegraphics[scale=\myAdjustableFigScale]{\myFigPath materials10_RIR_fafb_c_monpolslope_and_constant_rho0_psi_pi_1_5_sig_1_dt_25}}
\caption{Monetary policy shock}
\end{figure}

 \begin{figure}[h!]
\subfigure[Decreasing gain, $t=25$]{\includegraphics[scale=\myAdjustableFigScale]{\myFigPath materials10_RIR_fafb_d_monpolslope_and_constant_rho0_psi_pi_1_5_sig_1_dt_25}}
\subfigure[Constant gain, $t=25$]{\includegraphics[scale=\myAdjustableFigScale]{\myFigPath materials10_RIR_fafb_c_monpolslope_and_constant_rho0_psi_pi_1_5_sig_1_dt_25}}
\caption{Cost-push shock}
\end{figure}

\newpage
\section{How observables respond to expectations - RE vs. learning}
Ignoring shocks and setting $\psi_x = 0$, so the Taylor rule is just $i_t =  \psi_{\pi}\pi_t$, the two systems are
(throughout I'm using blue to denote negative values).
\begin{align*}
RE& \\
x_t &= \textcolor{blue}{- \sigma \psi_{\pi}} \pi_t+  \E_t x_{t+1}   + \sigma \E_t \pi_{t+1}  \\
\pi_t &= \kappa x_t +\beta \E_t \pi_{t+1} \\
Learning & \\
x_t &= \textcolor{blue}{ -\sigma \psi_{\pi}} \pi_t +\hat{\E}_t \sum_{T=t}^{\infty} \beta^{T-t }\big( (1-\beta) x_{T+1} + \textcolor{blue}{\sigma(1- \beta \psi_{\pi})}\pi_{T+1} \big)  \\
\pi_t &= \kappa x_t +\hat{\E}_t \sum_{T=t}^{\infty} (\alpha\beta)^{T-t }\big( \kappa \alpha \beta x_{T+1} + (1-\alpha)\beta \pi_{T+1} \big)
\end{align*}
Expressing $x, \pi$ as functions of expectations alone, this gives:
\begin{align*}
RE& \\
x_t &=  \textcolor{blue}{ \frac{\sigma(1-\beta\psi_{\pi})}{1+\sigma\psi_{\pi}\kappa}}\E_t \pi_{t+1}   +   \frac{1}{1+\sigma\psi_{\pi}\kappa}\E_t x_{t+1}  \\
\pi_t &=   \overbrace{\bigg( \textcolor{blue}{\frac{\kappa\sigma(1-\beta\psi_{\pi})}{1+\sigma\psi_{\pi}\kappa}}+\beta\bigg)}^{+}\E_t \pi_{t+1} +   \frac{\kappa}{1+\sigma\psi_{\pi}\kappa}\E_t x_{t+1} \\
Learning & \\
x_t & = \textcolor{blue}{\frac{-\sigma\psi_{\pi}}{w}} \begin{bmatrix} (1-\alpha)\beta&\kappa\alpha\beta & 0 \end{bmatrix} f_a + \frac{1}{w}\begin{bmatrix} \textcolor{blue}{\sigma(1-\beta\psi_{\pi})}&1-\beta & 0\end{bmatrix} f_b \\
\pi_t & = (1-\frac{\kappa\sigma\psi_{\pi}}{w}) \begin{bmatrix} (1-\alpha)\beta&\kappa\alpha\beta & 0 \end{bmatrix} f_a + \frac{\kappa}{w}\begin{bmatrix}  \textcolor{blue}{\sigma(1-\beta\psi_{\pi})}&1-\beta & 0\end{bmatrix} f_b
\end{align*}
This yields the stylized representation of how endogenous variables respond to expectations in the two formulations:

\vspace{-0.9cm}

\begin{multicols}{2}
\begin{align*}
& RE \\
x_t & = \overset{\textcolor{blue}{-}}{\E(\pi)} + \overset{+}{\E(x)} \\  
\pi_t & = \overset{+}{\E(\pi)} + \overset{+}{\E(x)} 
\end{align*}
\columnbreak

\vspace{0.1cm}

\begin{align*}
& Learning \\
x_t & = \overset{\textcolor{blue}{-}}{\E_a(\pi)}  + \overset{\textcolor{blue}{-}}{\E_b(\pi)}  
+ \overbrace{\overset{\textcolor{blue}{-}}{\E_a(x)}   + \overset{+}{\E_b(x)}}^{+ \; \text{since} \; f_a < f_b}   \\  
\pi_t & = \underbrace{\overset{+}{\E_a(\pi)} + \overset{\textcolor{blue}{-}}{\E_b(\pi)}}_{+ \; \text{since} \; \kappa \; \text{tiny}} 
+ \overset{+}{\E_a(x)} + \overset{+}{\E_b(x)}
\end{align*}
\end{multicols}

Where $f_a$ and $f_b$ denote long-horizon expectations and are given by
\begin{align}
f_a(t)  \equiv  \hat{\E}_t\sum_{T=t}^{\infty} (\alpha\beta)^{T-t } \begin{bmatrix} \pi_{T+1} \\ x_{T+1} \\ i_{T+1} \end{bmatrix} \quad \quad \quad \quad f_b(t)  \equiv \hat{\E}_t\sum_{T=t}^{\infty} (\beta)^{T-t } \begin{bmatrix} \pi_{T+1} \\ x_{T+1} \\ i_{T+1} \end{bmatrix}
\end{align}
\begin{equation}
f_a(t) = \frac{1}{1-\alpha\beta}\begin{bmatrix} \bar{\pi}_t \\ 0 \\ 0 \end{bmatrix} + b(I_4 - \alpha\beta h_x)^{-1}s_t \quad \quad \quad f_b(t) = \frac{1}{1-\beta}\begin{bmatrix} \bar{\pi}_t \\ 0 \\ 0 \end{bmatrix}  + b(I_4 - \beta h_x)^{-1}s_t  \label{fafb_obvious}
\end{equation}

(And $b = g_x h_x$, where $h_x$ is the state transition matrix and $g_x$ is the observation matrix from the RE model solution.)


\end{document}





