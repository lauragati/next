\documentclass{beamer}
\usepackage[english]{babel}
\usepackage[latin1]{inputenc}
%\usepackage[T1]{fontenc} % don't know what this does: it sets font encoding
%\usepackage{gfsartemisia} % this chooses the font
%%\usepackage{txfonts}
%% the above two seem to give the same, the below two as well
%%\usepackage{accanthis} 
%%\usepackage{concmath}
% Another way to go is the following
%  The available themes are: structurebold, structurebolditalic, structuresmallcapsserif, structureitalicsserif, serif and default.
%\usefonttheme{default}
%\usefonttheme{serif}
%\usefonttheme{structurebold}
%%\usefonttheme{structurebolditalic} % .sty wasn't found
%\usefonttheme{structuresmallcaps}
\usefonttheme{structuresmallcapsserif}
%%%%%%\usefonttheme{structureitalicsserif} % .sty wasn't found
%%%%%\usefonttheme{serif} % the font theme has to correspond to the font used
%%%%\usefonttheme{structurebold} % it seems you can use a combo of these
%%%%% Here comes the font command itself:
%%%%% The available fonts depend on your LATEX installation, the most common are: mathptmx, helvet, avat, bookman, chancery, charter, culer, mathtime, mathptm, newcent, palatino, pifont and utopia.
%%%%% I also have ccfonts
%%%%\usepackage{helvet}
%%%%\usepackage[T1]{fontenc}  % some fonts, like concrete math, require a specific font encoding (concrete math requires T1)
\usepackage{subfigure}
\usepackage[default]{lato}

\usepackage{amssymb}
\usepackage{amsmath}
\usepackage{booktabs}
\usepackage{verbatim}
\usepackage{caption}
\usepackage{float}
\usepackage{csquotes}
\usepackage{sansmathaccent}
\usepackage{subfigure}
\usepackage{multicol}
\pdfmapfile{+sansmathaccent.map}
\usepackage{pgfplots,tikz}
\usetikzlibrary{tikzmark,calc}
\usepackage{overpic}
\usepackage{color,soul}
\usepackage{stackengine}

\def \myFigPath {../../figures/} 
\def \myTablePath {../../tables/} 

\beamertemplatenavigationsymbolsempty % this removes the blue navigation bar that used to be on the bottom right
\setbeamersize{text margin left=5mm,text margin right=12mm} 
%
\font\reali=msbm10 at 12pt

%


\author[]{Laura G\'ati}
% note: in the [] you can put a short name to be displayed in the footer

\institute[]{Boston College}
% note: in the [] you can put a short affiliation to be displayed in the footer

\title[]{Monetary Policy \& Anchored Expectations \\
Why should we care?}
% note: in the [] you can put a short title to be displayed in the footer

\date[]{October 3, 2019}
% note: in the [] you can put a short date to be displayed in the footer

%\usetheme{CambridgeUS} % CambridgeUS, Madrid
\usetheme{Madrid}
\usecolortheme{dove}



\AtBeginSection[]
{
  \begin{frame}<beamer>
    \frametitle{Structure of talk}
    \tableofcontents[currentsection]
  \end{frame}
}

% New Commands
\newcommand{\real}{\hbox{\reali R}}
\newcommand{\realp}{\hbox{\reali R}_{\scriptscriptstyle +}}
\newcommand{\realpp}{\hbox{\reali R}_{\scriptscriptstyle ++}}
\newcommand{\R}{\mathbb{R}}
\DeclareMathOperator{\E}{\mathbb{E}}
\DeclareMathOperator{\argmin}{arg\,min}
\newcommand\w{3.0in}
\newcommand\wnum{3.0}
\def\myFigWidth{5.3in}
\def\mySmallerFigWidth{2.1in}
\def\myEvenBiggerFigScale{0.8}
\def\myPointSixFigScale{0.6}
\def\myBiggerFigScale{0.4}
\def\myFigScale{0.3}
\def\mySmallFigScale{0.25}
\def\mySmallerFigScale{0.18}
\def\myTinyFigScale{0.16}
\def\myPointFourteenFigScale{0.14}
\def\myTinierFigScale{0.12}
\def\myEvenTinierFigScale{0.10}
\def\myCrazyTinyFigScale{0.09}

\newtheorem{prop}{Proposition}
\newtheorem*{remark}{Remark}

\newcommand\numberthis{\addtocounter{equation}{1}\tag{\theequation}} % this defines a command to make align only number this line

% creating colored lines for legends
% green solid
\newcommand{\greenline}{\raisebox{2pt}{\tikz{\draw[-,black!40!green,solid,line width = 1.5pt](0,0) -- (5mm,0);}}}
% red solid, dashed, dotted
\newcommand{\redline}{\raisebox{2pt}{\tikz{\draw[-,red,solid,line width = 1.5pt](0,0) -- (5mm,0);}}}
\newcommand{\reddashedline}{\raisebox{2pt}{\tikz{\draw[-,red,dashed,line width = 1.5pt](0,0) -- (5mm,0);}}}
\newcommand{\reddottedline}{\raisebox{2pt}{\tikz{\draw[-,red,densely dotted,line width = 1.5pt](0,0) -- (5mm,0);}}}

% blue solid and dashed
\newcommand{\blueline}{\raisebox{2pt}{\tikz{\draw[-,blue,solid,line width = 1.5pt](0,0) -- (5mm,0);}}}
\newcommand{\bluedashedline}{\raisebox{2pt}{\tikz{\draw[-,blue,dashed,line width = 1.5pt](0,0) -- (5mm,0);}}}
% black solid and dashed
\newcommand{\blackline}{\raisebox{2pt}{\tikz{\draw[-,black,solid,line width = 1.5pt](0,0) -- (5mm,0);}}}
\newcommand{\blackdashedline}{\raisebox{2pt}{\tikz{\draw[-,black,dashed,line width = 1.5pt](0,0) -- (5mm,0);}}}


% These are my colors -- there are many like them, but these ones are mine (Paul Goldsmith-Pinkham's)
\definecolor{blue_PGP}{RGB}{0,114,178}
\definecolor{red_PGP}{RGB}{213,94,0}
\definecolor{yellow_PGP}{RGB}{240,228,66}
\definecolor{green_PGP}{RGB}{0,158,115}
%% I use a beige off white for my background 
\definecolor{MyBackground}{RGB}{255,253,218}
%% Uncomment this if you want to change the background color to something else 
%\setbeamercolor{background canvas}{bg=MyBackground} % this is a very nice beige
%% Change the bg color to adjust your transition slide background color! 
%\newenvironment{transitionframe}{
%\setbeamercolor{background canvas}{bg=yellow} \begin{frame}}{
%\end{frame} }

%\setbeamercolor{frametitle}{fg=blue_PGP} 
%\setbeamercolor{title}{fg=black} 
%\setbeamertemplate{footline}[frame number]  % this sets the footer the same color as the background
%\setbeamertemplate{navigation symbols}{} % don't know what this does
%\setbeamercolor{button}{bg=MyBackground,fg=blue_PGP}

% footnote suppressing number
\newcommand\blfootnote[1]{%
  \begingroup
  \renewcommand\thefootnote{}\footnote{#1}%
  \addtocounter{footnote}{-1}%
  \endgroup
}

% try to force appendix to be numbered differently
\newcommand{\backupbegin}{
   \newcounter{framenumberappendix}
   \setcounter{framenumberappendix}{\value{framenumber}}
}
\newcommand{\backupend}{
   \addtocounter{framenumberappendix}{-\value{framenumber}}
   \addtocounter{framenumber}{\value{framenumberappendix}} 
}

%%%%%%%%%%%%            BEGIN DOCUMENT         %%%%%%%%%%%%%%%%%%%%%
\begin{document}


\begin{frame}

\maketitle


\end{frame}



%%%%%%% Slide %%%%%%
\begin{frame}
	\frametitle{In the transition to RE, dynamics deviate from RE}

\begin{enumerate}
\item Once expectations have converged to RE, no difference between RE and AM
\item[$\rightarrow$] The difference is in the transition. Can't assume transition away because that $=$ assuming that all agents are born with knowledge of the model

\


\item What's happening in the transition and why do we care?
\begin{itemize}
\item In NK model, current variables depend on expectations of future variables (self-referentiality)

\

\item In transition, AM expectations do not coincide with RE expectations

\

\item[$\Rightarrow$] Law of motion of endogenous variables will be different!
\end{itemize}

\end{enumerate}



\end{frame}
%%%%%%%%%%%%%%%%%

%%%%%%% Slide %%%%%%
\begin{frame}
	\frametitle{Monetary policy based on RE gets things wrong}

\begin{enumerate}
\item[] Monetary policy based on the RE model will in the transition
\begin{enumerate}
\item Have the wrong expectations 


\


\item Conduct policy based on the wrong model of the endogenous variables (wrong law of motion)

\end{enumerate}

\

\item[] More concretely, it will
\begin{enumerate}
\item respond to shocks according to Taylor-rule and states of the RE model (none in NK)


\


\item will therefore ignore that the economy's law of motion is characterized by a novel state variable: expectations

\

\item and use the wrong optimal Taylor-rule parameters (CGG parameters optimal for RE law of motion)

\end{enumerate}

\item[$\Rightarrow$] smooth the wrong output gap, unemployment, inflation
\end{enumerate}

\end{frame}
%%%%%%%%%%%%%%%%%

%%%%%%% Slide %%%%%%
\begin{frame}
	\frametitle{Concrete example}

\begin{enumerate}
\item[] Uncertainty shock: $x\downarrow$, econ otherwise in a boom
\begin{enumerate}
\item If state variable beliefs is anchored (at Fed's target) and it's been there for a while with no sign of moving \\$\rightarrow$ RE NK dynamics is a good model, Fed can base decision of whether to lower $i\downarrow$ on business cycle indicators ($u, \pi$). \\
Assume: no need to lower interest rates.


\


\item If beliefs far below target consistently for a while, then that will amplify $x\downarrow$ (self-referentiality) \\
$\rightarrow$ need to $i\downarrow$ a lot

\

\item Key: if expectations anchored, but drifting down, then even though the RE dynamics is a good description, Fed still needs to make sure that it stays that way: \\
$i\downarrow$ \textbf{only to keep expectations anchored}

\end{enumerate}
\end{enumerate}

\end{frame}
%%%%%%%%%%%%%%%%%







%%%%%%%%%%%%%%%%%%%%%%%%%%%%%%%%%%%%%%%%%%%%%%%%%%%%%%%%%%%%%%%%%%%%%%%%
%%%%%%%%                     APPENDIX  
%%%%%%%%%%%%%%%%%%%%%%%%%%%%%%%%%%%%%%%%%%%%%%%%%%%%%%%%%%%%%%%%%%%%%%%%
%\appendix
%\backupbegin
%%%%%%%%% Slide %%%%%%
%\begin{frame}
%	\frametitle{Derivations}
%	\label{derivations}
%
%Household FOCs
% \begin{align*}
%\hat{C}_{t}^i & = \hat{\E}^i_t \hat{C}_{t+1}^i - \sigma(\hat{i}_{t} -\hat{\E}^i_t \hat{\pi}_{t+1})    \label{EE} \numberthis \\
%\\
%\hat{\E}^i_t \sum_{s=0}^{\infty}\beta^s \hat{C}^i_t& =\omega^i_t + \hat{\E}^i_t \sum_{s=0}^{\infty}\beta^s \hat{Y}^i_t \label{IBC} \numberthis
%\end{align*}
%where a hat denotes log-linear approximation and $\omega_t^i \equiv \frac{(1+i_{t-1})B^i_{t-1} }{P_t Y^*}$.
%
%\begin{enumerate}
%\item Solve (\ref{EE}) backward to some date $t$, take expectations at $t$ 
%\item Sub in (\ref{IBC})
%\item Aggregate over households $i$
%\item[$\rightarrow$] Obtain (\ref{prestons18})
%\end{enumerate}
%
%
%
%\hyperlink{NK}{\beamerreturnbutton{Return}}	
%
%
%\end{frame}
%%%%%%%%%%%%%%%%%%%
%
%%%%%%%%% Slide %%%%%%
%\begin{frame}
%	\frametitle{Compact notation}
%	\label{compact}
%
% \begin{align*}
%z_t & = A_1 f_{a,t} + A_2 f_{b,t} + A_3 s_t \label{LOM_LR} \numberthis \\
%\\
%s_t & = P s_{t-1} + \epsilon_t \label{exog} \numberthis
%\end{align*}
%where
%
%\begin{equation}
% z_t \equiv \begin{pmatrix} \pi_t \\ x_t \\ i_t
% \end{pmatrix} 
% \quad \quad \quad 
%  s_t  \equiv \begin{pmatrix} r_t^n \\ \bar{i}_t \\ u_t 
% \end{pmatrix} 
%\end{equation}
%and
%
%  \begin{align}
%f_{a,t}  \equiv  \hat{\E}_t\sum_{T=t}^{\infty} (\alpha\beta)^{T-t } z_{T+1} \quad \quad \quad \quad f_{b,t}  \equiv \hat{\E}_t\sum_{T=t}^{\infty} (\beta)^{T-t } z_{T+1} \label{fafb}
%\end{align}
%
%\hyperlink{NK}{\beamerreturnbutton{Return}}	
%
%
%\end{frame}
%%%%%%%%%%%%%%%%%%%
%%
%%%%%%%%% Slide %%%%%%
%%\begin{frame}
%%\frametitle{A beamer button template, how to get back to main text}
%%\label{Steps}
%%
%%\begin{equation}
%%D = \begin{bmatrix}
%%d_{11} & \gamma_{12} & \gamma_{13} & d_{14} & \cdots \\
%%d_{21} & \gamma_{22} & \gamma_{23} & d_{24} & \cdots \\
%%\vdots & \vdots & \vdots & \ddots & \vdots 
%%\end{bmatrix}
%%\end{equation}
%%
%%\hyperlink{calcCBloss}{\beamerreturnbutton{Return}}	
%%\end{frame}
%%%%%%%%%%%%%%%%%%%
%
%\backupend


\end{document}
