\documentclass[11pt]{beamer}
\usepackage[english]{babel}
\usepackage[latin1]{inputenc}
\usepackage{tikz}
\usetikzlibrary{calc}
\usepackage{subfigure}
\usepackage{amssymb}
\usepackage{amsmath}
\usepackage{booktabs}
\usepackage{verbatim}
\usepackage{caption}
\usepackage{float}
\usepackage{csquotes}
\usepackage{sansmathaccent}
\usepackage{subfigure}
\usepackage{multicol}
\pdfmapfile{+sansmathaccent.map}
\usepackage{pgfplots,tikz}
\usetikzlibrary{tikzmark,calc}
\usepackage{overpic}
\usepackage{color,soul}
\usepackage{stackengine}

% Layout and style
\usefonttheme{serif}
\usepackage{palatino} 

\usetheme{Madrid}
\usecolortheme{dove}

\def \myFigPath {../../../figures/} 
\def \myTablePath {../../../tables/} 

%%%% PROGRESS BAR %%%%%%%%%%%%%%%%%%%%%%%%%%%%%%%%%%%%%%%%%%
\definecolor{pbblue}{HTML}{0A75A8}% filling color for the progress bar
\definecolor{pbgray}{HTML}{575757}% background color for the progress bar

% a set of nice blue, nice red and nice green
\definecolor{dodgerblue}{RGB}{16,78,139}
\definecolor{aquamarinegreen}{RGB}{69,139,116}
\definecolor{brownredlight}{RGB}{238,59,59}
\definecolor{brownreddark}{RGB}{205,51,51}



\makeatletter
\def\progressbar@progressbar{} % the progress bar
\newcount\progressbar@tmpcounta% auxiliary counter
\newcount\progressbar@tmpcountb% auxiliary counter
\newdimen\progressbar@pbht %progressbar height
\newdimen\progressbar@pbwd %progressbar width
\newdimen\progressbar@tmpdim % auxiliary dimension

\progressbar@pbwd=1.15\linewidth % change width of progress bar
\progressbar@pbht=0.5ex % change thickness of progress bar

% the progress bar
\def\progressbar@progressbar{%

    \progressbar@tmpcounta=\insertframenumber
    \progressbar@tmpcountb=\inserttotalframenumber
    \progressbar@tmpdim=\progressbar@pbwd
    \multiply\progressbar@tmpdim by \progressbar@tmpcounta
    \divide\progressbar@tmpdim by \progressbar@tmpcountb

  \begin{tikzpicture}[rounded corners=2pt,very thin]

    \shade[top color=pbgray!10,bottom color=pbgray!10,middle color=pbgray!10]
      (0pt, 0pt) rectangle ++ (\progressbar@pbwd, \progressbar@pbht);

      \shade[draw=dodgerblue,top color=dodgerblue,bottom color=dodgerblue,middle color=dodgerblue] %
        (0pt, 0pt) rectangle ++ (\progressbar@tmpdim, \progressbar@pbht);

% ADD COMPLETION PERCENTAGE
%    \draw[color=normal text.fg!50]  
%      (0pt, 0pt) rectangle (\progressbar@pbwd, \progressbar@pbht) 
%        node[pos=0.5,color=normal text.fg] {\textnormal{%
%             \pgfmathparse{\insertframenumber*100/\inserttotalframenumber}%
%             \pgfmathprintnumber[fixed,precision=2]{\pgfmathresult}\,\%%
%        }%
%    };
  \end{tikzpicture}%
}

\addtobeamertemplate{headline}{}
{%
  \begin{beamercolorbox}[wd=\paperwidth,ht=4ex,center,dp=1ex]{white}%
    \progressbar@progressbar%
  \end{beamercolorbox}%
}
\makeatother

%%%% PROGRESS BAR %%%%%%%%%%%%%%%%%%%%%%%%%%%%%%%%%%%%%%%%%%


% footnote suppressing number
\newcommand\blfootnote[1]{%
  \begingroup
  \renewcommand\thefootnote{}\footnote{#1}%
  \addtocounter{footnote}{-1}%
  \endgroup
}

% try to force appendix to be numbered differently
\newcommand{\backupbegin}{
   \newcounter{framenumberappendix}
   \setcounter{framenumberappendix}{\value{framenumber}}
}
\newcommand{\backupend}{
   \addtocounter{framenumberappendix}{-\value{framenumber}}
   \addtocounter{framenumber}{\value{framenumberappendix}} 
}


% Set colors of butthons
\setbeamercolor{button}{bg=dodgerblue,fg=white}

% this removes the blue navigation bar that used to be on the bottom right
\beamertemplatenavigationsymbolsempty 
\setbeamersize{text margin left=5mm,text margin right=5mm} 


\AtBeginSection[]
{
  \begin{frame}<beamer>
    \frametitle{Structure of talk}
    \tableofcontents[currentsection]
  \end{frame}
}
\useinnertheme{circles}

% resets the bullet points of enumerate and itemize to defaults
\setbeamertemplate{itemize items}[bullet]
\setbeamertemplate{enumerate items}[default]

% this removes the ugly bullets for the table of contents and just adds numbers
\setbeamertemplate{section in toc}{\inserttocsectionnumber.~\inserttocsection}

\newcommand\numberthis{\addtocounter{equation}{1}\tag{\theequation}} % this defines a command to make align only number this line

% Colored lines for legends
% green solid
\newcommand{\greenline}{\raisebox{2pt}{\tikz{\draw[-,black!40!green,solid,line width = 1.5pt](0,0) -- (5mm,0);}}}
% red solid, dashed, dotted
\newcommand{\redline}{\raisebox{2pt}{\tikz{\draw[-,red,solid,line width = 1.5pt](0,0) -- (5mm,0);}}}
\newcommand{\reddashedline}{\raisebox{2pt}{\tikz{\draw[-,red,dashed,line width = 1.5pt](0,0) -- (5mm,0);}}}
\newcommand{\reddottedline}{\raisebox{2pt}{\tikz{\draw[-,red,densely dotted,line width = 1.5pt](0,0) -- (5mm,0);}}}

% blue solid and dashed
\newcommand{\blueline}{\raisebox{2pt}{\tikz{\draw[-,blue,solid,line width = 1.5pt](0,0) -- (5mm,0);}}}
\newcommand{\bluedashedline}{\raisebox{2pt}{\tikz{\draw[-,blue,dashed,line width = 1.5pt](0,0) -- (5mm,0);}}}
% black solid and dashed
\newcommand{\blackline}{\raisebox{2pt}{\tikz{\draw[-,black,solid,line width = 1.5pt](0,0) -- (5mm,0);}}}
\newcommand{\blackdashedline}{\raisebox{2pt}{\tikz{\draw[-,black,dashed,line width = 1.5pt](0,0) -- (5mm,0);}}}


% New Commands
\newcommand{\real}{\hbox{\reali R}}
\newcommand{\realp}{\hbox{\reali R}_{\scriptscriptstyle +}}
\newcommand{\realpp}{\hbox{\reali R}_{\scriptscriptstyle ++}}
\newcommand{\R}{\mathbb{R}}
\DeclareMathOperator{\E}{\mathbb{E}}
\DeclareMathOperator{\argmin}{arg\,min}
\newcommand\w{3.0in}
\newcommand\wnum{3.0}
\def\myFigWidth{5.3in}
\def\mySmallerFigWidth{2.1in}
\def\myEvenBiggerFigScale{0.8}
\def\myPointSixFigScale{0.6}
\def\myBiggerFigScale{0.4}
\def\myFigScale{0.3}
\def\mySmallFigScale{0.25}
\def\mySmallerFigScale{0.18}
\def\myTinyFigScale{0.16}
\def\myPointFourteenFigScale{0.14}
\def\myTinierFigScale{0.12}
\def\myEvenTinierFigScale{0.10}
\def\myCrazyTinyFigScale{0.09}

\newtheorem{prop}{Proposition}
\newtheorem*{remark}{Remark}
\newtheorem{result}{Result}
%\newtheorem{lemma}{Lemma}
%\newtheorem{corollary}{Corollary}


\author[]{Laura G\'ati}
% note: in the [] you can put a short name to be displayed in the footer

\institute[]{Boston College}
% note: in the [] you can put a short affiliation to be displayed in the footer

\title[]{Monetary Policy \& Anchored Expectations \\
An Endogenous Gain Learning Model}
% note: in the [] you can put a short title to be displayed in the footer

\date[]{April 15, 2020}
% note: in the [] you can put a short date to be displayed in the footer

%%%%%%%%%%%%            BEGIN DOCUMENT         %%%%%%%%%%%%%%%%%%%%%

\begin{document}

\begin{frame}[plain] % use the argument plain here to remove the progress bar

\maketitle

\end{frame}

%%%%%%% Slide %%%%%%
\begin{frame}
	\frametitle{Puzzling Fed behavior fall 2019}

\begin{figure}[h!]
\subfigure[Unemployment rate, \%]{\includegraphics[scale = 0.12]{\myFigPath urate_2020_02_09}}
\hfil \subfigure[Fed funds rate target, upper limit, \%]{\includegraphics[scale = 0.12]{\myFigPath frr_2020_02_09}}
\subfigure[Market-based inflation expectations, 10 year, \% average]{\includegraphics[scale = 0.12]{\myFigPath epi10_2020_02_09}}
\end{figure}


\end{frame}
%%%%%%%%%%%%%%%%%
%%%%%%%%%%%%%%%%%%%%%%%%%%%%%%%%%%%%%%%%%%%%%%%%%%%%%%
\section{Model}
%%%%%%%%%%%%%%%%%%%%%%%%%%%%%%%%%%%%%%%%%%%%%%%%%%%%%%

	%%%%%%%% Slide %%%%%%
\begin{frame}
	\frametitle{Preview of results}
	
	\begin{enumerate}
	\item Two layers of new intertemporal tradeoffs 

	
	\
	
	\
	
	\item Optimal monetary policy time-inconsistent
	
	\
	
	\
	
	
	\item[$\rightarrow$] Illustrate analytically in special case: target criterion 
	
	\
	
	\
	\item Not today: short-run costs vs. long-run benefits of anchoring expectations

	\end{enumerate}


\end{frame}
%%%%%%%%%%%%%%%%%%


%%%%%%%% Slide %%%%%%
\begin{frame}
	\frametitle{Related literature}

\begin{itemize}
\item \textbf{Optimal monetary policy in New Keynesian models}
\item[] Clarida, Gali \& Gertler (1999), Woodford (2003)

\

\

\item \textbf{Econometric learning}
\item[] Evans \& Honkapohja (2001), Preston (2005), Moln\'ar \& Santoro (2014)

\

\

\item \textbf{Anchoring / endogenous gain} 
\item[] Carvalho et al (2019), Svensson (2015), Hooper et al (2019), Milani (2014)
\end{itemize}



\end{frame}
%%%%%%%%%%%%%%%%%%


%%%%%%%% Slide %%%%%%
\begin{frame}
	\frametitle{Expectations: $\hat{\E}$ instead of $\E$}

\begin{itemize}
\item If use $\E$ (rational expectations, RE) \\

\

Model solution 
 \begin{align}
 s_t & = h s_{t-1} + \epsilon_t \quad \quad \quad \epsilon_t \sim \mathcal{N}(\mathbf{0},\Sigma) \label{state} \\
 y_t & = g s_t \label{obs_RE}
 \end{align}


$s_t \equiv (r^n_t, u_t)' \quad $  (states) \\
$y_t \equiv (\pi_t, x_t, i_t)' \quad $ (jumps)

\

\item If use $\hat{\E} \rightarrow$ private sector does not know $g$ \\
$\rightarrow$ estimate using observed states \& knowledge of (\ref{state})

\

\item Households and firms don't know they are identical
\end{itemize}



\end{frame}
%%%%%%%%%%%%%%%%%%

%%%%%%%% Slide %%%%%%
\begin{frame}
	\frametitle{Conclusion}




\end{frame}
%%%%%%%%%%%%%%%%%%



%%%%%%%%%%%%%%%%%%%%%%%%%%%%%%%%%%%%%%%%%%%%%%%%%%%%%%%%%%%%%%%%%%%%%%%
%%%%%%%                     APPENDIX  
%%%%%%%%%%%%%%%%%%%%%%%%%%%%%%%%%%%%%%%%%%%%%%%%%%%%%%%%%%%%%%%%%%%%%%%
\appendix
\backupbegin

%%%%%%%% Slide %%%%%%
\begin{frame}[plain]  % remove progress bar from appendix
	\frametitle{Short-run costs, long-run benefits}
	\label{SRLRtradeoffs}

\
	
Assume Taylor rule and no concern for output gap stabilization	
	
\begin{equation*}
i_t = \psi_{\pi}\pi_t  \quad \quad \lambda_x = 0 
\end{equation*}

\begin{figure}[h!]
\subfigure[RE]{\includegraphics[scale = 0.14]{\myFigPath plot_sim_loss_loss_RE_params_psi_pi_1_5_psi_x_0_gbar_0_145_thetbar_4_thettilde_2_5_kap_0_8_lamx_0_lami_0_2020_02_09}}
\subfigure[Anchoring]{\includegraphics[scale = 0.14]{\myFigPath plot_sim_loss_loss_again_critCUSUM_constant_only_params_psi_pi_1_5_psi_x_0_gbar_0_145_thetbar_4_thettilde_2_5_kap_0_8_lamx_0_lami_0_2020_02_09}}
\caption{Central bank loss as a function of $\psi_{\pi}$}
%\floatfoot{}
\label{fig_loss}
\end{figure}	
	
\hfill \hyperlink{SRcosts}{\beamergotobutton{Short-run costs}}

\end{frame}
%%%%%%%%%%%%%%%%%%


%%%%%%%% Slide %%%%%%
\begin{frame}[plain]  % remove progress bar from appendix
	\frametitle{Functional forms for $\mathbf{g}$}
	\label{g}
\begin{itemize}
\item Smooth anchoring function
\begin{equation}
k_t = k_{t-1} - c + d fe_{t|t-1}^2
\end{equation}
$c,d > 0$

\

\item Kinked anchoring function
 \begin{align*}
k_t & = \begin{cases} \frac{1}{t} \quad \text{when} \quad \theta_t < \bar{\theta}  \\ k \quad \text{otherwise.}\numberthis
\end{cases} 
\end{align*}
$\theta_t$ criterion, $\bar{\theta}$ threshold value

\end{itemize}

\vfill 
\hyperlink{anchoring1}{\beamerreturnbutton{Return}}	

\end{frame}
%%%%%%%%%%%%%%%%%%

\begin{frame}[plain]  % remove progress bar from appendix
	\frametitle{Choices for criterion $\theta_t$}
	\label{g}
\begin{itemize}
\item Carvalho et al. (2019)'s criterion  
\begin{equation}
\theta_t^{CEMP} = \max | \Sigma^{-1} ( \phi_{t-1} - T(\phi_{t-1})) |
\end{equation}


$\Sigma$ variance-covariance matrix of shocks \\
$T(\phi)$ mapping from PLM to ALM

\

\

\item CUSUM-criterion
\begin{align}
\omega_t & =  \omega_{t-1} + \kappa k_{t-1}(fe_{t|t-1} fe_{t|t-1}'  -\omega_{t-1})\\
\theta_t^{CUSUM} & =  \theta_{t-1} + \kappa k_{t-1}(fe_{t|t-1}'\omega_t^{-1}fe_{t|t-1} -\theta_{t-1})
\end{align}

\

$\omega_t$ estimated forecast-error variance
\end{itemize}




\vfill

\hyperlink{anchoring1}{\beamerreturnbutton{Return}}	


\end{frame}
%%%%%%%%%%%%%%%%%%

\backupend % NEED TO CONCLUDE WITH THIS TO HAVE SEPARATE NUMBERING FOR APPENDIX

\end{document}