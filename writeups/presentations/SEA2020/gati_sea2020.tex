\documentclass[10pt]{beamer}
\usepackage[english]{babel}
\usepackage[latin1]{inputenc}
\usepackage{tikz}
\usetikzlibrary{calc}
\usepackage{subfigure}
\usepackage{amssymb}
\usepackage{amsmath}
\usepackage{booktabs}
\usepackage{verbatim}
\usepackage{caption}
\usepackage{float}
\usepackage{csquotes}
\usepackage{sansmathaccent}
\usepackage{subfigure}
\usepackage{multicol}
\pdfmapfile{+sansmathaccent.map}
\usepackage{pgfplots,tikz}
\usetikzlibrary{tikzmark,calc}
\usepackage{overpic}
\usepackage{color,soul}
\usepackage{stackengine}

% Layout and style
\usefonttheme{serif}
\usepackage{palatino} 

\usetheme{Madrid}
\usecolortheme{dove}

\def \myFigPath {../../../figures/} 
\def \myTablePath {../../../tables/} 


% footnote suppressing number
\newcommand\blfootnote[1]{%
  \begingroup
  \renewcommand\thefootnote{}\footnote{#1}%
  \addtocounter{footnote}{-1}%
  \endgroup
}

% try to force appendix to be numbered differently
\newcommand{\backupbegin}{
   \newcounter{framenumberappendix}
   \setcounter{framenumberappendix}{\value{framenumber}}
}
\newcommand{\backupend}{
   \addtocounter{framenumberappendix}{-\value{framenumber}}
   \addtocounter{framenumber}{\value{framenumberappendix}} 
}


% Set colors of buttons
\setbeamercolor{button}{bg=dodgerblue,fg=white}

% a set of nice blue, nice red and nice green
\definecolor{dodgerblue}{RGB}{16,78,139}
\definecolor{aquamarinegreen}{RGB}{69,139,116}
\definecolor{brownredlight}{RGB}{238,59,59}
\definecolor{brownreddark}{RGB}{205,51,51}

% this removes the blue navigation bar that used to be on the bottom right
\beamertemplatenavigationsymbolsempty 
\setbeamersize{text margin left=5mm,text margin right=5mm} 


\AtBeginSection[]
{
  \begin{frame}<beamer>
    \frametitle{Structure of talk}
    \tableofcontents[currentsection]
  \end{frame}
}
\useinnertheme{circles}

% resets the bullet points of enumerate and itemize to defaults
\setbeamertemplate{itemize items}[bullet]
\setbeamertemplate{enumerate items}[default]

% this removes the ugly bullets for the table of contents and just adds numbers
\setbeamertemplate{section in toc}{\inserttocsectionnumber.~\inserttocsection}

\newcommand\numberthis{\addtocounter{equation}{1}\tag{\theequation}} % this defines a command to make align only number this line


% New Commands
\newcommand{\real}{\hbox{\reali R}}
\newcommand{\realp}{\hbox{\reali R}_{\scriptscriptstyle +}}
\newcommand{\realpp}{\hbox{\reali R}_{\scriptscriptstyle ++}}
\newcommand{\R}{\mathbb{R}}
\DeclareMathOperator{\E}{\mathbb{E}}
\DeclareMathOperator{\argmin}{arg\,min}
\newcommand\w{3.0in}
\newcommand\wnum{3.0}
\newtheorem{prop}{Proposition}
\newtheorem*{remark}{Remark}
\newtheorem{result}{Result}


%%%%%%%%%%%%%%%%%%%%
% DEFINE ALL VARIABLE NUMBERS HERE
%%%%%%%%%%%%%%%%%%%%

% numbers update: 17 Sept 2020 using ``complete'' or ``21 Sept" Estimation (N=1000) from Materials 44

% Empirical numbers
% mean estimated alpha
\def\meanalph{(0.82;    0.61;    0;    0.33;    0.45)}
\def\calibCalph{(0.8; 0.4; 0; 0.4; 0.8)}
% for 1% fe, you get a gain of
\def\oneppFEgives{0.08}
% pp forecast error that unanchors (that causes a gain of 0.05)
\def\ppFEunanchors{0.5 }
\def\bpNegFEunanchors{25}
\def\bpPosFEunanchors{50}


% raise/lower interest rate by x bp when when LR-exp move by y bp
\def\movepibar{5 }
\def\movei{12 }

%%%%%%%%%%%%%%%%%%%%
% DEFINE ALL FIGNAMES HERE
%%%%%%%%%%%%%%%%%%%%

% Market-based expectations (aren't used in prezi right now)
\def\fignameMarketEPi{epi10_2020_06_04}
% cleaned from liq premium
\def\fignameMarketEPiCleaned{cleaned_epi10_2020_07_28}
\def\fignameMarketEPiMoreHorizons{epi_be_in_data_command_anchoring_in_data_14_Sep_2020_18_12_59}


% SPF long-run-e
\def\fignameSPFLiv{epi_in_data_command_anchoring_in_data_14_Sep_2020_18_12_59}
\def\fignameRolling{rolling_overlapping_command_anchoring_in_data_individual_23_Sep_2020_21_04_48}
\def\fignameRollingPi{rolling_overlapping_pi_command_anchoring_in_data_individual_25_Sep_2020_10_37_43} % regressing on pi
\def\fignameRollingPCEcore{rolling_overlapping_pce_core_command_anchoring_in_data_individual_PCE_20_Oct_2020_11_16_01}
\def\fignameRollingControlPi{rolling_overlapping_pi_controls_pilevel_command_anchoring_in_data_individual_20_Oct_2020_11_42_26}
\def\fignameRollingControlPiIQR{rolling_overlapping_pi_controls_pilevel_iqrcommand_anchoring_in_data_individual_20_Oct_2020_11_42_26}


% Unanchoring in the data
%\def\fignameEpiData{epi_in_data_command_anchoring_in_data_29_Aug_2020_12_10_41}
%\def\fignameFeData{fe_in_data_command_anchoring_in_data_29_Aug_2020_12_10_41}
\def\fignameRegressionPlot{regression_plot_command_anchoring_in_data_14_Sep_2020_18_12_59}
\def\fignameSCEdistrib{SCE_distrib_topbottom_command_anchoring_in_data_14_Sep_2020_18_12_59}
\def\fignameLivIQR{Livingston_IQR_command_anchoring_in_data_14_Sep_2020_18_12_59}
\def\fignamePCEcore{PCE_core_target_command_anchoring_in_data_19_Oct_2020_15_24_30}

% Estimated coefficients alpha
\def\fignameAlphaHat{alph_opt_N_100_nfe_5_gridspacing_manual_Wdiffs2_100000_Wmid_1000_Nsimulations_command_sigmas_15_Oct_2020_15_02_07}
\def\fignameGdotFe{gdot_feN_100_nfe_5_gridspacing_manual_Wdiffs2_100000_Wmid_1000_Nsimulations_command_sigmas_13_Nov_2020_13_50_48}
% Autocovariogram
\def\fignameAutocov{autocovariogram_N_100_nfe_5_gridspacing_manual_Wdiffs2_100000_Wmid_1000_Nsimulations_command_sigmas_17_Sep_2020_12_05_56}

\def\fignameFeSPF{fe_SPF_command_anchoring_in_data_19_Sep_2020_11_45_59}
\def\fignameFeSPFhist{fe_SPF_hist_command_anchoring_in_data_19_Sep_2020_11_45_59}
\def\fignameGainSPF{gain_SPF_command_anchoring_in_data_11_Oct_2020_11_08_39}
\def\fignameGainSPFhist{gain_SPF_hist_command_anchoring_in_data_11_Oct_2020_11_08_39}
\def\fignameFeGainGdotFeSPF{fe_gain_gdot_fe_SPF_command_anchoring_in_data_18_Oct_2020_13_11_56}
%\def\fignameFeGainGdotFeSPF{fe_gain_gdot_fe_SPF_command_anchoring_in_data_20_Oct_2020_19_30_43} % pushes gain and fe further apart


% Comparative statics of optimal policy
% di/dpibar
% not annualized
%\def\fignameDiDpibar{analyze_opt_policy_ip17_Sep_2020}
\def\fignameDiDpibar{analyze_opt_policy_ip18_Oct_2020} % equals Sep 17, just prettier axis labels
\def\fignameHistPib{analyze_opt_policy_hist_pib_18_Oct_2020}
% not annualized with stabilizing factor of 2.6 (25 Sept 2020)
\def\fignameDiDpibarStab{analyze_opt_policy_ip25_Sep_2020}


% Observables in PEA for TR
% annualized
%\def\fignamePEAobsTR{implement_anchTC_obs_TR_approx_40q_17_Sep_2020}
% not annualized
%\def\fignamePEAobsTR{implement_anchTC_obs_TR_approx_40q_17_Sep_2020_19_29_35}
% not annualized, with psi_pi=1.1083 (opt)
\def\fignamePEAobsTR{implement_anchTC_obs_TR_approx24_Sep_2020_09_32_19}

%
% Observables in PEA for anchoring
% annualized
% cutting to 40 periods to 
%\def\fignamePEAobsAnch{implement_anchTC_obs_approx_40q19_Sep_2020_09_30_01}
% not annualized
%\def\fignamePEAobsAnch{implement_anchTC_obs_approx_40q17_Sep_2020_19_19_19}
% not annualized, with psi_pi=1.1083 (opt) and knowTR=1
\def\fignamePEAobsAnch{implement_anchTC_obs_approx24_Sep_2020_09_35_43}



% Compare PEA and VFI policy for sequence of shocks X1 (rng 2)
% annualized
\def\fignamePEAvsVFIfirstX{compare_value_pea_results_approx_value_outputs_approx17_Sep_2020_14_01_16_pea_outputs_approx17_Sep_2020_13_47_33_pretty_19_Sep_2020_09_30_01}
% not annualized
%\def\fignamePEAvsVFIfirstX{compare_value_pea_results_approx_value_outputs_approx17_Sep_2020_14_01_16_pea_outputs_approx17_Sep_2020_13_47_33_pretty_40q_17_Sep_2020_19_23_04}

% Central bank loss as a function of psi_pi in RE vs. anchoring
% Note 30 July 2020: objective_CB_approx.m should use sim_learn_approx_univariate.m to use the univariate anchoring function, but it leads to a bunch of explosions, so I'm not using the correction.
% Note 23 August 2020: now using sim_learn_approx_univariate.m and it simply leads to lots of explosions (I guess alpha and sig are high, and they can't tolerate psi_pi>1.4). But that's fine.
% Note 25 August: update with convention: RE red, anchoring blue.
% retired
%\def\fignameCBlossnilnil{plot_sim_loss_approx_pretty_losses_again_critsmooth_constant_only_pi_only_lamx0_lami0_2020_09_12}
%\def\fignameCBlossonenil{plot_sim_loss_approx_pretty_losses_again_critsmooth_constant_only_pi_only_lamx1_lami0_2020_09_12}
%\def\fignameCBlossnilone{plot_sim_loss_approx_pretty_losses_again_critsmooth_constant_only_pi_only_lamx0_lami1_2020_09_12}
%\def\fignameCBlossoneone{plot_sim_loss_approx_pretty_losses_again_critsmooth_constant_only_pi_only_lamx1_lami1_2020_09_12}
\def\fignameCBlossbaseline{plot_sim_loss_approx_pretty_losses_again_critsmooth_constant_only_pi_only_lamx0_05_lami0_2020_09_19}
\def\fignameCBlossbaselineWithOptimal{plot_sim_loss_approx_pretty_losses_with_optimal_again_critsmooth_constant_only_pi_only_lamx0_05_lami0_2020_09_19}
\def\fignameCBlossbaselineTwoY{plot_sim_loss_approx_2y_losses_with_optimal_again_critsmooth_constant_only_pi_only_lamx0_05_lami0_2020_10_14}

% IRFs anchored vs unanchored
% annualized
\def\fignameIRFanchored{RIR_anch_psi_pi1_5_command_IRFs_approx_pretty_2020_10_15}
\def\fignameIRFunanchored{RIR_unanch_psi_pi1_5_command_IRFs_approx_pretty_2020_10_15}
\def\fignameIRFanchUnanchTogether{RIR_together_psi_pi1_5_command_IRFs_approx_pretty_2020_10_15}
\def\fignameIRFanchUnanchTogetherCostPush{RIR_together_psi_pi1_5_command_IRFs_approx_pretty_2020_10_25}


% Cross-sectional average gains for various psi_pi
% not annualized
\def\fignameGainPsiSmall{gain_sim_psi_pi_1_01_command_IRFs_approx_pretty_2020_09_17}
\def\fignameGainPsiMedium{gain_sim_psi_pi_1_5_command_IRFs_approx_pretty_2020_09_17}
\def\fignameGainPsiBig{gain_sim_psi_pi_2_command_IRFs_approx_pretty_2020_09_17}

% IRFs unanchored for various psi_pi
% annualized
\def\fignameIRFpsipiSmall{RIR_unanch_psi_pi1_01_command_IRFs_approx_pretty_2020_10_15}
\def\fignameIRFpsipiMedium{RIR_unanch_psi_pi1_5_command_IRFs_approx_pretty_2020_10_15}
\def\fignameIRFpsipiBig{RIR_unanch_psi_pi2_command_IRFs_approx_pretty_2020_10_15}
\def\fignameIRFpsipiTogether{RIR_together_psi_pi2_command_IRFs_approx_pretty_together_2020_10_15}
\def\fignameIRFpsipiTogetherCostPush{RIR_together_psi_pi2_command_IRFs_approx_pretty_together_2020_10_25}


%%%%%%%%%%%%%%%%%%%%


\author[]{Laura G\'ati}
% note: in the [] you can put a short name to be displayed in the footer

\institute[]{Boston College}
% note: in the [] you can put a short affiliation to be displayed in the footer

\title[]{Monetary Policy \& Anchored Expectations \\
An Endogenous Gain Learning Model}
% note: in the [] you can put a short title to be displayed in the footer

\date[]{\\15\textsuperscript{th} End-of-Year Conference \\
\vspace{0.2cm}
Swiss Economists Abroad \\
\vspace{0.2cm}
December 22, 2020}
% note: in the [] you can put a short date to be displayed in the footer






%%%%%%%%%%%%            BEGIN DOCUMENT         %%%%%%%%%%%%%%%%%%%%%

\begin{document}

\begin{frame}[plain] % use the argument plain here to remove the progress bar

\maketitle

\end{frame}


%%%%%%%%%%%%%%%%


%%%%%%%% Slide %%%%%%
%\begin{frame}\label{motivation}
%	
%\begin{quote}
%Inflation that runs below its desired level can lead to an unwelcome fall in \textbf{longer-term inflation expectations}, which, in turn, can pull actual inflation even lower, resulting in an adverse cycle of ever-lower inflation and inflation expectations.
%[...]  \textbf{Well-anchored inflation expectations} are critical[.]  \\
%Jerome Powell, Chairman of the Federal Reserve \footnote{``New Economic Challenges and the Fed's Monetary Policy Review,''  August 27, 2020.} \\
%(Emphases added.)
%% Jackson Hole 
%\end{quote}	
%
%
%
%\end{frame}
%%%%%%%%%%%%%%%%%%



%%%%%%%% Slide %%%%%%
\begin{frame}
	\frametitle{Anchoring}
	\label{central_bankers}
	``Essential to anchor inflation expectations at some low level.''
	
	\vspace{0.2cm}
	
	\hfill ``We don't see a de-anchoring.''

\begin{figure}[h!]
\includegraphics[scale = 0.445]{\myFigPath bernanke}
\includegraphics[scale = 0.224]{\myFigPath yellen}
%\includegraphics[scale = 0.1]{\myFigPath bernanke2}
%\includegraphics[scale = 0.3]{\myFigPath powell}\\
\includegraphics[scale = 0.334]{\myFigPath powell2}
%\includegraphics[scale = 0.22]{\myFigPath draghi}
\includegraphics[scale = 0.22]{\myFigPath draghi2}
\label{central_bankers_talk}
\end{figure}

``Failure of the Fed to stably achieve its 2 percent target could de-anchor inflation expectations.''

	\vspace{0.4cm}
	
``Long-run inflation expectations [...] are not perfectly anchored in real economies; moreover, the extent to which they are anchored can change.''

%Yellen: ``But inflation expectations are now well anchored, possibly even a bit too low'' or ``Such well-anchored inflation expectations may have been fostered by the Fed's adoption of a numerical inflation objective of 2 percent in 2012'' or ``Well-anchored inflation expectations, in effect mitigate what might otherwise be painful conflicts between the Fed?s inflation and employment objectives.'' or ``a chronic failure of the Fed to stably achieve its 2 percent target could de-anchor inflation expectations on the downside''
%
%
%Draghi: ``we don't see a de-anchoring'' or ``On inflation expectations I think I've covered it before: of course we are concerned but we don't see signs of de-anchoring,''
%
%Powell: ``Policymakers came out of the Great Inflation era with a clear understanding that it was essential to anchor inflation expectations at some low level. '' or ``Well-anchored inflation expectations are critical for giving the Fed the latitude to support employment when necessary without destabilizing inflation.'' or ``Our statement emphasizes that our actions to achieve both sides of our dual mandate will be most effective if longer-term inflation expectations remain well anchored at 2 percent.''
%
%Bernanke: ``Long-run inflation expectations do vary over time. That is, they are not perfectly anchored in real economies; moreover, the extent to which they are anchored can change''
	\end{frame}
%%%%%%%%%%%%%%%%%


%%%%%%%% Slide %%%%%%
\begin{frame}
	\frametitle{This paper}

%\emph{How does a concern to anchor expectations affect the conduct of monetary policy?}

\

	
	\begin{enumerate}
%	\item[$\bullet$] How does a concern to anchor expectations affect the conduct of monetary policy?
%	
%	\
%	
%	\

	\item A model of expectations:  \\
	$\hookrightarrow$ \emph{unanchored expectations}: sensitivity of long-run expectations to short-run fluctuations 	
	
	\
	
	\
	
	\item Estimate how unanchoring takes place in data \\
	$\hookrightarrow$ quantify novel anchoring channel
	
	\
	
	\
	
	\item Analyze monetary policy \\
	$\hookrightarrow$ analytically and numerically using novel model disciplined by data \\

        \
	
	\
	
	\item Key takeaway \\
	$\hookrightarrow$ monetary policy anchors expectations to inflation target by not tolerating deviations in long-run expectations from target \\

	\end{enumerate}
	\end{frame}
%%%%%%%%%%%%%%%%%


%%%%%%%%% Slide %%%%%%
%\begin{frame}
%	\frametitle{The paper}
%	
%	\vspace{0.5cm}
%\setbeamercovered{transparent}	
%	\begin{enumerate}
%	\item The model
%	
%		\vspace{0.1cm}
%	
%		\begin{itemize}
%		\item Adaptive learning with endogenous gain \\ (Marcet \& Nicolini 2003, Carvalho et al 2019)
%		
%		\vspace{0.1cm}
%		
%		\item Novelty: continuous gain function
%		
%		\vspace{0.1cm}
%		
%		\item Embedded in general equilibrium New Keynesian model
%		
%		\end{itemize}
%		
%		\
%		
%	\item	 Estimation by simulated method of moments
%	
%		\vspace{0.1cm}
%	
%		\begin{itemize}
%		\item Expectations process nonlinear: more sensitive to large inflation surprises
%		\end{itemize}
%		
%		\
%		
%	\item Optimal Ramsey policy
%	
%	\vspace{0.1cm}
%	
%		\begin{itemize}
%		\item Solved analytically and numerically (parameterized expectations)
%		
%		\vspace{0.1cm}
%		
%		\item Interest rate responds aggressively to inflation when unanchored, accommodates inflation when anchored
%		\end{itemize}
%		
%		\
%		
%	\item Taylor rule
%		\begin{itemize}
%		\item Less aggressive on inflation than under rational expectations
%		\end{itemize}
%	\end{enumerate}
%
%	
%
%\end{frame}
%%%%%%%%%%%%%%%%%%%


%%%%%%%% Slide %%%%%%
\begin{frame}
	\frametitle{Related literature}

\

%\footnotesize{
\begin{itemize}
\item \textbf{Optimal monetary policy in the New Keynesian model}
\item[] Clarida, Gali \& Gertler (1999), Woodford (2003)

\

\

\item \textbf{Adaptive learning}
\item[] Evans \& Honkapohja (2001, 2006), Sargent (1999), Primiceri (2006), Lubik \& Matthes (2018), Bullard \& Mitra (2002), Preston (2005, 2008), Ferrero (2007), Moln\'ar \& Santoro (2014), Mele et al (2019), Eusepi \& Preston (2011), Milani (2007, 2014), Marcet \& Nicolini (2003)

\

\

\item \textbf{Anchoring and the Phillips curve} 
\item[] Goodfriend (1993), Svensson (2015), Hooper et al (2019), Afrouzi \& Yang (2020), Reis (2020), Hebden et al 2020, Gobbi et al (2019), Carvalho et al (2019)

%\
%
%\
%
%\item \textbf{Reputation} 
%\item[] Barro (1986), Cho \& Matsui (1995)
\end{itemize}

%}

\end{frame}
%%%%%%%%%%%%%%%%%%%

%%%%%%%%% Slide %%%%%%
%\begin{frame}
%	\frametitle{Structure of talk}
%
%% redefine margin locally - nice. finally it lines up with the TOC
%\setlength{\leftmargini}{12pt}
%\begin{enumerate}
%\item \hspace{-0.15cm} Model of anchoring expectations
%
%\vspace{0.75cm}
%
%\item \hspace{-0.15cm} Quantification of learning channel
%
%\vspace{0.75cm}
%
%\item \hspace{-0.15cm} Solving the Ramsey problem
%
%\vspace{0.75cm}
%
%\item \hspace{-0.15cm} Implementing optimal policy
%
%\vspace{0.75cm}
%
%\item \hspace{-0.15cm} Approximating optimal policy with a Taylor rule
%
%
%
%\end{enumerate}
%
%
%\end{frame}
%%%%%%%%%%%%%%%%%%%
%
%%%%%%%%%%%%%%%%%%%%%%%%%%%%%%%%%%%%%%%%%%%%%%%%%%%%%%%
%\section{Model of anchoring expectations}
%%%%%%%%%%%%%%%%%%%%%%%%%%%%%%%%%%%%%%%%%%%%%%%%%%%%%%%




%%%%%%%%% Slide %%%%%%
\begin{frame}
	\frametitle{Model overview}
	\label{model_overview}
\begin{itemize}
\item New Keynesian core: standard IS and Phillips curves	\hfill \hyperlink{HH}{\beamergotobutton{Microfoundations}}

 \begin{align}
x_t &=  -\sigma i_t +\textcolor{brownreddark}{\hat{\E}_t} \sum_{T=t}^{\infty} \beta^{T-t }\big( (1-\beta)x_{T+1} - \sigma(\beta i_{T+1} - \pi_{T+1}) +\sigma r_T^n \big)  \label{NKIS}  \\
\pi_t &= \kappa x_t +\textcolor{brownreddark}{\hat{\E}_t} \sum_{T=t}^{\infty} (\alpha\beta)^{T-t }\big( \kappa \alpha \beta x_{T+1} + (1-\alpha)\beta \pi_{T+1} + u_T\big) \label{NKPC} 
\end{align}
%\hfill \hyperlink{derivations}{\beamergotobutton{Derivations}} \hyperlink{ALMs}{\beamergotobutton{Actual laws of motion}}

Observables: $(\pi, x, i)$ inflation, output gap, interest rate \\
Exogenous states: $(r^n, u)$ natural rate and cost-push shock
\
\pause
\item \textcolor{brownreddark}{Novelty of the paper}: inflation expectations process 
\begin{align}
\textcolor{brownreddark}{\hat{\E}_{t}}\pi_{t+1} & = \textcolor{brownreddark}{\bar{\pi}_{t}} + \E_t\pi_{t+1}
%\bar{\pi}_{t}  &=\bar{\pi}_{t-1} +\mathbf{g}(f_{t|t-1}) \; f_{t|t-1}
\end{align}

$\E$: rational (model-consistent) expectations \\
$\textcolor{brownreddark}{\hat{\E}}$: nonrational expectations $\rightarrow$ long-run inflation expectations $ \textcolor{brownreddark}{\bar{\pi}_{t-1}} $

\end{itemize}



\end{frame}
%%%%%%%%%%%%%%%%%%%

%%%%%%%% Slide %%%%%%
\begin{frame}
	\frametitle{Evolution of long-run inflation expectations}
	\label{adaptive_learning}

 One-period ahead inflation forecast:
 \begin{equation}
 \hat{\E}_{t-1}\pi_{t} = \bar{\pi}_{t-1}+\E_{t-1}\pi_t
 \end{equation}

\
\pause
One-period ahead inflation forecast error:

 \begin{equation}
f_{t|t-1}  = \pi_{t} -  \hat{\E}_{t-1}\pi_{t}
 \end{equation}

\
\pause

\
$\rightarrow$ Update for long-run inflation expectations:	

\begin{equation}
\bar{\pi}_{t}  =\bar{\pi}_{t-1} +k_t  f_{t|t-1} 
\end{equation}
 
 \
 
 $k_t \in (0,1)$ learning gain \\
\vfill 

%\hfill \hyperlink{RLS}{\beamergotobutton{General algorithm}}







\end{frame}
%%%%%%%%%%%%%%%%%%


%%%%%%% Slide %%%%%%  	
\begin{frame}
	\frametitle{ Alternatives for the gain }
	\label{cgain_dgain}

%\begin{small}
\begin{enumerate}
\item Decreasing gain:

\begin{equation}
\bar{\pi}_{t}  =\bar{\pi}_{t-1} +\frac{1}{t} \; f_{t|t-1}
\end{equation}

 
 


 \only<3-5>{\textcolor{dodgerblue}{Optimal monetary policy: Mele et al (2019)}}

 \
 

 
 \
 
 \item Constant gain:
\begin{equation}
\bar{\pi}_{t}  =\bar{\pi}_{t-1} +k \; f_{t|t-1}
\end{equation}
 



  \only<3-5>{\textcolor{dodgerblue}{Optimal monetary policy: Moln\'ar \& Santoro (2014)}}

  
    
 \pause
 
 \
 
 \
 
 \item Endogenous gain:
 \begin{equation}
\bar{\pi}_{t}  =\bar{\pi}_{t-1} +\mathbf{g}(f_{t|t-1}) \; f_{t|t-1}
\end{equation}

 \only<4-5>{\textcolor{dodgerblue}{Marcet \& Nicolini (2003), Carvalho et al (2019)}}
 
  \only<5>{\textcolor{dodgerblue}{Optimal monetary policy: -}}

\vfill 

\hfill \hyperlink{ass_g}{\beamergotobutton{Assumptions on $\mathbf{g}(\cdot)$}}

\end{enumerate}
	

%\end{small}



\end{frame}
%%%%%%%%%%%%%%%%%




%%%%%%%%% Slide %%%%%%
%\begin{frame}
%	\frametitle{Calibration - parameters from the literature}
%
%\begin{center}
%\begin{table}
%\begin{tabular}{ c | c  | l }
%\hline
% $\beta$ & 0.98 & stochastic discount factor \\  \hline
% $\sigma$ & 1  & intertemporal elasticity of substitution \\  \hline
% $\alpha$ & 0.5 &  Calvo probability of not adjusting prices \\\hline
% $\kappa$ & 0.0842 &  slope of the Phillips curve \\\hline
% $\psi_{\pi} $& 1.5  & coefficient of inflation in Taylor rule\\\hline
% $\psi_x$ & 0.3   & coefficient of the output gap in Taylor rule  \\\hline 
%    $\sigma_r$ & 0.01 & standard deviation, natural rate shock  \\ \hline
%    $\sigma_i$ &  0.01  &standard deviation, monetary policy shock  \\ \hline
%    $\sigma_u$ & 0.5 & standard deviation, cost-push shock   \\ \hline  
% $\bar{g}$ & $0.145$  & initial value of the gain \\\hline 
%% $\lambda_x$ & 0.05 & weight on the output gap in central bank loss   \\ \hline  
%% $\rho_r$ & 0 &   persistence of natural rate shock \\ \hline 
%% $\rho_i$ & 0 &  persistence of monetary policy shock*  \\ \hline
%% $\rho_u$ & 0  &  persistence of cost-push shock  \\ \hline
%   \end{tabular}     
%   \label{calibration_lit}
% \end{table}
%\end{center}
%% TABLES WERE UPDATED 29 AUGUST 2020. 
%
%\
%
%\
%
% Chari et al (2000), Woodford (2003), Nakamura \& Steinsson (2008) \\
% Carvalho et al (2019)
%
%\
%
%\
%
%Parameterize $\mathbf{g}(\cdot)$ by estimating a flexible functional form \hfill \hyperlink{estimation_general}{\beamergotobutton{Estimating $\mathbf{g}(\cdot)$}}
% 
%
%\end{frame}
%
%%%%%%%%%%%%%%%%%%%
%
%%%%%%%%%%%%%%%%%%%%%%%%%%%%%%%%%%%%%%%%%%%%%%%%%%%%%%%%
%\section{Optimal monetary policy}
%%%%%%%%%%%%%%%%%%%%%%%%%%%%%%%%%%%%%%%%%%%%%%%%%%%%%%%%
%
%
%%%%%%%%% Slide %%%%%%
%\begin{frame}
%	\frametitle{Ramsey problem}\label{ramsey}
%	 \begin{align*}
%& \min_{ \{\pi_t, x_t, i_t, \bar{\pi}_{t-1}, k_t \}_{t=t_0}^{\infty}} \E_{t_0}\sum_{t=t_0}^{\infty} \beta^{t-t_0} (\pi_t^2  + \lambda_x x_t^2 )  \\
%& \text{s.t. model equations} \\
%& \text{s.t. evolution of expectations} 
%\end{align*}
%
%\
%
%\
%
%\begin{itemize}
%\item $\E$ is the central bank's (CB) expectation
%
%\
%
%\item Assumption: CB observes private expectations and knows the model
%
%\end{itemize}
%
%\vfill
% \hfill \hyperlink{anchTC}{\beamergotobutton{Analytical results}}
%
%\end{frame}
%%%%%%%%%%%%%%%%%%%
%
%%%%%%%%% Slide %%%%%%
%\begin{frame}[plain]  % remove progress bar from appendix
%	\frametitle{Numerical solution procedure}
%	
%\begin{itemize}
%\item[] Solve system of model equations + first-order conditions to Ramsey problem
%
%\
%
%\item[] For calibrated model with $\lambda_x = 0.05$ (Rotemberg \& Woodford 1997),
%
%
%\
%
%\item[] $\hookrightarrow$ solve using parameterized expectations algorithm
%
%\
%
%
%\item[] $\hookrightarrow$ obtain a cubic spline approximation to optimal policy function
%
%\
%
%\end{itemize}
%
%
%\end{frame}
%%%%%%%%%%%%%%%%%%%


%%%%%%%% Slide %%%%%%
\begin{frame}
	\frametitle{Optimal policy - responding to unanchoring }
	
\begin{figure}[h!]
\footnotesize
%\subfigure[$i(\bar{\pi}, \text{all other states at their means})$]{\includegraphics[scale=0.15]{\myFigPath \fignameDiDpibar}}
\stackunder[5pt]{\includegraphics[scale=0.15]{\myFigPath \fignameDiDpibar}}{$i(\bar{\pi}, \text{all other states at their means})$}
\stackunder[5pt]{\includegraphics[scale=0.15]{\myFigPath \fignameHistPib}}{Stabilizing $\bar{\pi}$}
\end{figure} 

\



\only<1-2>{5 bp movement in $\bar{\pi}$ $\rightarrow $ 250 bp movement in $i$}
 
 \

\only<2>{Mode: 0.3 bp movement in $\bar{\pi}$}


\end{frame}
%%%%%%%%%%%%%%%%%%

%%%%%%%% Slide %%%%%%
\begin{frame}
	\frametitle{Unanchoring causes volatility}\label{IRF1}

\begin{figure}[h!]
\includegraphics[scale = 0.22]{\myFigPath \fignameIRFanchUnanchTogetherCostPush}
\caption{Impulse responses after a cost-push shock when policy follows a Taylor rule}
\label{IRF}
\end{figure}	

\vfill
\vspace{-0.9cm}
\hfill \hyperlink{oscillatory}{\beamergotobutton{Why oscillatory?}}
\end{frame}
%%%%%%%%%%%%%%%%%%
%
%%%%%%%% Slide %%%%%%  	
%\begin{frame}
%	\frametitle{Volatility comes from endogenous gain}
%	\label{anchoring_vs_cgain}
%
%%\begin{small}
%\begin{itemize}
%\item Constant gain:
%\begin{equation}
%\bar{\pi}_{t}  =\bar{\pi}_{t-1} +k \; f_{t|t-1} \tag{13}
%\end{equation}
% 
% \
% 
% \
% 
% \item Endogenous gain:
% \begin{equation}
%\bar{\pi}_{t}  =\bar{\pi}_{t-1} +\textcolor<2>{brownreddark}{\mathbf{g}(f_{t|t-1})} \; f_{t|t-1} \tag{14}
%\end{equation}
%
%\
%
%\item[]<2>{\textcolor{brownreddark}{Shocks raise the gain $\rightarrow$ central bank needs to anchor}}
%\end{itemize}
%	
%
%%\end{small}
%
%
%
%\end{frame}
%%%%%%%%%%%%%%%%%%
%
%%%%%%%%% Slide %%%%%%
%\begin{frame}
%	\frametitle{... and from positive feedback} 
%	\label{pos_feedback}
%\begin{itemize}
%\item[] IS curve:	
% \begin{align*}
%\textcolor<2>{brownreddark}{x_t} &=  -\sigma i_t +\hat{\E}_t \sum_{T=t}^{\infty} \beta^{T-t }\big( (1-\beta)x_{T+1} - \sigma(\beta i_{T+1} - \textcolor<1>{brownreddark}{\pi_{T+1}}) +\sigma r_T^n \big)   %\\
%%\pi_t &= \kappa x_t +\hat{\E}_t \sum_{T=t}^{\infty} (\alpha\beta)^{T-t }\big( \kappa \alpha \beta x_{T+1} + (1-\alpha)\beta \pi_{T+1} + u_T\big) 
%\end{align*}
%
%\
%
%\
%
% 
%
%\begin{itemize}
%\item<1-2>  \textcolor<1>{brownreddark}{Unanchored $\rightarrow \bar{\pi}$ volatile $\rightarrow \hat{\E}_t \pi_{T+1}$ volatile}
%
%\
%
%\item<2> \textcolor<2>{brownreddark}{$\rightarrow x_t$ volatile}
%
%
%\end{itemize}
%
%
%
%\end{itemize}
%
%\vfill
% \hfill \hyperlink<2>{taylor_results}{\beamergotobutton{Results for Taylor rule}}
%
%\end{frame}
%%%%%%%%%%%%%%%%%%%

%%%%%%% Slide %%%%%%
\begin{frame}
	\frametitle{Conclusion}

\vspace{0.5cm}
	
\begin{itemize}
\item[] First theory of monetary policy for potentially unanchored expectations 

\

\item[] Estimation of novel unanchoring channel
\vspace{0.1cm}
	\begin{itemize}
	\item Expectations process nonlinear
%	\vspace{0.1cm}
%
%	\item Estimated gain time series: on average, people only use the last 8 quarters of data
	\end{itemize}
\


\item[] Monetary policy
\vspace{0.1cm}

	\begin{itemize}
	\item \textbf{Key:} Optimal policy aggressive when unanchored, accommodates otherwise
	\vspace{0.1cm}
	
	\item Degree of expectations unanchoring determines extent of smoothing shocks
	\vspace{0.1cm}



	\item Taylor rule less aggressive than under rational expectations
	\end{itemize}
	
\


\item[] Future work
\vspace{0.1cm}

\begin{itemize}
\item[$\hookrightarrow$]  How to anchor at zero-lower bound?
\vspace{0.1cm}

\item[$\hookrightarrow$]  Other applications: currency crises
\end{itemize}


%\item For a 0.1 pp drift in long-run inflation expectations, changes interest rate by \movei pp
\end{itemize}


\end{frame}
%%%%%%%%%%%%%%%%%
%
%
%%%%%%%%%%%%%%%%%%
%%         TEMPLATES
%%%%%%%%%%%%%%%%%%
%
%%%%%%%%% Slide %%%%%%
%%\begin{frame}
%%	\frametitle{Slide template}
%%
%%\end{frame}
%%%%%%%%%%%%%%%%%%%
%
%%%%%%%%% Slide %%%%%%  	
%%\begin{frame}
%%	\frametitle{A beamer button template}
%%	\label{identification}
%%	
%%\hyperlink{Technicalities}{\beamergotobutton{Technicalities}}
%%
%%\end{frame}
%%%%%%%%%%%%%%%%%%%
%
%
%
%%%%%%%%%%%%%%%%%%%%%%%%%%%%%%%%%%%%%%%%%%%%%%%%%%%%%%%%%%%%%%%%%%%%%%%%
%%%%%%%%                     APPENDIX  
%%%%%%%%%%%%%%%%%%%%%%%%%%%%%%%%%%%%%%%%%%%%%%%%%%%%%%%%%%%%%%%%%%%%%%%%
\appendix
\backupbegin

%%%%%%%%%%%%%%%%%%




%%%%%%%% Slide %%%%%%
\begin{frame}[plain]  % remove progress bar from appendix

\begin{centering}
\hfill \textbf{Appendix} \hfill
\end{centering}


\end{frame}
%%%%%%%%%%%%%%%%%%

%%%%%%% Slide %%%%%%
\begin{frame}[plain]  % remove progress bar from appendix
\frametitle{Long-run expectations: responsive to short-run conditions?}

\

Individual-level Survey of Professional Forecasters (SPF): for 1991-Q4 onward, estimate rolling regression

\begin{align*}
\Delta\bar{\pi}_t = & \beta_0 + \beta^w_1 f_{t|t-1} + \epsilon_t \numberthis \\
\end{align*}


\

$\bar{\pi}_t \quad $ 10-year ahead inflation expectation

\

$f_{t|t-1} \equiv \pi_t - \E_{t-1}\pi_t \quad $ individual one-year-ahead forecast error

\

$w$ indexes windows of 20 quarters







\end{frame}
%%%%%%%%%%%%%%%%%

%%%%%%% Slide %%%%%%
\begin{frame}[plain]  % remove progress bar from appendix
\frametitle{Time-varying responsiveness}\label{rolling}

\begin{align*}
\Delta\bar{\pi}_t = & \beta_0 + \beta^w_1 f_{t|t-1} + \epsilon_t \tag{1} \\
\end{align*}

\vspace{-0.7cm}

\begin{figure}[h!]
%\caption{Responsiveness of long-run expectations}
\includegraphics[scale = 0.22]{\myFigPath \fignameRolling}
\caption{Time series of $\hat{\beta}_1^w$}
\label{rolling}
\end{figure}


\vfill
\vspace{-0.9cm}

\hfill \hyperlink{further_evidence}{\beamergotobutton{Robustness}}
\end{frame}
%%%%%%%%%%%%%%%%%

%%%%%%%% Slide %%%%%%
\begin{frame}[plain]  % remove progress bar from appendix
\frametitle{Breakeven inflation}
	\label{app_TIPS}

\begin{figure}[h!]
\includegraphics[scale = 0.25]{\myFigPath \fignameMarketEPiMoreHorizons} % \fignameMarketEPiCleaned
\caption{Market-based inflation expectations, various horizons, \%}
%\floatfoot{Breakeven inflation, constructed as the difference between the yields of 10-year Treasuries and 10-year TIPS (blue line), difference between 10-year Treasury and 10-year TIPS, the latter cleaned from liquidity risk (red line).}
\label{epi_cleaned}
\end{figure}

\vfill 
\hyperlink{LRE_drifting_down}{\beamerreturnbutton{Return}}	

\end{frame}
%%%%%%%%%%%%%%%%%%

%%%%%%%% Slide %%%%%%
\begin{frame}[plain]  % remove progress bar from appendix
\frametitle{Correcting the TIPS from liquidity risk}

\begin{figure}[h!]
\includegraphics[scale = 0.25]{\myFigPath \fignameMarketEPiCleaned} % \fignameMarketEPiCleaned
\caption{Market-based inflation expectations, 10 year, \%}
%\floatfoot{Breakeven inflation, constructed as the difference between the yields of 10-year Treasuries and 10-year TIPS (blue line), difference between 10-year Treasury and 10-year TIPS, the latter cleaned from liquidity risk (red line).}
\label{epi_cleaned}
\end{figure}

\vfill 
\hyperlink{LRE_drifting_down}{\beamerreturnbutton{Return}}	

\end{frame}
%%%%%%%%%%%%%%%%%%

%%%%%%% Slide %%%%%%
\begin{frame}[plain]  % remove progress bar from appendix
\frametitle{Robustness checks}
\label{further_evidence}

\begin{align*}
\Delta\bar{\pi}_t = & \beta_0 + \beta^w_1\textcolor{brownreddark}{\pi_t} + \epsilon_t \tag{1} \\
\end{align*}

\vspace{-0.7cm}

\begin{figure}[h!]
%\caption{Responsiveness of long-run expectations}
\includegraphics[scale = 0.22]{\myFigPath \fignameRollingPi}
\caption{Time series of $\hat{\beta}_1^w$}
\label{rollingPi}
\end{figure}


\vfill
\vspace{-0.9cm}

\hyperlink{rolling}{\beamerreturnbutton{Return}}	\end{frame}
%%%%%%%%%%%%%%%%%

%%%%%%% Slide %%%%%%
\begin{frame}[plain]  % remove progress bar from appendix
\frametitle{Robustness checks - PCE core}

\begin{align*}
\Delta\bar{\pi}_t = & \beta^w_0+ \beta^w_1f_{t|t-1} + \epsilon_t \tag{1} \\
\end{align*}

\vspace{-0.7cm}

\begin{figure}[h!]
%\caption{Responsiveness of long-run expectations}
\includegraphics[scale = 0.22]{\myFigPath \fignameRollingPCEcore}
\caption{Time series of $\hat{\beta}_1^w$}
\label{rollingPi}
\end{figure}


\vfill
\vspace{-0.9cm}

\hyperlink{rolling}{\beamerreturnbutton{Return}}	\end{frame}
%%%%%%%%%%%%%%%%%

%%%%%%% Slide %%%%%%
\begin{frame}[plain]  % remove progress bar from appendix
\frametitle{Robustness checks - controlling for inflation levels}

\begin{align*}
\Delta\bar{\pi}_t = & \beta^w_0 + \beta^w_1f_{t|t-1} + \beta^w_2 \pi_t + \epsilon_t \tag{1} \\
\end{align*}

\vspace{-0.7cm}

\begin{figure}[h!]
%\caption{Responsiveness of long-run expectations}
\includegraphics[scale = 0.22]{\myFigPath \fignameRollingControlPi}
\caption{Time series of $\hat{\beta}_1^w$}
\label{rollingPi}
\end{figure}


\vfill
\vspace{-0.9cm}

\hyperlink{rolling}{\beamerreturnbutton{Return}}	\end{frame}
%%%%%%%%%%%%%%%%%

%%%%%%%% Slide %%%%%%
%\begin{frame}[plain]  % remove progress bar from appendix
%\frametitle{Robustness checks - controlling for inflation levels and uncertainty}
%
%\begin{align*}
%\Delta\bar{\pi}_t = & \beta^w_0+ \beta^w_1f_{t|t-1} + \beta^w_2 \pi_t + \beta^w_3 IQR_t+ \epsilon_t \tag{1} \\
%\end{align*}
%
%\vspace{-0.7cm}
%
%\begin{figure}[h!]
%%\caption{Responsiveness of long-run expectations}
%\includegraphics[scale = 0.22]{\myFigPath \fignameRollingControlPiIQR}
%\caption{Time series of $\hat{\beta}_1^w$}
%\label{rollingPi}
%\end{figure}
%
%
%\vfill
%\vspace{-0.9cm}
%
%\hyperlink{rolling}{\beamerreturnbutton{Return}}	\end{frame}
%%%%%%%%%%%%%%%%%%



%%%%%%% Slide %%%%%%
\begin{frame}[plain]  % remove progress bar from appendix
	\frametitle{Further evidence: disagreement}


\begin{figure}[h!]
\caption{Livingston Survey of Firms: \newline Interquartile range of 10-year ahead inflation expectations}
\includegraphics[scale = 0.24]{\myFigPath \fignameLivIQR}
\label{LivIQR}
\end{figure}

\vfill 
\hyperlink{rolling}{\beamerreturnbutton{Return}}	

\end{frame}
%%%%%%%%%%%%%%%%%

%%%%%%% Slide %%%%%%
\begin{frame}
%	\frametitle{Unanchoring in the data}

\begin{figure}[h!]
\caption{New York Fed Survey of Consumers: \newline Percent of respondents indicating 3-year ahead inflation will be in a particular range}
\includegraphics[scale = 0.24]{\myFigPath \fignameSCEdistrib}
\label{SCEdistrib}
\end{figure}

\vfill 
\hyperlink{rolling}{\beamerreturnbutton{Return}}	

\end{frame}
%%%%%%%%%%%%%%%%%

%%%%%%% Slide %%%%%%
\begin{frame}[plain]  % remove progress bar from appendix
	\frametitle{Further evidence: introspection}


\begin{figure}[h!]
\caption{PCE core inflation against the Fed's target}
\includegraphics[scale = 0.24]{\myFigPath \fignamePCEcore}
\label{PCEcore}
\end{figure}

\vfill 
\hyperlink{rolling}{\beamerreturnbutton{Return}}	

\end{frame}
%%%%%%%%%%%%%%%%%

%%%%%%% Slide %%%%%%
\begin{frame}[plain]  % remove progress bar from appendix
	\frametitle{Households: standard up to $\hat{\E}$}
	\label{HH}

Maximize lifetime expected utility
\begin{equation}
\textcolor{brownreddark}{\hat{\E}^i_t}\sum^{\infty}_{T=t}\beta^{T-t} \bigg[ U(C^i_T) - \int_0^1 v(h^i_T(j)) dj \bigg]
\label{lifetime_U}
\end{equation}	

Budget constraint
\begin{equation}
 B^i_t \leq (1+i_{t-1})B^i_{t-1} + \int_0^1 w_t(j)h^i_t(j) dj + \Pi_t^i(j)  dj-T_t -P_tC^i_t
 \label{BC}
\end{equation}




\vfill 
\hyperlink{model_overview}{\beamerreturnbutton{Return}}	
\end{frame}
%%%%%%%%%%%%%%%%%

%%%%%%% Slide %%%%%%
\begin{frame}[plain]  % remove progress bar from appendix
	\frametitle{Firms: standard up to $\hat{\E}$}

Maximize present value of profits
\begin{equation}
\textcolor{brownreddark}{\hat{\E}^j_t}\sum^{\infty}_{T=t}\alpha^{T-t} Q_{t,T} \bigg[ \Pi^j_t(p_t(j))\bigg]
\label{lifetime_profits}
\end{equation}

subject to demand
\begin{equation}
y_t(j) = Y_t \bigg(\frac{p_t(j)}{P_t}\bigg)^{-\theta}
\end{equation}


\vfill 
\hyperlink{model_overview}{\beamerreturnbutton{Return}}	

\end{frame}
%%%%%%%%%%%%%%%%%




%%%%%%%% Slide %%%%%%
\begin{frame}[plain]  % remove progress bar from appendix
\frametitle{Oscillatory dynamics in adaptive learning}
	\label{oscillatory}

Consider a stylized adaptive learning model in two equations:
\begin{align}
\pi_t & = \beta f_t + u_t  \\
f_t & = f_{t-1} + k(\pi_t - f_{t-1}) 
\end{align}

Solve for the time series of expectations $f_t$
\begin{equation}
f_t = \underbrace{\frac{1-k^{-1}}{1-k^{-1}\beta}}_{\approx 1}f_{t-1} + \frac{k^{-1}}{1-k^{-1}\beta}u_t
\end{equation}

Solve for forecast error $f_t \equiv \pi_t - f_{t-1}$:
\begin{equation}
f_t = \underbrace{-\frac{1-\beta}{1-k\beta}}_{\lim_{k \to 1} = -1}f_{t-1} + \frac{1}{1-k\beta}u_t 
\end{equation}

\vfill 
\hyperlink{IRF1}{\beamerreturnbutton{Return}}	

\end{frame}
%%%%%%%%%%%%%%%%%%


%%%%%%%% Slide %%%%%%
\begin{frame}[plain]  % remove progress bar from appendix
	\frametitle{Functional forms for $\mathbf{g}$ in the literature}
	\label{g}
\begin{itemize}
\item Smooth anchoring function (Gobbi et al, 2019)
\begin{equation}
 p = h(y_{t-1}) = A + \frac{B C e^{-D y_{t-1}}}{( C e^{-D y_{t-1}}+1)^2}
\end{equation}
$p \equiv Prob(\text{liquidity trap regime}) $ \\
$y_{t-1}$ output gap \\


\

\item Kinked anchoring function (Carvalho et al, 2019)
 \begin{align*}
k_t & = \begin{cases} \frac{1}{t} \quad \text{when} \quad \theta_t < \bar{\theta}  \\ k \quad \text{otherwise.}\numberthis
\end{cases} 
\end{align*}
$\theta_t$ criterion, $\bar{\theta}$ threshold value

\end{itemize}

\vfill 
\hyperlink{anchoring1}{\beamerreturnbutton{Return}}	

\end{frame}
%%%%%%%%%%%%%%%%%%


%%%%%%%%%%%%%%%%%%%%% Slide
\begin{frame}[plain]  % remove progress bar from appendix
	\frametitle{Choices for criterion $\theta_t$}
	\label{g}
\begin{itemize}
\item Carvalho et al. (2019)'s criterion  
\begin{equation}
\theta_t^{CEMP} = \max | \Sigma^{-1} ( \phi_{t-1} - T(\phi_{t-1})) |
\end{equation}


$\Sigma$ variance-covariance matrix of shocks \\
$T(\phi)$ mapping from PLM to ALM

\

\

\item CUSUM-criterion
\begin{align}
\omega_t & =  \omega_{t-1} + \kappa k_{t-1}(f_{t|t-1} f_{t|t-1}'  -\omega_{t-1})\\
\theta_t^{CUSUM} & =  \theta_{t-1} + \kappa k_{t-1}(f_{t|t-1}'\omega_t^{-1}f_{t|t-1} -\theta_{t-1})
\end{align}

\

$\omega_t$ estimated forecast-error variance
\end{itemize}




\vfill

\hyperlink{anchoring1}{\beamerreturnbutton{Return}}	


\end{frame}
%%%%%%%%%%%%%%%%%%

%%%%%%%% Slide %%%%%%
\begin{frame}[plain]  % remove progress bar from appendix
	\frametitle{General updating algorithm}
	\label{RLS}


\begin{align}
\phi_t  & = \bigg( \phi_{t-1}' + k_t R_t^{-1}\begin{bmatrix} 1 \\ s_{t-1} \end{bmatrix}\bigg(y_{t} - \phi_{t-1} \begin{bmatrix} 1 \\ s_{t-1} \end{bmatrix} \bigg)' \bigg)' \\
R_t &= R_{t-1} +  k_t \bigg( \begin{bmatrix} 1 \\ s_{t-1} \end{bmatrix} \begin{bmatrix} 1 & s_{t-1} \end{bmatrix}  - R_{t-1} \bigg)
\end{align}


\vfill

\hyperlink{adaptive_learning}{\beamerreturnbutton{Return}}	


\end{frame}
%%%%%%%%%%%%%%%%%%

%%%%%%%% Slide %%%%%%
\begin{frame}[plain]  % remove progress bar from appendix
	\frametitle{Assumptions on $\mathbf{g}(\cdot)$}
	\label{ass_g}


\begin{align}
\mathbf{g}_{ff} & \geq 0 \label{ass_g2}
\end{align}

\

$\mathbf{g}(\cdot)$ convex in forecast errors.

\vfill

\hyperlink{cgain_dgain}{\beamerreturnbutton{Return}}	


\end{frame}
%%%%%%%%%%%%%%%%%%

%%%%%%% Slide %%%%%%  	
\begin{frame}[plain]  % remove progress bar from appendix
	\frametitle{Estimating form of gain function}\label{estimation_general}




\


\begin{itemize}
\item Calibrate parameters of New Keynesian core to literature 

\

\

\item Estimate flexible form of expectations process via simulated method of moments \\ (Duffie \& Singleton 1990, Lee \& Ingram 1991, Smith 1993)

\begin{align}
%\hat{\E}_{t}\pi_{t+1} & = \bar{\pi}_{t-1} + b_1 s_t \\
\bar{\pi}_{t}  &=\bar{\pi}_{t-1} +\mathbf{g}(f_{t|t-1}) \; f_{t|t-1} \tag{18}
\end{align}
\

\

\item Moments: autocovariances of inflation, output gap, federal funds rate and 1-year ahead Survey of Professional Forecasters (SPF) inflation expectations at lags $0, \dots, 4$
\end{itemize}

\

\

\vfill

 \hyperlink{calibration_lit}{\beamerreturnbutton{Return}}	 \hfill \hyperlink{estimation_details}{\beamergotobutton{Details}}




\end{frame}
%%%%%%%%%%%%%%%%%
%%%%%%% Slide %%%%%%
\begin{frame}[plain]  % remove progress bar from appendix
	\frametitle{Estimated expectations process}

\begin{align}
%\hat{\E}_{t}\pi_{t+1} & = \bar{\pi}_{t-1} + b_1 s_t \\
\bar{\pi}_{t}  - \bar{\pi}_{t-1} &=\mathbf{g}(f_{t|t-1}) \; f_{t|t-1} \tag{18}
\end{align}

%\

\begin{figure}[h!]
\includegraphics[scale = 0.22]{\myFigPath \fignameGdotFe}
\caption{Changes in long-run inflation expectations as a function of forecast errors}
\label{epi}
\end{figure}

\vspace{-0.5cm}

 \hyperlink{calibration_lit}{\beamerreturnbutton{Return}}	
\end{frame}
%%%%%%%%%%%%%%%%%

%%%%%%%% Slide %%%%%%
\begin{frame}[plain]  % remove progress bar from appendix
	\frametitle{Details on households and firms}
	\label{details_HHs_firms}

\

Consumption:	
\begin{equation}
C^i_t =  \bigg[  \int_0^1 c^i_t(j)^{\frac{\theta-1}{\theta}} dj \bigg]^{\frac{\theta}{\theta-1}}\label{dixit}
\end{equation}
$\theta>1$: elasticity of substitution between varieties

\

Aggregate price level:
\begin{equation}
P_t =  \bigg[  \int_0^1 p_t(j)^{1-\theta} dj \bigg]^{\frac{1}{\theta-1}}
\label{agg_price}
\end{equation}

Profits:
\begin{equation}
\Pi_t^j = p_t(j)y_t(j) -w_t(j)f^{-1}(y_t(j)/A_t)
\end{equation}

Stochastic discount factor
\begin{equation}
Q_{t,T} = \beta^{T-t} \frac{P_t U_c(C_T)}{P_T U_c(C_t)}
\end{equation}

\vspace{-0.5cm}

\hyperlink{HH}{\beamerreturnbutton{Return}}	


\end{frame}
%%%%%%%%%%%%%%%%%%
%%%%%%%% Slide %%%%%%
\begin{frame}[plain]  % remove progress bar from appendix
	\frametitle{Derivations}
	\label{derivations}

Household FOCs
 \begin{align*}
\hat{C}_{t}^i & = \hat{\E}^i_t \hat{C}_{t+1}^i - \sigma(\hat{i}_{t} -\hat{\E}^i_t \hat{\pi}_{t+1})    \label{EE} \numberthis \\
\\
\hat{\E}^i_t \sum_{s=0}^{\infty}\beta^s \hat{C}^i_t& =\omega^i_t + \hat{\E}^i_t \sum_{s=0}^{\infty}\beta^s \hat{Y}^i_t \label{IBC} \numberthis
\end{align*}
where `hats' denote log-linear approximation and $\omega_t^i \equiv \frac{(1+i_{t-1})B^i_{t-1} }{P_t Y^*}$.

\begin{enumerate}
\item Solve (\ref{EE}) backward to some date $t$, take expectations at $t$ 
\item Sub in (\ref{IBC})
\item Aggregate over households $i$
\item[$\rightarrow$] Obtain (\ref{NKIS})
\end{enumerate}



\hyperlink{aggregate_LOMS}{\beamerreturnbutton{Return}}	


\end{frame}
%%%%%%%%%%%%%%%%%%

%%%%%%%% Slide %%%%%%
\begin{frame}[plain]  % remove progress bar from appendix
	\frametitle{Actual laws of motion}
	\label{ALMs}

 \begin{align*}
y_t & = A_1 f_{a,t} + A_2 f_{b,t} + A_3 s_t \label{LOM_LR} \numberthis \\
\\
s_t & = h s_{t-1} + \epsilon_t \label{exog} \numberthis
\end{align*}
where

\begin{equation}
 y_t \equiv \begin{pmatrix} \pi_t \\ x_t \\ i_t
 \end{pmatrix} 
 \quad \quad \quad 
  s_t  \equiv \begin{pmatrix} r_t^n \\ u_t 
 \end{pmatrix} 
\end{equation}
and

  \begin{align}
f_{a,t}  \equiv  \hat{\E}_t\sum_{T=t}^{\infty} (\alpha\beta)^{T-t } y_{T+1} \quad \quad \quad \quad 
f_{b,t}  \equiv \hat{\E}_t\sum_{T=t}^{\infty} (\beta)^{T-t } y_{T+1} \label{fafb}
\end{align}

\hyperlink{aggregate_LOMS}{\beamerreturnbutton{Return}}	


\end{frame}
%%%%%%%%%%%%%%%%%%

%%%%%%% Slide %%%%%%  	
\begin{frame}[plain]  % remove progress bar from appendix
	\frametitle{Piecewise linear approximation to gain function}\label{estimation_details}



\begin{equation}
 \mathbf{g}(f_{t|t-1}) = \sum_i\gamma_i b_i(f_{t|t-1})
\end{equation}

\

\begin{itemize}
\item $ b_i(f_{t|t-1}) = $ piecewise linear basis

\

\item $\gamma_i  = $ approximating coefficient at node $i$

\

\item[$\hookrightarrow$] Estimate $\hat{\gamma}$ via simulated method of moments
\end{itemize}



\vfill


\hyperlink{estimation_general}{\beamerreturnbutton{Return}}	



\end{frame}
%%%%%%%%%%%%%%%%%

%%%%%%% Slide %%%%%%
\begin{frame}[plain]  % remove progress bar from appendix
	\frametitle{The expectation process over time}

\begin{figure}[h!]
%\subfigure{\includegraphics[scale = 0.15]{\myFigPath \fignameFeSPF}}
%\subfigure{\includegraphics[scale = 0.15]{\myFigPath \fignameGainSPF}}
\includegraphics[scale = 0.22]{\myFigPath \fignameFeGainGdotFeSPF}
\caption{Time series of forecast errors, changes in long-run expectations and gain}
\label{fe_in_data}
\end{figure}

%\
%
%\pause
%Median change in long-run expectations $\approx -0.8$ basis points
%
%\pause
%Largest change in absolute value $\approx -240$ basis points
%Mean gain $\approx 0.12$ \pause $\quad \rightarrow$ discount forecast errors older than 8 quarters 


\end{frame}
%%%%%%%%%%%%%%%%%

%%%%%%%%% Slide %%%%%%
\begin{frame}[plain]  % remove progress bar from appendix
	\frametitle{Target criterion}
	\label{anchTC}
	
	\begin{prop} 

\

Let $\; \mathbf{g}_{z,t} \equiv \frac{\partial \mathbf{g}}{\partial z}\;$ at $t$. Then monetary policy optimally brings about the following target relationship between inflation and the output gap

%\begin{small}	
\begin{align*}
& \pi_t \; \only<2>{\textcolor{brownreddark}{- \Gamma(k)
\E_t\sum_{i=1}^{\infty}x_{t+i}}}
\only<3>{- \Omega \bigg(k_{\textcolor{brownreddark}{t}}\textcolor{brownreddark}{+f_{t|t-1}\mathbf{g}_{\pi,t}}\bigg)
\bigg(\E_t\sum_{i=1}^{\infty}x_{t+i}\textcolor{brownreddark}{\prod_{j=0}^{i-1}(1-k_{t+1+j} - f_{t+1+j|t+j}\mathbf{g_{\bar{\pi}, t+j}})} \bigg)}  = -\frac{\lambda_x}{\kappa}x_t 
 %\label{target}
\end{align*}
%\end{small}
	\end{prop}


\

 \onslide<1>{RE (discretion): move $\pi_t$ and $x_t$ to offset cost-push shocks \\}


\onslide<2>{\textcolor<2>{brownreddark}{Adaptive learning: can move $\E_t x_{t+i}$ too if $k > 0$} \hfill \hyperlink{fullTC}{\beamergotobutton{$\Gamma(k)$}} \\}


\onslide<3>{\textcolor<3>{brownreddark}{Endogenous gain: ability to move $\E_t x_{t+i}$ depends on present and future degree of unanchoring}}

\vfill

\hfill  \onslide<3>{\hyperlink{fullTC}{\beamergotobutton{Full expression, $\Omega$}} \hyperlink{no_commitment}{\beamergotobutton{No commitment}}}

\vfill


\hyperlink{ramsey}{\beamerreturnbutton{Return}}

\end{frame}
%%%%%%%%%%%%%%%%%%%





%%%%%%%% Slide %%%%%%
\begin{frame}[plain]  % remove progress bar from appendix
	\frametitle{}
	\label{no_commitment}

\begin{lemma} The discretion and commitment solutions of the Ramsey problem coincide. 
\end{lemma}

\

\hfill \hyperlink{no_commitment_intuition}{\beamergotobutton{Why no commitment?}}

\

\

\begin{corollary} Optimal policy under adaptive learning is time-consistent. 
\end{corollary}







\end{frame}
%%%%%%%%%%%%%%%%%%

%%%%%%%% Slide %%%%%%
\begin{frame}[plain]  % remove progress bar from appendix
	\frametitle{No commitment - no lagged multipliers}
	\label{no_commitment_intuition}
	
	Simplified version of the model: planner chooses $\{\pi_t, x_t, f_t, k_t\}_{t=t_0}^{\infty}$ to minimize
 \begin{align*}
\mathcal{L} &= \E_{t_0}\sum_{t=t_0}^{\infty} \beta^{t-t_0}\bigg\{ \pi_t^2  + \lambda x_t^2 + \varphi_{1,t} (\pi_t -\kappa x_t- \beta f_t +u_t) \\ &+ \varphi_{2,t}(f_t - f_{t-1} -k_t(\pi_t - f_{t-1})) + \varphi_{3,t}(k_t- \mathbf{g}(\pi_t - f_{t-1})) \bigg\}
 \end{align*}

 \begin{align}
  2\pi_t +2\frac{\lambda}{\kappa}x_t -\textcolor{brownreddark}{\varphi_{2,t}}(k_t + \mathbf{g_{\pi}}(\pi_t -f_{t-1}))& = 0 \label{simpleFOC1} \\
  -2\beta\frac{\lambda}{\kappa}x_t + \textcolor{brownreddark}{\varphi_{2,t}} -\textcolor{brownreddark}{\varphi_{2,t+1}}(1-k_{t+1} -\mathbf{g_{f}}(\pi_{t+1} -f_{t})) & = 0 \label{simpleFOC2} 
 \end{align}

  
\hyperlink{no_commitment}{\beamerreturnbutton{Return}}	


\end{frame}
%%%%%%%%%%%%%%%%%%



%%%%%%%% Slide %%%%%%
\begin{frame}[plain]  % remove progress bar from appendix
	\frametitle{Target criterion system for anchoring function as changes of the gain}
	\label{generalTC}

\begin{align*}
\varphi_{6,t} & = -c f_{t|t-1} x_{t+1} + \bigg(1+ \frac{f_{t|t-1}}{f_{t+1|t}}(1-k_{t+1}) -f_{t|t-1} \mathbf{g}_{\bar{\pi},t} \bigg) \varphi_{6,t+1} \\
& -\frac{f_{t|t-1}}{f_{t+1|t}}(1-k_{t+1})\varphi_{6,t+2} \numberthis \label{6'} \\
0 & = 2\pi_t + 2\frac{\lambda_x}{\kappa}x_t   - \bigg( \frac{k_t}{f_{t|t-1}} + \mathbf{g}_{\pi,t}\bigg)\varphi_{6,t} + \frac{k_t}{f_{t|t-1}}\varphi_{6,t+1}\numberthis \label{1'}
\end{align*}
$\varphi_{6,t}$ Lagrange multiplier on anchoring function

\

The solution to (\ref{1'}) is given by:
\begin{equation}
\varphi_{6,t} = -2\E_t\sum_{i=0}^{\infty}(\pi_{t+i}+\frac{\lambda_x}{\kappa}x_{t+i})\prod_{j=0}^{i-1}\frac{\frac{k_{t+j}}{f_{t+j|t+j-1}}}{\frac{k_{t+j}}{f_{t+j|t+j-1}} + \mathbf{g}_{\pi, t+j}} \label{sol1'}
\end{equation}


\vspace{-0.5cm}
 
\hyperlink{anchTC}{\beamerreturnbutton{Return}}	


\end{frame}
%%%%%%%%%%%%%%%%%%

%%%%%%%% Slide %%%%%%
\begin{frame}[plain]  % remove progress bar from appendix
	\frametitle{Optimal Taylor-coefficient on inflation}\label{taylor_results}
	
	\begin{equation}
	i_t = \textcolor{aquamarinegreen}{\psi_{\pi}} \pi_t + \psi_x x_t
	\end{equation}

\pause	
\centering
\begin{tikzpicture}
    \draw (0, 0) node[inner sep=0] {\includegraphics[scale=0.22]{\myFigPath \fignameCBlossbaselineTwoY}};
    \draw (0, 3) node {Figure: Central bank loss as a function of $\psi_{\pi}$};
    \draw<3>[-,dodgerblue,solid,line width = 1.2pt](-3.4,-1.82) -- (-3.4,2.52);
    \draw<3>[-,brownreddark,solid,line width = 1.2pt](2.625,-1.82) -- (2.625,2.52);
\end{tikzpicture}

%%\psi_pi lines
%\newcommand{\psipilineAnch}{\raisebox{2pt}{\tikz{\draw[-,dodgerblue,solid,line width = 1.5pt](0,0) -- (5cm,5cm);}}}
%\newcommand{\psipilineRE}{\raisebox{2pt}{\tikz{\draw[-,brownredlight,solid,line width = 1.5pt](0,0) -- (5mm,0);}}}



\pause
\small{
\textcolor{dodgerblue}{Anchoring-optimal coefficient: $\psi_{\pi}^A =1.1$} $\quad \quad $\textcolor{brownreddark}{ RE-optimal coefficient: $\psi_{\pi}^{RE} =2.2$}
}



\end{frame}
%%%%%%%%%%%%%%%%%%


%%%%%%%% Slide %%%%%%
\begin{frame}[plain]  % remove progress bar from appendix
	\frametitle{Why less aggressive? Future interest rate expectations} 
	\label{less_aggressive}
\begin{itemize}
\item[] IS curve:	
 \begin{align*}
x_t &=  -\sigma \textcolor<1>{brownreddark}{i_t} +\hat{\E}_t \sum_{T=t}^{\infty} \beta^{T-t }\big( (1-\beta)x_{T+1} - \sigma(\beta \textcolor<2-3>{brownreddark}{i_{T+1}} - \pi_{T+1}) +\sigma r_T^n \big)   %\\
%\pi_t &= \kappa x_t +\hat{\E}_t \sum_{T=t}^{\infty} (\alpha\beta)^{T-t }\big( \kappa \alpha \beta x_{T+1} + (1-\alpha)\beta \pi_{T+1} + u_T\big) 
\end{align*}

\

\

 

\begin{itemize}
\item<1-3>  \textcolor<1>{brownreddark}{Current interest rate $i_t$: one channel of policy}

\

\item<2-3> Taylor rule implies interest rate expectation
	\begin{equation}
	\textcolor<2>{brownreddark}{\hat{\E}_ti_{t+k}} = \psi_{\pi}\hat{\E}_t \pi_{t+k}+ \psi_x \hat{\E}_t x_{t+k}
	\end{equation}

\item<3>  \textcolor<3>{brownreddark}{If private sector understands and believes Taylor rule, expected future interest rates additional channel of policy} \\
 \textcolor<3>{brownreddark}{(Eusepi, Giannoni \& Preston 2018)}


\end{itemize}



\end{itemize}


\vfill \hyperlink{pos_feedback}{\beamerreturnbutton{Return}}	 \hfill  \hyperlink{IRFs_function_psipi}{\beamergotobutton{IRFs for various $\psi_{\pi}$}}

\end{frame}
%%%%%%%%%%%%%%%%%%

%%%%%%%% Slide %%%%%%
\begin{frame}[plain]  % remove progress bar from appendix
	\frametitle{Respond but not too much}\label{IRFs_function_psipi}

\begin{figure}[h!]
%\subfigure[$\psi_{\pi} = 1.01$]{\includegraphics[scale = 0.13]{\myFigPath \fignameIRFpsipiSmall}}
%%\hfill
%\subfigure[$\psi_{\pi} = 1.5$]{\includegraphics[scale = 0.13]{\myFigPath \fignameIRFpsipiMedium}}
%\subfigure[$\psi_{\pi} = 2$]{\includegraphics[scale = 0.13]{\myFigPath \fignameIRFpsipiBig}}
\includegraphics[scale = 0.22]{\myFigPath \fignameIRFpsipiTogetherCostPush}
\caption{Impulse responses for unanchored expectations for various values of $\psi_{\pi}$}
%\floatfoot{Shock imposed at $t=25$ of a sample length of $T=400$ (with 5 initial burn-in periods), cross-sectional average with a cross-section size of $N=1000$.}
\label{IRF_unanchored_psi}
\end{figure}

\vfill
\hyperlink{pos_feedback}{\beamerreturnbutton{Return}}	


\end{frame}
%%%%%%%%%%%%%%%%%%



%
%%%%%%%% Slide %%%%%%
%\begin{frame}
%\frametitle{A beamer button template, how to get back to main text}
%\label{Steps}
%
%\begin{equation}
%D = \begin{bmatrix}
%d_{11} & \gamma_{12} & \gamma_{13} & d_{14} & \cdots \\
%d_{21} & \gamma_{22} & \gamma_{23} & d_{24} & \cdots \\
%\vdots & \vdots & \vdots & \ddots & \vdots 
%\end{bmatrix}
%\end{equation}
%
%\hyperlink{calcCBloss}{\beamerreturnbutton{Return}}	
%\end{frame}
%%%%%%%%%%%%%%%%%%

\backupend


\end{document}
