\documentclass{beamer}
\usepackage[english]{babel}
\usepackage[latin1]{inputenc}
%\usepackage[T1]{fontenc} % don't know what this does: it sets font encoding
%\usepackage{gfsartemisia} % this chooses the font
%%\usepackage{txfonts}
%% the above two seem to give the same, the below two as well
%%\usepackage{accanthis} 
%%\usepackage{concmath}
% Another way to go is the following
%  The available themes are: structurebold, structurebolditalic, structuresmallcapsserif, structureitalicsserif, serif and default.
%\usefonttheme{default}
%\usefonttheme{serif}
%\usefonttheme{structurebold}
%%\usefonttheme{structurebolditalic} % .sty wasn't found
%\usefonttheme{structuresmallcaps}
\usefonttheme{structuresmallcapsserif}
%%%%%%\usefonttheme{structureitalicsserif} % .sty wasn't found
%%%%%\usefonttheme{serif} % the font theme has to correspond to the font used
%%%%\usefonttheme{structurebold} % it seems you can use a combo of these
%%%%% Here comes the font command itself:
%%%%% The available fonts depend on your LATEX installation, the most common are: mathptmx, helvet, avat, bookman, chancery, charter, culer, mathtime, mathptm, newcent, palatino, pifont and utopia.
%%%%% I also have ccfonts
%%%%\usepackage{helvet}
%%%%\usepackage[T1]{fontenc}  % some fonts, like concrete math, require a specific font encoding (concrete math requires T1)
\usepackage{subfigure}
\usepackage[default]{lato}

\usepackage{amssymb}
\usepackage{amsmath}
\usepackage{booktabs}
\usepackage{verbatim}
\usepackage{caption}
\usepackage{float}
\usepackage{csquotes}
\usepackage{sansmathaccent}
\usepackage{subfigure}
\usepackage{multicol}
\pdfmapfile{+sansmathaccent.map}
\usepackage{pgfplots,tikz}
\usetikzlibrary{tikzmark,calc}
\usepackage{overpic}
\usepackage{color,soul}
\usepackage{stackengine}

\def \myFigPath {../../figures/} 
\def \myTablePath {../../tables/} 

\beamertemplatenavigationsymbolsempty % this removes the blue navigation bar that used to be on the bottom right
\setbeamersize{text margin left=5mm,text margin right=12mm} 
%
\font\reali=msbm10 at 12pt

%


\author[]{Laura G\'ati}
% note: in the [] you can put a short name to be displayed in the footer

\institute[]{Boston College}
% note: in the [] you can put a short affiliation to be displayed in the footer

\title[]{Monetary Policy \& Anchored Expectations}
% note: in the [] you can put a short title to be displayed in the footer

\date[]{October 1, 2019}
% note: in the [] you can put a short date to be displayed in the footer

%\usetheme{CambridgeUS} % CambridgeUS, Madrid
\usetheme{Madrid}
\usecolortheme{dove}



\AtBeginSection[]
{
  \begin{frame}<beamer>
    \frametitle{Structure of talk}
    \tableofcontents[currentsection]
  \end{frame}
}

% New Commands
\newcommand{\real}{\hbox{\reali R}}
\newcommand{\realp}{\hbox{\reali R}_{\scriptscriptstyle +}}
\newcommand{\realpp}{\hbox{\reali R}_{\scriptscriptstyle ++}}
\newcommand{\R}{\mathbb{R}}
\DeclareMathOperator{\E}{\mathbb{E}}
\DeclareMathOperator{\argmin}{arg\,min}
\newcommand\w{3.0in}
\newcommand\wnum{3.0}
\def\myFigWidth{5.3in}
\def\mySmallerFigWidth{2.1in}
\def\myEvenBiggerFigScale{0.8}
\def\myPointSixFigScale{0.6}
\def\myBiggerFigScale{0.4}
\def\myFigScale{0.3}
\def\mySmallFigScale{0.25}
\def\mySmallerFigScale{0.18}
\def\myTinyFigScale{0.16}
\def\myPointFourteenFigScale{0.14}
\def\myTinierFigScale{0.12}
\def\myEvenTinierFigScale{0.10}
\def\myCrazyTinyFigScale{0.09}

\newtheorem{prop}{Proposition}
\newtheorem*{remark}{Remark}

\newcommand\numberthis{\addtocounter{equation}{1}\tag{\theequation}} % this defines a command to make align only number this line

% creating colored lines for legends
% green solid
\newcommand{\greenline}{\raisebox{2pt}{\tikz{\draw[-,black!40!green,solid,line width = 1.5pt](0,0) -- (5mm,0);}}}
% red solid, dashed, dotted
\newcommand{\redline}{\raisebox{2pt}{\tikz{\draw[-,red,solid,line width = 1.5pt](0,0) -- (5mm,0);}}}
\newcommand{\reddashedline}{\raisebox{2pt}{\tikz{\draw[-,red,dashed,line width = 1.5pt](0,0) -- (5mm,0);}}}
\newcommand{\reddottedline}{\raisebox{2pt}{\tikz{\draw[-,red,densely dotted,line width = 1.5pt](0,0) -- (5mm,0);}}}

% blue solid and dashed
\newcommand{\blueline}{\raisebox{2pt}{\tikz{\draw[-,blue,solid,line width = 1.5pt](0,0) -- (5mm,0);}}}
\newcommand{\bluedashedline}{\raisebox{2pt}{\tikz{\draw[-,blue,dashed,line width = 1.5pt](0,0) -- (5mm,0);}}}
% black solid and dashed
\newcommand{\blackline}{\raisebox{2pt}{\tikz{\draw[-,black,solid,line width = 1.5pt](0,0) -- (5mm,0);}}}
\newcommand{\blackdashedline}{\raisebox{2pt}{\tikz{\draw[-,black,dashed,line width = 1.5pt](0,0) -- (5mm,0);}}}


% These are my colors -- there are many like them, but these ones are mine (Paul Goldsmith-Pinkham's)
\definecolor{blue_PGP}{RGB}{0,114,178}
\definecolor{red_PGP}{RGB}{213,94,0}
\definecolor{yellow_PGP}{RGB}{240,228,66}
\definecolor{green_PGP}{RGB}{0,158,115}
%% I use a beige off white for my background 
\definecolor{MyBackground}{RGB}{255,253,218}
%% Uncomment this if you want to change the background color to something else 
%\setbeamercolor{background canvas}{bg=MyBackground} % this is a very nice beige
%% Change the bg color to adjust your transition slide background color! 
%\newenvironment{transitionframe}{
%\setbeamercolor{background canvas}{bg=yellow} \begin{frame}}{
%\end{frame} }

%\setbeamercolor{frametitle}{fg=blue_PGP} 
%\setbeamercolor{title}{fg=black} 
%\setbeamertemplate{footline}[frame number]  % this sets the footer the same color as the background
%\setbeamertemplate{navigation symbols}{} % don't know what this does
%\setbeamercolor{button}{bg=MyBackground,fg=blue_PGP}

% footnote suppressing number
\newcommand\blfootnote[1]{%
  \begingroup
  \renewcommand\thefootnote{}\footnote{#1}%
  \addtocounter{footnote}{-1}%
  \endgroup
}

% try to force appendix to be numbered differently
\newcommand{\backupbegin}{
   \newcounter{framenumberappendix}
   \setcounter{framenumberappendix}{\value{framenumber}}
}
\newcommand{\backupend}{
   \addtocounter{framenumberappendix}{-\value{framenumber}}
   \addtocounter{framenumber}{\value{framenumberappendix}} 
}

%%%%%%%%%%%%            BEGIN DOCUMENT         %%%%%%%%%%%%%%%%%%%%%
\begin{document}


\begin{frame}

\maketitle


\end{frame}



%%%%%%% Slide %%%%%%
\begin{frame}
	\frametitle{Inflation expectations moving down}

\begin{figure}[h!]
\subfigure[Market-based inflation expectations, 10 year, average]{\includegraphics[scale = \mySmallerFigScale]{\myFigPath dw_Epi10}}
\end{figure}


\end{frame}
%%%%%%%%%%%%%%%%%

%%%%%%% Slide %%%%%%
\begin{frame}
	\frametitle{}

\begin{figure}[h!]
\subfigure[Unemployment rate]{\includegraphics[scale = \myTinierFigScale]{\myFigPath dw_urate}}
\subfigure[Fed funds rate target, upper limit]{\includegraphics[scale = \myTinierFigScale]{\myFigPath dw_ffr}}
\end{figure}


\end{frame}
%%%%%%%%%%%%%%%%%



%%%%%%%% Slide %%%%%%
\begin{frame}
	\frametitle{This project}
	
	\begin{enumerate}
	\item[] I embed an endogenous anchoring mechanism (AM) in a standard model of monetary policy
	
	\
	
	\
	
	\item[] Results 
	
	\
	
	\begin{itemize}
	\item Anchoring expectations is a new objective of monetary policy
	
	\
	
	\item Great Inflation was a period of unanchored expectations
	
	\
	
	\item Optimal policy should take recent economic environment into account when responding to current shocks
	\end{itemize}

	
	\
	
	\end{enumerate}


\end{frame}
%%%%%%%%%%%%%%%%%%


\section{Related literature}
%%%%%%%% Slide %%%%%%
\begin{frame}
	\frametitle{Related literature}

\begin{itemize}
\item \textbf{Optimal monetary policy in New Keynesian models}
\item[] Clarida, Gali \& Gertler (1999), Woodford (2003)

\

\

\item \textbf{Econometric learning}
\item[] Evans \& Honkapohja (2001), Preston (2005), Graham (2011)

\

\

\item \textbf{Anchoring} 
\item[] Carvalho et al (2019), Svensson (2015), Hooper et al (2019)
\end{itemize}



\end{frame}
%%%%%%%%%%%%%%%%%%

\section{Intuition: what is anchoring and why should it matter?}
%%%%%%%% Slide %%%%%%
\begin{frame}
	\frametitle{Phillips Curve}

\begin{align*}
\pi_t & = \beta \hat{\E}_t\pi_{t+1} + \kappa x_t
\end{align*}

\begin{itemize}
\item $\pi_t = $ inflation

\

\item $x_t =$ output gap

\

\item $\hat{\E}_t =$ expectation-operator (not necessarily rational)


\end{itemize}



\end{frame}
%%%%%%%%%%%%%%%%%%

%%%%%%%% Slide %%%%%%
\begin{frame}
	\frametitle{}

Suppose a negative demand shock:

\

\begin{align*}
\pi_t & = \beta \hat{\E}_t\pi_{t+1} + \kappa \underset{\downarrow}{x_t} 
\end{align*}




\end{frame}
%%%%%%%%%%%%%%%%%%

%%%%%%%% Slide %%%%%%
\begin{frame}
	\frametitle{}

If expectations do not move:

\

\begin{align*}
\underset{\downarrow}{\pi_t} & = \beta \hat{\E}_t\pi_{t+1}+ \kappa \underset{\downarrow}{x_t} 
\end{align*}




\end{frame}
%%%%%%%%%%%%%%%%%%

%%%%%%%% Slide %%%%%%
\begin{frame}
	\frametitle{}

If seeing $\underset{\downarrow}{\pi_t}$, expectations adjust:

\

\begin{align*}
\underset{\downarrow \color{red}{\downarrow}}{\pi_t} & = \beta \underset{\color{blue}{\downarrow}}{\hat{\E}_t\pi_{t+1}}+ \kappa \underset{\downarrow}{x_t} 
\end{align*}

\

\

\

\begin{itemize}
\item[] Keeping expectations stable may be desirable

\

\item[$\rightarrow$]  ``Anchored'': notion of stable expectations

\


\

%\item[] \small{(Flattening PC due to anchored expectations, Hooper et al (2019))}
\end{itemize}

\


\end{frame}
%%%%%%%%%%%%%%%%%%

\section{A model of anchoring}
%%%%%%%% Slide %%%%%%
\begin{frame}
	\frametitle{A learning model of expectation formation}
Suppose firms and households

\

\begin{itemize}
\item observe everything up to time $t$ 

\

\

\item do not observe future variables

\

\

\item KEY: are unsure about the long-run mean of inflation, $\bar{\pi}$
\end{itemize}


\



\end{frame}
%%%%%%%%%%%%%%%%%%

%%%%%%%% Slide %%%%%%
\begin{frame}
	\frametitle{}

Agents construct one-period-ahead inflation forecasts as

\

\begin{equation}
\hat{\E}_{t}\pi_{t+1} =  \bar{\pi}_{t-1}+bs_{t} \label{PLM}
\end{equation}

\

\begin{itemize}
\item[] $\bar{\pi} =$ estimate of inflation drift ($=$ long-run mean, ``target")

\



\item[] $\hat{\E} =$ subjective expectation operator (not rational expectations, $\E$)

\



\item[] $b =$ matrix of constants 

\

\item[] $s =$ shocks
\end{itemize}

\



\end{frame}
%%%%%%%%%%%%%%%%%%

%%%%%%%% Slide %%%%%%
\begin{frame}
	\frametitle{Anchoring mechanism}
%\vspace{-1cm}
	
 \begin{align*}
\bar{\pi}_{t} & = \bar{\pi}_{t-1} +k_t\overbrace{\big(\pi_{t} -(\bar{\pi}_{t-1}+bs_{t-1}) \big)}^{\text{short-run forecast error}}  \label{RLS_anchoring} \numberthis \\
\\
k_t &= \begin{cases}  \frac{1}{k_{t-1} +1} \quad \quad \text{if} \quad \overbrace{|\hat{\E}_{t-1}\pi_t - \E_{t-1}\pi_t| / \sigma_s}^{\equiv \theta_t} \leq \bar{\theta} \\ \\\bar{g} \quad \quad \quad \quad \text{otherwise}\end{cases} \label{gain} \numberthis
\end{align*}

\

Equation (\ref{gain}): \textbf{endogenous} gain

\

\begin{itemize}
\item Carvalho et al (2019)

\

\item Difference to standard econometric learning
\end{itemize}


\end{frame}
%%%%%%%%%%%%%%%%%%

%%%%%%%% Slide %%%%%%
\begin{frame}
	\frametitle{}
	
	


\begin{itemize}
\item Expectations anchored $ = $ when agents choose \textbf{decreasing} gains

\

\

\item Expectations unanchored $ = $ when agents choose \textbf{constant} gains
\end{itemize}



\end{frame}
%%%%%%%%%%%%%%%%%%



\section{Full model with anchoring mechanism}

%%%%%%%% Slide %%%%%%
\begin{frame}
	\frametitle{The model}

Households maximize
\begin{equation}
\hat{\E}^i_t \sum_{T=t}^{\infty} \beta^{T-t} \bigg( U(C^i_{T}) - v(H^i_{T})\bigg)
\end{equation}

\

\

Household budget constraint:
\begin{equation}
 B^i_t \leq (1+i_{t-1})B^i_{t-1} + W_tH^i_t + \Pi_t^i -T_t -P_tC^i_t
\end{equation}

%M^i_t + B^i_t \leq (1+i^m_{t-1})M^i_{t-1} + (1+i_{t-1})B^i_{t-1} + W_tH^i_t + \Pi_t^i -T_t -P_tC^i_t

\

\

Firms: monopolistic competition in varieties $C^j$, Calvo price setting

\

\

Expectations: $\hat{\E}$ as in (\ref{PLM})

\end{frame}
%%%%%%%%%%%%%%%%%%

%%%%%%%% Slide %%%%%%
\begin{frame}
	\frametitle{3-Equation New Keynesian Model}
	\label{NK}

\begin{align*}
x_t &=  -\sigma i_t +\hat{\E}_t \sum_{T=t}^{\infty} \beta^{T-t }\big( (1-\beta)x_{T+1} - \sigma(\beta i_{T+1} - \pi_{T+1}) +\sigma r_T^n \big) \label{prestons18}  \numberthis \\
\\
\pi_t &= \kappa x_t +\hat{\E}_t \sum_{T=t}^{\infty} (\alpha\beta)^{T-t }\big( \kappa \alpha \beta x_{T+1} + (1-\alpha)\beta \pi_{T+1} + u_T\big) \label{prestons19}  \numberthis \\
\\
i_t &= \psi_{\pi}\pi_t + \psi_{x} x_t + \bar{i}_t \label{TR} \numberthis
\end{align*}

\

``Long-horizon forecasts'' $\rightarrow$ agents do not know the model\\
$\; $Preston (2005)

\

\hyperlink{derivations}{\beamergotobutton{Derivations}} \\
\hyperlink{compact}{\beamergotobutton{Compact notation}}	


\end{frame}
%%%%%%%%%%%%%%%%%%



\section{Simulations}

%%%%%%%% Slide %%%%%%
\begin{frame}
	\frametitle{Calibration}

\begin{center}
\begin{tabular}{ c | c  | l }
 $\beta$ & 0.98 & stochastic discount factor \\  \hline
 $\sigma$ & 0.5  & intertemporal elasticity of substitution \\  \hline
 $\alpha$ & 0.5 &  Calvo probability of not adjusting prices \\\hline
 $\psi_{\pi} $& 1.5  & coefficient of inflation in Taylor rule \\\hline
 $\psi_x$ & 1.5   & coefficient of the output gap in Taylor rule  \\\hline
 $\bar{g}$ & $0.145^{*}$  & value of the constant gain \\\hline
& & \\ [-1em] % this adds an extra empty row, and decreases its size, so it looks as if thetbar's row was higher
 $\bar{\theta}$ &  5*  & threshold deviation between subjective \& objective $\E$ \\ \hline
    $\rho_r$ & 0 &   persistence of natural rate shock \\ \hline
    $\rho_i$ & 0.877* &  persistence of monetary policy shock  \\ \hline
    $\rho_u$ & 0  &  persistence of cost-push shock  \\ \hline
    $\sigma_i$ & 0.359* & standard deviation of natural rate shock  \\ \hline
    $\sigma_r$ &  0.1  &standard deviation of monetary policy shock  \\ \hline
    $\sigma_u$ & 0.277* & standard deviation of cost-push shock   \\ \hline
\end{tabular}
\end{center}

%\begin{tabular}{cc}
%\hline
%hi & tall one\\[10ex]
%\hline
%hi & medium one\\[5ex]
%\hline
%hi & standard one\\
%\hline
%\end{tabular}

\blfootnote{* Carvalho et al (2019)'s estimates.  Exception: $\bar{\theta} = 0.029$.} 
\end{frame}
%%%%%%%%%%%%%%%%%%
%%%%%%%% Slide %%%%%%
\begin{frame}
	\frametitle{When always anchored, AM = learning}
% learning models against RE

\begin{figure}[h!]
\addtocounter{subfigure}{-3} % the a,b,c was starting at d, so I move it back 3 steps to a
\subfigure[Inflation]{\includegraphics[scale = \myTinierFigScale]{\myFigPath dw_role_of_learning_pi}}
%\stackunder[5pt]{\includegraphics[width=2in,height=.7in]{\myFigPath dw_role_of_learning_pi}}{NorESM1-M}
\subfigure[Output gap]{\includegraphics[scale = \myTinierFigScale]{\myFigPath dw_role_of_learning_x}}
\subfigure[Interest rate]{\includegraphics[scale = \myTinierFigScale]{\myFigPath dw_role_of_learning_i}}
\put (-180, 50){\makebox[0.7\textwidth][r]{\blackline $\; $ RE}}
\put (-152, 30){\makebox[0.7\textwidth][r]{\blueline $\; $ Learning}}
\put (-176, 10){\makebox[0.7\textwidth][r]{\reddashedline $\; $ AM}}
\caption{Rational expectations (RE), learning and anchoring mechanism (AM)}
\end{figure}

\end{frame}
%%%%%%%%%%%%%%%%%%

%%%%%%%% Slide %%%%%%
\begin{frame}
	\frametitle{}
% gain and drift in pi in this case

\begin{figure}[h!]
\subfigure[Gain]{\includegraphics[scale = \myTinierFigScale]{\myFigPath dw2_gain}}
\subfigure[Inflation drift]{\includegraphics[scale = \myTinierFigScale]{\myFigPath dw2_drift}}
\caption{Well anchored expectations: decreasing gain}
\end{figure}

\end{frame}
%%%%%%%%%%%%%%%%%%

%%%%%%%% Slide %%%%%%
\begin{frame}
	\frametitle{A lower $\bar{\theta}$: a brief unanchored period}
% gain and drift in pi in this case

\begin{figure}[h!]
\subfigure[Gain]{\includegraphics[scale = \myTinierFigScale]{\myFigPath dw4_gain}}
\subfigure[Inflation drift]{\includegraphics[scale = \myTinierFigScale]{\myFigPath dw4_drift}}
\caption{$\bar{\theta}=1$. Short unanchored episode: constant gain}
\end{figure}

\end{frame}
%%%%%%%%%%%%%%%%%%

%%%%%%%% Slide %%%%%%
\begin{frame}
% learning model with and w/o anchoring, potentially for different thetbars
\begin{figure}[h!]
\subfigure[Inflation]{\includegraphics[scale = \myTinierFigScale]{\myFigPath dw3_pi}}
\subfigure[Output gap]{\includegraphics[scale = \myTinierFigScale]{\myFigPath dw3_x}}
\subfigure[Interest rate]{\includegraphics[scale = \myTinierFigScale]{\myFigPath dw3_i}}
\put (-180, 50){\makebox[0.7\textwidth][r]{\blackline $\; $ RE}}
\put (-152, 30){\makebox[0.7\textwidth][r]{\blueline $\; $ Learning}}
\put (-176, 10){\makebox[0.7\textwidth][r]{\reddashedline $\; $ AM}}
\caption{$\bar{\theta}=1$}
\end{figure}

\end{frame}
%%%%%%%%%%%%%%%%%%

%%%%%%%% Slide %%%%%%
\begin{frame}
	\frametitle{A much lower $\bar{\theta}$}
% gain and drift in pi in this case

\begin{figure}[h!]
\subfigure[Gain]{\includegraphics[scale = \myTinierFigScale]{\myFigPath dw6_gain}}
%\subfigure[Inflation drift]{\includegraphics[scale = \myTinierFigScale]{\myFigPath dw6_drift}}
\caption{$\bar{\theta}=0.029$. Carvalho et al's estimate extremely unanchored!}
\end{figure}

\end{frame}
%%%%%%%%%%%%%%%%%%


%%%%%%%% Slide %%%%%%
\begin{frame}
	\frametitle{Gain when Varying Taylor-rule coefficients}
% gain and drift in pi in this case


\blfootnote{ $\bar{\theta}=1$}
\begin{figure}[h!]
\subfigure[$(\psi_{\pi}, \psi_x) = (1.5, 1.5)$]{\includegraphics[scale = \myTinierFigScale]{\myFigPath dw4_gain}}
\subfigure[$(\psi_{\pi}, \psi_x) = (\colorbox{yellow}{1.1}, 1.5)$]{\includegraphics[scale = \myTinierFigScale]{\myFigPath dw5_gain_psipi1_1}}
%\begin{tikzpicture}[remember picture,overlay]
%\draw [red,thick, rounded corners] (-2.55,-0.6) rectangle (-2.05,-0.1);
%\end{tikzpicture}
\caption{Less aggressive on inflation}
\end{figure}

\end{frame}
%%%%%%%%%%%%%%%%%%

%%%%%%%% Slide %%%%%%
\begin{frame}
	\frametitle{}
\begin{figure}[h!]
\subfigure[$(\psi_{\pi}, \psi_x) = (1.5, 1.5)$]{\includegraphics[scale = \myTinierFigScale]{\myFigPath dw4_gain}}
\subfigure[$(\psi_{\pi}, \psi_x) = (\colorbox{yellow}{3}, 1.5)$]{\includegraphics[scale = \myTinierFigScale]{\myFigPath dw5_gain_psipi3}}
%\begin{tikzpicture}[remember picture,overlay]
%\draw [red,thick, rounded corners] (-2.55,-0.6) rectangle (-2.05,-0.1);
%\end{tikzpicture}
\caption{More aggressive on inflation}
\end{figure}

\end{frame}
%%%%%%%%%%%%%%%%%%

%%%%%%%% Slide %%%%%%
\begin{frame}
	\frametitle{}
\begin{figure}[h!]
\subfigure[$(\psi_{\pi}, \psi_x) = (1.5, 1.5)$]{\includegraphics[scale = \myTinierFigScale]{\myFigPath dw4_gain}}
\subfigure[$(\psi_{\pi}, \psi_x) = (\colorbox{yellow}{5}, 1.5)$]{\includegraphics[scale = \myTinierFigScale]{\myFigPath dw5_gain_psipi5}}
%\begin{tikzpicture}[remember picture,overlay]
%\draw [red,thick, rounded corners] (-2.55,-0.6) rectangle (-2.05,-0.1);
%\end{tikzpicture}
\caption{Too aggressive on inflation?}
\end{figure}

\end{frame}
%%%%%%%%%%%%%%%%%%


%%%%%%% Slide %%%%%%
\begin{frame}
	\frametitle{Today's conclusion and work ahead}
\vspace{-1cm}
	
\begin{itemize}
\item Model of anchoring $+$ macro model with monetary policy

\

\begin{itemize}
\item[] $\rightarrow$ investigation of new constraint on monetary policy
\end{itemize}

\

\item Next steps

\

\begin{itemize}
\item Write and solve monetary policy problem

\

\item Estimate model
\end{itemize}

\end{itemize}


\end{frame}
%%%%%%%%%%%%%%%%%

%%%%%%% Slide %%%%%%
\begin{frame}
	\frametitle{}
\vspace{-1cm}
	
\centering Thank you!


\end{frame}
%%%%%%%%%%%%%%%%%

%%%%%%%%%%%%%%%%%
%         TEMPLATES
%%%%%%%%%%%%%%%%%

%%%%%%%% Slide %%%%%%
%\begin{frame}
%	\frametitle{Slide template}
%
%\end{frame}
%%%%%%%%%%%%%%%%%%

%%%%%%%% Slide %%%%%%  	
%\begin{frame}
%	\frametitle{A beamer button template}
%	\label{identification}
%	
%\hyperlink{Technicalities}{\beamergotobutton{Technicalities}}
%
%\end{frame}
%%%%%%%%%%%%%%%%%%



%%%%%%%%%%%%%%%%%%%%%%%%%%%%%%%%%%%%%%%%%%%%%%%%%%%%%%%%%%%%%%%%%%%%%%%
%%%%%%%                     APPENDIX  
%%%%%%%%%%%%%%%%%%%%%%%%%%%%%%%%%%%%%%%%%%%%%%%%%%%%%%%%%%%%%%%%%%%%%%%
\appendix
\backupbegin
%%%%%%%% Slide %%%%%%
\begin{frame}
	\frametitle{Derivations}
	\label{derivations}

Household FOCs
 \begin{align*}
\hat{C}_{t}^i & = \hat{\E}^i_t \hat{C}_{t+1}^i - \sigma(\hat{i}_{t} -\hat{\E}^i_t \hat{\pi}_{t+1})    \label{EE} \numberthis \\
\\
\hat{\E}^i_t \sum_{s=0}^{\infty}\beta^s \hat{C}^i_t& =\omega^i_t + \hat{\E}^i_t \sum_{s=0}^{\infty}\beta^s \hat{Y}^i_t \label{IBC} \numberthis
\end{align*}
where a hat denotes log-linear approximation and $\omega_t^i \equiv \frac{(1+i_{t-1})B^i_{t-1} }{P_t Y^*}$.

\begin{enumerate}
\item Solve (\ref{EE}) backward to some date $t$, take expectations at $t$ 
\item Sub in (\ref{IBC})
\item Aggregate over households $i$
\item[$\rightarrow$] Obtain (\ref{prestons18})
\end{enumerate}



\hyperlink{NK}{\beamerreturnbutton{Return}}	


\end{frame}
%%%%%%%%%%%%%%%%%%

%%%%%%%% Slide %%%%%%
\begin{frame}
	\frametitle{Compact notation}
	\label{compact}

 \begin{align*}
z_t & = A_1 f_{a,t} + A_2 f_{b,t} + A_3 s_t \label{LOM_LR} \numberthis \\
\\
s_t & = P s_{t-1} + \epsilon_t \label{exog} \numberthis
\end{align*}
where

\begin{equation}
 z_t \equiv \begin{pmatrix} \pi_t \\ x_t \\ i_t
 \end{pmatrix} 
 \quad \quad \quad 
  s_t  \equiv \begin{pmatrix} r_t^n \\ \bar{i}_t \\ u_t 
 \end{pmatrix} 
\end{equation}
and

  \begin{align}
f_{a,t}  \equiv  \hat{\E}_t\sum_{T=t}^{\infty} (\alpha\beta)^{T-t } z_{T+1} \quad \quad \quad \quad f_{b,t}  \equiv \hat{\E}_t\sum_{T=t}^{\infty} (\beta)^{T-t } z_{T+1} \label{fafb}
\end{align}

\hyperlink{NK}{\beamerreturnbutton{Return}}	


\end{frame}
%%%%%%%%%%%%%%%%%%
%
%%%%%%%% Slide %%%%%%
%\begin{frame}
%\frametitle{A beamer button template, how to get back to main text}
%\label{Steps}
%
%\begin{equation}
%D = \begin{bmatrix}
%d_{11} & \gamma_{12} & \gamma_{13} & d_{14} & \cdots \\
%d_{21} & \gamma_{22} & \gamma_{23} & d_{24} & \cdots \\
%\vdots & \vdots & \vdots & \ddots & \vdots 
%\end{bmatrix}
%\end{equation}
%
%\hyperlink{calcCBloss}{\beamerreturnbutton{Return}}	
%\end{frame}
%%%%%%%%%%%%%%%%%%

\backupend


\end{document}
