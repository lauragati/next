\documentclass{beamer}
\usepackage[english]{babel}
\usepackage[latin1]{inputenc}
%\usepackage[T1]{fontenc} % don't know what this does: it sets font encoding
%\usepackage{gfsartemisia} % this chooses the font
%%\usepackage{txfonts}
%% the above two seem to give the same, the below two as well
%%\usepackage{accanthis} 
%%\usepackage{concmath}
% Another way to go is the following
%  The available themes are: structurebold, structurebolditalic, structuresmallcapsserif, structureitalicsserif, serif and default.
%\usefonttheme{default}
%\usefonttheme{serif}
%\usefonttheme{structurebold}
%%\usefonttheme{structurebolditalic} % .sty wasn't found
%\usefonttheme{structuresmallcaps}
\usefonttheme{structuresmallcapsserif}
%%%%%%\usefonttheme{structureitalicsserif} % .sty wasn't found
%%%%%\usefonttheme{serif} % the font theme has to correspond to the font used
%%%%\usefonttheme{structurebold} % it seems you can use a combo of these
%%%%% Here comes the font command itself:
%%%%% The available fonts depend on your LATEX installation, the most common are: mathptmx, helvet, avat, bookman, chancery, charter, culer, mathtime, mathptm, newcent, palatino, pifont and utopia.
%%%%% I also have ccfonts
%%%%\usepackage{helvet}
%%%%\usepackage[T1]{fontenc}  % some fonts, like concrete math, require a specific font encoding (concrete math requires T1)

\usepackage[default]{lato}

\usepackage{amssymb}
\usepackage{amsmath}
\usepackage{booktabs}
\usepackage{verbatim}
\usepackage{caption}
\usepackage{float}
\usepackage{csquotes}
\usepackage{sansmathaccent}
\usepackage{subfigure}
\usepackage{multicol}
\pdfmapfile{+sansmathaccent.map}
\usepackage{pgfplots,tikz}
\usepackage{overpic}

\def \myFigPath {../../figures/} 
\def \myTablePath {../../tables/} 

\beamertemplatenavigationsymbolsempty % this removes the blue navigation bar that used to be on the bottom right
\setbeamersize{text margin left=5mm,text margin right=12mm} 
%
\font\reali=msbm10 at 12pt

%


\author[]{Laura G\'ati}
% note: in the [] you can put a short name to be displayed in the footer

\institute[]{Boston College}
% note: in the [] you can put a short affiliation to be displayed in the footer

\title[]{Monetary Policy \& Anchored Expectations}
% note: in the [] you can put a short title to be displayed in the footer

\date[]{October 1, 2019}
% note: in the [] you can put a short date to be displayed in the footer

%\usetheme{CambridgeUS} % CambridgeUS, Madrid
\usetheme{Madrid}
\usecolortheme{dove}



\AtBeginSection[]
{
  \begin{frame}<beamer>
    \frametitle{Roadmap}
    \tableofcontents[currentsection]
  \end{frame}
}

% New Commands
\newcommand{\real}{\hbox{\reali R}}
\newcommand{\realp}{\hbox{\reali R}_{\scriptscriptstyle +}}
\newcommand{\realpp}{\hbox{\reali R}_{\scriptscriptstyle ++}}
\newcommand{\R}{\mathbb{R}}
\DeclareMathOperator{\E}{\mathbb{E}}
\DeclareMathOperator{\argmin}{arg\,min}
\newcommand\w{3.0in}
\newcommand\wnum{3.0}
\def\myFigWidth{5.3in}
\def\mySmallerFigWidth{2.1in}
\def\myEvenBiggerFigScale{0.8}
\def\myPointSixFigScale{0.6}
\def\myBiggerFigScale{0.4}
\def\myFigScale{0.3}
\def\mySmallFigScale{0.25}
\def\mySmallerFigScale{0.18}
\def\myTinyFigScale{0.16}
\def\myPointFourteenFigScale{0.14}
\def\myTinierFigScale{0.12}
\def\myEvenTinierFigScale{0.10}
\def\myCrazyTinyFigScale{0.09}

\newtheorem{prop}{Proposition}
\newtheorem*{remark}{Remark}

\newcommand\numberthis{\addtocounter{equation}{1}\tag{\theequation}} % this defines a command to make align only number this line

% creating colored lines for legends
% green solid
\newcommand{\greenline}{\raisebox{2pt}{\tikz{\draw[-,black!40!green,solid,line width = 1.5pt](0,0) -- (5mm,0);}}}
% red solid and dashed
\newcommand{\redline}{\raisebox{2pt}{\tikz{\draw[-,red,solid,line width = 1.5pt](0,0) -- (5mm,0);}}}
\newcommand{\reddashedline}{\raisebox{2pt}{\tikz{\draw[-,red,dashed,line width = 1.5pt](0,0) -- (5mm,0);}}}
% blue solid and dashed
\newcommand{\blueline}{\raisebox{2pt}{\tikz{\draw[-,blue,solid,line width = 1.5pt](0,0) -- (5mm,0);}}}
\newcommand{\bluedashedline}{\raisebox{2pt}{\tikz{\draw[-,blue,dashed,line width = 1.5pt](0,0) -- (5mm,0);}}}
% black solid and dashed
\newcommand{\blackline}{\raisebox{2pt}{\tikz{\draw[-,black,solid,line width = 1.5pt](0,0) -- (5mm,0);}}}
\newcommand{\blackdashedline}{\raisebox{2pt}{\tikz{\draw[-,black,dashed,line width = 1.5pt](0,0) -- (5mm,0);}}}


% These are my colors -- there are many like them, but these ones are mine (Paul Goldsmith-Pinkham's)
\definecolor{blue_PGP}{RGB}{0,114,178}
\definecolor{red_PGP}{RGB}{213,94,0}
\definecolor{yellow_PGP}{RGB}{240,228,66}
\definecolor{green_PGP}{RGB}{0,158,115}
%% I use a beige off white for my background 
\definecolor{MyBackground}{RGB}{255,253,218}
%% Uncomment this if you want to change the background color to something else 
%\setbeamercolor{background canvas}{bg=MyBackground} % this is a very nice beige
%% Change the bg color to adjust your transition slide background color! 
%\newenvironment{transitionframe}{
%\setbeamercolor{background canvas}{bg=yellow} \begin{frame}}{
%\end{frame} }

%\setbeamercolor{frametitle}{fg=blue_PGP} 
%\setbeamercolor{title}{fg=black} 
%\setbeamertemplate{footline}[frame number]  % this sets the footer the same color as the background
%\setbeamertemplate{navigation symbols}{} % don't know what this does
%\setbeamercolor{button}{bg=MyBackground,fg=blue_PGP}

%%%%%%%%%%%%            BEGIN DOCUMENT         %%%%%%%%%%%%%%%%%%%%%
\begin{document}


\begin{frame}

\maketitle


\end{frame}


%%%%%%% Slide %%%%%%
\begin{frame}
	\frametitle{Motivation}

A quote or a plot, something about how policy-makers worry about anchored inflation expectations
\end{frame}
%%%%%%%%%%%%%%%%%

%%%%%%%% Slide %%%%%%
\begin{frame}
	\frametitle{Anchoring - a concern for monetary policy?}

\begin{itemize}
\item What is anchoring?

\

\

\item (Why) do we want expectations to be anchored?
\end{itemize}

\end{frame}
%%%%%%%%%%%%%%%%%%

%%%%%%%% Slide %%%%%%
\begin{frame}
	\frametitle{This paper}
	
	\begin{enumerate}
	\item Defines anchoring from the lens of a learning model
	
	\
	
	\
	
	\item Embeds anchoring in a New Keynesian (NK) model with econometric learning
	
	\
	
	\
	
	\item Goal: Derive optimal monetary policy, and contrast it with rational expectations (RE) 
	
	\
	
	\
	
	\item Today: some initial simulations with different specifications for monetary policy
	\end{enumerate}


\end{frame}
%%%%%%%%%%%%%%%%%%

%%%%%%%% Slide %%%%%%
\begin{frame}
	\frametitle{In words}
	
	\begin{enumerate}
	\item Expectations anchored if unresponsive to short-run fluctuations
	
	\
	
	\
	
	\item Blessing or curse for monetary policy?

	\end{enumerate}


\end{frame}
%%%%%%%%%%%%%%%%%%


%%%%%%%% Slide %%%%%%
\begin{frame}
	\frametitle{Related literature}

\begin{itemize}
\item Optimal monetary policy in New Keynesian models
\item[] Clarida, Gali \& Gertler (1999), Woodford (2003)

\

\

\item Econometric learning
\item[] Evans \& Honkapohja (2001), Preston (2005), Graham (2011)

\

\

\item Anchoring
\item[] Carvalho et al (2019), Svensson (2015), Hooper et al (2019)
\end{itemize}



\end{frame}
%%%%%%%%%%%%%%%%%%

\section{Intuition: what is anchoring and why should it matter?}
%%%%%%%% Slide %%%%%%
\begin{frame}
	\frametitle{New Keynesian Phillips Curve}

\begin{align*}
\pi_t & = \beta \hat{\E}_t\pi_{t+1} + \kappa x_t
\end{align*}

\begin{itemize}
\item $\pi_t = $ inflation

\

\item $x_t =$ output gap

\

\item $\hat{\E}_t =$ expectation-operator (not necessarily rational)


\end{itemize}



\end{frame}
%%%%%%%%%%%%%%%%%%

%%%%%%%% Slide %%%%%%
\begin{frame}
	\frametitle{}

Suppose a negative demand shock:

\

\begin{align*}
\pi_t & = \beta \hat{\E}_t\pi_{t+1} + \kappa \underset{\downarrow}{x_t} 
\end{align*}




\end{frame}
%%%%%%%%%%%%%%%%%%

%%%%%%%% Slide %%%%%%
\begin{frame}
	\frametitle{}

If expectations do not move:

\

\begin{align*}
\underset{\downarrow}{\pi_t} & = \beta \hat{\E}_t\pi_{t+1}+ \kappa \underset{\downarrow}{x_t} 
\end{align*}




\end{frame}
%%%%%%%%%%%%%%%%%%

%%%%%%%% Slide %%%%%%
\begin{frame}
	\frametitle{}

If seeing $\underset{\downarrow}{\pi_t}$, expectations adjust:

\

\begin{align*}
\underset{\downarrow \color{red}{\downarrow}}{\pi_t} & = \beta \underset{\color{blue}{\downarrow}}{\hat{\E}_t\pi_{t+1}}+ \kappa \underset{\downarrow}{x_t} 
\end{align*}

\

\

\

\begin{itemize}
\item[] Keeping expectations stable may be desirable

\

\item[$\rightarrow$]  Anchoring as a notion of stable expectations
\end{itemize}

\


\end{frame}
%%%%%%%%%%%%%%%%%%

\section{A formal notion of anchoring}
%%%%%%%% Slide %%%%%%
\begin{frame}
	\frametitle{Anchoring definition}
Suppose firms

\

\begin{itemize}
\item observe everything up to time $t$ 

\

\

\item do not observe future variables

\

\

\item KEY: are unsure about the long-run mean of inflation, $\bar{\pi}$
\end{itemize}


\



\end{frame}
%%%%%%%%%%%%%%%%%%

%%%%%%%% Slide %%%%%%
\begin{frame}
	\frametitle{Anchoring definition II}

Firms construct one-period-ahead inflation forecasts as

\
\color{red}{CHECK}
\begin{equation}
\hat{\E}_t\pi_{t+1} =  \bar{\pi}_{t-1}+bs_{t}
\end{equation}

\

\begin{itemize}
\item[] $\bar{\pi} =$ drift in inflation ($=$ long-run mean, ``target")

\



\item[] $\hat{\E} =$ subjective expectation operator (not rational expectations, $\E$)

\



\item[] $b =$ matrix of constants 

\

\item[] $s =$ shocks
\end{itemize}

\



\end{frame}
%%%%%%%%%%%%%%%%%%

%%%%%%%% Slide %%%%%%
\begin{frame}
	\frametitle{Anchoring definition III}
	
And update their estimate of the inflation drift as \\
(Carvalho et al, 2019)
\
\color{red}{CHECK}
 \begin{align*}
\bar{\pi}_{t} & = \bar{\pi}_{t-1} +k_t\overbrace{\big(\pi_{t} -(\bar{\pi}_{t-1}+bs_{t}) \big)}^{\text{short-run forecast error}}  \label{RLS_anchoring} \numberthis \\
\\
k_t &= \mathbb{I} \times \frac{1}{k_{t-1} +1} + (1-\mathbb{I})\times \bar{g} \label{gain} \numberthis\\
\end{align*}

\begin{itemize}
\item[] $\bar{g} =$ constant

\

\item[] $k =$ gain $\rightarrow$ sensitivity to short-run forecast errors
\end{itemize}
\

Anchoring: when $k$ decreases over time.

\

\end{frame}
%%%%%%%%%%%%%%%%%%

%%%%%%%% Slide %%%%%%
\begin{frame}
	\frametitle{Anchoring definition IV}
	
	
 \begin{align*}
\mathbb{I} & = \begin{cases} 1 \quad \text{if} \; \theta_t \leq \bar{\theta}  \\ 0 \quad \text{otherwise.}\numberthis\\
\end{cases} \\
\\
\theta_t & = |\hat{\E}_{t-1}\pi_t - \E_{t-1}\pi_t| / \sigma_s \label{criterion}\numberthis
\end{align*}

\begin{itemize}
\item[] $\bar{\theta} = $ constant

\

\item[] $\theta = $ difference between subjective and objective (model-consistent) expectations, scaled by noise
\end{itemize}

\

Anchoring $\equiv$ when the deviation between objective and subjective expectations is small enough such that firms choose decreasing gains

\end{frame}
%%%%%%%%%%%%%%%%%%

%%%%%%%% Slide %%%%%%
\begin{frame}
	\frametitle{Intuition}

\begin{itemize}
\item When my expectation far from what is implied by the model, I update my estimate of the drift strongly 

\

\

\item When the two are close, I load less on my forecast error because it matters less

\

\

\item Unanchored if: $\pi$ deviates from target 
\item[] i) strongly enough
\item [] ii) long enough
\end{itemize}



\end{frame}
%%%%%%%%%%%%%%%%%%


\section{NK model with anchoring}

%%%%%%%% Slide %%%%%%
\begin{frame}
	\frametitle{3-Equation New Keynesian Model}

\begin{align*}
x_t &=  -\sigma i_t +\hat{\E}_t \sum_{T=t}^{\infty} \beta^{T-t }\big( (1-\beta)x_{T+1} - \sigma(\beta i_{T+1} - \pi_{T+1}) +\sigma r_T^n \big) \label{prestons18}  \numberthis \\
\\
\pi_t &= \kappa x_t +\hat{\E}_t \sum_{T=t}^{\infty} (\alpha\beta)^{T-t }\big( \kappa \alpha \beta x_{T+1} + (1-\alpha)\beta \pi_{T+1} + u_T\big) \label{prestons19}  \numberthis \\
\\
i_t &= \psi_{\pi}\pi_t + \psi_{x} x_t + \bar{i}_t \label{TR} \numberthis
\end{align*}

\

``Long-horizon forecasts'' $\rightarrow$ firms do not know beliefs of others \\
(Preston, 2005)

\end{frame}
%%%%%%%%%%%%%%%%%%

%%%%%%%% Slide %%%%%%
\begin{frame}
	\frametitle{Compact notation}

 \begin{align*}
z_t & = A_1 f_{a,t} + A_2 f_{b,t} + A_3 s_t \label{LOM_LR} \numberthis \\
\\
s_t & = P s_{t-1} + \epsilon_t \label{exog} \numberthis
\end{align*}
where

\begin{equation}
 z_t \equiv \begin{pmatrix} \pi_t \\ x_t \\ i_t
 \end{pmatrix} 
 \quad \quad \quad 
  s_t  \equiv \begin{pmatrix} r_t^n \\ \bar{i}_t \\ u_t 
 \end{pmatrix} 
\end{equation}
and

  \begin{align}
f_{a,t}  \equiv  \hat{\E}_t\sum_{T=t}^{\infty} (\alpha\beta)^{T-t } z_{T+1} \quad \quad \quad \quad f_{b,t}  \equiv \hat{\E}_t\sum_{T=t}^{\infty} (\beta)^{T-t } z_{T+1} \label{fafb}
\end{align}




\end{frame}
%%%%%%%%%%%%%%%%%%

\section{Simulations}

%%%%%%%% Slide %%%%%%
\begin{frame}
	\frametitle{Calibration}

\begin{center}
\begin{tabular}{ c | c }
 $\beta$ & 0.98  \\ \hline
 $\sigma$ & 0.5  \\  \hline
 $\alpha$ & 0.5    \\\hline
 $\psi_{\pi} $& 1.5    \\\hline
 $\psi_x$ & 1.5    \\\hline
 $\bar{g}$ & $0.145^{-1}$    \\ \hline
 $\bar{\theta}$ & 1    \\ \hline
    $\rho_r$ & 0.9    \\ \hline
    $\rho_i$ & 0.9    \\ \hline
    $\rho_u$ & 0.9    \\ \hline
    $\sigma_i$ & 0.1    \\ \hline
    $\sigma_r$ & 0.359    \\ \hline
    $\sigma_u$ & 0.277    \\ \hline
\end{tabular}
\end{center}

Carvalho et al, 2019
\end{frame}
%%%%%%%%%%%%%%%%%%
%%%%%%%% Slide %%%%%%
\begin{frame}
	\frametitle{Role of learning}
% learning models against RE


\end{frame}
%%%%%%%%%%%%%%%%%%

%%%%%%%% Slide %%%%%%
\begin{frame}
	\frametitle{Varying $\bar{\theta}$}
% learning model with and w/o anchoring, potentially for different thetbars


\end{frame}
%%%%%%%%%%%%%%%%%%

%%%%%%%% Slide %%%%%%
\begin{frame}
	\frametitle{Varying Taylor-rule coefficients}
% learning model with and w/o anchoring, for different TR coefficients


\end{frame}
%%%%%%%%%%%%%%%%%%



%%%%%%%% Slide %%%%%%
%\begin{frame}
%	\frametitle{Slide template}
%
%\end{frame}
%%%%%%%%%%%%%%%%%%

%%%%%%%%%%%%%%%%%
%         TEMPLATES
%%%%%%%%%%%%%%%%%

%%%%%%%% Slide %%%%%%
%\begin{frame}
%	\frametitle{Slide template}
%
%\end{frame}
%%%%%%%%%%%%%%%%%%

%%%%%%%% Slide %%%%%%  	
%\begin{frame}
%	\frametitle{A beamer button template}
%	\label{identification}
%	
%\hyperlink{Technicalities}{\beamergotobutton{Technicalities}}
%
%\end{frame}
%%%%%%%%%%%%%%%%%%



%%%%%%%%%%%%%%%%%%%%%%%%%%%%%%%%%%%%%%%%%%%%%%%%%%%%%%%%%%%%%%%%%%%%%%
%%%%%%                     APPENDIX  
%%%%%%%%%%%%%%%%%%%%%%%%%%%%%%%%%%%%%%%%%%%%%%%%%%%%%%%%%%%%%%%%%%%%%%
\appendix

%%%%%%% Slide %%%%%%
\begin{frame}
\frametitle{A beamer button template, how to get back to main text}
\label{Steps}

\begin{equation}
D = \begin{bmatrix}
d_{11} & \gamma_{12} & \gamma_{13} & d_{14} & \cdots \\
d_{21} & \gamma_{22} & \gamma_{23} & d_{24} & \cdots \\
\vdots & \vdots & \vdots & \ddots & \vdots 
\end{bmatrix}
\end{equation}

\hyperlink{calcCBloss}{\beamerreturnbutton{Return}}	
\end{frame}
%%%%%%%%%%%%%%%%%




\end{document}
