\documentclass[11pt]{beamer}
\usepackage[english]{babel}
\usepackage[latin1]{inputenc}
\usepackage{tikz}
\usetikzlibrary{calc}
\usepackage{subfigure}
\usepackage{amssymb}
\usepackage{amsmath}
\usepackage{booktabs}
\usepackage{verbatim}
\usepackage{caption}
\usepackage{float}
\usepackage{csquotes}
\usepackage{sansmathaccent}
\usepackage{subfigure}
\usepackage{multicol}
\pdfmapfile{+sansmathaccent.map}
\usepackage{pgfplots,tikz}
\usetikzlibrary{tikzmark,calc}
\usepackage{overpic}
\usepackage{color,soul}
\usepackage{stackengine}

% Layout and style
\usefonttheme{serif}
\usepackage{palatino} 

\usetheme{Madrid}
\usecolortheme{dove}

\def \myFigPath {../../../figures/} 
\def \myTablePath {../../../tables/} 

%%%% PROGRESS BAR %%%%%%%%%%%%%%%%%%%%%%%%%%%%%%%%%%%%%%%%%%
\definecolor{pbblue}{HTML}{0A75A8}% filling color for the progress bar
\definecolor{pbgray}{HTML}{575757}% background color for the progress bar

% a set of nice blue, nice red and nice green
\definecolor{dodgerblue}{RGB}{16,78,139}
\definecolor{aquamarinegreen}{RGB}{69,139,116}
\definecolor{brownredlight}{RGB}{238,59,59}
\definecolor{brownreddark}{RGB}{205,51,51}



\makeatletter
\def\progressbar@progressbar{} % the progress bar
\newcount\progressbar@tmpcounta% auxiliary counter
\newcount\progressbar@tmpcountb% auxiliary counter
\newdimen\progressbar@pbht %progressbar height
\newdimen\progressbar@pbwd %progressbar width
\newdimen\progressbar@tmpdim % auxiliary dimension

\progressbar@pbwd=1.15\linewidth % change width of progress bar
\progressbar@pbht=0.5ex % change thickness of progress bar

% the progress bar
\def\progressbar@progressbar{%

    \progressbar@tmpcounta=\insertframenumber
    \progressbar@tmpcountb=\inserttotalframenumber
    \progressbar@tmpdim=\progressbar@pbwd
    \multiply\progressbar@tmpdim by \progressbar@tmpcounta
    \divide\progressbar@tmpdim by \progressbar@tmpcountb

  \begin{tikzpicture}[rounded corners=2pt,very thin]

    \shade[top color=pbgray!10,bottom color=pbgray!10,middle color=pbgray!10]
      (0pt, 0pt) rectangle ++ (\progressbar@pbwd, \progressbar@pbht);

      \shade[draw=dodgerblue,top color=dodgerblue,bottom color=dodgerblue,middle color=dodgerblue] %
        (0pt, 0pt) rectangle ++ (\progressbar@tmpdim, \progressbar@pbht);

% ADD COMPLETION PERCENTAGE
%    \draw[color=normal text.fg!50]  
%      (0pt, 0pt) rectangle (\progressbar@pbwd, \progressbar@pbht) 
%        node[pos=0.5,color=normal text.fg] {\textnormal{%
%             \pgfmathparse{\insertframenumber*100/\inserttotalframenumber}%
%             \pgfmathprintnumber[fixed,precision=2]{\pgfmathresult}\,\%%
%        }%
%    };
  \end{tikzpicture}%
}

\addtobeamertemplate{headline}{}
{%
  \begin{beamercolorbox}[wd=\paperwidth,ht=4ex,center,dp=1ex]{white}%
    \progressbar@progressbar%
  \end{beamercolorbox}%
}
\makeatother

%%%% PROGRESS BAR %%%%%%%%%%%%%%%%%%%%%%%%%%%%%%%%%%%%%%%%%%


% footnote suppressing number
\newcommand\blfootnote[1]{%
  \begingroup
  \renewcommand\thefootnote{}\footnote{#1}%
  \addtocounter{footnote}{-1}%
  \endgroup
}

% try to force appendix to be numbered differently
\newcommand{\backupbegin}{
   \newcounter{framenumberappendix}
   \setcounter{framenumberappendix}{\value{framenumber}}
}
\newcommand{\backupend}{
   \addtocounter{framenumberappendix}{-\value{framenumber}}
   \addtocounter{framenumber}{\value{framenumberappendix}} 
}


% Set colors of butthons
\setbeamercolor{button}{bg=dodgerblue,fg=white}

% this removes the blue navigation bar that used to be on the bottom right
\beamertemplatenavigationsymbolsempty 
\setbeamersize{text margin left=5mm,text margin right=5mm} 


\AtBeginSection[]
{
  \begin{frame}<beamer>
    \frametitle{Structure of talk}
    \tableofcontents[currentsection]
  \end{frame}
}
\useinnertheme{circles}

% resets the bullet points of enumerate and itemize to defaults
\setbeamertemplate{itemize items}[bullet]
\setbeamertemplate{enumerate items}[default]

% this removes the ugly bullets for the table of contents and just adds numbers
\setbeamertemplate{section in toc}{\inserttocsectionnumber.~\inserttocsection}

\newcommand\numberthis{\addtocounter{equation}{1}\tag{\theequation}} % this defines a command to make align only number this line

% Colored lines for legends
% green solid
\newcommand{\greenline}{\raisebox{2pt}{\tikz{\draw[-,black!40!green,solid,line width = 1.5pt](0,0) -- (5mm,0);}}}
% red solid, dashed, dotted
\newcommand{\redline}{\raisebox{2pt}{\tikz{\draw[-,red,solid,line width = 1.5pt](0,0) -- (5mm,0);}}}
\newcommand{\reddashedline}{\raisebox{2pt}{\tikz{\draw[-,red,dashed,line width = 1.5pt](0,0) -- (5mm,0);}}}
\newcommand{\reddottedline}{\raisebox{2pt}{\tikz{\draw[-,red,densely dotted,line width = 1.5pt](0,0) -- (5mm,0);}}}

% blue solid and dashed
\newcommand{\blueline}{\raisebox{2pt}{\tikz{\draw[-,blue,solid,line width = 1.5pt](0,0) -- (5mm,0);}}}
\newcommand{\bluedashedline}{\raisebox{2pt}{\tikz{\draw[-,blue,dashed,line width = 1.5pt](0,0) -- (5mm,0);}}}
% black solid and dashed
\newcommand{\blackline}{\raisebox{2pt}{\tikz{\draw[-,black,solid,line width = 1.5pt](0,0) -- (5mm,0);}}}
\newcommand{\blackdashedline}{\raisebox{2pt}{\tikz{\draw[-,black,dashed,line width = 1.5pt](0,0) -- (5mm,0);}}}


% New Commands
\newcommand{\real}{\hbox{\reali R}}
\newcommand{\realp}{\hbox{\reali R}_{\scriptscriptstyle +}}
\newcommand{\realpp}{\hbox{\reali R}_{\scriptscriptstyle ++}}
\newcommand{\R}{\mathbb{R}}
\DeclareMathOperator{\E}{\mathbb{E}}
\DeclareMathOperator{\argmin}{arg\,min}
\newcommand\w{3.0in}
\newcommand\wnum{3.0}
\def\myFigWidth{5.3in}
\def\mySmallerFigWidth{2.1in}
\def\myEvenBiggerFigScale{0.8}
\def\myPointSixFigScale{0.6}
\def\myBiggerFigScale{0.4}
\def\myFigScale{0.3}
\def\mySmallFigScale{0.25}
\def\mySmallerFigScale{0.18}
\def\myTinyFigScale{0.16}
\def\myPointFourteenFigScale{0.14}
\def\myTinierFigScale{0.12}
\def\myEvenTinierFigScale{0.10}
\def\myCrazyTinyFigScale{0.09}

\newtheorem{prop}{Proposition}
\newtheorem*{remark}{Remark}
\newtheorem{result}{Result}
%\newtheorem{lemma}{Lemma}
%\newtheorem{corollary}{Corollary}


%%%%%%%%%%%%%%%%%%%%
% DEFINE ALL FIGNAMES HERE
%%%%%%%%%%%%%%%%%%%%

% Market-based expectations
\def\fignameMarketEPi{epi10_2020_06_04}
% cleaned from liq premium
\def\fignameMarketEPiCleaned{cleaned_epi10_2020_07_28}


% Estimated coefficients alpha
\def\fignameAlphaHat{alph_opt_constant_only_pi_only_N_100_nfe_5_loss_0_gridspacing_uniform_Wdiffs2_100000_Wmid_0_Nestimations_command_GMM_LOMgain_univariate_23_Jul_2020}

% Comparative statics of optimal policy
% di/dk
\def\fignameDiDk{analyze_opt_policy_ik}
% di/dpibar
\def\fignameDiDpibar{analyze_opt_policy_ip}

% Observables in PEA for TR
\def\fignamePEAobsTR{implement_anchTC_obs_TR}
% Observables in PEA for anchoring
\def\fignamePEAobsAnch{implement_anchTC_obs}

% Central bank loss as a function of psi_pi in RE
\def\fignameCBlossRE{plot_sim_loss_pretty_loss_RE_again_critCUSUM_constant_only_params_psi_pi_1_5_psi_x_0_gbar_0_145_thetbar_16_thettilde_2_5_kap_0_8_lamx_0_lami_0_2020_06_05}
% Central bank loss as a function of psi_pi in anchoring
\def\fignameCBlossAnch{plot_sim_loss_pretty_loss_again_critCUSUM_constant_only_T_400_N_100_burnin_0_params_psi_pi_1_5_psi_x_0_gbar_0_145_thetbar_16_thettilde_2_5_kap_0_8_lamx_0_lami_0_date_2020_06_05}

% Autocovariogram
\def\fignameAutocov{autocovariogram_constant_only_pi_only_N_100_nfe_5_loss_0_gridspacing_uniform_Wdiffs2_100000_Wmid_0_Nestimations_command_GMM_LOMgain_univariate_23_Jul_2020}

% Compare PEA and VFI policy for sequence of shocks X1
\def\fignamePEAvsVFIfirstX{compare_value_pea_results_value_outputs_server_inputs_pretty}
% Compare PEA and VFI policy for sequence of shocks X2
\def\fignamePEAvsVFIsecondX{compare_value_pea_results_value_outputs_server02_Jun_2020_14_58_12_pea_outputs_30_May_2020_10_18_28_pretty}

% IRFs anchored vs unanchored
\def\fignameIRFanchored{command_IFS_anchoring_pretty_RIR_LH_anch_monpol_again_critCUSUM_constant_only_T_400_N_1000_burnin_5_params_psi_pi_1_5_psi_x_0_gbar_0_145_thetbar_16_thettilde_2_5_kap_0_8_lamx_0_lami_0_date_2020_06_05}
\def\fignameIRFunanchored{command_IFS_anchoring_pretty_RIR_LH_unanch_monpol_again_critCUSUM_constant_only_T_400_N_1000_burnin_5_params_psi_pi_1_5_psi_x_0_gbar_0_145_thetbar_16_thettilde_2_5_kap_0_8_lamx_0_lami_0_date_2020_06_05}

% Cross-sectional average gains for various psi_pi
\def\fignameGainPsiSmall{command_IFS_anchoring_pretty_invgain_again_critCUSUM_constant_only_params_psi_pi_1_01_psi_x_0_gbar_0_145_thetbar_16_thettilde_2_5_kap_0_8_lamx_0_lami_0_2020_06_05}
\def\fignameGainPsiMedium{command_IFS_anchoring_pretty_invgain_again_critCUSUM_constant_only_params_psi_pi_1_5_psi_x_0_gbar_0_145_thetbar_16_thettilde_2_5_kap_0_8_lamx_0_lami_0_2020_06_05}
\def\fignameGainPsiBig{command_IFS_anchoring_pretty_invgain_again_critCUSUM_constant_only_params_psi_pi_2_psi_x_0_gbar_0_145_thetbar_16_thettilde_2_5_kap_0_8_lamx_0_lami_0_2020_06_05}

% IRFs unanchored for various psi_pi
\def\fignameIRFpsipiSmall{command_IFS_anchoring_pretty_RIR_LH_unanch_monpol_again_critCUSUM_constant_only_T_400_N_1000_burnin_5_params_psi_pi_1_01_psi_x_0_gbar_0_145_thetbar_16_thettilde_2_5_kap_0_8_lamx_0_lami_0_date_2020_06_05}
\def\fignameIRFpsipiMedium{command_IFS_anchoring_pretty_RIR_LH_unanch_monpol_again_critCUSUM_constant_only_T_400_N_1000_burnin_5_params_psi_pi_1_5_psi_x_0_gbar_0_145_thetbar_16_thettilde_2_5_kap_0_8_lamx_0_lami_0_date_2020_06_05}
\def\fignameIRFpsipiBig{command_IFS_anchoring_pretty_RIR_LH_unanch_monpol_again_critCUSUM_constant_only_T_400_N_1000_burnin_5_params_psi_pi_2_psi_x_0_gbar_0_145_thetbar_16_thettilde_2_5_kap_0_8_lamx_0_lami_0_date_2020_06_05}

%%%%%%%%%%%%%%%%%%%%


\author[]{Laura G\'ati}
% note: in the [] you can put a short name to be displayed in the footer

\institute[]{Boston College}
% note: in the [] you can put a short affiliation to be displayed in the footer

\title[]{Monetary Policy \& Anchored Expectations \\
An Endogenous Gain Learning Model}
% note: in the [] you can put a short title to be displayed in the footer

\date[]{TO DO April 15, 2020}
% note: in the [] you can put a short date to be displayed in the footer

%%%%%%%%%%%%            BEGIN DOCUMENT         %%%%%%%%%%%%%%%%%%%%%

\begin{document}

\begin{frame}[plain] % use the argument plain here to remove the progress bar

\maketitle

\end{frame}


%%%%%%%%%%%%%%%%


%%%%%%% Slide %%%%%%
\begin{frame}\label{motivation}
	
\begin{quote}
Policymakers came out of the Great Inflation era with a clear understanding that it was essential to anchor inflation expectations at some low level. \\
Jerome Powell, Chairman of the Federal Reserve \footnote{Federal Reserve ``Challenges for Monetary Policy,''  August 23, 2019.}
\end{quote}	

\begin{figure}[h!]
\includegraphics[scale = \myTinyFigScale]{\myFigPath \fignameMarketEPi}
\caption{Market-based inflation expectations, 10 year, average, \%}
\label{epi}
\end{figure}


%\vfill

\vspace{-0.5cm}
\hfill \hyperlink{app_TIPS}{\beamergotobutton{TIPS}}
\end{frame}
%%%%%%%%%%%%%%%%%



%%%%%%%% Slide %%%%%%
\begin{frame}
	\frametitle{This project}
	
	\begin{itemize}
	\item Estimation of the anchoring function: when do expectations become unanchored?
	
	\
	
	\

	\item Model of anchoring expectation formation as an endogenous gain adaptive learning scheme
	
	\
	
	\
	
	\item[$\rightarrow$] How to conduct optimal monetary policy in interaction with the anchoring expectation formation?
	%How does a concern to anchor expectations affect the conduct of monetary policy?
	\end{itemize}
	\end{frame}
	%%%%%%%% Slide %%%%%%
\begin{frame}
	\frametitle{Preview of results}
	
	\begin{itemize}
	\item A \colorbox{yellow}{1\%-point} forecast error unanchors expectations

	
	\
	
	\
	
	\item Optimal monetary policy responsiveness time-varying 
	
	\
	
	\
	
	
	\item[$\hookrightarrow$] Unanchored expectations introduce an intertemporal volatility tradeoff 
	
	\
	
	\
	\item[$\hookrightarrow$] Illustrate analytically in special case: target criterion 

	\end{itemize}


\end{frame}
%%%%%%%%%%%%%%%%%%


%%%%%%%% Slide %%%%%%
\begin{frame}
	\frametitle{Related literature}

\begin{itemize}
\item \textbf{Optimal monetary policy in New Keynesian models}
\item[] Clarida, Gali \& Gertler (1999), Woodford (2003)

\

\

\item \textbf{Econometric learning}
\item[] Evans \& Honkapohja (2001, 2006), Bullard \& Mitra (2002), Preston (2005, 2008), Ferrero (2007), Moln\'ar \& Santoro (2014), Eusepi \& Preston (2011), Milani (2007, 2014), Lubik \& Matthes (2018), Mele et al (2019)

\

\

\item \textbf{Anchoring and the Phillips curve} 
\item[] Sargent (1999), Svensson (2015), Hooper et al (2019), Afrouzi \& Yang (2020), Gobbi et al (2019), Carvalho et al (2019)
\end{itemize}



\end{frame}
%%%%%%%%%%%%%%%%%%

%%%%%%%%%%%%%%%%%%%%%%%%%%%%%%%%%%%%%%%%%%%%%%%%%%%%%%
\section{Unanchoring in the data}
%%%%%%%%%%%%%%%%%%%%%%%%%%%%%%%%%%%%%%%%%%%%%%%%%%%%%%

%%%%%%% Slide %%%%%%
\begin{frame}
	
	\colorbox{yellow}{TO DO}

\begin{figure}[h!]
\includegraphics[scale = \mySmallerFigScale]{\myFigPath \fignameAlphaHat}
\caption{Unanchoring as a function of forecast errors in inflation (\%-point)}
\label{epi}
\end{figure}


\end{frame}
%%%%%%%%%%%%%%%%%

%%%%%%%%%%%%%%%%%%%%%%%%%%%%%%%%%%%%%%%%%%%%%%%%%%%%%%
\section{Model of anchoring expectations}
%%%%%%%%%%%%%%%%%%%%%%%%%%%%%%%%%%%%%%%%%%%%%%%%%%%%%%

%%%%%%% Slide %%%%%%
\begin{frame}
	\frametitle{Households: standard up to $\hat{\E}$}
	\label{HH}

Maximize lifetime expected utility
\begin{equation}
\textcolor{brownreddark}{\hat{\E}_t}\sum^{\infty}_{T=t}\beta^{T-t} \bigg[ U(C^i_T) - \int_0^1 v(h^i_T(j)) dj \bigg]
\label{lifetime_U}
\end{equation}	

Budget constraint
\begin{equation}
 B^i_t \leq (1+i_{t-1})B^i_{t-1} + \int_0^1 w_t(j)h^i_t(j) + \Pi_t^i(j)  dj-T_t -P_tC^i_t
 \label{BC}
\end{equation}



\vfill

%\vspace{2cm}
\hfill \hyperlink{details_HHs_firms}{\beamergotobutton{Consumption, price level}}
\end{frame}
%%%%%%%%%%%%%%%%%

%%%%%%% Slide %%%%%%
\begin{frame}
	\frametitle{Firms: standard up to $\hat{\E}$}

Maximize present value of profits
\begin{equation}
\textcolor{brownreddark}{\hat{\E}_t}\sum^{\infty}_{T=t}\alpha^{T-t} Q_{t,T} \bigg[ \Pi^j_t(p_t(j))\bigg]
\label{lifetime_profits}
\end{equation}

subject to demand
\begin{equation}
y_t(j) = Y_t \bigg(\frac{p_t(j)}{P_t}\bigg)^{-\theta}
\end{equation}


\vfill

\hfill \hyperlink{details_HHs_firms}{\beamergotobutton{Profits, stochastic discount factor}}

\end{frame}
%%%%%%%%%%%%%%%%%

%%%%%%%% Slide %%%%%%
\begin{frame}
	\frametitle{Expectations: $\hat{\E}$ instead of $\E$}

\begin{itemize}
\item If use $\E$ (rational expectations, RE) \\

\

Model solution 
 \begin{align}
 s_t & = h s_{t-1} + \epsilon_t \quad \quad \quad \epsilon_t \sim \mathcal{N}(\mathbf{0},\Sigma) \label{state} \\
 y_t & = g s_t \label{obs_RE}
 \end{align}


$s_t \equiv $ states \\
$y_t \equiv $ jumps \\
$\epsilon_t \equiv $ disturbances
\

\pause

\

\item If use $\hat{\E} \rightarrow$ private sector does not know (\ref{obs_RE}) \\

\

$\hookrightarrow$ estimate using observed states \& knowledge of (\ref{state})

\end{itemize}



\end{frame}
%%%%%%%%%%%%%%%%%%

%%%%%%%% Slide %%%%%%
\begin{frame}
	\frametitle{Adaptive learning}
	\label{adaptive_learning}

\begin{itemize}
\item Postulate linear functional relationship instead of (\ref{obs_RE}):

\begin{equation}
\hat{\E}_t y_{t+1} = \textcolor{brownreddark}{a_{t-1}} + b_{t-1}s_t  \label{PLM_fcst_general}
\end{equation}

\

\pause

In RE, $\textcolor{brownreddark}{a_{t-1} = \mathbf{0}}, \quad b_{t-1} = g\;h \quad \forall t$

\

\pause
  


\item Note: \textcolor{brownreddark}{misspecified}	$\rightarrow$ not model-consistent (not RE) 

\

\pause


\item Estimate $a, b$ using recursive least squares (RLS)


\end{itemize}


\end{frame}
%%%%%%%%%%%%%%%%%%

%%%%%%% Slide %%%%%%  	
\begin{frame}
	\frametitle{Recursive least squares}
	\label{RLS_special}
	
	\
	
Jumps are: $(\pi, x, i)'$ 	

\
\pause
\

Special case: learn only intercept of inflation:

\begin{equation}
a_{t-1} = ( \textcolor{brownreddark}{\bar{\pi}_{t-1}},0,0)', \quad b_{t-1} = g\; h \quad \forall t
\end{equation}

\
\pause
\

$\rightarrow$ RLS
\begin{equation}
\bar{\pi}_{t}  =\bar{\pi}_{t-1} +k_t \underbrace{\big(\pi_{t} -(\bar{\pi}_{t-1}+b_1 s_{t-1}) \big)}_{\equiv \; fe_{t|t-1} \text{, forecast error} } 
\end{equation}
 
 \
 
 $k_t \in (0,1)$ gain \\
 $b_1$ first row of $b$
\vfill 

\hfill \hyperlink{RLS}{\beamergotobutton{General RLS algorithm}}




\end{frame}
%%%%%%%%%%%%%%%%%

%%%%%%% Slide %%%%%%  	
\begin{frame}
	\frametitle{Anchoring mechanism: endogenous gain}
	\label{anchoring1}


\begin{equation}
\bar{\pi}_{t}  =\bar{\pi}_{t-1} + \textcolor{brownreddark}{k_t}\big(\pi_{t} -(\bar{\pi}_{t-1}+b_1 s_{t-1}) \big)
\end{equation}

\

\

$k_t = \textcolor{brownreddark}{\mathbf{g}(fe_{t|t-1})}$: anchoring function
\pause
\begin{equation}
 \textcolor{brownreddark}{\mathbf{g}(fe_{t|t-1})} = \alpha b(fe_{t|t-1})\label{gain}
\end{equation}

\

$ b(fe_{t|t-1}) = $ basis, here: second order spline (piecewise linear) \\

\

$\alpha  = $ approximating coefficients, here: use $\hat{\alpha}$ from estimation

\vfill 

\hfill \hyperlink{g}{\beamergotobutton{Functional forms in literature}}



\end{frame}
%%%%%%%%%%%%%%%%%

%%%%%%%% Slide %%%%%%
\begin{frame}
	\frametitle{Model summary}
	\label{aggregate_LOMS}
\begin{itemize}
\item IS- and Phillips curve:	
 \begin{align}
x_t &=  -\sigma i_t +\hat{\E}_t \sum_{T=t}^{\infty} \beta^{T-t }\big( (1-\beta)x_{T+1} - \sigma(\beta i_{T+1} - \pi_{T+1}) +\sigma r_T^n \big)  \label{NKIS}  \\
\pi_t &= \kappa x_t +\hat{\E}_t \sum_{T=t}^{\infty} (\alpha\beta)^{T-t }\big( \kappa \alpha \beta x_{T+1} + (1-\alpha)\beta \pi_{T+1} + u_T\big) \label{NKPC} 
\end{align}
\hfill \hyperlink{derivations}{\beamergotobutton{Derivations}} \hyperlink{ALMs}{\beamergotobutton{Actual laws of motion}}

\

\item  Expectations evolve according to RLS with the endogenous gain given by (\ref{gain})

\

\item[$\rightarrow$] How should $\{ i_t \}$ be set?
\end{itemize}



\end{frame}
%%%%%%%%%%%%%%%%%%


%%%%%%%%%%%%%%%%%%%%%%%%%%%%%%%%%%%%%%%%%%%%%%%%%%%%%%
\section{Solving the Ramsey problem}
%%%%%%%%%%%%%%%%%%%%%%%%%%%%%%%%%%%%%%%%%%%%%%%%%%%%%%

%%%%%%%% Slide %%%%%%
\begin{frame}
	\frametitle{Ramsey problem}
	 \begin{align*}
& \min_{ \{y_t, \bar{\pi}_{t-1}, k_t \}_{t=t_0}^{\infty}} \E_{t_0}\sum_{t=t_0}^{\infty} \beta^{t-t_0} (\pi_t^2  + \lambda_x x_t^2 )  \\
& \text{s.t. model equations} \\
& \text{s.t. evolution of expectations} 
\end{align*}

\

\

\begin{itemize}
\item $\E$ is the central bank's (CB) expectation

\

\item Assumption: CB observes private expectations and knows the model
\end{itemize}

 

\end{frame}
%%%%%%%%%%%%%%%%%%


%%%%%%%% Slide %%%%%%
\begin{frame}
	\frametitle{Target criterion}
	\label{anchTC}
	
	\begin{result} 

\

In the model with anchoring, monetary policy optimally brings about the following target relationship between inflation and the output gap
	
\begin{align*}
\pi_t  = -\frac{\lambda_x}{\kappa}x_t + \frac{\lambda_x}{\kappa}\frac{(1-\alpha)\beta}{1-\alpha\beta} \bigg(k_t+((\pi_t - \bar{\pi}_{t-1}-b_1 s_{t-1}))\mathbf{g}_{\pi,t}\bigg) \\
\\
\bigg(\E_t\sum_{i=1}^{\infty}x_{t+i}\prod_{j=0}^{i-1}(1-k_{t+1+j} - (\pi_{t+1+j} - \bar{\pi}_{t+j}-b_1 s_{t+j})\mathbf{g_{\bar{\pi}, t+j}}) \bigg)
 \label{target}
\end{align*}

\

\

where $\; \mathbf{g}_{z,t} \equiv \frac{\partial \mathbf{g}}{\partial z}\;$ at $t$, $\; \prod_{j=0}^{0} \equiv 1 \; $ and $b_1$ is the first row of $b$.
	\end{result}

\vspace{-0.1cm}

\hfill \hyperlink{generalTC}{\beamergotobutton{General case}}

\end{frame}
%%%%%%%%%%%%%%%%%%

%%%%%%%% Slide %%%%%%
\begin{frame}
	\frametitle{Two layers of intertemporal tradeoffs}
\small{
\begin{align*}
& \pi_t  =  \; \textcolor{brownreddark}{-\frac{\lambda_x}{\kappa} x_t} \textcolor{dodgerblue}{ \; + \frac{\lambda_x}{\kappa} \frac{(1-\alpha)\beta}{1-\alpha\beta} \bigg(k_t+ fe_{t|t-1}\mathbf{g}_{\pi,t} \bigg)\E_t\sum_{i=1}^{\infty}x_{t+i}}  \\
& \textcolor{aquamarinegreen}{- \frac{\lambda_x}{\kappa} \frac{(1-\alpha)\beta}{1-\alpha\beta} \bigg(k_t+ fe_{t|t-1}\mathbf{g}_{\pi,t} \bigg)\E_t\sum_{i=1}^{\infty}x_{t+i}\prod_{j=0}^{i-1}(k_{t+1+j}+ fe_{t+1+j|t+j}\mathbf{g_{\bar{\pi},t+j}} )}
\end{align*}

\

\

\textcolor{brownreddark}{Intratemporal tradeoffs in RE (discretion)} \\

\

\

\textcolor{dodgerblue}{Intertemporal tradeoff: current level and change of the gain } \\

\

\

\textcolor{aquamarinegreen}{Intertemporal tradeoff: future expected levels and changes of the gain}

}
\end{frame}
%%%%%%%%%%%%%%%%%%

%%%%%%%% Slide %%%%%%
\begin{frame}
	\frametitle{}
	\label{no_commitment}

\begin{lemma} The discretion and commitment solutions of the Ramsey problem coincide. 
\end{lemma}

\

\hfill \hyperlink{no_commitment_intuition}{\beamergotobutton{Why no commitment?}}

\

\

\begin{corollary} Optimal policy under adaptive learning is time-consistent. 
\end{corollary}


\

\

$\hookrightarrow $ Foreshadow: optimal policy aggressiveness time-varying

\vfill




\end{frame}
%%%%%%%%%%%%%%%%%%

%%%%%%%%%%%%%%%%%%%%%%%%%%%%%%%%%%%%%%%%%%%%%%%%%%%%%%
\section{Implementing optimal policy}
%%%%%%%%%%%%%%%%%%%%%%%%%%%%%%%%%%%%%%%%%%%%%%%%%%%%%%


%%%%%%%% Slide %%%%%%
\begin{frame}
	\frametitle{Solution procedure}
	
\begin{itemize}
\item[] Solve system of model equations + target criterion 

\

\item[] $\hookrightarrow$ solve using parameterized expectations (PEA) and value function iteration (VFI)

\


\item[] $\hookrightarrow$ obtain a cubic spline approximation to optimal policy function

\

\end{itemize}


\end{frame}
%%%%%%%%%%%%%%%%%%

%%%%%%%% Slide %%%%%%
\begin{frame}
	\frametitle{Optimal policy I - responding to unanchoring }
\colorbox{yellow}{TO DO} 
	
\begin{figure}[h!]
\includegraphics[scale=\mySmallerFigScale]{\myFigPath \fignameDiDpibar}
\caption{Comparative statics of the policy function: $\partial i / \partial \bar{\pi}$ if all other states are kept at their mean}
\end{figure} % Materials 35

\end{frame}
%%%%%%%%%%%%%%%%%%

%%%%%%%% Slide %%%%%%
\begin{frame}
	\frametitle{Optimal policy II - a particular history }
\colorbox{yellow}{TO DO} 
	
\begin{figure}[h!]
%\subfigure[Policy function: PEA against VFI]{\includegraphics[scale=\myTinyFigScale]{\myFigPath compare_value_pea_results_approx_value_outputs_approx17_Jul_2020_14_27_00_pea_outputs_approx17_Jul_2020_15_28_03_pretty_17_Jul_2020_15_34_46}}
%\hfill
\subfigure[Observables]{\includegraphics[scale=0.09]{\myFigPath implement_anchTC_obs_approx17_Jul_2020_15_34_46}}
\subfigure[Gain]{\includegraphics[scale=0.09]{\myFigPath implement_anchTC_invgain17_Jul_2020_15_34_46}}
\hfill
\subfigure[$\bar{\pi}$]{\includegraphics[scale=0.09]{\myFigPath implement_anchTC_pibar17_Jul_2020_15_34_46}}
\caption{Optimal policy solution conditional on a particular evolution of shocks} % from Materials 37
\end{figure}

\end{frame}
%%%%%%%%%%%%%%%%%%

%%%%%%%% Slide %%%%%%
\begin{frame}
	\frametitle{Optimal policy III - optimal Taylor-rule coefficients}
\colorbox{yellow}{TO DO} 
	
\begin{figure}[h!]
\subfigure[$\lambda_x = 0, \lambda_i = 0$]{\includegraphics[scale=0.09]{\myFigPath plot_sim_loss_approx_pretty_losses_again_critsmooth_constant_only_pi_only_params_psi_pi_1_5_psi_x_0_gbar_0_145_lamx_0_lami_0_2020_07_01}}
\hfill
\subfigure[$\lambda_x = 1, \lambda_i = 0$]{\includegraphics[scale=0.09]{\myFigPath plot_sim_loss_approx_pretty_losses_again_critsmooth_constant_only_pi_only_params_psi_pi_1_5_psi_x_0_gbar_0_145_lamx_1_lami_0_2020_07_01}}
\subfigure[$\lambda_x = 0, \lambda_i = 1$]{\includegraphics[scale=0.09]{\myFigPath plot_sim_loss_approx_pretty_losses_again_critsmooth_constant_only_pi_only_params_psi_pi_1_5_psi_x_0_gbar_0_145_lamx_0_lami_1_2020_07_01}}
\hfill
\subfigure[$\lambda_x = 1, \lambda_i = 1$]{\includegraphics[scale=0.09]{\myFigPath plot_sim_loss_approx_pretty_losses_again_critsmooth_constant_only_pi_only_params_psi_pi_1_5_psi_x_0_gbar_0_145_lamx_1_lami_1_2020_07_01}}
\caption{CB loss as a function of $\psi_{\pi}$ for various weights on $x$ and $i$ in the central bank's loss}
\end{figure} % Materials 35

\end{frame}
%%%%%%%%%%%%%%%%%%

%%%%%%% Slide %%%%%%
\begin{frame}
	\frametitle{Conclusion}
	
\begin{itemize}
\item Interaction between monetary policy and anchoring 

\

\

\item Optimal policy conditions on stance of current and expected future anchoring
\item[] \hspace{4cm} $\hookrightarrow$ determine intertemporal tradeoffs

\

\

\item Frontloads aggressive interest rate response to suppress potential unanchoring

\

\

\item For a 1\%-point positive (negative) forecast error, raises (lowers) interest rate by \colorbox{yellow}{\%-point}
\end{itemize}


\end{frame}
%%%%%%%%%%%%%%%%%


%%%%%%%%%%%%%%%%%
%         TEMPLATES
%%%%%%%%%%%%%%%%%

%%%%%%%% Slide %%%%%%
%\begin{frame}
%	\frametitle{Slide template}
%
%\end{frame}
%%%%%%%%%%%%%%%%%%

%%%%%%%% Slide %%%%%%  	
%\begin{frame}
%	\frametitle{A beamer button template}
%	\label{identification}
%	
%\hyperlink{Technicalities}{\beamergotobutton{Technicalities}}
%
%\end{frame}
%%%%%%%%%%%%%%%%%%



%%%%%%%%%%%%%%%%%%%%%%%%%%%%%%%%%%%%%%%%%%%%%%%%%%%%%%%%%%%%%%%%%%%%%%%
%%%%%%%                     APPENDIX  
%%%%%%%%%%%%%%%%%%%%%%%%%%%%%%%%%%%%%%%%%%%%%%%%%%%%%%%%%%%%%%%%%%%%%%%
\appendix
\backupbegin

%%%%%%%%%%%%%%%%%%


%%%%%%%% Slide %%%%%%
\begin{frame}[plain]  % remove progress bar from appendix

\begin{centering}
\hfill \textbf{Appendix} \hfill
\end{centering}


\end{frame}
%%%%%%%%%%%%%%%%%%

%%%%%%%% Slide %%%%%%
\begin{frame}[plain]  % remove progress bar from appendix
\frametitle{Correcting the TIPS from liquidity risk}
	\label{app_TIPS}

\begin{figure}[h!]
\includegraphics[scale = 0.25]{\myFigPath \fignameMarketEPiCleaned} % \fignameMarketEPiCleaned
\caption{Market-based inflation expectations, 10 year, average, \%}
%\floatfoot{Breakeven inflation, constructed as the difference between the yields of 10-year Treasuries and 10-year TIPS (blue line), difference between 10-year Treasury and 10-year TIPS, the latter cleaned from liquidity risk (red line).}
\label{epi_cleaned}
\end{figure}

\vfill 
\hyperlink{motivation}{\beamerreturnbutton{Return}}	

\end{frame}
%%%%%%%%%%%%%%%%%%

%%%%%%%% Slide %%%%%%
\begin{frame}[plain]  % remove progress bar from appendix
\frametitle{Oscillatory dynamics in adaptive learning}
	\label{oscillatory}

Consider a stylized adaptive learning model in two equations:
\begin{align}
\pi_t & = \beta f_t + u_t  \\
f_t & = f_{t-1} + k(\pi_t - f_{t-1}) 
\end{align}

Solve for the time series of expectations $f_t$
\begin{equation}
f_t = \underbrace{\frac{1-k^{-1}}{1-k^{-1}\beta}}_{\approx 1}f_{t-1} + \frac{k^{-1}}{1-k^{-1}\beta}u_t
\end{equation}

Solve for forecast error $fe_t \equiv \pi_t - f_{t-1}$:
\begin{equation}
fe_t = \underbrace{-\frac{1-\beta}{1-k\beta}}_{\lim_{k \to 1} = -1}f_{t-1} + \frac{1}{1-k\beta}u_t 
\end{equation}

\end{frame}
%%%%%%%%%%%%%%%%%%


%%%%%%%% Slide %%%%%%
\begin{frame}[plain]  % remove progress bar from appendix
	\frametitle{Functional forms for $\mathbf{g}$ in the literature}
	\label{g}
\begin{itemize}
\item Smooth anchoring function (Gobbi et al, 2019)
\begin{equation}
 p = h(y_{t-1}) = A + \frac{B C e^{-D y_{t-1}}}{( C e^{-D y_{t-1}}+1)^2}
\end{equation}
$p \equiv Prob(\text{liquidity trap regime}) $ \\
$y_{t-1}$ output gap \\


\

\item Kinked anchoring function (Carvalho et al, 2019)
 \begin{align*}
k_t & = \begin{cases} \frac{1}{t} \quad \text{when} \quad \theta_t < \bar{\theta}  \\ k \quad \text{otherwise.}\numberthis
\end{cases} 
\end{align*}
$\theta_t$ criterion, $\bar{\theta}$ threshold value

\end{itemize}

\vfill 
\hyperlink{anchoring1}{\beamerreturnbutton{Return}}	

\end{frame}
%%%%%%%%%%%%%%%%%%


%%%%%%%%%%%%%%%%%%%%% Slide
\begin{frame}[plain]  % remove progress bar from appendix
	\frametitle{Choices for criterion $\theta_t$}
	\label{g}
\begin{itemize}
\item Carvalho et al. (2019)'s criterion  
\begin{equation}
\theta_t^{CEMP} = \max | \Sigma^{-1} ( \phi_{t-1} - T(\phi_{t-1})) |
\end{equation}


$\Sigma$ variance-covariance matrix of shocks \\
$T(\phi)$ mapping from PLM to ALM

\

\

\item CUSUM-criterion
\begin{align}
\omega_t & =  \omega_{t-1} + \kappa k_{t-1}(fe_{t|t-1} fe_{t|t-1}'  -\omega_{t-1})\\
\theta_t^{CUSUM} & =  \theta_{t-1} + \kappa k_{t-1}(fe_{t|t-1}'\omega_t^{-1}fe_{t|t-1} -\theta_{t-1})
\end{align}

\

$\omega_t$ estimated forecast-error variance
\end{itemize}




\vfill

\hyperlink{anchoring1}{\beamerreturnbutton{Return}}	


\end{frame}
%%%%%%%%%%%%%%%%%%

%%%%%%%% Slide %%%%%%
\begin{frame}[plain]  % remove progress bar from appendix
	\frametitle{Recursive least squares algorithm}
	\label{RLS}


\begin{align}
\phi_t  & = \bigg( \phi_{t-1}' + k_t R_t^{-1}\begin{bmatrix} 1 \\ s_{t-1} \end{bmatrix}\bigg(y_{t} - \phi_{t-1} \begin{bmatrix} 1 \\ s_{t-1} \end{bmatrix} \bigg)' \bigg)' \\
R_t &= R_{t-1} +  k_t \bigg( \begin{bmatrix} 1 \\ s_{t-1} \end{bmatrix} \begin{bmatrix} 1 & s_{t-1} \end{bmatrix}  - R_{t-1} \bigg)
\end{align}


\vfill

\hyperlink{RLS_special}{\beamerreturnbutton{Return}}	


\end{frame}
%%%%%%%%%%%%%%%%%%


%%%%%%%% Slide %%%%%%
\begin{frame}[plain]  % remove progress bar from appendix
	\frametitle{Actual laws of motion}
	\label{ALMs}

 \begin{align*}
y_t & = A_1 f_{a,t} + A_2 f_{b,t} + A_3 s_t \label{LOM_LR} \numberthis \\
\\
s_t & = h s_{t-1} + \epsilon_t \label{exog} \numberthis
\end{align*}
where

\begin{equation}
 y_t \equiv \begin{pmatrix} \pi_t \\ x_t \\ i_t
 \end{pmatrix} 
 \quad \quad \quad 
  s_t  \equiv \begin{pmatrix} r_t^n \\ u_t 
 \end{pmatrix} 
\end{equation}
and

  \begin{align}
f_{a,t}  \equiv  \hat{\E}_t\sum_{T=t}^{\infty} (\alpha\beta)^{T-t } y_{T+1} \quad \quad \quad \quad 
f_{b,t}  \equiv \hat{\E}_t\sum_{T=t}^{\infty} (\beta)^{T-t } y_{T+1} \label{fafb}
\end{align}

\hyperlink{aggregate_LOMS}{\beamerreturnbutton{Return}}	


\end{frame}
%%%%%%%%%%%%%%%%%%



%%%%%%%% Slide %%%%%%
\begin{frame}[plain]  % remove progress bar from appendix
	\frametitle{No commitment - no lagged multipliers}
	\label{no_commitment_intuition}
	
	Simplified version of the model: planner chooses $\{\pi_t, x_t, f_t, k_t\}_{t=t_0}^{\infty}$ to minimize
 \begin{align*}
\mathcal{L} &= \E_{t_0}\sum_{t=t_0}^{\infty} \beta^{t-t_0}\bigg\{ \pi_t^2  + \lambda x_t^2 + \varphi_{1,t} (\pi_t -\kappa x_t- \beta f_t +u_t) \\ &+ \varphi_{2,t}(f_t - f_{t-1} -k_t(\pi_t - f_{t-1})) + \varphi_{3,t}(k_t- \mathbf{g}(\pi_t - f_{t-1})) \bigg\}
 \end{align*}

 \begin{align}
  2\pi_t +2\frac{\lambda}{\kappa}x_t -\textcolor{brownreddark}{\varphi_{2,t}}(k_t + \mathbf{g_{\pi}}(\pi_t -f_{t-1}))& = 0 \label{simpleFOC1} \\
  -2\beta\frac{\lambda}{\kappa}x_t + \textcolor{brownreddark}{\varphi_{2,t}} -\textcolor{brownreddark}{\varphi_{2,t+1}}(1-k_{t+1} -\mathbf{g_{f}}(\pi_{t+1} -f_{t})) & = 0 \label{simpleFOC2} 
 \end{align}

  
\hyperlink{no_commitment}{\beamerreturnbutton{Return}}	


\end{frame}
%%%%%%%%%%%%%%%%%%



%%%%%%%% Slide %%%%%%
\begin{frame}[plain]  % remove progress bar from appendix
	\frametitle{Target criterion system for anchoring function as changes of the gain}
	\label{generalTC}

\begin{align*}
\varphi_{6,t} & = -c fe_{t|t-1} x_{t+1} + \bigg(1+ \frac{fe_{t|t-1}}{fe_{t+1|t}}(1-k_{t+1}) -fe_{t|t-1} \mathbf{g}_{\bar{\pi},t} \bigg) \varphi_{6,t+1} \\
& -\frac{fe_{t|t-1}}{fe_{t+1|t}}(1-k_{t+1})\varphi_{6,t+2} \numberthis \label{6'} \\
0 & = 2\pi_t + 2\frac{\lambda_x}{\kappa}x_t   - \bigg( \frac{k_t}{fe_{t|t-1}} + \mathbf{g}_{\pi,t}\bigg)\varphi_{6,t} + \frac{k_t}{fe_{t|t-1}}\varphi_{6,t+1}\numberthis \label{1'}
\end{align*}
$\varphi_{6,t}$ Lagrange multiplier on anchoring function

\

The solution to (\ref{1'}) is given by:
\begin{equation}
\varphi_{6,t} = -2\E_t\sum_{i=0}^{\infty}(\pi_{t+i}+\frac{\lambda_x}{\kappa}x_{t+i})\prod_{j=0}^{i-1}\frac{\frac{k_{t+j}}{fe_{t+j|t+j-1}}}{\frac{k_{t+j}}{fe_{t+j|t+j-1}} + \mathbf{g}_{\pi, t+j}} \label{sol1'}
\end{equation}


\vspace{-0.5cm}
 
\hyperlink{anchTC}{\beamerreturnbutton{Return}}	


\end{frame}
%%%%%%%%%%%%%%%%%%


%%%%%%%% Slide %%%%%%
\begin{frame}[plain]  % remove progress bar from appendix
	\frametitle{Details on households and firms}
	\label{details_HHs_firms}

\

Consumption:	
\begin{equation}
C^i_t =  \bigg[  \int_0^1 c^i_t(j)^{\frac{\theta-1}{\theta}} dj \bigg]^{\frac{\theta}{\theta-1}}\label{dixit}
\end{equation}
$\theta>1$: elasticity of substitution between varieties

\

Aggregate price level:
\begin{equation}
P_t =  \bigg[  \int_0^1 p_t(j)^{1-\theta} dj \bigg]^{\frac{1}{\theta-1}}
\label{agg_price}
\end{equation}

Profits:
\begin{equation}
\Pi_t^j = p_t(j)y_t(j) -w_t(j)f^{-1}(y_t(j)/A_t)
\end{equation}

Stochastic discount factor
\begin{equation}
Q_{t,T} = \beta^{T-t} \frac{P_t U_c(C_T)}{P_T U_c(C_t)}
\end{equation}

\vspace{-0.5cm}

\hyperlink{HH}{\beamerreturnbutton{Return}}	


\end{frame}
%%%%%%%%%%%%%%%%%%
%%%%%%%% Slide %%%%%%
\begin{frame}[plain]  % remove progress bar from appendix
	\frametitle{Derivations}
	\label{derivations}

Household FOCs
 \begin{align*}
\hat{C}_{t}^i & = \hat{\E}^i_t \hat{C}_{t+1}^i - \sigma(\hat{i}_{t} -\hat{\E}^i_t \hat{\pi}_{t+1})    \label{EE} \numberthis \\
\\
\hat{\E}^i_t \sum_{s=0}^{\infty}\beta^s \hat{C}^i_t& =\omega^i_t + \hat{\E}^i_t \sum_{s=0}^{\infty}\beta^s \hat{Y}^i_t \label{IBC} \numberthis
\end{align*}
where `hats' denote log-linear approximation and $\omega_t^i \equiv \frac{(1+i_{t-1})B^i_{t-1} }{P_t Y^*}$.

\begin{enumerate}
\item Solve (\ref{EE}) backward to some date $t$, take expectations at $t$ 
\item Sub in (\ref{IBC})
\item Aggregate over households $i$
\item[$\rightarrow$] Obtain (\ref{NKIS})
\end{enumerate}



\hyperlink{aggregate_LOMS}{\beamerreturnbutton{Return}}	


\end{frame}
%%%%%%%%%%%%%%%%%%
%
%%%%%%%% Slide %%%%%%
%\begin{frame}
%\frametitle{A beamer button template, how to get back to main text}
%\label{Steps}
%
%\begin{equation}
%D = \begin{bmatrix}
%d_{11} & \gamma_{12} & \gamma_{13} & d_{14} & \cdots \\
%d_{21} & \gamma_{22} & \gamma_{23} & d_{24} & \cdots \\
%\vdots & \vdots & \vdots & \ddots & \vdots 
%\end{bmatrix}
%\end{equation}
%
%\hyperlink{calcCBloss}{\beamerreturnbutton{Return}}	
%\end{frame}
%%%%%%%%%%%%%%%%%%

\backupend


\end{document}
