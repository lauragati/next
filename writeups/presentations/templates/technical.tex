\documentclass[10pt]{beamer}
\usepackage[english]{babel}
\usepackage[latin1]{inputenc}
\usepackage{tikz}
\usetikzlibrary{calc}
\usepackage{subfigure}
\usepackage{amssymb}
\usepackage{amsmath}
\usepackage{booktabs}
\usepackage{verbatim}
\usepackage{caption}
\usepackage{float}
\usepackage{csquotes}
\usepackage{sansmathaccent}
\usepackage{subfigure}
\usepackage{multicol}
\pdfmapfile{+sansmathaccent.map}
\usepackage{pgfplots,tikz}
\usetikzlibrary{tikzmark,calc}
\usepackage{overpic}
\usepackage{color,soul}
\usepackage{stackengine}

% Layout and style
\usefonttheme{serif}
\usepackage{palatino} 

\usetheme{Madrid}
\usecolortheme{dove}

\def \myFigPath {../../../figures/} 
\def \myTablePath {../../../tables/} 

%%%% PROGRESS BAR %%%%%%%%%%%%%%%%%%%%%%%%%%%%%%%%%%%%%%%%%%
\definecolor{pbblue}{HTML}{0A75A8}% filling color for the progress bar
\definecolor{pbgray}{HTML}{575757}% background color for the progress bar

% a set of nice blue, nice red and nice green
\definecolor{dodgerblue}{RGB}{16,78,139}
\definecolor{aquamarinegreen}{RGB}{69,139,116}
\definecolor{brownredlight}{RGB}{238,59,59}
\definecolor{brownreddark}{RGB}{205,51,51}



\makeatletter
\def\progressbar@progressbar{} % the progress bar
\newcount\progressbar@tmpcounta% auxiliary counter
\newcount\progressbar@tmpcountb% auxiliary counter
\newdimen\progressbar@pbht %progressbar height
\newdimen\progressbar@pbwd %progressbar width
\newdimen\progressbar@tmpdim % auxiliary dimension

\progressbar@pbwd=1.15\linewidth % change width of progress bar
\progressbar@pbht=0.7ex % change thickness of progress bar

% the progress bar
\def\progressbar@progressbar{%

    \progressbar@tmpcounta=\insertframenumber
    \progressbar@tmpcountb=\inserttotalframenumber
    \progressbar@tmpdim=\progressbar@pbwd
    \multiply\progressbar@tmpdim by \progressbar@tmpcounta
    \divide\progressbar@tmpdim by \progressbar@tmpcountb

  \begin{tikzpicture}[rounded corners=2pt,very thin]

    \shade[top color=pbgray!10,bottom color=pbgray!10,middle color=pbgray!10]
      (0pt, 0pt) rectangle ++ (\progressbar@pbwd, \progressbar@pbht);

      \shade[draw=dodgerblue,top color=dodgerblue,bottom color=dodgerblue,middle color=dodgerblue] %
        (0pt, 0pt) rectangle ++ (\progressbar@tmpdim, \progressbar@pbht);

% ADD COMPLETION PERCENTAGE
%    \draw[color=normal text.fg!50]  
%      (0pt, 0pt) rectangle (\progressbar@pbwd, \progressbar@pbht) 
%        node[pos=0.5,color=normal text.fg] {\textnormal{%
%             \pgfmathparse{\insertframenumber*100/\inserttotalframenumber}%
%             \pgfmathprintnumber[fixed,precision=2]{\pgfmathresult}\,\%%
%        }%
%    };
  \end{tikzpicture}%
}

\addtobeamertemplate{headline}{}
{%
  \begin{beamercolorbox}[wd=\paperwidth,ht=4ex,center,dp=1ex]{white}%
    \progressbar@progressbar%
  \end{beamercolorbox}%
}
\makeatother

%%%% PROGRESS BAR %%%%%%%%%%%%%%%%%%%%%%%%%%%%%%%%%%%%%%%%%%



% footnote suppressing number
\newcommand\blfootnote[1]{%
  \begingroup
  \renewcommand\thefootnote{}\footnote{#1}%
  \addtocounter{footnote}{-1}%
  \endgroup
}

% try to force appendix to be numbered differently
\newcommand{\backupbegin}{
   \newcounter{framenumberappendix}
   \setcounter{framenumberappendix}{\value{framenumber}}
}
\newcommand{\backupend}{
   \addtocounter{framenumberappendix}{-\value{framenumber}}
   \addtocounter{framenumber}{\value{framenumberappendix}} 
}


% Set colors of buttons
\setbeamercolor{button}{bg=dodgerblue,fg=white}

% this removes the blue navigation bar that used to be on the bottom right
\beamertemplatenavigationsymbolsempty 
\setbeamersize{text margin left=5mm,text margin right=5mm} 


\AtBeginSection[]
{
  \begin{frame}<beamer>
    \frametitle{Structure of talk}
    \tableofcontents[currentsection]
  \end{frame}
}
\useinnertheme{circles}

% resets the bullet points of enumerate and itemize to defaults
\setbeamertemplate{itemize items}[bullet]
\setbeamertemplate{enumerate items}[default]

% this removes the ugly bullets for the table of contents and just adds numbers
\setbeamertemplate{section in toc}{\inserttocsectionnumber.~\inserttocsection}

\newcommand\numberthis{\addtocounter{equation}{1}\tag{\theequation}} % this defines a command to make align only number this line

% Colored lines for legends
% green solid
\newcommand{\greenline}{\raisebox{2pt}{\tikz{\draw[-,black!40!green,solid,line width = 1.5pt](0,0) -- (5mm,0);}}}
% red solid, dashed, dotted
\newcommand{\redline}{\raisebox{2pt}{\tikz{\draw[-,red,solid,line width = 1.5pt](0,0) -- (5mm,0);}}}
\newcommand{\reddashedline}{\raisebox{2pt}{\tikz{\draw[-,red,dashed,line width = 1.5pt](0,0) -- (5mm,0);}}}
\newcommand{\reddottedline}{\raisebox{2pt}{\tikz{\draw[-,red,densely dotted,line width = 1.5pt](0,0) -- (5mm,0);}}}

% blue solid and dashed
\newcommand{\blueline}{\raisebox{2pt}{\tikz{\draw[-,blue,solid,line width = 1.5pt](0,0) -- (5mm,0);}}}
\newcommand{\bluedashedline}{\raisebox{2pt}{\tikz{\draw[-,blue,dashed,line width = 1.5pt](0,0) -- (5mm,0);}}}
% black solid and dashed
\newcommand{\blackline}{\raisebox{2pt}{\tikz{\draw[-,black,solid,line width = 1.5pt](0,0) -- (5mm,0);}}}
\newcommand{\blackdashedline}{\raisebox{2pt}{\tikz{\draw[-,black,dashed,line width = 1.5pt](0,0) -- (5mm,0);}}}


% New Commands
\newcommand{\real}{\hbox{\reali R}}
\newcommand{\realp}{\hbox{\reali R}_{\scriptscriptstyle +}}
\newcommand{\realpp}{\hbox{\reali R}_{\scriptscriptstyle ++}}
\newcommand{\R}{\mathbb{R}}
\DeclareMathOperator{\E}{\mathbb{E}}
\DeclareMathOperator{\argmin}{arg\,min}
\newcommand\w{3.0in}
\newcommand\wnum{3.0}
\def\myFigWidth{5.3in}
\def\mySmallerFigWidth{2.1in}
\def\myEvenBiggerFigScale{0.8}
\def\myPointSixFigScale{0.6}
\def\myBiggerFigScale{0.4}
\def\myFigScale{0.3}
\def\mySmallFigScale{0.25}
\def\mySmallerFigScale{0.18}
\def\myTinyFigScale{0.16}
\def\myPointFourteenFigScale{0.14}
\def\myTinierFigScale{0.12}
\def\myEvenTinierFigScale{0.10}
\def\myCrazyTinyFigScale{0.09}

\newtheorem{prop}{Proposition}
\newtheorem*{remark}{Remark}
\newtheorem{result}{Result}
%\newtheorem{lemma}{Lemma}
%\newtheorem{corollary}{Corollary}


%%%%%%%%%%%%%%%%%%%%
% DEFINE ALL VARIABLE NUMBERS HERE
%%%%%%%%%%%%%%%%%%%%

% numbers update: 17 Sept 2020 using ``complete'' or ``21 Sept" Estimation (N=1000) from Materials 44

% Empirical numbers
% mean estimated alpha
\def\meanalph{(0.82;    0.61;    0;    0.33;    0.45)}
\def\calibCalph{(0.8; 0.4; 0; 0.4; 0.8)}
% for 1% fe, you get a gain of
\def\oneppFEgives{0.08}
% pp forecast error that unanchors (that causes a gain of 0.05)
\def\ppFEunanchors{0.5 }
\def\bpNegFEunanchors{25}
\def\bpPosFEunanchors{50}


% raise/lower interest rate by x bp when when LR-exp move by y bp
\def\movepibar{5 }
\def\movei{12 }

%%%%%%%%%%%%%%%%%%%%
% DEFINE ALL FIGNAMES HERE
%%%%%%%%%%%%%%%%%%%%

% Market-based expectations (aren't used in prezi right now)
\def\fignameMarketEPi{epi10_2020_06_04}
% cleaned from liq premium
\def\fignameMarketEPiCleaned{cleaned_epi10_2020_07_28}
\def\fignameMarketEPiMoreHorizons{epi_be_in_data_command_anchoring_in_data_14_Sep_2020_18_12_59}


% SPF long-run-e
\def\fignameSPFLiv{epi_in_data_command_anchoring_in_data_14_Sep_2020_18_12_59}
\def\fignameRolling{rolling_overlapping_command_anchoring_in_data_individual_23_Sep_2020_21_04_48}
\def\fignameRollingPi{rolling_overlapping_pi_command_anchoring_in_data_individual_25_Sep_2020_10_37_43} % regressing on pi


% Unanchoring in the data
%\def\fignameEpiData{epi_in_data_command_anchoring_in_data_29_Aug_2020_12_10_41}
%\def\fignameFeData{fe_in_data_command_anchoring_in_data_29_Aug_2020_12_10_41}
\def\fignameRegressionPlot{regression_plot_command_anchoring_in_data_14_Sep_2020_18_12_59}
\def\fignameSCEdistrib{SCE_distrib_topbottom_command_anchoring_in_data_14_Sep_2020_18_12_59}
\def\fignameLivIQR{Livingston_IQR_command_anchoring_in_data_14_Sep_2020_18_12_59}

% Estimated coefficients alpha
\def\fignameAlphaHat{alph_opt_N_100_nfe_5_gridspacing_manual_Wdiffs2_100000_Wmid_1000_Nsimulations_command_sigmas_09_Oct_2020_15_08_45}
% Autocovariogram
\def\fignameAutocov{autocovariogram_N_100_nfe_5_gridspacing_manual_Wdiffs2_100000_Wmid_1000_Nsimulations_command_sigmas_17_Sep_2020_12_05_56}

\def\fignameFeSPF{fe_SPF_command_anchoring_in_data_19_Sep_2020_11_45_59}
\def\fignameFeSPFhist{fe_SPF_hist_command_anchoring_in_data_19_Sep_2020_11_45_59}
\def\fignameGainSPF{gain_SPF_command_anchoring_in_data_11_Oct_2020_11_08_39}
\def\fignameGainSPFhist{gain_SPF_hist_command_anchoring_in_data_11_Oct_2020_11_08_39}


% Comparative statics of optimal policy
% di/dpibar
% not annualized
\def\fignameDiDpibar{analyze_opt_policy_ip17_Sep_2020}
% not annualized with stabilizing factor of 2.6 (25 Sept 2020)
\def\fignameDiDpibarStab{analyze_opt_policy_ip25_Sep_2020}


% Observables in PEA for TR
% annualized
%\def\fignamePEAobsTR{implement_anchTC_obs_TR_approx_40q_17_Sep_2020}
% not annualized
%\def\fignamePEAobsTR{implement_anchTC_obs_TR_approx_40q_17_Sep_2020_19_29_35}
% not annualized, with psi_pi=1.1083 (opt)
\def\fignamePEAobsTR{implement_anchTC_obs_TR_approx24_Sep_2020_09_32_19}

%
% Observables in PEA for anchoring
% annualized
% cutting to 40 periods to 
%\def\fignamePEAobsAnch{implement_anchTC_obs_approx_40q19_Sep_2020_09_30_01}
% not annualized
%\def\fignamePEAobsAnch{implement_anchTC_obs_approx_40q17_Sep_2020_19_19_19}
% not annualized, with psi_pi=1.1083 (opt) and knowTR=1
\def\fignamePEAobsAnch{implement_anchTC_obs_approx24_Sep_2020_09_35_43}



% Compare PEA and VFI policy for sequence of shocks X1 (rng 2)
% annualized
\def\fignamePEAvsVFIfirstX{compare_value_pea_results_approx_value_outputs_approx17_Sep_2020_14_01_16_pea_outputs_approx17_Sep_2020_13_47_33_pretty_19_Sep_2020_09_30_01}
% not annualized
%\def\fignamePEAvsVFIfirstX{compare_value_pea_results_approx_value_outputs_approx17_Sep_2020_14_01_16_pea_outputs_approx17_Sep_2020_13_47_33_pretty_40q_17_Sep_2020_19_23_04}

% Central bank loss as a function of psi_pi in RE vs. anchoring
% Note 30 July 2020: objective_CB_approx.m should use sim_learn_approx_univariate.m to use the univariate anchoring function, but it leads to a bunch of explosions, so I'm not using the correction.
% Note 23 August 2020: now using sim_learn_approx_univariate.m and it simply leads to lots of explosions (I guess alpha and sig are high, and they can't tolerate psi_pi>1.4). But that's fine.
% Note 25 August: update with convention: RE red, anchoring blue.
% retired
%\def\fignameCBlossnilnil{plot_sim_loss_approx_pretty_losses_again_critsmooth_constant_only_pi_only_lamx0_lami0_2020_09_12}
%\def\fignameCBlossonenil{plot_sim_loss_approx_pretty_losses_again_critsmooth_constant_only_pi_only_lamx1_lami0_2020_09_12}
%\def\fignameCBlossnilone{plot_sim_loss_approx_pretty_losses_again_critsmooth_constant_only_pi_only_lamx0_lami1_2020_09_12}
%\def\fignameCBlossoneone{plot_sim_loss_approx_pretty_losses_again_critsmooth_constant_only_pi_only_lamx1_lami1_2020_09_12}
\def\fignameCBlossbaseline{plot_sim_loss_approx_pretty_losses_again_critsmooth_constant_only_pi_only_lamx0_05_lami0_2020_09_19}
\def\fignameCBlossbaselineWithOptimal{plot_sim_loss_approx_pretty_losses_with_optimal_again_critsmooth_constant_only_pi_only_lamx0_05_lami0_2020_09_19}
\def\fignameCBlossbaselineTwoY{plot_sim_loss_approx_2y_losses_with_optimal_again_critsmooth_constant_only_pi_only_lamx0_05_lami0_2020_10_14}

% IRFs anchored vs unanchored
% annualized
\def\fignameIRFanchored{RIR_anch_psi_pi1_5_command_IRFs_approx_pretty_2020_09_17}
\def\fignameIRFunanchored{RIR_unanch_psi_pi1_5_command_IRFs_approx_pretty_2020_09_17}
\def\fignameIRFanchUnanchTogether{RIR_together_psi_pi1_5_command_IRFs_approx_pretty_2020_10_15}


% Cross-sectional average gains for various psi_pi
% not annualized
\def\fignameGainPsiSmall{gain_sim_psi_pi_1_01_command_IRFs_approx_pretty_2020_09_17}
\def\fignameGainPsiMedium{gain_sim_psi_pi_1_5_command_IRFs_approx_pretty_2020_09_17}
\def\fignameGainPsiBig{gain_sim_psi_pi_2_command_IRFs_approx_pretty_2020_09_17}

% IRFs unanchored for various psi_pi
% annualized
\def\fignameIRFpsipiSmall{RIR_unanch_psi_pi1_01_command_IRFs_approx_pretty_2020_09_17}
\def\fignameIRFpsipiMedium{RIR_unanch_psi_pi1_5_command_IRFs_approx_pretty_2020_09_17}
\def\fignameIRFpsipiBig{RIR_unanch_psi_pi2_command_IRFs_approx_pretty_2020_09_17}

%%%%%%%%%%%%%%%%%%%%




\author[]{Laura G\'ati}
% note: in the [] you can put a short name to be displayed in the footer

\institute[]{Boston College}
% note: in the [] you can put a short affiliation to be displayed in the footer

\title[]{Monetary Policy \& Anchored Expectations \\
An Endogenous Gain Learning Model}
% note: in the [] you can put a short title to be displayed in the footer

\date[]{TO DO April 15, 2020}
% note: in the [] you can put a short date to be displayed in the footer

%%%%%%%%%%%%            BEGIN DOCUMENT         %%%%%%%%%%%%%%%%%%%%%

\begin{document}

\begin{frame}[plain] % use the argument plain here to remove the progress bar

\maketitle

\end{frame}


%%%%%%%%%%%%%%%%


%%%%%%% Slide %%%%%%
\begin{frame}\label{motivation}
	
\begin{quote}
Inflation that runs below its desired level can lead to an unwelcome fall in \textbf{longer-term inflation expectations}, which, in turn, can pull actual inflation even lower, resulting in an adverse cycle of ever-lower inflation and inflation expectations.
[...]  \textbf{Well-anchored inflation expectations} are critical[.]  \\
Jerome Powell, Chairman of the Federal Reserve \footnote{``New Economic Challenges and the Fed's Monetary Policy Review,''  August 27, 2020.} \\
(Emphases added.)
% Jackson Hole 
\end{quote}	



\end{frame}
%%%%%%%%%%%%%%%%%
%%%%%%% Slide %%%%%%


%%%%%%% Slide %%%%%%
\begin{frame}
\frametitle{Long-run expectations: capturing responsiveness to short-run conditions}

\

Individual SPF forecasts: for 1991-Q4 onward, estimate rolling regression

\begin{align*}
\Delta\bar{\pi}_t = & \beta_0 + \beta^w_1 f_{t|t-1} + \epsilon_t \numberthis \\
\end{align*}

\

where $w$ indexes windows of 20 quarters length,

\

$f_{t|t-1} \equiv \pi_t - \E_{t-1}\pi_t \quad $ individual one-year-ahead forecast error






\end{frame}
%%%%%%%%%%%%%%%%%

%%%%%%% Slide %%%%%%
\begin{frame}
\frametitle{Time-varying responsiveness}\label{rolling}

\begin{align*}
\Delta\bar{\pi}_t = & \beta_0 + \beta^w_1 f_{t|t-1} + \epsilon_t \tag{1} \\
\end{align*}

\vspace{-0.7cm}

\begin{figure}[h!]
%\caption{Responsiveness of long-run expectations}
\includegraphics[scale = 0.25]{\myFigPath \fignameRolling}
%\caption{Time series of responsiveness of long-run inflation expectations to inflation surprises}
\label{rolling}
\end{figure}



%\vspace{-0.9cm}

\hfill \hyperlink{further_evidence}{\beamergotobutton{Disagreement measures}}
\end{frame}
%%%%%%%%%%%%%%%%%



%%%%%%%% Slide %%%%%%
\begin{frame}
	\frametitle{This project}
	
	\begin{itemize}
	\item How to conduct monetary policy in interaction with the anchoring expectation formation?
	%How does a concern to anchor expectations affect the conduct of monetary policy?	
	
	\
	
	\

	\item Model of expectations unanchoring  \\
	$\hookrightarrow$ extension to adaptive learning that captures time-varying responsiveness of long-run expectations
	
	
	\
	
	\
	
	\item Estimate how unanchoring takes place in data \\
	$\hookrightarrow$ quantify novel anchoring channel and use for monetary policy analysis

	\end{itemize}
	\end{frame}
%%%%%%%%%%%%%%%%%


%%%%%%%% Slide %%%%%%
\begin{frame}
	\frametitle{Preview of results}
	
	\vspace{0.5cm}
	
	\begin{enumerate}
	\item Estimation
		\begin{itemize}
		\item Larger mistakes unanchor more
		
		\
		
		\item Overestimating inflation unanchors more than underestimating it (Hebden et al 2020)
		
		\
		
		\item On average, people discount observations older than 8 quarters
		\end{itemize}
		
		\
		
	\item Optimal policy
		\begin{itemize}
		\item Responds aggressively to inflation when unanchored, accommodates inflation when anchored
		\end{itemize}
		
		\
		
	\item Taylor rule
		\begin{itemize}
		\item Less aggressive on inflation than under rational expectations
		\end{itemize}
	\end{enumerate}

	

\end{frame}
%%%%%%%%%%%%%%%%%%


%%%%%%%% Slide %%%%%%
\begin{frame}
	\frametitle{Related literature}

\

\footnotesize{
\begin{itemize}
\item \textbf{Optimal monetary policy in the New Keynesian model}
\item[] Clarida, Gali \& Gertler (1999), Woodford (2003)

\

\

\item \textbf{Adaptive learning}
\item[] Evans \& Honkapohja (2001, 2006), Sargent (1999), Primiceri (2006), Lubik \& Matthes (2018), Bullard \& Mitra (2002), Preston (2005, 2008), Ferrero (2007), Moln\'ar \& Santoro (2014), Mele et al (2019), Eusepi \& Preston (2011), Milani (2007, 2014)

\

\

\item \textbf{Anchoring and the Phillips curve} 
\item[] Svensson (2015), Hooper et al (2019), Afrouzi \& Yang (2020), Reis (2020), Hebden et al 2020, Gobbi et al (2019), Carvalho et al (2019)

\

\

\item \textbf{Reputation} 
\item[] Barro (1986), Cho \& Matsui (1995)
\end{itemize}

}

\end{frame}
%%%%%%%%%%%%%%%%%%

%%%%%%%% Slide %%%%%%
\begin{frame}
	\frametitle{Structure of talk}

% redefine margin locally - nice. finally it lines up with the TOC
\setlength{\leftmargini}{12pt}
\begin{enumerate}
\item \hspace{-0.15cm} Model of anchoring expectations

\vspace{0.75cm}

\item \hspace{-0.15cm} Quantification of learning channel

\vspace{0.75cm}

\item \hspace{-0.15cm} Solving the Ramsey problem

\vspace{0.75cm}

\item \hspace{-0.15cm} Implementing optimal policy

\vspace{0.75cm}

\item \hspace{-0.15cm} Approximating optimal policy with a Taylor rule



\end{enumerate}


\end{frame}
%%%%%%%%%%%%%%%%%%

%%%%%%%%%%%%%%%%%%%%%%%%%%%%%%%%%%%%%%%%%%%%%%%%%%%%%%
\section{Model of anchoring expectations}
%%%%%%%%%%%%%%%%%%%%%%%%%%%%%%%%%%%%%%%%%%%%%%%%%%%%%%

%%%%%%% Slide %%%%%%
\begin{frame}
	\frametitle{Households: standard up to $\hat{\E}$}
	\label{HH}

Maximize lifetime expected utility
\begin{equation}
\textcolor{brownreddark}{\hat{\E}_t}\sum^{\infty}_{T=t}\beta^{T-t} \bigg[ U(C^i_T) - \int_0^1 v(h^i_T(j)) dj \bigg]
\label{lifetime_U}
\end{equation}	

Budget constraint
\begin{equation}
 B^i_t \leq (1+i_{t-1})B^i_{t-1} + \int_0^1 w_t(j)h^i_t(j) dj + \Pi_t^i(j)  dj-T_t -P_tC^i_t
 \label{BC}
\end{equation}



\vfill

%\vspace{2cm}
\hfill \hyperlink{details_HHs_firms}{\beamergotobutton{Consumption, price level}}
\end{frame}
%%%%%%%%%%%%%%%%%

%%%%%%% Slide %%%%%%
\begin{frame}
	\frametitle{Firms: standard up to $\hat{\E}$}

Maximize present value of profits
\begin{equation}
\textcolor{brownreddark}{\hat{\E}_t}\sum^{\infty}_{T=t}\alpha^{T-t} Q_{t,T} \bigg[ \Pi^j_t(p_t(j))\bigg]
\label{lifetime_profits}
\end{equation}

subject to demand
\begin{equation}
y_t(j) = Y_t \bigg(\frac{p_t(j)}{P_t}\bigg)^{-\theta}
\end{equation}


\vfill

\hfill \hyperlink{details_HHs_firms}{\beamergotobutton{Profits, stochastic discount factor}}

\end{frame}
%%%%%%%%%%%%%%%%%

%%%%%%%% Slide %%%%%%
\begin{frame}
	\frametitle{Expectations: $\hat{\E}$ instead of $\E$}

\

\begin{itemize}
\item Model implies mapping between exogenous states $s_t$ and observables $y_t \equiv (\pi, x, i)'$

 \begin{align}
% s_t & = h s_{t-1} + \epsilon_t \quad \quad \quad \epsilon_t \sim \mathcal{N}(\mathbf{0},\Sigma) \label{state} \\
 y_t & = g s_t \label{obs_RE}
 \end{align}
 
\

\item Under rational expectations (RE), private sector knows model \\
$\rightarrow$ knows (\ref{obs_RE}) 

\begin{equation}
\E_t y_{t+1} = g \E s_{t+1}
\end{equation}
\

\item $\hat{\E}$: agents do not internalize that identical $\rightarrow$ do not know aggregate model $\rightarrow$ do not know (\ref{obs_RE}) \\



\end{itemize}



\end{frame}
%%%%%%%%%%%%%%%%%%

%%%%%%%% Slide %%%%%%
\begin{frame}
	\frametitle{Adaptive learning}
	\label{adaptive_learning}

\

\begin{itemize}
\item Agents know exogenous evolution of states

 \begin{align}
 s_t & = h s_{t-1} + \epsilon_t \quad \quad \quad \epsilon_t \sim \mathcal{N}(\mathbf{0},\Sigma) \label{state} 
% y_t & = g s_t \label{obs_RE}
 \end{align}


\pause

\

\item Postulate linear functional relationship instead of (\ref{obs_RE}):

\begin{equation}
\hat{\E}_t y_{t+1} = \textcolor{brownreddark}{a_{t-1}} + b_{t-1}s_t  \label{PLM_fcst_general}
\end{equation}


\

\pause
  


\item[] \textcolor{brownreddark}{$a$}	$\rightarrow$ concept of long-run expectations in the model

\


\pause


\item Estimate $a, b$ using recursive least squares (RLS) using observed states and knowledge of (\ref{state})



\end{itemize}


\end{frame}
%%%%%%%%%%%%%%%%%%

%%%%%%% Slide %%%%%%  	
\begin{frame}
	\frametitle{Recursive least squares (RLS)}
	\label{RLS_special}
	
	\
	
Observables are: $(\pi, x, i)'$ 	

\
\

Assumption: learn only intercept of inflation:

\begin{equation}
a_{t-1} = ( \textcolor{brownreddark}{\bar{\pi}_{t-1}},0,0)', \quad b_{t-1} = g\; h \quad \forall t
\end{equation}

\
\pause
\

 \textcolor{brownreddark}{$\bar{\pi}_{t-1}$}: long-run inflation expectations $\rightarrow$ anchoring

\
\pause
\

$\rightarrow$ RLS
\begin{equation}
\bar{\pi}_{t}  =\bar{\pi}_{t-1} +k_t \underbrace{\big(\pi_{t} -(\bar{\pi}_{t-1}+b_1 s_{t-1}) \big)}_{\equiv \; f_{t|t-1} \text{, forecast error} } 
\end{equation}
 
 \
 
 $k_t \in (0,1)$ gain \\
 $b_1$ first row of $b$
\vfill 

\hfill \hyperlink{RLS}{\beamergotobutton{General RLS algorithm}}




\end{frame}
%%%%%%%%%%%%%%%%%

%%%%%%% Slide %%%%%%  	
\begin{frame}
	\frametitle{ Alternatives for the gain }
	\label{cgain_dgain}

%\begin{small}
\begin{enumerate}
\item Decreasing gain:

\begin{equation}
\bar{\pi}_{t}  =\bar{\pi}_{t-1} +\frac{1}{t} \; f_{t|t-1}
\end{equation}

 
 \only<1-3>{$\bar{\pi}_{t}$ sample mean of full sample of forecast errors}
 


 \only<4-7>{\textcolor{dodgerblue}{Optimal monetary policy: Mele et al 2019}}

 \
 
 \pause
 
 \
 
 \item Constant gain:
\begin{equation}
\bar{\pi}_{t}  =\bar{\pi}_{t-1} +k \; f_{t|t-1}
\end{equation}
 
 \only<2-3>{$\bar{\pi}_{t}$ sample mean of most recent forecast errors only} 



  \only<5-7>{\textcolor{dodgerblue}{Optimal monetary policy: Moln\'ar \& Santoro 2014}}

  
    
 \pause
 
 \
 
 \
 
 \item Endogenous gain:
 \begin{equation}
\bar{\pi}_{t}  =\bar{\pi}_{t-1} +\mathbf{g}(f_{t|t-1}) \; f_{t|t-1}
\end{equation}

 \only<6-7>{\textcolor{dodgerblue}{Marcet \& Nicolini 2003, Carvalho et al 2019}}
 
  \only<7>{\textcolor{dodgerblue}{Optimal monetary policy: -}}



\end{enumerate}
	

%\end{small}



\end{frame}
%%%%%%%%%%%%%%%%%

%%%%%%%% Slide %%%%%%
\begin{frame}
	\frametitle{Model summary}
	\label{aggregate_LOMS}
\begin{itemize}
\item New Keynesian core: IS and Phillips curves	
 \begin{align}
x_t &=  -\sigma i_t +\hat{\E}_t \sum_{T=t}^{\infty} \beta^{T-t }\big( (1-\beta)x_{T+1} - \sigma(\beta i_{T+1} - \pi_{T+1}) +\sigma r_T^n \big)  \label{NKIS}  \\
\pi_t &= \kappa x_t +\hat{\E}_t \sum_{T=t}^{\infty} (\alpha\beta)^{T-t }\big( \kappa \alpha \beta x_{T+1} + (1-\alpha)\beta \pi_{T+1} + u_T\big) \label{NKPC} 
\end{align}
\hfill \hyperlink{derivations}{\beamergotobutton{Derivations}} \hyperlink{ALMs}{\beamergotobutton{Actual laws of motion}}

\
\pause
\item Expectations: 
\begin{align}
\hat{\E}_{t}\pi_{t+1} & = \bar{\pi}_{t-1} + b_1 s_t \\
\bar{\pi}_{t}  &=\bar{\pi}_{t-1} +\mathbf{g}(f_{t|t-1}) \; f_{t|t-1}
\end{align}

\
\pause
\item[$\rightarrow$] How should $\{ i_t \}$ be set?
\end{itemize}



\end{frame}
%%%%%%%%%%%%%%%%%%



%%%%%%%%%%%%%%%%%%%%%%%%%%%%%%%%%%%%%%%%%%%%%%%%%%%
\section{Quantification of learning channel}
%%%%%%%%%%%%%%%%%%%%%%%%%%%%%%%%%%%%%%%%%%%%%%%%%%%

%%%%%%% Slide %%%%%%  	
\begin{frame}
	\frametitle{Estimating form of gain function}
	\label{anchoring1}



\begin{equation}
 \mathbf{g}(f_{t|t-1}) = \sum_i\gamma_i b_i(f_{t|t-1})\label{gain}
\end{equation}

\

\begin{itemize}
\item $ b_i(f_{t|t-1}) = $ piecewise linear basis

\

\item $\gamma_i  = $ approximating coefficient at node $i$

\
\pause
\item[$\hookrightarrow$] Estimate $\hat{\gamma}$ via simulated method of moments \\ (Duffie \& Singleton 1990, Lee \& Ingram 1991, Smith 1993)

\vspace{0.2cm}

\begin{itemize}
\item Calibrate parameters of New Keynesian core to literature 

\

\item Calibrate variances of disturbances to match moments

\
\item Estimate $\hat{\gamma}$ to match moments
\end{itemize}


\
\pause
\item Moments: autocovariances of inflation, output gap, federal funds rate and 1-year ahead SPF inflation expectations at lags $0, \dots, 4$
\end{itemize}





\end{frame}
%%%%%%%%%%%%%%%%%

%%%%%%%% Slide %%%%%%
\begin{frame}
	\frametitle{Calibration - parameters from the literature}

\begin{center}
\begin{table}
\begin{tabular}{ c | c  | l }
\hline
 $\beta$ & 0.98 & stochastic discount factor \\  \hline
 $\sigma$ & 1  & intertemporal elasticity of substitution \\  \hline
 $\alpha$ & 0.5 &  Calvo probability of not adjusting prices \\\hline
 $\kappa$ & 0.0842 &  slope of the Phillips curve \\\hline
 $\psi_{\pi} $& 1.5  & coefficient of inflation in Taylor rule\\\hline
 $\bar{g}$ & $0.145$  & initial value of the gain \\\hline 
 $\lambda_x$ & 0.05 & weight on the output gap in central bank loss   \\ \hline  
% $\rho_r$ & 0 &   persistence of natural rate shock \\ \hline 
% $\rho_i$ & 0 &  persistence of monetary policy shock*  \\ \hline
% $\rho_u$ & 0  &  persistence of cost-push shock  \\ \hline
   \end{tabular}     
   \label{calibration_lit}
 \end{table}
\end{center}
% TABLES WERE UPDATED 29 AUGUST 2020. 



\

\

Main sources: Chari et al 2000, Woodford 2003, Nakamura \& Steinsson 2008



\end{frame}

%%%%%%%%%%%%%%%%%%

%%%%%%%% Slide %%%%%%
\begin{frame}
	\frametitle{Calibration - matching moments}

\small{
\begin{center}
\begin{table}
\begin{tabular}{ c | c  | l }
\hline
 $\psi_x$ & 0.3   & coefficient of the output gap in Taylor rule  \\\hline 
    $\sigma_r$ & 0.01 & standard deviation, natural rate shock  \\ \hline
    $\sigma_i$ &  0.01  &standard deviation, monetary policy shock  \\ \hline
    $\sigma_u$ & 0.5 & standard deviation, cost-push shock   \\ \hline  
    $\hat{\gamma}_i$ &  \meanalph & coefficients in anchoring function \\ \hline   
\end{tabular}     
       \label{calibration}
 \end{table}
\end{center}
}



\end{frame}
%%%%%%%%%%%%%%%%%%

%%%%%%% Slide %%%%%%
\begin{frame}
	\frametitle{Estimated form for $\mathbf{g(\cdot)}$}

\begin{figure}[h!]
\includegraphics[scale = \mySmallerFigScale]{\myFigPath \fignameAlphaHat}
\caption{Gain as a function of forecast errors in inflation in \%}
\label{epi}
\end{figure}


\end{frame}
%%%%%%%%%%%%%%%%%

%%%%%%% Slide %%%%%%
\begin{frame}
	\frametitle{Forecast errors in the data}

\begin{figure}[h!]
\caption{Time series of 1-year ahead forecast errors and implied gain in the SPF}
\subfigure{\includegraphics[scale = 0.15]{\myFigPath \fignameFeSPF}}
\subfigure{\includegraphics[scale = 0.15]{\myFigPath \fignameGainSPF}}
\label{fe_in_data}
\end{figure}

\

\pause
Mean gain $\approx 0.12$ \pause $\quad \rightarrow$ discount forecast errors older than 8 quarters 


\end{frame}
%%%%%%%%%%%%%%%%%





%%%%%%%%%%%%%%%%%%%%%%%%%%%%%%%%%%%%%%%%%%%%%%%%%%%%%%
\section{Solving the Ramsey problem}
%%%%%%%%%%%%%%%%%%%%%%%%%%%%%%%%%%%%%%%%%%%%%%%%%%%%%%

%%%%%%%% Slide %%%%%%
\begin{frame}
	\frametitle{Ramsey problem}
	 \begin{align*}
& \min_{ \{y_t, \bar{\pi}_{t-1}, k_t \}_{t=t_0}^{\infty}} \E_{t_0}\sum_{t=t_0}^{\infty} \beta^{t-t_0} (\pi_t^2  + \lambda_x x_t^2 )  \\
& \text{s.t. model equations} \\
& \text{s.t. evolution of expectations} 
\end{align*}

\

\

\begin{itemize}
\item $\E$ is the central bank's (CB) expectation

\

\item Assumption: CB observes private expectations and knows the model
\end{itemize}

 

\end{frame}
%%%%%%%%%%%%%%%%%%


%%%%%%%% Slide %%%%%%
\begin{frame}
	\frametitle{Target criterion}
	\label{anchTC}
	
	\begin{prop} 

\

In the model with anchoring, monetary policy optimally brings about the following target relationship between inflation and the output gap

%\begin{small}	
\begin{align*}
& \pi_t  = -\frac{\lambda_x}{\kappa}x_t \\
& + \frac{\lambda_x}{\kappa}\frac{(1-\alpha)\beta}{1-\alpha\beta} \bigg(k_t+f_{t|t-1}\mathbf{g}_{\pi,t}\bigg)
\bigg(\E_t\sum_{i=1}^{\infty}x_{t+i}\prod_{j=0}^{i-1}(1-k_{t+1+j} - f_{t+1+j|t+j}\mathbf{g_{\bar{\pi}, t+j}}) \bigg)
 \label{target}
\end{align*}
%\end{small}
\

\

where $\; \mathbf{g}_{z,t} \equiv \frac{\partial \mathbf{g}}{\partial z}\;$ at $t$, and $b_1$ is the first row of $b$.
	\end{prop}

\vfill

\hfill \hyperlink{generalTC}{\beamergotobutton{General case}}

\end{frame}
%%%%%%%%%%%%%%%%%%

%%%%%%%% Slide %%%%%%
\begin{frame}
	\frametitle{Responding to cost-push shocks}
	
	

\small{
%\begin{align*}
%& \pi_t  =  \; \textcolor{brownreddark}{-\frac{\lambda_x}{\kappa} x_t} \textcolor{dodgerblue}{ \; + \frac{\lambda_x}{\kappa} \frac{(1-\alpha)\beta}{1-\alpha\beta} \bigg(k_t+ f_{t|t-1}\mathbf{g}_{\pi,t} \bigg)\E_t\sum_{i=1}^{\infty}x_{t+i}}  \\
%& \textcolor{aquamarinegreen}{- \frac{\lambda_x}{\kappa} \frac{(1-\alpha)\beta}{1-\alpha\beta} \bigg(k_t+ f_{t|t-1}\mathbf{g}_{\pi,t} \bigg)\E_t\sum_{i=1}^{\infty}x_{t+i}\prod_{j=0}^{i-1}(k_{t+1+j}+ f_{t+1+j|t+j}\mathbf{g_{\bar{\pi},t+j}} )}
%\end{align*}

% makes the covered stuff appear with a low facealpha (only works with some overlay commands such as onslide (not with only))
\setbeamercovered{transparent}

\begin{align*}
& \pi_t  = \only<1>{\textcolor{brownreddark}{-\frac{\lambda_x}{\kappa}x_t} + \frac{\lambda_x}{\kappa}\frac{(1-\alpha)\beta}{1-\alpha\beta} \bigg(k_t+f_{t|t-1}\mathbf{g}_{\pi,t}\bigg)
\bigg(\E_t\sum_{i=1}^{\infty}x_{t+i}\prod_{j=0}^{i-1}(1-k_{t+1+j} - f_{t+1+j|t+j}\mathbf{g_{\bar{\pi}, t+j}}) \bigg)}
\only<2>{-\frac{\lambda_x}{\kappa}x_t \; \textcolor{brownreddark}{ + \frac{\lambda_x}{\kappa}\frac{(1-\alpha)\beta}{1-\alpha\beta} \bigg(k_t+f_{t|t-1}\mathbf{g}_{\pi,t}\bigg)
\bigg(\E_t\sum_{i=1}^{\infty}x_{t+i}}\prod_{j=0}^{i-1}(1-k_{t+1+j} - f_{t+1+j|t+j}\mathbf{g_{\bar{\pi}, t+j}}) \bigg)}
\only<3>{-\frac{\lambda_x}{\kappa}x_t  + \frac{\lambda_x}{\kappa}\frac{(1-\alpha)\beta}{1-\alpha\beta} \bigg(k_t+f_{t|t-1}\mathbf{g}_{\pi,t}\bigg)
\bigg(\E_t\sum_{i=1}^{\infty}x_{t+i}\textcolor{brownreddark}{\prod_{j=0}^{i-1}(1-k_{t+1+j} - f_{t+1+j|t+j}\mathbf{g_{\bar{\pi}, t+j}})} \bigg)}
\end{align*}

\

\

 \onslide<1>{\textcolor<1>{brownreddark}{Intratemporal tradeoffs in RE (discretion)} \\}

\

\

\onslide<2>{\textcolor<2>{brownreddark}{Postpone current tradeoff to future as long as gain $>$ 0} \\}

\

\

\onslide<3>{\textcolor<3>{brownreddark}{Extent to which can postpone depends on not unanchoring too much in future}}

}
\end{frame}
%%%%%%%%%%%%%%%%%%

%%%%%%%% Slide %%%%%%
\begin{frame}
	\frametitle{}
	\label{no_commitment}

\begin{lemma} The discretion and commitment solutions of the Ramsey problem coincide. 
\end{lemma}

\

\hfill \hyperlink{no_commitment_intuition}{\beamergotobutton{Why no commitment?}}

\

\

\begin{corollary} Optimal policy under adaptive learning is time-consistent. 
\end{corollary}







\end{frame}
%%%%%%%%%%%%%%%%%%

%%%%%%%%%%%%%%%%%%%%%%%%%%%%%%%%%%%%%%%%%%%%%%%%%%%%%%
\section{Implementing optimal policy}
%%%%%%%%%%%%%%%%%%%%%%%%%%%%%%%%%%%%%%%%%%%%%%%%%%%%%%


%%%%%%%% Slide %%%%%%
\begin{frame}
	\frametitle{Solution procedure}
	
\begin{itemize}
\item[] Solve system of model equations + target criterion 

\

\item[] $\hookrightarrow$ solve using parameterized expectations (PEA)

\


\item[] $\hookrightarrow$ obtain a cubic spline approximation to optimal policy function

\

\end{itemize}


\end{frame}
%%%%%%%%%%%%%%%%%%



%%%%%%%% Slide %%%%%%
\begin{frame}
	\frametitle{Optimal policy - responding to unanchoring }
	
\begin{figure}[h!]
\includegraphics[scale=0.2]{\myFigPath \fignameDiDpibar}
\caption{Policy function: $i(\bar{\pi}, \text{all other states at their means})$}
\end{figure} 

\

\pause

$\rightarrow$ For 5 bp drop in $\bar{\pi}$, lower $i$ by 2.5 pp

\end{frame}
%%%%%%%%%%%%%%%%%%

%%%%%%%% Slide %%%%%%
\begin{frame}
	\frametitle{Unanchoring causes volatility}

\begin{figure}[h!]
\includegraphics[scale = 0.25]{\myFigPath \fignameIRFanchUnanchTogether}
\caption{Impulse responses after a contractionary monetary policy shock when policy follows a Taylor rule}
\label{IRF}
\end{figure}	

\end{frame}
%%%%%%%%%%%%%%%%%%

%%%%%%%% Slide %%%%%%
\begin{frame}
	\frametitle{Why so volatile? Term structure of expectations}
\begin{itemize}
\item[] IS- and Phillips curve:	
 \begin{align*}
x_t &=  -\sigma i_t +\textcolor{brownreddark}{\hat{\E}_t \sum_{T=t}^{\infty}} \beta^{T-t }\big( (1-\beta)x_{T+1} - \sigma(\beta i_{T+1} - \pi_{T+1}) +\sigma r_T^n \big)   \\
\pi_t &= \kappa x_t +\textcolor{brownreddark}{\hat{\E}_t \sum_{T=t}^{\infty}} (\alpha\beta)^{T-t }\big( \kappa \alpha \beta x_{T+1} + (1-\alpha)\beta \pi_{T+1} + u_T\big) 
\end{align*}


\end{itemize}



\end{frame}
%%%%%%%%%%%%%%%%%%

%%%%%%%% Slide %%%%%%
\begin{frame}
	\frametitle{Why oscillatory? Intertemporal anticipation effects}
\begin{itemize}
\item[] IS- and Phillips curve:	
 \begin{align*}
x_t &=  -\sigma \textcolor{dodgerblue}{i_t} +\hat{\E}_t \sum_{T=t}^{\infty} \beta^{T-t }\big( (1-\beta)x_{T+1} - \sigma(\beta \textcolor{brownreddark}{i_{T+1}} - \pi_{T+1}) +\sigma r_T^n \big)   \\
\pi_t &= \kappa x_t +\hat{\E}_t \sum_{T=t}^{\infty} (\alpha\beta)^{T-t }\big( \kappa \alpha \beta x_{T+1} + (1-\alpha)\beta \pi_{T+1} + u_T\big) 
\end{align*}

\

\

\pause 

\begin{itemize}
\item  \textcolor{brownreddark}{Additional channel of policy}

\

\pause 
\item  \textcolor{brownreddark}{Only if policy reaction function internalized}
\end{itemize}




\end{itemize}



\end{frame}
%%%%%%%%%%%%%%%%%%

%%%%%%%%%%%%%%%%%%%%%%%%%%%%%%%%%%%%%%%%%%%%%%%%%%%%%%
\section{Approximating optimal policy with a Taylor rule}
%%%%%%%%%%%%%%%%%%%%%%%%%%%%%%%%%%%%%%%%%%%%%%%%%%%%%%

%%%%%%%% Slide %%%%%%
\begin{frame}
	\frametitle{Optimal Taylor-coefficient on inflation}
	
\begin{figure}[h!]
\caption{Central bank loss as a function of $\psi_{\pi}$}
\includegraphics[scale=0.24]{\myFigPath \fignameCBlossbaselineTwoY}
\end{figure} 


\small{
Anchoring-optimal coefficient: $\psi_{\pi}^A =1.09 \quad \quad $ RE-optimal coefficient: $\psi_{\pi}^{RE} =2.21$
}


\end{frame}
%%%%%%%%%%%%%%%%%%

%%%%%%%% Slide %%%%%%
\begin{frame}
	\frametitle{Respond but not too much}

\begin{figure}[h!]
\subfigure[$\psi_{\pi} = 1.01$]{\includegraphics[scale = 0.13]{\myFigPath \fignameIRFpsipiSmall}}
%\hfill
\subfigure[$\psi_{\pi} = 1.5$]{\includegraphics[scale = 0.13]{\myFigPath \fignameIRFpsipiMedium}}
\subfigure[$\psi_{\pi} = 2$]{\includegraphics[scale = 0.13]{\myFigPath \fignameIRFpsipiBig}}
\caption{Impulse responses for unanchored expectations for various values of $\psi_{\pi}$}
%\floatfoot{Shock imposed at $t=25$ of a sample length of $T=400$ (with 5 initial burn-in periods), cross-sectional average with a cross-section size of $N=1000$.}
\label{IRF_unanchored_psi}
\end{figure}

\end{frame}
%%%%%%%%%%%%%%%%%%

%%%%%%%% Slide %%%%%%
%\begin{frame}
%	\frametitle{Losses for optimal Taylor-rule coefficient on inflation}
%	
%\begin{center}
%\begin{table}[h!]
%      \caption{Loss for RE and anchoring models for choice of RE- or anchoring-optimal $\psi_{\pi}$}  
%
%\begin{tabular}{ c | c | c | c}
%%\hline
%Anchoring, $\psi_{\pi}^{RE}$ & Anchoring, $\psi_{\pi}^{A}$ & RE, $\psi_{\pi}^{RE}$ & Optimal policy \\  \hline
% 14.2633  &  6.0228 & 4.4866 & 6.0985 \\  \hline
%\end{tabular}     
%      \label{table_welfare}
% \end{table}
%\end{center}
%
%\
%
%\
%
%\pause
%
%$\rightarrow$ Optimal policy or anchoring-optimal $\psi_{\pi}^A$ gets 84\% of the distance to RE-optimal $\psi_{\pi}^{RE}$ under RE
%
%\
%
%\pause
%
%\
%
%Optimal policy vs. Taylor rule? $\rightarrow$ is reaction function internalized or not
%
%
%\end{frame}
%%%%%%%%%%%%%%%%%%

%%%%%%% Slide %%%%%%
\begin{frame}
	\frametitle{Conclusion}

\vspace{0.5cm}
	
\begin{itemize}
\item[] First theory of monetary policy for potentially unanchored expectations 

\

\item[] Estimation of unanchoring in the data
\vspace{0.1cm}
	\begin{itemize}
	\item Large and negative surprises unanchor more
	\vspace{0.1cm}

	\item Estimated gain time series: on average, people only use the last 8 quarters of data
	\end{itemize}
\


\item[] Monetary policy
\vspace{0.1cm}

	\begin{itemize}
	\item Expectations unanchoring makes smoothing shocks over time possible
	\vspace{0.1cm}

	\item Optimal policy frontloads aggressive interest rate response to suppress potential unanchoring
	\vspace{0.1cm}

	\item Taylor rule less aggressive than under rational expectations
	\end{itemize}
	
\


\item[] Future work
\vspace{0.1cm}

\begin{itemize}
\item[$\hookrightarrow$]  How to anchor at zero-lower bound?
\vspace{0.1cm}

\item[$\hookrightarrow$]  Other applications: currency crises
\end{itemize}


%\item For a 0.1 pp drift in long-run inflation expectations, changes interest rate by \movei pp
\end{itemize}


\end{frame}
%%%%%%%%%%%%%%%%%


%%%%%%%%%%%%%%%%%
%         TEMPLATES
%%%%%%%%%%%%%%%%%

%%%%%%%% Slide %%%%%%
%\begin{frame}
%	\frametitle{Slide template}
%
%\end{frame}
%%%%%%%%%%%%%%%%%%

%%%%%%%% Slide %%%%%%  	
%\begin{frame}
%	\frametitle{A beamer button template}
%	\label{identification}
%	
%\hyperlink{Technicalities}{\beamergotobutton{Technicalities}}
%
%\end{frame}
%%%%%%%%%%%%%%%%%%



%%%%%%%%%%%%%%%%%%%%%%%%%%%%%%%%%%%%%%%%%%%%%%%%%%%%%%%%%%%%%%%%%%%%%%%
%%%%%%%                     APPENDIX  
%%%%%%%%%%%%%%%%%%%%%%%%%%%%%%%%%%%%%%%%%%%%%%%%%%%%%%%%%%%%%%%%%%%%%%%
\appendix
\backupbegin

%%%%%%%%%%%%%%%%%%


%%%%%%%% Slide %%%%%%
\begin{frame}[plain]  % remove progress bar from appendix

\begin{centering}
\hfill \textbf{Appendix} \hfill
\end{centering}


\end{frame}
%%%%%%%%%%%%%%%%%%

%%%%%%%% Slide %%%%%%
\begin{frame}[plain]  % remove progress bar from appendix
\frametitle{Breakeven inflation}
	\label{app_TIPS}

\begin{figure}[h!]
\includegraphics[scale = 0.25]{\myFigPath \fignameMarketEPiMoreHorizons} % \fignameMarketEPiCleaned
\caption{Market-based inflation expectations, various horizons, \%}
%\floatfoot{Breakeven inflation, constructed as the difference between the yields of 10-year Treasuries and 10-year TIPS (blue line), difference between 10-year Treasury and 10-year TIPS, the latter cleaned from liquidity risk (red line).}
\label{epi_cleaned}
\end{figure}

\vfill 
\hyperlink{LRE_drifting_down}{\beamerreturnbutton{Return}}	

\end{frame}
%%%%%%%%%%%%%%%%%%

%%%%%%%% Slide %%%%%%
\begin{frame}[plain]  % remove progress bar from appendix
\frametitle{Correcting the TIPS from liquidity risk}

\begin{figure}[h!]
\includegraphics[scale = 0.25]{\myFigPath \fignameMarketEPiCleaned} % \fignameMarketEPiCleaned
\caption{Market-based inflation expectations, 10 year, \%}
%\floatfoot{Breakeven inflation, constructed as the difference between the yields of 10-year Treasuries and 10-year TIPS (blue line), difference between 10-year Treasury and 10-year TIPS, the latter cleaned from liquidity risk (red line).}
\label{epi_cleaned}
\end{figure}

\vfill 
\hyperlink{LRE_drifting_down}{\beamerreturnbutton{Return}}	

\end{frame}
%%%%%%%%%%%%%%%%%%


%%%%%%% Slide %%%%%%
\begin{frame}
	\frametitle{Further evidence}\label{further_evidence}


\begin{figure}[h!]
\caption{Livingston Survey of Firms: \newline Interquartile range of 10-year ahead inflation expectations}
\includegraphics[scale = 0.24]{\myFigPath \fignameLivIQR}
\label{LivIQR}
\end{figure}

\vfill 
\hyperlink{rolling}{\beamerreturnbutton{Return}}	

\end{frame}
%%%%%%%%%%%%%%%%%

%%%%%%% Slide %%%%%%
\begin{frame}
%	\frametitle{Unanchoring in the data}

\begin{figure}[h!]
\caption{New York Fed Survey of Consumers: \newline Percent of respondents indicating 3-year ahead inflation will be in a particular range}
\includegraphics[scale = 0.24]{\myFigPath \fignameSCEdistrib}
\label{SCEdistrib}
\end{figure}

\vfill 
\hyperlink{rolling}{\beamerreturnbutton{Return}}	

\end{frame}
%%%%%%%%%%%%%%%%%


%%%%%%%% Slide %%%%%%
\begin{frame}[plain]  % remove progress bar from appendix
\frametitle{Oscillatory dynamics in adaptive learning}
	\label{oscillatory}

Consider a stylized adaptive learning model in two equations:
\begin{align}
\pi_t & = \beta f_t + u_t  \\
f_t & = f_{t-1} + k(\pi_t - f_{t-1}) 
\end{align}

Solve for the time series of expectations $f_t$
\begin{equation}
f_t = \underbrace{\frac{1-k^{-1}}{1-k^{-1}\beta}}_{\approx 1}f_{t-1} + \frac{k^{-1}}{1-k^{-1}\beta}u_t
\end{equation}

Solve for forecast error $f_t \equiv \pi_t - f_{t-1}$:
\begin{equation}
f_t = \underbrace{-\frac{1-\beta}{1-k\beta}}_{\lim_{k \to 1} = -1}f_{t-1} + \frac{1}{1-k\beta}u_t 
\end{equation}

\end{frame}
%%%%%%%%%%%%%%%%%%


%%%%%%%% Slide %%%%%%
\begin{frame}[plain]  % remove progress bar from appendix
	\frametitle{Functional forms for $\mathbf{g}$ in the literature}
	\label{g}
\begin{itemize}
\item Smooth anchoring function (Gobbi et al, 2019)
\begin{equation}
 p = h(y_{t-1}) = A + \frac{B C e^{-D y_{t-1}}}{( C e^{-D y_{t-1}}+1)^2}
\end{equation}
$p \equiv Prob(\text{liquidity trap regime}) $ \\
$y_{t-1}$ output gap \\


\

\item Kinked anchoring function (Carvalho et al, 2019)
 \begin{align*}
k_t & = \begin{cases} \frac{1}{t} \quad \text{when} \quad \theta_t < \bar{\theta}  \\ k \quad \text{otherwise.}\numberthis
\end{cases} 
\end{align*}
$\theta_t$ criterion, $\bar{\theta}$ threshold value

\end{itemize}

\vfill 
\hyperlink{anchoring1}{\beamerreturnbutton{Return}}	

\end{frame}
%%%%%%%%%%%%%%%%%%


%%%%%%%%%%%%%%%%%%%%% Slide
\begin{frame}[plain]  % remove progress bar from appendix
	\frametitle{Choices for criterion $\theta_t$}
	\label{g}
\begin{itemize}
\item Carvalho et al. (2019)'s criterion  
\begin{equation}
\theta_t^{CEMP} = \max | \Sigma^{-1} ( \phi_{t-1} - T(\phi_{t-1})) |
\end{equation}


$\Sigma$ variance-covariance matrix of shocks \\
$T(\phi)$ mapping from PLM to ALM

\

\

\item CUSUM-criterion
\begin{align}
\omega_t & =  \omega_{t-1} + \kappa k_{t-1}(f_{t|t-1} f_{t|t-1}'  -\omega_{t-1})\\
\theta_t^{CUSUM} & =  \theta_{t-1} + \kappa k_{t-1}(f_{t|t-1}'\omega_t^{-1}f_{t|t-1} -\theta_{t-1})
\end{align}

\

$\omega_t$ estimated forecast-error variance
\end{itemize}




\vfill

\hyperlink{anchoring1}{\beamerreturnbutton{Return}}	


\end{frame}
%%%%%%%%%%%%%%%%%%

%%%%%%%% Slide %%%%%%
\begin{frame}[plain]  % remove progress bar from appendix
	\frametitle{Recursive least squares algorithm}
	\label{RLS}


\begin{align}
\phi_t  & = \bigg( \phi_{t-1}' + k_t R_t^{-1}\begin{bmatrix} 1 \\ s_{t-1} \end{bmatrix}\bigg(y_{t} - \phi_{t-1} \begin{bmatrix} 1 \\ s_{t-1} \end{bmatrix} \bigg)' \bigg)' \\
R_t &= R_{t-1} +  k_t \bigg( \begin{bmatrix} 1 \\ s_{t-1} \end{bmatrix} \begin{bmatrix} 1 & s_{t-1} \end{bmatrix}  - R_{t-1} \bigg)
\end{align}


\vfill

\hyperlink{RLS_special}{\beamerreturnbutton{Return}}	


\end{frame}
%%%%%%%%%%%%%%%%%%


%%%%%%%% Slide %%%%%%
\begin{frame}[plain]  % remove progress bar from appendix
	\frametitle{Actual laws of motion}
	\label{ALMs}

 \begin{align*}
y_t & = A_1 f_{a,t} + A_2 f_{b,t} + A_3 s_t \label{LOM_LR} \numberthis \\
\\
s_t & = h s_{t-1} + \epsilon_t \label{exog} \numberthis
\end{align*}
where

\begin{equation}
 y_t \equiv \begin{pmatrix} \pi_t \\ x_t \\ i_t
 \end{pmatrix} 
 \quad \quad \quad 
  s_t  \equiv \begin{pmatrix} r_t^n \\ u_t 
 \end{pmatrix} 
\end{equation}
and

  \begin{align}
f_{a,t}  \equiv  \hat{\E}_t\sum_{T=t}^{\infty} (\alpha\beta)^{T-t } y_{T+1} \quad \quad \quad \quad 
f_{b,t}  \equiv \hat{\E}_t\sum_{T=t}^{\infty} (\beta)^{T-t } y_{T+1} \label{fafb}
\end{align}

\hyperlink{aggregate_LOMS}{\beamerreturnbutton{Return}}	


\end{frame}
%%%%%%%%%%%%%%%%%%



%%%%%%%% Slide %%%%%%
\begin{frame}[plain]  % remove progress bar from appendix
	\frametitle{No commitment - no lagged multipliers}
	\label{no_commitment_intuition}
	
	Simplified version of the model: planner chooses $\{\pi_t, x_t, f_t, k_t\}_{t=t_0}^{\infty}$ to minimize
 \begin{align*}
\mathcal{L} &= \E_{t_0}\sum_{t=t_0}^{\infty} \beta^{t-t_0}\bigg\{ \pi_t^2  + \lambda x_t^2 + \varphi_{1,t} (\pi_t -\kappa x_t- \beta f_t +u_t) \\ &+ \varphi_{2,t}(f_t - f_{t-1} -k_t(\pi_t - f_{t-1})) + \varphi_{3,t}(k_t- \mathbf{g}(\pi_t - f_{t-1})) \bigg\}
 \end{align*}

 \begin{align}
  2\pi_t +2\frac{\lambda}{\kappa}x_t -\textcolor{brownreddark}{\varphi_{2,t}}(k_t + \mathbf{g_{\pi}}(\pi_t -f_{t-1}))& = 0 \label{simpleFOC1} \\
  -2\beta\frac{\lambda}{\kappa}x_t + \textcolor{brownreddark}{\varphi_{2,t}} -\textcolor{brownreddark}{\varphi_{2,t+1}}(1-k_{t+1} -\mathbf{g_{f}}(\pi_{t+1} -f_{t})) & = 0 \label{simpleFOC2} 
 \end{align}

  
\hyperlink{no_commitment}{\beamerreturnbutton{Return}}	


\end{frame}
%%%%%%%%%%%%%%%%%%



%%%%%%%% Slide %%%%%%
\begin{frame}[plain]  % remove progress bar from appendix
	\frametitle{Target criterion system for anchoring function as changes of the gain}
	\label{generalTC}

\begin{align*}
\varphi_{6,t} & = -c f_{t|t-1} x_{t+1} + \bigg(1+ \frac{f_{t|t-1}}{f_{t+1|t}}(1-k_{t+1}) -f_{t|t-1} \mathbf{g}_{\bar{\pi},t} \bigg) \varphi_{6,t+1} \\
& -\frac{f_{t|t-1}}{f_{t+1|t}}(1-k_{t+1})\varphi_{6,t+2} \numberthis \label{6'} \\
0 & = 2\pi_t + 2\frac{\lambda_x}{\kappa}x_t   - \bigg( \frac{k_t}{f_{t|t-1}} + \mathbf{g}_{\pi,t}\bigg)\varphi_{6,t} + \frac{k_t}{f_{t|t-1}}\varphi_{6,t+1}\numberthis \label{1'}
\end{align*}
$\varphi_{6,t}$ Lagrange multiplier on anchoring function

\

The solution to (\ref{1'}) is given by:
\begin{equation}
\varphi_{6,t} = -2\E_t\sum_{i=0}^{\infty}(\pi_{t+i}+\frac{\lambda_x}{\kappa}x_{t+i})\prod_{j=0}^{i-1}\frac{\frac{k_{t+j}}{f_{t+j|t+j-1}}}{\frac{k_{t+j}}{f_{t+j|t+j-1}} + \mathbf{g}_{\pi, t+j}} \label{sol1'}
\end{equation}


\vspace{-0.5cm}
 
\hyperlink{anchTC}{\beamerreturnbutton{Return}}	


\end{frame}
%%%%%%%%%%%%%%%%%%


%%%%%%%% Slide %%%%%%
\begin{frame}[plain]  % remove progress bar from appendix
	\frametitle{Details on households and firms}
	\label{details_HHs_firms}

\

Consumption:	
\begin{equation}
C^i_t =  \bigg[  \int_0^1 c^i_t(j)^{\frac{\theta-1}{\theta}} dj \bigg]^{\frac{\theta}{\theta-1}}\label{dixit}
\end{equation}
$\theta>1$: elasticity of substitution between varieties

\

Aggregate price level:
\begin{equation}
P_t =  \bigg[  \int_0^1 p_t(j)^{1-\theta} dj \bigg]^{\frac{1}{\theta-1}}
\label{agg_price}
\end{equation}

Profits:
\begin{equation}
\Pi_t^j = p_t(j)y_t(j) -w_t(j)f^{-1}(y_t(j)/A_t)
\end{equation}

Stochastic discount factor
\begin{equation}
Q_{t,T} = \beta^{T-t} \frac{P_t U_c(C_T)}{P_T U_c(C_t)}
\end{equation}

\vspace{-0.5cm}

\hyperlink{HH}{\beamerreturnbutton{Return}}	


\end{frame}
%%%%%%%%%%%%%%%%%%
%%%%%%%% Slide %%%%%%
\begin{frame}[plain]  % remove progress bar from appendix
	\frametitle{Derivations}
	\label{derivations}

Household FOCs
 \begin{align*}
\hat{C}_{t}^i & = \hat{\E}^i_t \hat{C}_{t+1}^i - \sigma(\hat{i}_{t} -\hat{\E}^i_t \hat{\pi}_{t+1})    \label{EE} \numberthis \\
\\
\hat{\E}^i_t \sum_{s=0}^{\infty}\beta^s \hat{C}^i_t& =\omega^i_t + \hat{\E}^i_t \sum_{s=0}^{\infty}\beta^s \hat{Y}^i_t \label{IBC} \numberthis
\end{align*}
where `hats' denote log-linear approximation and $\omega_t^i \equiv \frac{(1+i_{t-1})B^i_{t-1} }{P_t Y^*}$.

\begin{enumerate}
\item Solve (\ref{EE}) backward to some date $t$, take expectations at $t$ 
\item Sub in (\ref{IBC})
\item Aggregate over households $i$
\item[$\rightarrow$] Obtain (\ref{NKIS})
\end{enumerate}



\hyperlink{aggregate_LOMS}{\beamerreturnbutton{Return}}	


\end{frame}
%%%%%%%%%%%%%%%%%%
%
%%%%%%%% Slide %%%%%%
%\begin{frame}
%\frametitle{A beamer button template, how to get back to main text}
%\label{Steps}
%
%\begin{equation}
%D = \begin{bmatrix}
%d_{11} & \gamma_{12} & \gamma_{13} & d_{14} & \cdots \\
%d_{21} & \gamma_{22} & \gamma_{23} & d_{24} & \cdots \\
%\vdots & \vdots & \vdots & \ddots & \vdots 
%\end{bmatrix}
%\end{equation}
%
%\hyperlink{calcCBloss}{\beamerreturnbutton{Return}}	
%\end{frame}
%%%%%%%%%%%%%%%%%%

\backupend


\end{document}
