%2multibyte Version: 5.50.0.2960 CodePage: 65001
\documentclass{article}%
\usepackage{amsmath}%
\setcounter{MaxMatrixCols}{30}%
\usepackage{amsfonts}%
\usepackage{amssymb}%
\usepackage{graphicx}
%TCIDATA{OutputFilter=latex2.dll}
%TCIDATA{Version=5.50.0.2960}
%TCIDATA{Codepage=65001}
%TCIDATA{CSTFile=40 LaTeX article.cst}
%TCIDATA{Created=Friday, September 11, 2020 11:55:52}
%TCIDATA{LastRevised=Saturday, September 12, 2020 12:39:10}
%TCIDATA{<META NAME="GraphicsSave" CONTENT="32">}
%TCIDATA{<META NAME="SaveForMode" CONTENT="1">}
%TCIDATA{BibliographyScheme=Manual}
%TCIDATA{<META NAME="DocumentShell" CONTENT="Standard LaTeX\Blank - Standard LaTeX Article">}
%BeginMSIPreambleData
\providecommand{\U}[1]{\protect\rule{.1in}{.1in}}
%EndMSIPreambleData
\newtheorem{theorem}{Theorem}
\newtheorem{acknowledgement}[theorem]{Acknowledgement}
\newtheorem{algorithm}[theorem]{Algorithm}
\newtheorem{axiom}[theorem]{Axiom}
\newtheorem{case}[theorem]{Case}
\newtheorem{claim}[theorem]{Claim}
\newtheorem{conclusion}[theorem]{Conclusion}
\newtheorem{condition}[theorem]{Condition}
\newtheorem{conjecture}[theorem]{Conjecture}
\newtheorem{corollary}[theorem]{Corollary}
\newtheorem{criterion}[theorem]{Criterion}
\newtheorem{definition}[theorem]{Definition}
\newtheorem{example}[theorem]{Example}
\newtheorem{exercise}[theorem]{Exercise}
\newtheorem{lemma}[theorem]{Lemma}
\newtheorem{notation}[theorem]{Notation}
\newtheorem{problem}[theorem]{Problem}
\newtheorem{proposition}[theorem]{Proposition}
\newtheorem{remark}[theorem]{Remark}
\newtheorem{solution}[theorem]{Solution}
\newtheorem{summary}[theorem]{Summary}
\newenvironment{proof}[1][Proof]{\noindent\textbf{#1.} }{\ \rule{0.5em}{0.5em}}
\begin{document}
\section{Muth: rational and adaptive expectations}

JASA: adaptive expectations are rational if variable is sum of unobserved
permanent and transitory components. \ Adjustment coefficient is between 0 and
1, with higher values when permanent component's innovation is more important.

\section{Phelps: learning and optimal policy}

Today's expectation about tomorrow's inflation%
\[
e_{t}=e_{t-1}+g(\pi_{t}-e_{t-1})+z_{et}%
\]


Phelps' Phillips Curve:
\[
\pi_{t}=\kappa x_{t}+e_{t-1}+z_{\pi t}%
\]


Note that these specifications mean that $e_{t-1}$ is key state
variable.\bigskip

Policy objective (inflation bad, output good)%
\[
E_{t}\sum_{j=0}^{\infty}\beta^{j}u(\pi_{t+j},x_{t+j})
\]


Authority knows $z$ when making date t decisions (this is vector Markov).

\bigskip

Bellman equation
\[
V(e_{t-1},z_{t})=\max\{u(\pi_{t},x_{t})+\beta E_{t}V(e_{t},z_{t+1}%
)|(e_{t-1},z_{t})\}
\]
subject to (2) which is a flow constraint and (1) which is essentially a
transition equation. 

\bigskip

With standard $u$ approximation, problem looks LQ. With constant gain $g$ less
than 1, this is essentially a modern version of Phelp's original analysis. 



\subsection{Empirical questions: }

What is empirical performance of (1) on survey expectations for various $g$?
\ How important are expectations shocks?\bigskip

\subsection{Quantitative model questions:}

Is there an optimal steady state rate of inflation in the absence of shocks?
If so, how does it depend on model parameters?

What is the nature of transitional dynamics? How do these depend on $g$? 

What is the nature of response to expectations shocks?

What is the nature of response to "price shocks"?

\bigskip

What are the gaps between (1) and rational inflation expectations if the above
model is true?

\bigskip

In this model, would a rational observer ever have a reason to view inflation
as "unanchored?"

\bigskip

\section{Nonlinear Expectations}

Forecasters (Stock and Watson) have modified Muth's model to allow for
evolution of permanent and temporary components. \ 

With expectations measures, one can determine if agent's beliefs indicate a
role for nonlinearity. \ Seems like a natural benchmark exercise to motivate
further work with model.

How different is Phelps model if $g(\pi-e)$ is made nonlinear in manner
compatible with evidence from US and other countries? \ Is there a potential
for multiple steady states, i.e., inflation and expected inflation that are
permanently different? \ Is this "unanchored expectations?"

\bigskip

\section{Forward-looking Phillips Curve}
\[
\pi_{t}=\kappa x_{t}+e_{t}+z_{\pi t}%
\]
Key element is now that $e_{t}$ moves with $\pi_{t}$: 

(a)\qquad Change in "slope" with fixed gain, where slope effect depends on $g$

(b)\qquad Nonlinearity with variable gain. \ 

\bigskip

Can (b) lead to "unanchored expectations" as discussed previously?\ 


\end{document}