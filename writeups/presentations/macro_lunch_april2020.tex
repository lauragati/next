\documentclass[11pt]{beamer}
\usepackage[english]{babel}
\usepackage[latin1]{inputenc}
%\usepackage[T1]{fontenc} % don't know what this does: it sets font encoding
%\usepackage{gfsartemisia} % this chooses the font
%%\usepackage{txfonts}
%% the above two seem to give the same, the below two as well
%%\usepackage{accanthis} 
%%\usepackage{concmath}
% Another way to go is the following
%  The available themes are: structurebold, structurebolditalic, structuresmallcapsserif, structureitalicsserif, serif and default.
%\usefonttheme{default}
\usefonttheme{serif}
%\usefonttheme{structurebold}
%%\usefonttheme{structurebolditalic} % .sty wasn't found
%\usefonttheme{structuresmallcaps}
%\usefonttheme{structuresmallcapsserif} % <------
%%%%%%\usefonttheme{structureitalicsserif} % .sty wasn't found
%%%%%\usefonttheme{serif} % the font theme has to correspond to the font used
%%%%\usefonttheme{structurebold} % it seems you can use a combo of these
%%%%% Here comes the font command itself:
%%%%% The available fonts depend on your LATEX installation, the most common are: mathptmx, helvet, avat, bookman, chancery, charter, culer, mathtime, mathptm, newcent, palatino, pifont and utopia.
%%%%% I also have ccfonts
\usepackage{bookman}
%%%%\usepackage{helvet}
%%%%\usepackage[T1]{fontenc}  % some fonts, like concrete math, require a specific font encoding (concrete math requires T1)
\usepackage{subfigure}
%\usepackage[default]{lato}

\usepackage{amssymb}
\usepackage{amsmath}
\usepackage{booktabs}
\usepackage{verbatim}
\usepackage{caption}
\usepackage{float}
\usepackage{csquotes}
\usepackage{sansmathaccent}
\usepackage{subfigure}
\usepackage{multicol}
\pdfmapfile{+sansmathaccent.map}
\usepackage{pgfplots,tikz}
\usetikzlibrary{tikzmark,calc}
\usepackage{overpic}
\usepackage{color,soul}
\usepackage{stackengine}

\def \myFigPath {../../figures/} 
\def \myTablePath {../../tables/} 

\beamertemplatenavigationsymbolsempty % this removes the blue navigation bar that used to be on the bottom right
\setbeamersize{text margin left=30mm,text margin right=30mm} 
%
%\font\reali=msbm10 at 12pt

%


\author[]{Laura G\'ati}
% note: in the [] you can put a short name to be displayed in the footer

\institute[]{Boston College}
% note: in the [] you can put a short affiliation to be displayed in the footer

\title[]{Monetary Policy \& Anchored Expectations \\
An Endogenous Gain Learning Model}
% note: in the [] you can put a short title to be displayed in the footer

\date[]{April 15, 2020}
% note: in the [] you can put a short date to be displayed in the footer

%\usetheme{CambridgeUS} % CambridgeUS, Madrid
\usetheme{Madrid}
\usecolortheme{dove}



\AtBeginSection[]
{
  \begin{frame}<beamer>
    \frametitle{Structure of talk}
    \tableofcontents[currentsection]
  \end{frame}
}
\useinnertheme{circles}

% resets the bullet points of enumerate and itemize to defaults
\setbeamertemplate{itemize items}[bullet]
\setbeamertemplate{enumerate items}[default]

%\defbeamertemplate{section in toc}{square unnumbered}{%
%   \leavevmode\leftskip=1.75ex%
%   \llap{\textcolor{red!70!black}{\vrule width2.25ex
%     height1.85ex depth.4ex}}%
%   \kern1.5ex\inserttocsection\par}
%\setbeamertemplate{section in toc}[square unnumbered]

% this removes the ugly bullets for the table of contents and just adds numbers
\setbeamertemplate{section in toc}{\inserttocsectionnumber.~\inserttocsection}

% New Commands
\newcommand{\real}{\hbox{\reali R}}
\newcommand{\realp}{\hbox{\reali R}_{\scriptscriptstyle +}}
\newcommand{\realpp}{\hbox{\reali R}_{\scriptscriptstyle ++}}
\newcommand{\R}{\mathbb{R}}
\DeclareMathOperator{\E}{\mathbb{E}}
\DeclareMathOperator{\argmin}{arg\,min}
\newcommand\w{3.0in}
\newcommand\wnum{3.0}
\def\myFigWidth{5.3in}
\def\mySmallerFigWidth{2.1in}
\def\myEvenBiggerFigScale{0.8}
\def\myPointSixFigScale{0.6}
\def\myBiggerFigScale{0.4}
\def\myFigScale{0.3}
\def\mySmallFigScale{0.25}
\def\mySmallerFigScale{0.18}
\def\myTinyFigScale{0.16}
\def\myPointFourteenFigScale{0.14}
\def\myTinierFigScale{0.12}
\def\myEvenTinierFigScale{0.10}
\def\myCrazyTinyFigScale{0.09}

\newtheorem{prop}{Proposition}
\newtheorem*{remark}{Remark}
\newtheorem{result}{Result}
%\newtheorem{lemma}{Lemma}
%\newtheorem{corollary}{Corollary}



\newcommand\numberthis{\addtocounter{equation}{1}\tag{\theequation}} % this defines a command to make align only number this line

% creating colored lines for legends
% green solid
\newcommand{\greenline}{\raisebox{2pt}{\tikz{\draw[-,black!40!green,solid,line width = 1.5pt](0,0) -- (5mm,0);}}}
% red solid, dashed, dotted
\newcommand{\redline}{\raisebox{2pt}{\tikz{\draw[-,red,solid,line width = 1.5pt](0,0) -- (5mm,0);}}}
\newcommand{\reddashedline}{\raisebox{2pt}{\tikz{\draw[-,red,dashed,line width = 1.5pt](0,0) -- (5mm,0);}}}
\newcommand{\reddottedline}{\raisebox{2pt}{\tikz{\draw[-,red,densely dotted,line width = 1.5pt](0,0) -- (5mm,0);}}}

% blue solid and dashed
\newcommand{\blueline}{\raisebox{2pt}{\tikz{\draw[-,blue,solid,line width = 1.5pt](0,0) -- (5mm,0);}}}
\newcommand{\bluedashedline}{\raisebox{2pt}{\tikz{\draw[-,blue,dashed,line width = 1.5pt](0,0) -- (5mm,0);}}}
% black solid and dashed
\newcommand{\blackline}{\raisebox{2pt}{\tikz{\draw[-,black,solid,line width = 1.5pt](0,0) -- (5mm,0);}}}
\newcommand{\blackdashedline}{\raisebox{2pt}{\tikz{\draw[-,black,dashed,line width = 1.5pt](0,0) -- (5mm,0);}}}


% These are my colors -- there are many like them, but these ones are mine (Paul Goldsmith-Pinkham's)
\definecolor{blue_PGP}{RGB}{0,114,178}
\definecolor{red_PGP}{RGB}{213,94,0}
\definecolor{yellow_PGP}{RGB}{240,228,66}
\definecolor{green_PGP}{RGB}{0,158,115}
\definecolor{mygreen}{RGB}{0, 128, 0} % this is Laura's green
%% I use a beige off white for my background 
\definecolor{MyBackground}{RGB}{255,253,218}
%% Uncomment this if you want to change the background color to something else 
%\setbeamercolor{background canvas}{bg=MyBackground} % this is a very nice beige
%% Change the bg color to adjust your transition slide background color! 
%\newenvironment{transitionframe}{
%\setbeamercolor{background canvas}{bg=yellow} \begin{frame}}{
%\end{frame} }

%\setbeamercolor{frametitle}{fg=blue_PGP} 
%\setbeamercolor{title}{fg=black} 
%\setbeamertemplate{footline}[frame number]  % this sets the footer the same color as the background
%\setbeamertemplate{navigation symbols}{} % don't know what this does
%\setbeamercolor{button}{bg=MyBackground,fg=blue_PGP}

% footnote suppressing number
\newcommand\blfootnote[1]{%
  \begingroup
  \renewcommand\thefootnote{}\footnote{#1}%
  \addtocounter{footnote}{-1}%
  \endgroup
}

% try to force appendix to be numbered differently
\newcommand{\backupbegin}{
   \newcounter{framenumberappendix}
   \setcounter{framenumberappendix}{\value{framenumber}}
}
\newcommand{\backupend}{
   \addtocounter{framenumberappendix}{-\value{framenumber}}
   \addtocounter{framenumber}{\value{framenumberappendix}} 
}

%%%%%%%%%%%%            BEGIN DOCUMENT         %%%%%%%%%%%%%%%%%%%%%
\begin{document}


\begin{frame}

\maketitle


\end{frame}



%%%%%%% Slide %%%%%%
\begin{frame}
	\frametitle{Puzzling Fed behavior fall 2019}

\begin{figure}[h!]
\subfigure[Unemployment rate, \%]{\includegraphics[scale = 0.12]{\myFigPath urate_2020_02_09}}
\hfil \subfigure[Fed funds rate target, upper limit, \%]{\includegraphics[scale = 0.12]{\myFigPath frr_2020_02_09}}
\subfigure[Market-based inflation expectations, 10 year, \% average]{\includegraphics[scale = 0.12]{\myFigPath epi10_2020_02_09}}
\end{figure}


\end{frame}
%%%%%%%%%%%%%%%%%



%%%%%%%% Slide %%%%%%
\begin{frame}
	\frametitle{This project}
	
	\begin{enumerate}
	\item[] Model anchored expectations as an endogenous gain learning scheme
	
	\
	
	\
	
	\item[$\rightarrow$] How to conduct optimal monetary policy in interaction with the anchoring expectation formation?
	%How does a concern to anchor expectations affect the conduct of monetary policy?
	\end{enumerate}
	\end{frame}
	%%%%%%%% Slide %%%%%%
\begin{frame}
	\frametitle{Preview of results}
	
	\begin{enumerate}
	\item Two layers of new intertemporal tradeoffs 

	
	\
	
	\
	
	\item optimal monetary policy time-inconsistent
	
	\
	
	\
	
	
	\item[$\rightarrow$] illustrate analytically in special case: target criterion 
	
	\
	
	\
	\item Not today: short-run costs vs. long-run benefits of anchoring expectations

	\end{enumerate}


\end{frame}
%%%%%%%%%%%%%%%%%%


%%%%%%%% Slide %%%%%%
\begin{frame}
	\frametitle{Related literature}

\begin{itemize}
\item \textbf{Optimal monetary policy in New Keynesian models}
\item[] Clarida, Gali \& Gertler (1999), Woodford (2003)

\

\

\item \textbf{Econometric learning}
\item[] Evans \& Honkapohja (2001), Preston (2005), Moln\'ar \& Santoro (2014)

\

\

\item \textbf{Anchoring / endogenous gain} 
\item[] Carvalho et al (2019), Svensson (2015), Hooper et al (2019), Milani (2014)
\end{itemize}



\end{frame}
%%%%%%%%%%%%%%%%%%

%%%%%%%%%%%%%%%%%%%%%%%%%%%%%%%%%%%%%%%%%%%%%%%%%%%%%%
\section{Model}
%%%%%%%%%%%%%%%%%%%%%%%%%%%%%%%%%%%%%%%%%%%%%%%%%%%%%%

%%%%%%% Slide %%%%%%
\begin{frame}
	\frametitle{Households: standard up to $\hat{\E}$}
	\label{HH}

Maximize lifetime expected utility
\begin{equation}
\textcolor{blue}{\hat{\E}_t}\sum^{\infty}_{T=t}\beta^{T-t} \bigg[ U(C^i_T) - \int_0^1 v(h^i_T(j)) dj \bigg]
\label{lifetime_U}
\end{equation}	

Budget constraint
\begin{equation}
 B^i_t \leq (1+i_{t-1})B^i_{t-1} + \int_0^1 w_t(j)h^i_t(j) + \Pi_t^i(j)  dj-T_t -P_tC^i_t
 \label{BC}
\end{equation}



\vfill

%\vspace{2cm}
\hfill \hyperlink{details_HHs_firms}{\beamergotobutton{Consumption, price level}}
\end{frame}
%%%%%%%%%%%%%%%%%

%%%%%%% Slide %%%%%%
\begin{frame}
	\frametitle{Firms: standard up to $\hat{\E}$}

Maximize present value of profits
\begin{equation}
\textcolor{blue}{\hat{\E}_t}\sum^{\infty}_{T=t}\alpha^{T-t} Q_{t,T} \bigg[ \Pi^j_t(p_t(j))\bigg]
\label{lifetime_profits}
\end{equation}

subject to demand
\begin{equation}
y_t(j) = Y_t \bigg(\frac{p_t(j)}{P_t}\bigg)^{-\theta}
\end{equation}


\vfill

\hfill \hyperlink{details_HHs_firms}{\beamergotobutton{Profits, stochastic discount factor}}

\end{frame}
%%%%%%%%%%%%%%%%%

%%%%%%%% Slide %%%%%%
\begin{frame}
	\frametitle{Expectations: $\hat{\E}$ instead of $\E$}

\begin{itemize}
\item If use $\E$ (rational expectations, RE) \\

\

Model solution 
 \begin{align}
 s_t & = h s_{t-1} + \epsilon_t \quad \quad \quad \epsilon_t \sim \mathcal{N}(\mathbf{0},\Sigma) \label{state} \\
 y_t & = g s_t \label{obs_RE}
 \end{align}


$s_t \equiv (r^n_t, u_t)' \quad $  (states) \\
$y_t \equiv (\pi_t, x_t, i_t)' \quad $ (jumps)

\

\item If use $\hat{\E} \rightarrow$ don't know $g$ \\
$\rightarrow$ estimate using observed states \& knowledge of (\ref{state})
\end{itemize}



\end{frame}
%%%%%%%%%%%%%%%%%%

%%%%%%%% Slide %%%%%%
\begin{frame}
	\frametitle{Adaptive learning}
	\label{adaptive_learning}

\begin{itemize}
\item Estimate $g$ using recursive least squares (RLS)
  \\
  
  \
  
$\rightarrow$ nonrational expectations:
\begin{equation} 
\hat{\E}_t y_{t+1} = \phi_{t-1}\begin{bmatrix} \textcolor{blue}{1} \\ s_{t} \end{bmatrix} \label{PLMcompact}
\end{equation}

\item Note: \textcolor{blue}{misspecified}	\\

\

Can write:
\begin{equation}
\hat{\E}_t y_{t+1} = \textcolor{blue}{a_{t-1}} + b_{t-1}s_t  \label{PLM_fcst_general}
\end{equation}

\

In RE, $\textcolor{blue}{a_{t-1} = (0,0,0)'}, b_{t-1} = g\;h \quad \forall t$
\end{itemize}


\end{frame}
%%%%%%%%%%%%%%%%%%

%%%%%%% Slide %%%%%%  	
\begin{frame}
	\frametitle{Recursive least squares}
	\label{RLS_special}
	
	\
	
Special case: learn only intercept of inflation:

\begin{equation}
a_{t-1} = (\bar{\pi}_{t-1},0,0)', \quad b_{t-1} = g\; h \quad \forall t
\end{equation}

\

\

$\rightarrow$ RLS
\begin{equation}
\bar{\pi}_{t}  =\bar{\pi}_{t-1} +k_t \underbrace{\big(\pi_{t} -(\bar{\pi}_{t-1}+b_1 s_{t-1}) \big)}_{\equiv \; fe_{t|t-1} \text{, forecast error} } 
\end{equation}
 
 \
 
 $k_t \in (0,1)$ gain \\
 $b_1$ first row of $b$
\vfill 

\hfill \hyperlink{RLS}{\beamergotobutton{General RLS algorithm}}




\end{frame}
%%%%%%%%%%%%%%%%%

%%%%%%% Slide %%%%%%  	
\begin{frame}
	\frametitle{Anchoring mechanism: endogenous gain}
	\label{anchoring1}


Gain in literature usually exogenous: 
\begin{equation*}
k_t = \begin{cases}\frac{1}{t} \quad \quad \text{decreasing}\\
k \quad \quad \text{constant}
\end{cases}
\end{equation*}

\

\

Here instead
\begin{equation}
k_t = k_{t-1} + \mathbf{g}(fe_{t|t-1}) \label{gain}
\end{equation}


\vfill 

\hfill \hyperlink{g}{\beamergotobutton{Functional forms}}



\end{frame}
%%%%%%%%%%%%%%%%%

%%%%%%%% Slide %%%%%%
\begin{frame}
	\frametitle{Model summary}
	\label{aggregate_LOMS}
\begin{itemize}
\item IS- and Phillips curve:	
 \begin{align}
x_t &=  -\sigma i_t +\hat{\E}_t \sum_{T=t}^{\infty} \beta^{T-t }\big( (1-\beta)x_{T+1} - \sigma(\beta i_{T+1} - \pi_{T+1}) +\sigma r_T^n \big)  \label{NKIS}  \\
\pi_t &= \kappa x_t +\hat{\E}_t \sum_{T=t}^{\infty} (\alpha\beta)^{T-t }\big( \kappa \alpha \beta x_{T+1} + (1-\alpha)\beta \pi_{T+1} + u_T\big) \label{NKPC} 
\end{align}
\hfill \hyperlink{derivations}{\beamergotobutton{Derivations}}

\

\item  Expectations evolve according to RLS with the endogenous gain given by (\ref{gain})

\

\item[$\rightarrow$] How should $\{ i_t \}$ be set?
\end{itemize}



\end{frame}
%%%%%%%%%%%%%%%%%%


%%%%%%%%%%%%%%%%%%%%%%%%%%%%%%%%%%%%%%%%%%%%%%%%%%%%%%
\section{Solving the Ramsey problem}
%%%%%%%%%%%%%%%%%%%%%%%%%%%%%%%%%%%%%%%%%%%%%%%%%%%%%%

%%%%%%%% Slide %%%%%%
\begin{frame}
	\frametitle{Ramsey problem}
	 \begin{align*}
& \min_{ \{y_t, \bar{\pi}_{t-1}, k_t \}_{t=t_0}^{\infty}} \E_{t_0}\sum_{t=t_0}^{\infty} \beta^{t-t_0} (\pi_t^2  + \lambda_x x_t^2 )  \\
& \text{s.t. model equations}
\end{align*}

\

\

\begin{itemize}
\item $\E$ is the central bank's (CB) expectation

\

\item Assumption: CB observes private expectations and knows the model
\end{itemize}

 

\end{frame}
%%%%%%%%%%%%%%%%%%

%%%%%%%% Slide %%%%%%
\begin{frame}
	\frametitle{Special case}
	\label{special_case}

\begin{itemize}
\item Only inflation intercept learned

\

\item Anchoring function simplified to 
\begin{equation}
k_t = \mathbf{g}(fe_{t|t-1}) \label{gain_simple}
\end{equation}
\end{itemize}



\end{frame}
%%%%%%%%%%%%%%%%%%

%%%%%%%% Slide %%%%%%
\begin{frame}
	\frametitle{Target criterion for special case}
	\label{anchTC}
	
	\begin{result} 

\

In the simplified model with anchoring, monetary policy optimally brings about the following target relationship between inflation and the output gap
	
\begin{align*}
\pi_t  = -\frac{\lambda_x}{\kappa}\bigg\{x_t - \frac{(1-\alpha)\beta}{1-\alpha\beta} \bigg(k_t+((\pi_t - \bar{\pi}_{t-1}-b_1 s_{t-1}))\mathbf{g}_{\pi,t}\bigg) \\
\\
\bigg(\E_t\sum_{i=1}^{\infty}x_{t+i}\prod_{j=0}^{i-1}(1-k_{t+1+j} - (\pi_{t+1+j} - \bar{\pi}_{t+j}-b_1 s_{t+j})\mathbf{g_{\bar{\pi}, t+j}}) \bigg)
\bigg\}  \label{target}
\end{align*}

\

\

where $\; \mathbf{g}_{z,t} \equiv \frac{\partial \mathbf{g}}{\partial z}\;$ at $t$, $\; \prod_{j=0}^{0} \equiv 1 \; $ and $b_1$ is the first row of $b$.
	\end{result}
\vfill 

\hfill \hyperlink{generalTC}{\beamergotobutton{General case}}

\end{frame}
%%%%%%%%%%%%%%%%%%

%%%%%%%% Slide %%%%%%
\begin{frame}
	\frametitle{Two layers of intertemporal tradeoffs}
\small{
\begin{align*}
& \pi_t  =  \; \textcolor{red}{-\frac{\lambda_x}{\kappa} x_t} \textcolor{blue}{ \; + \frac{\lambda_x}{\kappa} \frac{(1-\alpha)\beta}{1-\alpha\beta} \bigg(k_t+ fe_{t|t-1}\mathbf{g}_{\pi,t} \bigg)\E_t\sum_{i=1}^{\infty}x_{t+i}}  \\
& \textcolor{mygreen}{- \frac{\lambda_x}{\kappa} \frac{(1-\alpha)\beta}{1-\alpha\beta} \bigg(k_t+ fe_{t|t-1}\mathbf{g}_{\pi,t} \bigg)\E_t\sum_{i=1}^{\infty}x_{t+i}\prod_{j=0}^{i-1}(k_{t+1+j}+ fe_{t+1+j|t+j}\mathbf{g_{\bar{\pi},t+j}} )}
\end{align*}

\

\

\textcolor{red}{Intratemporal tradeoffs in RE (discretion)} \\

\

\

\textcolor{blue}{Intertemporal tradeoff: current level and change of the gain } \\

\

\

\textcolor{mygreen}{Intertemporal tradeoff: future expected levels and changes of the gain}

}
\end{frame}
%%%%%%%%%%%%%%%%%%

%%%%%%%% Slide %%%%%%
\begin{frame}
	\frametitle{}
	\label{no_commitment}

\begin{lemma} The commitment solution of the Ramsey problem does not exist under adaptive learning. 
\end{lemma}

\

\

\

\begin{corollary} Optimal policy under adaptive learning is time-inconsistent. 
\end{corollary}

\

\vfill

\hfill \hyperlink{no_commitment_intuition}{\beamergotobutton{Why no commitment?}}


\end{frame}
%%%%%%%%%%%%%%%%%%





%%%%%%%%%%%%%%%%%%%%%
\section{Implications}
%%%%%%%%%%%%%%%%%%%%%


%%%%%%%% Slide %%%%%%
\begin{frame}
	\frametitle{How to implement?}

\	
	\begin{itemize}
	\item Related issue under RE: optimal interest rate sequence implies indeterminate equilibrium
	
	\
	
	\item[$\Rightarrow$] Reaction function stabilizes expectations
	
	\
	
	\
	
	\item[] Recall IS-curve:
	\begin{equation*}
	x_t =  -\sigma i_t +\hat{\E}_t \sum_{T=t}^{\infty} \beta^{T-t }\big( (1-\beta)x_{T+1} - \sigma(\beta \textcolor{blue}{i_{T+1}} - \pi_{T+1}) +\sigma r_T^n \big) 
	\end{equation*}
	
	\
	
	\
	
	\item E.g. Taylor rule disciplines expectations:
	
	 $$\; \hat{\E}_t i_T = \psi_{\pi}\hat{\E}_t\pi_{T}+\psi_{x}\hat{\E}_tx_{T}$$
	\end{itemize}


\end{frame}
%%%%%%%%%%%%%%%%%%

%%%%%%%% Slide %%%%%%
\begin{frame}
	\frametitle{Next steps: form of reaction function}
	
\begin{itemize}
\item Model suggests $\quad i_t = \mathbf{f}(\pi_t, k_{t}, \bar{\pi}_{t-1}; t)$ nonlinear

\

\

\item Explains deviations from Taylor rule


\

\

\item However: no commitment makes Taylor rule more viable than under RE as a rough approximation of optimal feedback rule

\

\

\item If Taylor rule, model prefers being less aggressive on inflation
\end{itemize}
	

\end{frame}
%%%%%%%%%%%%%%%%%%

%%%%%%% Slide %%%%%%
\begin{frame}
	\frametitle{Conclusion}
	
\begin{itemize}
\item Interaction between monetary policy and anchoring 

\

\

\item Optimal policy conditions on stance of current and expected future anchoring
\item[] \hspace{4cm} $\hookrightarrow$ determine intertemporal tradeoffs

\

\

\item Explain departures from the Taylor rule like US, fall 2019

\

\

\item If Fed acted to anchor expectations, then Missing Deflation and Inflation are not ``missing"
\end{itemize}


\end{frame}
%%%%%%%%%%%%%%%%%


%%%%%%%%%%%%%%%%%
%         TEMPLATES
%%%%%%%%%%%%%%%%%

%%%%%%%% Slide %%%%%%
%\begin{frame}
%	\frametitle{Slide template}
%
%\end{frame}
%%%%%%%%%%%%%%%%%%

%%%%%%%% Slide %%%%%%  	
%\begin{frame}
%	\frametitle{A beamer button template}
%	\label{identification}
%	
%\hyperlink{Technicalities}{\beamergotobutton{Technicalities}}
%
%\end{frame}
%%%%%%%%%%%%%%%%%%



%%%%%%%%%%%%%%%%%%%%%%%%%%%%%%%%%%%%%%%%%%%%%%%%%%%%%%%%%%%%%%%%%%%%%%%
%%%%%%%                     APPENDIX  
%%%%%%%%%%%%%%%%%%%%%%%%%%%%%%%%%%%%%%%%%%%%%%%%%%%%%%%%%%%%%%%%%%%%%%%
\appendix
\backupbegin

%%%%%%%% Slide %%%%%%
\begin{frame}
	\frametitle{Short-run costs, long-run benefits}
	\label{SRLRtradeoffs}

\
	
Assume Taylor rule and no concern for output gap stabilization	
	
\begin{equation*}
i_t = \psi_{\pi}\pi_t  \quad \quad \lambda_x = 0 
\end{equation*}

\begin{figure}[h!]
\subfigure[RE]{\includegraphics[scale = 0.14]{\myFigPath plot_sim_loss_loss_RE_params_psi_pi_1_5_psi_x_0_gbar_0_145_thetbar_4_thettilde_2_5_kap_0_8_lamx_0_lami_0_2020_02_09}}
\subfigure[Anchoring]{\includegraphics[scale = 0.14]{\myFigPath plot_sim_loss_loss_again_critCUSUM_constant_only_params_psi_pi_1_5_psi_x_0_gbar_0_145_thetbar_4_thettilde_2_5_kap_0_8_lamx_0_lami_0_2020_02_09}}
\caption{Central bank loss as a function of $\psi_{\pi}$}
%\floatfoot{}
\label{fig_loss}
\end{figure}	
	
\hfill \hyperlink{SRcosts}{\beamergotobutton{Short-run costs}}

\end{frame}
%%%%%%%%%%%%%%%%%%


%%%%%%%% Slide %%%%%%
\begin{frame}
	\frametitle{Functional forms for $\mathbf{g}$}
	\label{g}
\begin{itemize}
\item Smooth anchoring function
\begin{equation}
k_t = k_{t-1} - c + d fe_{t|t-1}^2
\end{equation}
$c,d > 0$

\

\item Kinked anchoring function
 \begin{align*}
k_t & = \begin{cases} \frac{1}{t} \quad \text{when} \quad \theta_t < \bar{\theta}  \\ k \quad \text{otherwise.}\numberthis
\end{cases} 
\end{align*}
$\theta_t$ criterion, $\bar{\theta}$ threshold value

\end{itemize}

\vfill 
\hyperlink{anchoring1}{\beamerreturnbutton{Return}}	

\end{frame}
%%%%%%%%%%%%%%%%%%

\begin{frame}
	\frametitle{Choices for criterion $\theta_t$}
	\label{g}
\begin{itemize}
\item Carvalho et al. (2019)'s criterion  
\begin{equation}
\theta_t^{CEMP} = \max | \Sigma^{-1} ( \phi_{t-1} - T(\phi_{t-1})) |
\end{equation}


$\Sigma$ variance-covariance matrix of shocks \\
$T(\phi)$ mapping from PLM to ALM

\

\

\item CUSUM-criterion
\begin{align}
\omega_t & =  \omega_{t-1} + \kappa k_{t-1}(fe_{t|t-1} fe_{t|t-1}'  -\omega_{t-1})\\
\theta_t^{CUSUM} & =  \theta_{t-1} + \kappa k_{t-1}(fe_{t|t-1}'\omega_t^{-1}fe_{t|t-1} -\theta_{t-1})
\end{align}

\

$\omega_t$ estimated forecast-error variance
\end{itemize}




\vfill

\hyperlink{anchoring1}{\beamerreturnbutton{Return}}	


\end{frame}
%%%%%%%%%%%%%%%%%%

%%%%%%%% Slide %%%%%%
\begin{frame}
	\frametitle{Recursive least squares algorithm}
	\label{RLS}


\begin{align}
\phi_t  & = \bigg( \phi_{t-1}' + k_t R_t^{-1}\begin{bmatrix} 1 \\ s_{t-1} \end{bmatrix}\bigg(y_{t} - \phi_{t-1} \begin{bmatrix} 1 \\ s_{t-1} \end{bmatrix} \bigg)' \bigg)' \\
R_t &= R_{t-1} +  k_t \bigg( \begin{bmatrix} 1 \\ s_{t-1} \end{bmatrix} \begin{bmatrix} 1 & s_{t-1} \end{bmatrix}  - R_{t-1} \bigg)
\end{align}


\vfill

\hyperlink{anchoring1}{\beamerreturnbutton{Return}}	


\end{frame}
%%%%%%%%%%%%%%%%%%


%%%%%%%% Slide %%%%%%
\begin{frame}
	\frametitle{Compact notation}
	\label{compact}

 \begin{align*}
y_t & = A_1 f_{a,t} + A_2 f_{b,t} + A_3 s_t \label{LOM_LR} \numberthis \\
\\
s_t & = h s_{t-1} + \epsilon_t \label{exog} \numberthis
\end{align*}
where

\begin{equation}
 y_t \equiv \begin{pmatrix} \pi_t \\ x_t \\ i_t
 \end{pmatrix} 
 \quad \quad \quad 
  s_t  \equiv \begin{pmatrix} r_t^n \\ \bar{i}_t \\ u_t 
 \end{pmatrix} 
\end{equation}
and

  \begin{align}
f_{a,t}  \equiv  \hat{\E}_t\sum_{T=t}^{\infty} (\alpha\beta)^{T-t } y_{T+1} \quad \quad \quad \quad 
f_{b,t}  \equiv \hat{\E}_t\sum_{T=t}^{\infty} (\beta)^{T-t } y_{T+1} \label{fafb}
\end{align}

\hyperlink{aggregate_LOMS}{\beamerreturnbutton{Return}}	


\end{frame}
%%%%%%%%%%%%%%%%%%



%%%%%%%% Slide %%%%%%
\begin{frame}
	\frametitle{No commitment - no lagged multipliers}
	\label{no_commitment_intuition}
	
	Simplified version of the model: planner chooses $\{\pi_t, x_t, f_t, k_t\}_{t=t_0}^{\infty}$ to minimize
 \begin{align*}
\mathcal{L} &= \E_{t_0}\sum_{t=t_0}^{\infty} \beta^{t-t_0}\bigg\{ \pi_t^2  + \lambda x_t^2 + \varphi_{1,t} (\pi_t -\kappa x_t- \beta f_t +u_t) \\ &+ \varphi_{2,t}(f_t - f_{t-1} -k_t(\pi_t - f_{t-1})) + \varphi_{3,t}(k_t- \mathbf{g}(\pi_t - f_{t-1})) \bigg\}
 \end{align*}

 \begin{align}
  2\pi_t +2\frac{\lambda}{\kappa}x_t -\textcolor{blue}{\varphi_{2,t}}(k_t + \mathbf{g_{\pi}}(\pi_t -f_{t-1}))& = 0 \label{simpleFOC1} \\
  -2\beta\frac{\lambda}{\kappa}x_t + \textcolor{blue}{\varphi_{2,t}} -\textcolor{blue}{\varphi_{2,t+1}}(1-k_{t+1} -\mathbf{g_{f}}(\pi_{t+1} -f_{t})) & = 0 \label{simpleFOC2} 
 \end{align}

  
\hyperlink{no_commitment}{\beamerreturnbutton{Return}}	


\end{frame}
%%%%%%%%%%%%%%%%%%

%%%%%%%% Slide %%%%%%
\begin{frame}
	\frametitle{Short-run costs from oscillatory dynamics}
	\label{SRcosts}

 Consider a stylized adaptive learning model in two equations:
\begin{align}
\pi_t & = \beta f_t + u_t \label{simple_NKPC} \\
f_t & = f_{t-1} + k(\pi_t - f_{t-1}) \label{simple_expectations}
\end{align}

Solve for the time series of expectations $f_t$
\begin{equation}
f_t = \underbrace{\frac{1-k^{-1}}{1-k^{-1}\beta}}_{\approx 1}f_{t-1} + \frac{k^{-1}}{1-k^{-1}\beta}u_t
\end{equation}

Solve for forecast error $fe_t \equiv \pi_t - f_{t-1}$:
\begin{equation}
fe_t = \underbrace{-\frac{1-\beta}{1-k\beta}}_{\lim_{k \to 1} = -1}f_{t-1} + \frac{1}{1-k\beta}u_t \label{oscillating_fe}
\end{equation}

 
\hyperlink{SRLRtradeoffs}{\beamerreturnbutton{Return}}	


\end{frame}
%%%%%%%%%%%%%%%%%%

%%%%%%%% Slide %%%%%%
\begin{frame}
	\frametitle{Target criterion system for anchoring function as changes of the gain}
	\label{generalTC}

\begin{align*}
\varphi_{6,t} & = -c fe_{t|t-1} x_{t+1} + \bigg(1+ \frac{fe_{t|t-1}}{fe_{t+1|t}}(1-k_{t+1}) -fe_{t|t-1} \mathbf{g}_{\bar{\pi},t} \bigg) \varphi_{6,t+1} \\
& -\frac{fe_{t|t-1}}{fe_{t+1|t}}(1-k_{t+1})\varphi_{6,t+2} \numberthis \label{6'} \\
0 & = 2\pi_t + 2\frac{\lambda_x}{\kappa}x_t   - \bigg( \frac{k_t}{fe_{t|t-1}} + \mathbf{g}_{\pi,t}\bigg)\varphi_{6,t} + \frac{k_t}{fe_{t|t-1}}\varphi_{6,t+1}\numberthis \label{1'}
\end{align*}
$\varphi_{6,t}$ Lagrange multiplier on anchoring function

\

The solution to (\ref{1'}) is given by:
\begin{equation}
\varphi_{6,t} = -2\E_t\sum_{i=0}^{\infty}(\pi_{t+i}+\frac{\lambda_x}{\kappa}x_{t+i})\prod_{j=0}^{i-1}\frac{\frac{k_{t+j}}{fe_{t+j|t+j-1}}}{\frac{k_{t+j}}{fe_{t+j|t+j-1}} + \mathbf{g}_{\pi, t+j}} \label{sol1'}
\end{equation}

 
\hyperlink{anchTC}{\beamerreturnbutton{Return}}	


\end{frame}
%%%%%%%%%%%%%%%%%%


%%%%%%%% Slide %%%%%%
\begin{frame}
	\frametitle{Details on households and firms}
	\label{details_HHs_firms}

\

Consumption:	
\begin{equation}
C^i_t =  \bigg[  \int_0^1 c^i_t(j)^{\frac{\theta-1}{\theta}} dj \bigg]^{\frac{\theta}{\theta-1}}\label{dixit}
\end{equation}
$\theta>1$: elasticity of substitution between varieties

\

Aggregate price level:
\begin{equation}
P_t =  \bigg[  \int_0^1 p_t(j)^{1-\theta} dj \bigg]^{\frac{1}{\theta-1}}
\label{agg_price}
\end{equation}

Profits:
\begin{equation}
\Pi_t^j = p_t(j)y_t(j) -w_t(j)f^{-1}(y_t(j)/A_t)
\end{equation}

Stochastic discount factor
\begin{equation}
Q_{t,T} = \beta^{T-t} \frac{P_t U_c(C_T)}{P_T U_c(C_t)}
\end{equation}


\hyperlink{HH}{\beamerreturnbutton{Return}}	


\end{frame}
%%%%%%%%%%%%%%%%%%
%%%%%%%% Slide %%%%%%
\begin{frame}
	\frametitle{Derivations}
	\label{derivations}

Household FOCs
 \begin{align*}
\hat{C}_{t}^i & = \hat{\E}^i_t \hat{C}_{t+1}^i - \sigma(\hat{i}_{t} -\hat{\E}^i_t \hat{\pi}_{t+1})    \label{EE} \numberthis \\
\\
\hat{\E}^i_t \sum_{s=0}^{\infty}\beta^s \hat{C}^i_t& =\omega^i_t + \hat{\E}^i_t \sum_{s=0}^{\infty}\beta^s \hat{Y}^i_t \label{IBC} \numberthis
\end{align*}
where `hats' denote log-linear approximation and $\omega_t^i \equiv \frac{(1+i_{t-1})B^i_{t-1} }{P_t Y^*}$.

\begin{enumerate}
\item Solve (\ref{EE}) backward to some date $t$, take expectations at $t$ 
\item Sub in (\ref{IBC})
\item Aggregate over households $i$
\item[$\rightarrow$] Obtain (\ref{NKIS})
\end{enumerate}



\hyperlink{aggregate_LOMS}{\beamerreturnbutton{Return}}	


\end{frame}
%%%%%%%%%%%%%%%%%%
%
%%%%%%%% Slide %%%%%%
%\begin{frame}
%\frametitle{A beamer button template, how to get back to main text}
%\label{Steps}
%
%\begin{equation}
%D = \begin{bmatrix}
%d_{11} & \gamma_{12} & \gamma_{13} & d_{14} & \cdots \\
%d_{21} & \gamma_{22} & \gamma_{23} & d_{24} & \cdots \\
%\vdots & \vdots & \vdots & \ddots & \vdots 
%\end{bmatrix}
%\end{equation}
%
%\hyperlink{calcCBloss}{\beamerreturnbutton{Return}}	
%\end{frame}
%%%%%%%%%%%%%%%%%%

\backupend


\end{document}
