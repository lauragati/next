\documentclass[11pt]{article}
\usepackage{amsmath, amsthm, amssymb,lscape, natbib}
\usepackage{mathtools}
\usepackage{subfigure}
\usepackage[font=footnotesize,labelfont=bf]{caption}
\usepackage{graphicx}
\usepackage{colortbl}
\usepackage{hhline}
\usepackage{multirow}
\usepackage{multicol}
\usepackage{setspace}
\usepackage[final]{pdfpages}
\usepackage[left=2.5cm,top=2.5cm,right=2.5cm, bottom=2.5cm]{geometry}
\usepackage{natbib} 
\usepackage{bibentry} 
\newcommand{\bibverse}[1]{\begin{verse} \bibentry{#1} \end{verse}}
\newcommand{\vs}{\vspace{.3in}}
\renewcommand{\ni}{\noindent}
\usepackage{xr-hyper}
\usepackage[]{hyperref}
\hypersetup{
    colorlinks=true,
    linkcolor=blue,
    filecolor=magenta,      
    urlcolor=cyan,
}
 
\urlstyle{same}
\usepackage[capposition=top]{floatrow}
\usepackage{amssymb}
\usepackage{relsize}
\usepackage[dvipsnames]{xcolor}
\usepackage{fancyhdr}
\usepackage{tikz}
 
\pagestyle{fancy} % customize header and footer
\fancyhf{} % clear initial header and footer
%\rhead{Overleaf}
\lhead{\centering \rightmark} % this adds subsection number and name
\lfoot{\centering \rightmark} 
\rfoot{\thepage} % put page number (the centering command puts it in the middle, don't matter if you put it in right or left footer)

\def \myFigPath {../figures/} 
% BE CAREFUL WITH FIGNAMES, IN LATEX THEY'RE NOT CASE SENSITIVE!!
\def \myTablePath {../tables/} 

%\definecolor{mygreen}{RGB}{0, 100, 0}
\definecolor{mygreen}{RGB}{0, 128, 0}

\definecolor{citec}{rgb}{0,0,.5}
\definecolor{linkc}{rgb}{0,0,.6}
\definecolor{bcolor}{rgb}{1,1,1}
\hypersetup{
%hidelinks = true
  colorlinks = true,
  urlcolor=linkc,
  linkcolor=linkc,
  citecolor = citec,
  filecolor = linkc,
  pdfauthor={Laura G\'ati},
}


\geometry{left=.83in,right=.89in,top=1in,
bottom=1in}
\linespread{1.5}
\renewcommand{\[}{\begin{equation}}
\renewcommand{\]}{\end{equation}}

% New Options
\newtheorem{prop}{Proposition}
\newtheorem{definition}{Definition}[section]
\newtheorem*{remark}{Remark}
\newtheorem{lemma}{Lemma}
\newtheorem{corollary}{Corollary}
\newtheorem{conjecture}{Conjecture}

%\newtheorem{theorem}{Theorem}[section] % the third argument specifies that their number will be adopted to the section
%\newtheorem{corollary}{Corollary}[theorem]
%\newtheorem{lemma}[theorem]{Lemma}
%\declaretheorem{proposition}
%\linespread{1.3}
%\raggedbottom
%\font\reali=msbm10 at 12pt

% New Commands
\newcommand{\real}{\hbox{\reali R}}
\newcommand{\realp}{\hbox{\reali R}_{\scriptscriptstyle +}}
\newcommand{\realpp}{\hbox{\reali R}_{\scriptscriptstyle ++}}
\newcommand{\R}{\mathbb{R}}
\DeclareMathOperator{\E}{\mathbb{E}}
\DeclareMathOperator{\argmin}{arg\,min}
\newcommand\w{3.0in}
\newcommand\wnum{3.0}
\def\myFigWidth{5.3in}
\def\mySmallerFigWidth{2.1in}
\def\myEvenBiggerFigScale{0.8}
\def\myPointSixFigScale{0.6}
\def\myBiggerFigScale{0.4}
\def\myFigScale{0.3}
\def\myMediumFigScale{0.25}
\def\mySmallFigScale{0.22}
\def\mySmallerFigScale{0.18}
\def\myTinyFigScale{0.16}
\def\myPointFourteenFigScale{0.14}
\def\myTinierFigScale{0.12}
\def\myAdjustableFigScale{0.14}
\newcommand\numberthis{\addtocounter{equation}{1}\tag{\theequation}} % this defines a command to make align only number this line
\newcommand{\code}[1]{\texttt{#1}} %code %

\renewcommand*\contentsname{Overview}
\setcounter{tocdepth}{2}

% define a command to make a huge question mark (it works in math mode)
\newcommand{\bigqm}[1][1]{\text{\larger[#1]{\textbf{?}}}}

\begin{document}

\linespread{1.0}

\title{Materials 19 - Limits on analytical results?}
\author{Laura G\'ati} 
\date{\today}
\maketitle

%%%%%%%%%%%%%%%%%%%%             DOCUMENT           %%%%%%%%%%%%%%%%%% 

%\tableofcontents

%\listoffigures

\section{A simplified optimal policy with anchoring problem}
Planner chooses $\{\pi_t, x_t, f_t, k_t^{-1}\}_{t=t_0}^{\infty}$ to minimize
 \begin{align*}
\mathcal{L} &= \E_{t_0}\sum_{t=t_0}^{\infty} \beta^{t-t_0}\bigg\{ \pi_t^2  + \lambda x_t^2 + \varphi_{1,t} (\pi_t -\kappa x_t- \beta f_t +u_t) \\ &+ \varphi_{2,t}(f_t - f_{t-1} -k_t^{-1}(\pi_t - f_{t-1})) + \varphi_{3,t}(k_t^{-1} - \mathbf{g}(\pi_t - f_{t-1})) \bigg\}
 \end{align*}
 where the IS-curve, $x_t = \E_t x_{t+1}+\sigma f_t -\sigma i_t +\sigma r_t^n$, is a non-binding constraint, and $\E_t x_{t+1}$ is rational. 
 After some manipulation, FOCs reduce to:
 \begin{align}
  2\pi_t +2\frac{\lambda}{\kappa}x_t -\varphi_{2,t}(k_t^{-1} + \mathbf{g_{\pi}}(\pi_t -f_{t-1}))& = 0 \label{FOC1learn} \\
  -2\beta\frac{\lambda}{\kappa}x_t + \varphi_{2,t} -\varphi_{2,t+1}(1-k_{t+1}^{-1} -\mathbf{g_{f}}(\pi_{t+1} -f_{t})) & = 0 \label{FOC2learn} 
 \end{align}
Combining FOCs with the three model equations, I obtain the following system in $\{\pi_t, x_t, f_t,k_t^{-1}, \varphi_t\}_{t=0}^{\infty}$ where I've relabeled $\varphi \equiv \varphi_2$ for simplicity:
 \begin{align}
  2\pi_t +2\frac{\lambda}{\kappa}x_t -\varphi_{t}(k_t^{-1} + \mathbf{g_{\pi}}(\pi_t -f_{t-1}))& = 0 \\
  -2\beta\frac{\lambda}{\kappa}x_t + \varphi_{t} -\varphi_{t+1}(1-k_{t+1}^{-1} -\mathbf{g_{f}}(\pi_{t+1} -f_{t})) & = 0 \\
  \pi_t -\kappa x_t- \beta f_t +u_t & = 0 \\
  f_t - f_{t-1} -k_t^{-1}(\pi_t - f_{t-1}) & = 0 \\
  k_t^{-1} - \mathbf{g}(\pi_t - f_{t-1}) & = 0 
 \end{align}

Unless I find some trick to convert this to a linear system, I can't solve it.

\newpage
So I see 3 options:
\begin{enumerate}
\item Find trick to linearize.
\item Discuss optimal policy only up to target criterion, which can be derived even for the more complex (but simplified) model and is given by:
\begin{align}
\pi_t  = -\frac{\lambda}{\kappa}\bigg\{x_t - \frac{(1-\alpha)\beta}{1-\alpha\beta} \bigg(k_t^{-1}+((\pi_t - \bar{\pi}_{t-1}-b s_{t-1}))f_{\pi}(t) \bigg) 
\bigg(\sum_{i=1}^{\infty}x_{t+i}\prod_{j=1}^{i-1}(1-k_{t+j}^{-1}(\pi_{t+1+j} - \bar{\pi}_{t+j}-b s_{t+j})) \bigg)
\bigg\} \label{target}
\end{align}
One could also be ``Woodfordian'' and try to find simplifying expressions for this criterion that are approximately as good as this.
\item Since no distinction between discretion and commitment, could investigate a purely forward-looking policy rule (Taylor rule), i.e. reexamine the noninertial plan. 
\end{enumerate}


\appendix
\newpage
\section{Procedure to obtain the optimal interest rate rule: NKIS vs. target criterion}
``NKIS approach" (Woodford 2003 and Handbook chapter, Moln\'ar \& Santoro)
\begin{enumerate}
\item Combine all FOCs to a single difference equation (system) in one variable (vector), solve it for the time path of the variable as a function of forcing terms.
\item Express optimal time paths for all endogenous variables, plug into NKIS.
\item Obtain optimal time path for the nominal interest rate as a function pf disturbances only.
\end{enumerate}

\noindent ``Target criterion approach" (Woodford 2003 and Handbook chapter)
\begin{enumerate}
\item From FOCs, express a target criterion as a relationship between endogenous variables that is optimal.
\item Derive the same (?) time paths for the endogenous variables, use NKIS and NKPC to express expectations for them.
\item Plug everything into the target criterion, rearrange to obtain the nominal interest rate path.
\end{enumerate}

Why? The criterion is robust, but the rule isn't. Both result in a rule that delivers an indeterminate equilibrium. (In fact, they result in the same rule (Hbook, p. 19).)

\section{Details}
Woodford: 
\begin{enumerate}
\item Set up the above problem with RE. Get FOCs.
\item Combine FOCs to get a relationship between $\pi, x$; the target criterion.
\item Obtain the following 2nd order difference equation in the multiplier:
\begin{equation}
\beta \E_t\varphi_{t+1} - \big(1+\beta+\frac{\kappa^2}{\lambda}\big)\varphi_t + \varphi_{t-1} = \frac{\kappa}{\lambda} x^* + u_t
\end{equation}
\item Obtain the optimal time path as the solution of the above equation:
\begin{equation}
\varphi_t = -\frac{\lambda}{\kappa}x^*(1-\mu_1^{t+1}) -\beta^{-1}\sum_{j=0}^{\infty}\mu_2^{-j-1}\E_t u_{t+j} \quad \quad 0 < \mu_1 < 1 < \mu_2
\end{equation}
(where I think $\beta^{-j}$ is missing in the sum, and we know from the Vi\`ete formulas that $\mu_1 = \mu_2^{-1}\beta^{-1}$) \\
Substitution into FOCs gives time path for $\pi,x$.
\item Sub time paths into NKPC and NKIS to solve for agents' expectations as functions of shocks only.
\item Evaluate target criterion at time $t$ by substituting in time paths and expectations of $\pi,x$. Rearrange to express $i_t$.
\end{enumerate}


\noindent Moln\'ar \& Santoro
\begin{enumerate}
\item Set up the above problem with learning. Get FOCs. 
\item Express a target criterion, but don't think of it as such explicitly.
\item Sub out $x$ from it using the NKIS to express $\E_t \pi_{t+1}$ as a function of current inflation, expectations and disturbances. 
\item At this point they think of the economy as a trivariate system (inflation, inflation beliefs, output gap beliefs) where $t+1$ values depend on $t$ ones and disturbances. Solve for the evolution of inflation as a function of inflation beliefs and disturbances only using a guess-and-verify method.
\item Once you have that for $\pi$, you can get the same for $x$.
\item Plug time paths for $\pi,x$ into NKPC. Express $i_t$.
\end{enumerate}



\end{document}





