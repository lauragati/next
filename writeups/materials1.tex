\documentclass[11pt]{article}
\usepackage{amsmath, amsthm, amssymb,lscape, natbib}
\usepackage{mathtools}
\usepackage{subfigure}
\usepackage[font=footnotesize,labelfont=bf]{caption}
\usepackage{graphicx}
\usepackage{colortbl}
\usepackage{hhline}
\usepackage{multirow}
\usepackage{multicol}
\usepackage{setspace}
\usepackage[final]{pdfpages}
\usepackage[left=2.5cm,top=2.5cm,right=2.5cm, bottom=2.5cm]{geometry}
\usepackage{natbib} 
\usepackage{bibentry} 
\newcommand{\bibverse}[1]{\begin{verse} \bibentry{#1} \end{verse}}
\newcommand{\vs}{\vspace{.3in}}
\renewcommand{\ni}{\noindent}
\usepackage{xr-hyper}
\usepackage[]{hyperref}
\usepackage[capposition=top]{floatrow}
\usepackage{amssymb}


\def \myFigPath {../figures/} 
% BE CAREFUL WITH FIGNAMES, IN LATEX THEY'RE NOT CASE SENSITIVE!!
\def \myTablePath {../tables/} 

\definecolor{citec}{rgb}{0,0,.5}
\definecolor{linkc}{rgb}{0,0,.6}
\definecolor{bcolor}{rgb}{1,1,1}
\hypersetup{
%hidelinks = true
  colorlinks = true,
  urlcolor=linkc,
  linkcolor=linkc,
  citecolor = citec,
  filecolor = linkc,
  pdfauthor={Laura G\'ati},
}


\geometry{left=.83in,right=.89in,top=1in,
bottom=1in}
\linespread{1.5}
\renewcommand{\[}{\begin{equation}}
\renewcommand{\]}{\end{equation}}

% New Options
\newtheorem{prop}{Proposition}
\newtheorem{definition}{Definition}[section]
\newtheorem*{remark}{Remark}
\newtheorem{lemma}{Lemma}
\newtheorem{corollary}{Corollary}
%\newtheorem{theorem}{Theorem}[section] % the third argument specifies that their number will be adopted to the section
%\newtheorem{corollary}{Corollary}[theorem]
%\newtheorem{lemma}[theorem]{Lemma}
%\declaretheorem{proposition}
%\linespread{1.3}
%\raggedbottom
%\font\reali=msbm10 at 12pt

% New Commands
\newcommand{\real}{\hbox{\reali R}}
\newcommand{\realp}{\hbox{\reali R}_{\scriptscriptstyle +}}
\newcommand{\realpp}{\hbox{\reali R}_{\scriptscriptstyle ++}}
\newcommand{\R}{\mathbb{R}}
\DeclareMathOperator{\E}{\mathbb{E}}
\DeclareMathOperator{\argmin}{arg\,min}
\newcommand\w{3.0in}
\newcommand\wnum{3.0}
\def\myFigWidth{5.3in}
\def\mySmallerFigWidth{2.1in}
\def\myEvenBiggerFigScale{0.8}
\def\myPointSixFigScale{0.6}
\def\myBiggerFigScale{0.4}
\def\myFigScale{0.3}
\def\mySmallFigScale{0.22}
\def\mySmallerFigScale{0.18}
\def\myTinyFigScale{0.16}
\def\myPointFourteenFigScale{0.14}
\def\myTinierFigScale{0.12}
\newcommand\numberthis{\addtocounter{equation}{1}\tag{\theequation}} % this defines a command to make align only number this line
\newcommand{\code}[1]{\texttt{#1}} %code %

\renewcommand*\contentsname{Overview}
\setcounter{tocdepth}{2}

\begin{document}

\linespread{1.0}

\title{Materials 1}
\author{Laura G\'ati} 
\date{\today}
\maketitle

%%%%%%%%%%%%%%%%%%%%             DOCUMENT           %%%%%%%%%%%%%%%%%% 

\tableofcontents

%\listoffigures


\section{Some inflation targeting countries}
\begin{figure}[h!]
\caption{4 inflation targeting countries (switch to IT year in brackets)}
\centering
\includegraphics[scale=\myBiggerFigScale]{\myFigPath materials1_infl_targeting_countries}
\end{figure}

\newpage
\section{Cost-push shock in NK model w/o and w/ learning - Intuition}

Same clarification (based on Justiniano \& Primiceri's Bank of Belgium presentation 2010):
\begin{itemize}
\item \underline{efficient output}: output under flexible prices and no distortions
\item \underline{potential output}: output under flexible prices but imperfect competition, yet constant markups
\item \underline{natural output}: output under flexible prices but imperfect competition AND nominal frictions (time-varying markups)
\end{itemize}
$\rightarrow$ Eric Sims: ``2 distortions in the NK model: LR distortion (imperfect competition) and SR distortion (price rigidity).''
I think:
\begin{enumerate}
\item In st.st, potential output = natural output. 
\item In an NK model, mon. policy stabilizes actual output to potential output (= st.st. natural output). In other words, mon. pol can't undo a LR distortion because money is neutral in the LR.
\end{enumerate}
A cost-push shock:
\begin{itemize}
\item is a shock to the steady-state markup $\mu$: $\mu = \frac{\theta}{\theta-1}$, where $\theta =$ el. of substitution between varieties $\rightarrow$ it's a shock to the level of LR distortions (to the distance to perfect competition).
\item Note that a cost-push shock isn't the only thing that can move markups in the NK model: out of st.st., marginal costs are an increasing function of demand (and thus of output gaps), and so demand shocks increase marginal costs, decreasing markups.
\item I think it is for this reason that Clarida, Gali \& Gertler (1999) call cost-push shocks ``all shocks that are not demand that move markups.''
\end{itemize}
Why think about cost-push shocks? $\rightarrow$ because they are the only shock that introduces an inflation-output gap tradeoff for the mon. authority. Key point: I think that this tradeoff is amplified is expectations are allowed to move around. That is, I think one can make a case for anchoring expectations here. I'm going to lay out the argument in three steps:
\begin{itemize}
\item Clarida, Gali \& Gertler (1999)'s Result 1: for mon. policy w/o commitment, cost-push shocks introduce an inflation-output gap tradeoff in a NK model. If expectations are allowed to adjust, the tradeoff is amplified.
\item Let's see the same intuition on the plain-vanilla 3-equation NK model w/ commitment (a Taylor rule). f expectations are allowed to adjust, the tradeoff is amplified.
\item Let's look at the CEMP economy: we can see the same intuition at work as in the 3-equation NK model.
\end{itemize}


\subsection{Clarida, Gali \& Gertler (1999)'s Result 1}
The CB's problem is the standard mon. policy problem under discretion:
\begin{align}
\max -\frac{1}{2}(\alpha x_t^2 + \pi_t^2 ) + F_t \quad \text{s.t} \quad \pi_t = \lambda x_t + f_t
\end{align}
where expectations $F_t$ and $f_t = \beta \E \pi_{t+1} + u_t$ are taken as given by the CB, and $u_t$ is a cost-push shock (appended to the NKPC). Optimality conditions to this problem, subbing $f_t$ in, are:
\begin{align}
x_t = -\frac{\lambda}{\alpha + \lambda^2}(\beta \E \pi_{t+1} + u_t) \\
\pi_t = \frac{\alpha}{\alpha + \lambda^2}(\beta \E \pi_{t+1} + u_t) 
\end{align}
\begin{itemize}
\item A favorable cost-push shock ($\theta  \uparrow, u_t \downarrow$, that is we move towards perfect competition) $\rightarrow x_t \uparrow, \pi_t \downarrow$.
\item If $\E \pi_{t+1}$ is allowed to move (unanchored), it will decrease over time, amplifying the shock. (I'm implicitly assuming RLS learning for the expectation formation.)
\end{itemize}

\subsection{The 3-equation NK model}
\begin{align}
\pi_t & = \beta \E_t\pi_{t+1} + \kappa x_t  + u_t\\
x_t & = \E_t x_{t+1} -\frac{1}{\sigma} (i_t - \E_t\pi_{t+1}) + \frac{1}{\sigma}r_t^n \\
i_t & = \delta_{\pi} \pi_t + \delta_x x_t
\end{align}
\begin{itemize}
\item $u_t \downarrow \rightarrow \pi_t \downarrow$ (my intuition tells me that $x_t$ should go up because less market power means higher quantities, lower prices, but I fail to see it here).
\item If mon policy reacts, as the TR tells it to, it lowers interest rates, leading to a rise in output gaps. Voila. 
\item Again, if expectations are allowed to move, assuming RLS learning, $\E\pi_{t+1} \downarrow$, pushing inflation down further, amplifying the shock. 
\item Although expectations moving also helps you a bit, because it balances some of the increase in $x_t$. (?)
\end{itemize}

	
\subsection{CEMP (w/o specified mon. policy)}
\begin{align}
k_t & = \mathbf{f_k} \\
\bar{\pi}_t &=\mathbf{f_{\bar{\pi}}} + \mathbf{f_k}^{-1}\eta_{t-1}\\
\xi_t & = \mathbf{f_{\xi}} + \mathbf{A_{\xi}} \xi_{t-1} + S_{\xi} \begin{pmatrix} \epsilon_t \\ \mu_t \end{pmatrix} 
\end{align}
where $\xi = (\eta, \varphi, \pi)'$, and $\pi$ is described by a hybrid Phillips-curve with marginal cost shocks (shock $\epsilon$ on $\varphi$) and markup shocks $\mu_t$, and $\eta_t = \mu_t + \epsilon_t$, i.e. it's a catch-all shock which mixes marginal cost and markup shocks $\rightarrow$ marginal cost shocks stand in for demand shocks (which is absent in the model), and $\mu_t$ is the cost-push shock. (Here a demand shock will have the same effect as a cost-push shock... mmm, I don't know if I like that...)
\begin{itemize}
\item $\mu_t \downarrow \rightarrow \eta_t \downarrow$
\item $\pi_t \downarrow$
\item If beliefs anchored, $\mathbf{f_k}^{-1} \rightarrow 0$, i.e. beliefs aren't allowed to fluctuate, so we stop here.
\item If unanchored, $\bar{\pi}_t \downarrow \rightarrow \pi_t \downarrow \dots$ Again, shock is amplified. 
\end{itemize}

\section{Questions}
\begin{itemize}
\item Why is the discretion and 3-equation intuition not quite the same? Does ignoring beliefs make their amplification role more acute?
\item Demand shocks? $\rightarrow$ it seems to me to be a general thing that expectations (if based on RLS learning) amplify shocks, making stabilization more costly. But cost-push shocks seem special because of the tradeoff they introduce. 
\item Cost-push shocks as supply shocks? The issue of stabilization when shocks permanent (new steady state).
\end{itemize}




 
\end{document}



