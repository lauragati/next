\documentclass[11pt]{article}
\usepackage{amsmath, amsthm, amssymb,lscape, natbib}
\usepackage{mathtools}
\usepackage{subfigure}
\usepackage[font=footnotesize,labelfont=bf]{caption}
\usepackage{graphicx}
\usepackage{colortbl}
\usepackage{hhline}
\usepackage{multirow}
\usepackage{multicol}
\usepackage{setspace}
\usepackage[final]{pdfpages}
\usepackage[left=2.5cm,top=2.5cm,right=2.5cm, bottom=2.5cm]{geometry}
\usepackage{natbib} 
\usepackage{bibentry} 
\newcommand{\bibverse}[1]{\begin{verse} \bibentry{#1} \end{verse}}
\newcommand{\vs}{\vspace{.3in}}
\renewcommand{\ni}{\noindent}
\usepackage{xr-hyper}
\usepackage[]{hyperref}
\hypersetup{
    colorlinks=true,
    linkcolor=blue,
    filecolor=magenta,      
    urlcolor=cyan,
}
 
\urlstyle{same}
\usepackage[capposition=top]{floatrow}
\usepackage{amssymb}
\usepackage{relsize}
\usepackage[dvipsnames]{xcolor}
\usepackage{fancyhdr}
\usepackage{tikz}
 
\pagestyle{fancy} % customize header and footer
\fancyhf{} % clear initial header and footer
%\rhead{Overleaf}
\lhead{\centering \rightmark} % this adds subsection number and name
\lfoot{\centering \rightmark} 
\rfoot{\thepage} % put page number (the centering command puts it in the middle, don't matter if you put it in right or left footer)

\def \myFigPath {../figures/} 
% BE CAREFUL WITH FIGNAMES, IN LATEX THEY'RE NOT CASE SENSITIVE!!
\def \myTablePath {../tables/} 

%\definecolor{mygreen}{RGB}{0, 100, 0}
\definecolor{mygreen}{RGB}{0, 128, 0}

\definecolor{citec}{rgb}{0,0,.5}
\definecolor{linkc}{rgb}{0,0,.6}
\definecolor{bcolor}{rgb}{1,1,1}
\hypersetup{
%hidelinks = true
  colorlinks = true,
  urlcolor=linkc,
  linkcolor=linkc,
  citecolor = citec,
  filecolor = linkc,
  pdfauthor={Laura G\'ati},
}


\geometry{left=.83in,right=.89in,top=1in,
bottom=1in}
\linespread{1.5}
\renewcommand{\[}{\begin{equation}}
\renewcommand{\]}{\end{equation}}

% New Options
\newtheorem{prop}{Proposition}
\newtheorem{definition}{Definition}[section]
\newtheorem*{remark}{Remark}
\newtheorem{lemma}{Lemma}
\newtheorem{corollary}{Corollary}
\newtheorem{conjecture}{Conjecture}

%\newtheorem{theorem}{Theorem}[section] % the third argument specifies that their number will be adopted to the section
%\newtheorem{corollary}{Corollary}[theorem]
%\newtheorem{lemma}[theorem]{Lemma}
%\declaretheorem{proposition}
%\linespread{1.3}
%\raggedbottom
%\font\reali=msbm10 at 12pt

% New Commands
\newcommand{\real}{\hbox{\reali R}}
\newcommand{\realp}{\hbox{\reali R}_{\scriptscriptstyle +}}
\newcommand{\realpp}{\hbox{\reali R}_{\scriptscriptstyle ++}}
\newcommand{\R}{\mathbb{R}}
\DeclareMathOperator{\E}{\mathbb{E}}
\DeclareMathOperator{\argmin}{arg\,min}
\newcommand\w{3.0in}
\newcommand\wnum{3.0}
\def\myFigWidth{5.3in}
\def\mySmallerFigWidth{2.1in}
\def\myEvenBiggerFigScale{0.8}
\def\myPointSixFigScale{0.6}
\def\myBiggerFigScale{0.4}
\def\myFigScale{0.3}
\def\myMediumFigScale{0.25}
\def\mySmallFigScale{0.22}
\def\mySmallerFigScale{0.18}
\def\myTinyFigScale{0.16}
\def\myPointFourteenFigScale{0.14}
\def\myTinierFigScale{0.12}
\def\myAdjustableFigScale{0.18}
\newcommand\numberthis{\addtocounter{equation}{1}\tag{\theequation}} % this defines a command to make align only number this line
\newcommand{\code}[1]{\texttt{#1}} %code %

\renewcommand*\contentsname{Overview}
\setcounter{tocdepth}{2}

% define a command to make a huge question mark (it works in math mode)
\newcommand{\bigqm}[1][1]{\text{\larger[#1]{\textbf{?}}}}

\begin{document}

\linespread{1.0}

\title{Materials 10 - Is overshooting endemic to constant gain learning?}
\author{Laura G\'ati} 
\date{\today}
\maketitle

%%%%%%%%%%%%%%%%%%%%             DOCUMENT           %%%%%%%%%%%%%%%%%% 

\tableofcontents

%\listoffigures

%\newpage


\newpage
\section{Model summary}
\begin{align}
x_t &=  -\sigma i_t +\hat{\E}_t \sum_{T=t}^{\infty} \beta^{T-t }\big( (1-\beta)x_{T+1} - \sigma(\beta i_{T+1} - \pi_{T+1}) +\sigma r_T^n \big)  \label{prestons18}  \\
\pi_t &= \kappa x_t +\hat{\E}_t \sum_{T=t}^{\infty} (\alpha\beta)^{T-t }\big( \kappa \alpha \beta x_{T+1} + (1-\alpha)\beta \pi_{T+1} + u_T\big) \label{prestons19}  \\
i_t &= \psi_{\pi}\pi_t + \psi_{x} x_t  + \rho i_{t-1} + \bar{i}_t \label{TR}
\end{align}
 I consider two variations of the learning rule. The first is a ``mean-only'' rule:
\begin{align}
\hat{\E}_t z_{t+h} & =  \begin{bmatrix}\bar{\pi}_{t-1} \\ 0 \\ 0 \end{bmatrix}+ bh_x^{h-1}s_t  \quad \forall h\geq 1 \quad b = g_x \; h_x ,\quad \quad  \text{PLM1} \label{PLM1} \\
 \text{but the first row of $b$ is } \quad b_1 & = \begin{bmatrix} 0& 0&0 \end{bmatrix}
 \end{align}
\begin{equation}
\bar{\pi}_{t} = \bar{\pi}_{t-1} +k_t^{-1}\underbrace{\big(\pi_{t} - \bar{\pi}_{t-1} \big)}_{\text{fcst error using (\ref{PLM1})} }  
\end{equation}
The second is a ``learning the slope too'' rule:
\begin{align}
\hat{\E}_t z_{t+h} & =  \begin{bmatrix}\bar{\pi}_{t-1} \\ 0 \\ 0 \end{bmatrix}+ b_{t-1}h_x^{h-1}s_t  \quad \forall h\geq 1 \quad b = g_x \; h_x ,\quad \quad  \text{PLM2} \label{PLM2} \\
 \text{but the first row of $b$ is } &b_{1,t} \; \text{and is also learned. Let} \; \phi_t = \begin{bmatrix} \bar{\pi}_{t}& b_{1,t} 
 \end{bmatrix}
 \end{align}
\begin{equation}
\phi_{t} = \bigg( \phi_{t-1}' +k_t^{-1}\underbrace{\big(\pi_{t} - \phi_{t-1}\begin{bmatrix} 1 \\ s_{t-1} \end{bmatrix}
 \big)}_{\text{fcst error using (\ref{PLM2})} }  \bigg)'
\end{equation}
% \begin{align*}
%k_t & = \mathbb{I}\times(k_{t-1}+1) + (1-\mathbb{I}) \times \bar{g}^{-1}  \label{gain} \numberthis\\
%\mathbb{I} & = \begin{cases} 1 \quad \text{if} \; \theta_t \leq \bar{\theta}  \\ 0 \quad \text{otherwise.}\numberthis
%\end{cases} \\
%\theta_t & = |\hat{\E}_{t-1}\pi_t - \E_{t-1}\pi_t| / \sigma_s \quad \quad \text{CEMP criterion for the gain}\label{criterion}\numberthis
%\end{align*}

%The alternative criterion for the choice of gain is a recursive variant of the CUSUM-test (Brown, Durbin, Evans 1975):
%\begin{enumerate}
%\item Let $FE_t$ denote the short-run forecast error, and $\omega_t$ firms' estimate of the FE variance.
%\item Let $\kappa \in (0,1)$ and $\tilde{\theta}$ be the new threshold value for the criterion.
%\item Then for initial $(\omega_0, \theta_0)$, firms in every period estimate the criterion and the FEV as:
%\begin{align}
% \omega_t & =  \omega_{t-1} + \kappa k_{t-1}^{-1}(FE_t^2 -\omega_{t-1})\\
%\theta_t & =  \theta_{t-1} + \kappa k_{t-1}^{-1}(FE_t^2/\omega_t -\theta_{t-1})\\
%k_t & = \mathbb{I}\times(k_{t-1}+1) + (1-\mathbb{I}) \times \bar{g}^{-1} \\
%\mathbb{I} & = 1 \quad \text{if} \quad  \theta_t \leq \tilde{\theta}
%\end{align}
%\end{enumerate}

\newpage
\section{Compact notation}
 \begin{align}
z_t & = A_p^{RE} \E_t z_{t+1} + A_s^{RE} s_t \label{LOM_RE} \\
z_t & = A_a^{LH} f_a(t) + A_b^{LH} f_b(t) + A_s^{LH} s_t \label{LOM_LH} \\
s_t & = P s_{t-1} + \epsilon_t \quad \quad \textcolor{blue}{\rightarrow} \quad s'_t  = hx \; s'_{t-1} + \epsilon'_t \label{exog} \\
 \quad \text{where} \quad 
 s'_t & \equiv \begin{pmatrix} r_t^n \\ \bar{i}_t \\ u_t \\ \textcolor{blue}{i_{t-1}}
 \end{pmatrix} \quad 
 hx  \equiv \begin{pmatrix} \rho_r & 0 & 0 & \textcolor{blue}{0} \\ 0& \rho_i & 0 & \textcolor{blue}{0} \\ 0&0& \rho_u & \textcolor{blue}{0}  \\ 
 \textcolor{blue}{gx_{3,1}}&\textcolor{blue}{gx_{3,2}}& \textcolor{blue}{gx_{3,3}} & \textcolor{blue}{gx_{3,4}}
 \end{pmatrix}  \quad 
 \epsilon'_t \equiv \begin{pmatrix}\varepsilon_t^{r} \\ \varepsilon_t^{i}  \\ \varepsilon_t^{u} \\ \textcolor{blue}{0} 
 \end{pmatrix}  \quad  \text{and } \quad \Sigma'  =  \begin{pmatrix} \sigma_r & 0 & 0 & \textcolor{blue}{0} \\ 0& \sigma_i & 0 & \textcolor{blue}{0}  \\ 0&0& \sigma_u & \textcolor{blue}{0}  \\ \textcolor{blue}{0}  & \textcolor{blue}{0} & \textcolor{blue}{0} & \textcolor{blue}{0} &
 \end{pmatrix} 
\end{align}

And the $A^{RE}_s$ and $A^{LH}_s$ are given by:
\begin{align}
A_s^{RE} &= \begin{pmatrix}   \frac{\kappa\sigma}{w}  &-\frac{\kappa\sigma}{w}  & 1-\frac{\kappa\sigma\psi_{\pi}}{w} & \textcolor{blue}{0}\\
 \frac{ \sigma}{w} &  -\frac{\sigma}{w} & -\frac{\sigma\psi_{\pi}}{w} & \textcolor{blue}{0}\\ 
 \psi_x( \frac{\sigma}{w}) + \psi_{\pi}( \frac{\kappa\sigma}{w}) & \psi_x(- \frac{\sigma}{w}) + \psi_{\pi}(- \frac{\kappa\sigma}{w}) +1 &  \psi_x(-\frac{\sigma\psi_{\pi}}{w}) + \psi_{\pi}( 1-\frac{\kappa\sigma\psi_{\pi}}{w}) & \textcolor{blue}{\rho}\end{pmatrix}  
\\
 A_s^{LH} & = \begin{pmatrix} g_{\pi s} \\ g_{x s} \\ \psi_{\pi}g_{\pi s} + \psi_xg_{x s} + \begin{bmatrix} 0 & 1& 0 & \textcolor{blue}{\rho}\end{bmatrix}
\end{pmatrix} \\
g_{\pi s} & = (1-\frac{\kappa\sigma\psi_{\pi}}{w} )\begin{bmatrix} 0&0&1 & \textcolor{blue}{0}\end{bmatrix} (I_{\textcolor{blue}{4}} - \alpha\beta \textcolor{blue}{hx})^{-1} -\frac{\kappa\sigma}{w}\begin{bmatrix} -1&1&0 & \textcolor{blue}{\rho} \end{bmatrix} (I_{\textcolor{blue}{4}} -\beta \textcolor{blue}{hx})^{-1}\\
g_{x s} & =  \frac{-\sigma\psi_{\pi}}{w} \begin{bmatrix} 0&0&1& \textcolor{blue}{0} \end{bmatrix}(I_{\textcolor{blue}{4}} - \alpha\beta \textcolor{blue}{hx})^{-1}  -\frac{\sigma}{w}\begin{bmatrix} -1&1&0 & \textcolor{blue}{\rho}\end{bmatrix}(I_{\textcolor{blue}{4}} -\beta \textcolor{blue}{hx})^{-1}
\end{align}

And long-horizon expectations are
\begin{align}
f_a(t)  \equiv  \hat{\E}_t\sum_{T=t}^{\infty} (\alpha\beta)^{T-t } \begin{bmatrix} \pi_{T+1} \\ x_{T+1} \\ i_{T+1} \end{bmatrix} \quad \quad \quad \quad f_b(t)  \equiv \hat{\E}_t\sum_{T=t}^{\infty} (\beta)^{T-t } \begin{bmatrix} \pi_{T+1} \\ x_{T+1} \\ i_{T+1} \end{bmatrix}\label{fafb}
\end{align}
\begin{equation}
f_a(t) = \frac{1}{1-\alpha\beta}\begin{bmatrix} \bar{\pi}_t \\ 0 \\ 0 \end{bmatrix} + b(I_4 - \alpha\beta h_x)^{-1}s_t \quad \quad \quad f_b(t) = \frac{1}{1-\beta}\begin{bmatrix} \bar{\pi}_t \\ 0 \\ 0 \end{bmatrix}  + b(I_4 - \beta h_x)^{-1}s_t  \label{fafb_analytical}
\end{equation}

\clearpage
\section{Recap of timing}

Define some objects: \emph{(I usually let t denote the time in which the variable is formed.)}
\begin{align}
f^j_t &= \hat{\E}_t(z_{t+1}) \quad \text{one-period-ahead forecast formed at time } t, \; j=m,e \; \text{(morning or evening)} \\
FE_t & = z_{t+1}-f_t \quad \text{one-period-ahead forecast error realized at time } t+1 \\
& = ALM(t+1) - PLM(t) \\
\theta_t &=  \hat{\E}_{t-1}(z_{t}) - \E_{t-1}(z_{t}) \quad \text{CEMP's criterion} \\
& = PLM(t-1) - \E_{t-1}ALM(t)
\end{align}


\begin{enumerate}
\item[] $PLM(t): \hat{\E}_t z_{t+1} = \bar{z}_{\color{red}{t-1}} + bs_t$
\item[] \textbf{Morning}: morning of time $t$ available: $\mathcal{I}_t^m = \{\bar{z}_{\color{red}{t-1}}, s_t, k_{t-1}, FE_{t-2}\}$
\item Form all future expectations using $PLM(t) \text{(morning forecast)} \rightarrow z_t$ realized, $\rightarrow FE_{t-1}$ realized 
\item Form $\theta_t \rightarrow k_t$ realized
\item \textbf{Evening}: Update $\bar{z}_t = \bar{z}_{t-1} + k_t^{-1}(FE^e_{t-1})$
\item[] where $FE^e_{t-1} = z_{t}-f^e_{t-1} = z_{t}-(\bar{z}_{\color{red}{t-1}} + bs_{t-1})$ is the most recent realized FE, so:
\item[] $\bar{z}_t = \bar{z}_{t-1} + k_t^{-1}(z_{t}-(\bar{z}_{\color{red}{t-1}} + bs_{t-1}))$
\item[] $\rightarrow$ evening of time $t$ available: $\mathcal{I}_t^e = \{\bar{z}_{\color{red}{t}}, s_t, k_{t}, FE_{t-1}\}$
\end{enumerate}

\clearpage
\section{Current set of baseline parameters}
%\begin{center}
\begin{tabular}{ c | c  | l | l}
 $\beta$ & 0.99 & stochastic discount factor & standard (Woodford 2003/2011) \\  \hline
 $\sigma$ & 1  & IES & consistent with balanced growth \\  \hline
 $\alpha$ & 0.5 &  Calvo probability of not adjusting  & match 6-month duration of prices (can increase to 0.75)\\\hline
 $\psi_{\pi} $& 1.5  & coefficient of inflation in Taylor rule & Taylor \\\hline
 $\psi_x$ & 0   & coefficient of output gap in Taylor rule  & focus on $\pi$\\\hline
 $\bar{g}$ & $0.145$  & value of the constant gain & CEMP \\\hline
& & \\ [-1em] % this adds an extra empty row, and decreases its size, so it looks as if thetbar's row was higher
 $\bar{\theta}$ &  1  & threshold deviation between $\hat{\E}$ \& $\E$ & CEMP: 0.029\\ \hline
    $\rho_r$ & 0 &   persistence of natural rate shock & n.a. \\ \hline
    $\rho_i$ & 0.6 &  persistence of monetary policy shock & CEMP: 0.877  (can increase to 0.78 if $\alpha=0.75$) \\ \hline
    $\rho_u$ & 0  &  persistence of cost-push shock  & CEMP\\ \hline
    $\sigma_r$ & 0.1 & standard deviation of natural rate shock & n.a. \\ \hline
    $\sigma_i$ &  0.359  &standard deviation of mon. policy shock & CEMP \\ \hline
    $\sigma_u$ & 0.277 & standard deviation of cost-push shock & CEMP  \\ \hline
    $\theta$ & 10 & price elasticity of demand &Woodford 2003/2011, Chari, Kehoe \& McGrattan 2000  \\ \hline
    $\omega$ & 1.25 & elasticity of marginal cost to output &Woodford 2003/2011,  Chari, Kehoe \& McGrattan 2000  \\ \hline
\end{tabular}
%\end{center}

\newpage
\section{Cross-sectional IRFs, mon. pol shock only, cgain \& dgain only, ``mean-only'' PLM $\blacktriangleleft$}	

\begin{figure}[h!]
\subfigure[Decreasing gain, $t=25$]{\includegraphics[scale=\myAdjustableFigScale]{\myFigPath materials10_RIR_y_d_monpolmean_only_PLM_rho0_psi_pi_1_5_sig_1_dt_25}}
\subfigure[Constant gain, $t=25$]{\includegraphics[scale=\myAdjustableFigScale]{\myFigPath materials10_RIR_y_c_monpolmean_only_PLM_rho0_psi_pi_1_5_sig_1_dt_25}}
\caption{IRF for observables, shock imposed at $t$}
\end{figure}

\begin{figure}[h!]
\subfigure[Forecasts, $t=25$]{\includegraphics[scale=\myAdjustableFigScale]{\myFigPath materials10_RIR_F_both_monpolmean_only_PLM_rho0_psi_pi_1_5_sig_1_dt_25}}
\subfigure[Forecast errors, $t=25$]{\includegraphics[scale=\myAdjustableFigScale]{\myFigPath materials10_RIR_FE_both_monpolmean_only_PLM_rho0_psi_pi_1_5_sig_1_dt_25}}
\caption{IRF for 1-period ahead forecasts and FEs, together, morning and evening, shock imposed at $t$}
\end{figure}

\begin{figure}[h!]
\subfigure[Decreasing gain, $t=25$]{\includegraphics[scale=\myAdjustableFigScale]{\myFigPath materials10_RIR_fafb_d_monpolmean_only_PLM_rho0_psi_pi_1_5_sig_1_dt_25}}
\subfigure[Constant gain, $t=25$]{\includegraphics[scale=\myAdjustableFigScale]{\myFigPath materials10_RIR_fafb_c_monpolmean_only_PLM_rho0_psi_pi_1_5_sig_1_dt_25}}
\caption{IRF for LH forecasts, shock imposed at $t$}
\end{figure}

\newpage
\section{Cross-sectional IRFs, mon. pol shock only, cgain \& dgain only, ``slope and constant'' PLM $\blacktriangleleft$}	

\begin{figure}[h!]
\subfigure[Decreasing gain, $t=25$]{\includegraphics[scale=\myAdjustableFigScale]{\myFigPath materials10_RIR_y_d_monpolslope_and_constant_rho0_psi_pi_1_5_sig_1_dt_25}}
\subfigure[Constant gain, $t=25$]{\includegraphics[scale=\myAdjustableFigScale]{\myFigPath materials10_RIR_y_c_monpolslope_and_constant_rho0_psi_pi_1_5_sig_1_dt_25}}
\caption{IRF for observables, shock imposed at $t$}
\end{figure}

\begin{figure}[h!]
\subfigure[Forecasts, $t=25$]{\includegraphics[scale=\myAdjustableFigScale]{\myFigPath materials10_RIR_F_both_monpolslope_and_constant_rho0_psi_pi_1_5_sig_1_dt_25}}
\subfigure[Forecast errors, $t=25$]{\includegraphics[scale=\myAdjustableFigScale]{\myFigPath materials10_RIR_FE_both_monpolslope_and_constant_rho0_psi_pi_1_5_sig_1_dt_25}}
\caption{IRF for 1-period ahead forecasts and FEs, together, morning and evening, shock imposed at $t$}
\end{figure}

\begin{figure}[h!]
\subfigure[Decreasing gain, $t=25$]{\includegraphics[scale=\myAdjustableFigScale]{\myFigPath materials10_RIR_fafb_d_monpolslope_and_constant_rho0_psi_pi_1_5_sig_1_dt_25}}
\subfigure[Constant gain, $t=25$]{\includegraphics[scale=\myAdjustableFigScale]{\myFigPath materials10_RIR_fafb_c_monpolslope_and_constant_rho0_psi_pi_1_5_sig_1_dt_25}}
\caption{IRF for LH forecasts, shock imposed at $t$}
\end{figure}

\begin{itemize}
\item is almost identical to constant-only learning b/c 1) they're only learning the slope of inflation 2) $f_a, f_b$ are still driven mainly by the constant.
\end{itemize}

\newpage
\section{Choosing $\bar{g}$ to minimize the FEV $\blacktriangleleft$}
Currently what I do is:
\begin{itemize}
\item For each simulated sequence of shocks $n$
\item calculate the FEV across time for that particular history as a function of $\bar{g}$
\item choose $\bar{g}^*_n$ to minimize the FEV
\item take an average across the simulations
\end{itemize}
For 500 simulated sequences, I obtain an average $\bar{g}^* = 2.5715 \times 10^{-4} \approx 0.0003$ . The maximum value for $\bar{g}^*_n$ is 0.0049, and I constrain $\bar{g}^*_n$ to lie between $[ 0.00001, 0.2]$. (Somewhat troubling is that without the constraint, $\bar{g}^*_n$ is often negative, although small.)

With a $\bar{g}^* = 0.0003$, the overshooting is completely killed and decreasing and constant gain learning look identical. FEs still switch sign one time, but barely because, after impact, they are extremely small. 

A note: Ideally, I wanted to take the FEV over the cross-section instead of across time. This is however much more computationally intensive because it requires for each history $n$ and each time period $t$ to resimulate all sequences $1,\dots, N$ up to time $t$ to compute the most recent, period $t$ FEV across the cross-section, and then for every proposed value of $\bar{g}$, to repeat this process until $\bar{g}^*_t$ is found. Thus I expect this to take at least $T$ times longer than the first approach (and likely more because fmincon will also take at least $N$ times longer). Considering that the ``across-time'' approach takes a little above 5 minutes to run, this second, ``cross-section'' approach would need \emph{at least} 50 minutes for a modest simulation length of 100. 

\newpage
\section{How observables respond to expectations - connecting RE and learning}

For this section, disregard the difference in the expectation operator in the two models. Pretend like it was the same operator. With this assumption, the RE model is just a recursive formulation of the learning model. My aim here is to show that both ways of writing the \emph{same} system embody the same channels of how observables respond to expectations, only these channels are more explicit in the non-recursive formulation.

Ignoring shocks and setting $\psi_x = 0$, so the Taylor rule is just $i_t =  \psi_{\pi}\pi_t$, the two systems are
\begin{align*}
RE& \\
x_t &= - \sigma i_t + \E_t x_{t+1}   + \sigma \E_t \pi_{t+1}  \\
\pi_t &= \kappa x_t +\beta \E_t \pi_{t+1} \\
Learning & \\
x_t &=  -\sigma i_t +\hat{\E}_t \sum_{T=t}^{\infty} \beta^{T-t }\big( (1-\beta)x_{T+1} - \sigma\beta i_{T+1} +\sigma \pi_{T+1}) \big)  \\
\pi_t &= \kappa x_t +\hat{\E}_t \sum_{T=t}^{\infty} (\alpha\beta)^{T-t }\big( \kappa \alpha \beta x_{T+1} + (1-\alpha)\beta \pi_{T+1} \big)
\end{align*}
When you plug in the interest rate, you see that the recursive representation hides the contractionary effect of future positive values on current $x_t$ (and thereby on inflation) coming from the anticipated interest rate effect. (Throughout I'm using blue to denote negative values.)
\begin{align*}
RE& \\
x_t &= \textcolor{blue}{- \sigma \psi_{\pi}} \pi_t+  \E_t x_{t+1}   + \sigma \E_t \pi_{t+1}  \\
\pi_t &= \kappa x_t +\beta \E_t \pi_{t+1} \\
Learning & \\
x_t &= \textcolor{blue}{ -\sigma \psi_{\pi}} \pi_t +\hat{\E}_t \sum_{T=t}^{\infty} \beta^{T-t }\big( (1-\beta) x_{T+1} + \textcolor{blue}{\sigma(1- \beta \psi_{\pi})}\pi_{T+1} \big)  \\
\pi_t &= \kappa x_t +\hat{\E}_t \sum_{T=t}^{\infty} (\alpha\beta)^{T-t }\big( \kappa \alpha \beta x_{T+1} + (1-\alpha)\beta \pi_{T+1} \big)
\end{align*}
This gives
\begin{align*}
RE& \\
x_t &=  \textcolor{blue}{ \frac{\sigma(1-\beta\psi_{\pi})}{1+\sigma\psi_{\pi}\kappa}}\E_t \pi_{t+1}   +   \frac{1}{1+\sigma\psi_{\pi}\kappa}\E_t x_{t+1}  \\
\pi_t &=   \overbrace{\bigg( \textcolor{blue}{\frac{\kappa\sigma(1-\beta\psi_{\pi})}{1+\sigma\psi_{\pi}\kappa}}+\beta\bigg)}^{+}\E_t \pi_{t+1} +   \frac{\kappa}{1+\sigma\psi_{\pi}\kappa}\E_t x_{t+1} \\
Learning & \\
x_t & =  \bigg( \textcolor{blue}{-\frac{\sigma\psi_{\pi}}{1+\sigma\psi_{\pi}\kappa}(1-\alpha)\beta + \frac{\sigma(1-\beta\psi_{\pi})}{1+\sigma\psi_{\pi}\kappa}}\bigg) \E^{\alpha,\beta}_t \pi_{\infty} 
+\overbrace{\bigg( \textcolor{blue}{-\frac{\sigma\psi_{\pi}}{1+\sigma\psi_{\pi}\kappa}\kappa\alpha\beta} + \frac{1-\beta}{1+\sigma\psi_{\pi}\kappa}\bigg)}^{\textcolor{blue}{-}} \E^{\alpha,\beta}_t x_{\infty} \\
\pi_t & =  \overbrace{\bigg( \big(1-\frac{\kappa\sigma\psi_{\pi}}{1+\sigma\psi_{\pi}\kappa} \big)(1-\alpha)\beta + \textcolor{blue}{\frac{\kappa\sigma(1-\beta\psi_{\pi})}{1+\sigma\psi_{\pi}\kappa}}\bigg)}^{+}  \E^{\alpha,\beta}_t \pi_{\infty} 
+\bigg(   \big(1-\frac{\kappa\sigma\psi_{\pi}}{1+\sigma\psi_{\pi}\kappa} \big) \kappa\alpha\beta + \frac{\kappa(1-\beta)}{1+\sigma\psi_{\pi}\kappa}\bigg)\E^{\alpha,\beta}_t x_{\infty} 
\end{align*}
where I write $\E^{\alpha,\beta}_t x_{\infty}$ for $\E_t\sum_{T=t}^{\infty} (\alpha\beta)^{T-t}x_{T+1} + \E_t\sum_{T=t}^{\infty} (\beta)^{T-t} x_{T+1}$

\newpage
\subsection{A more concise rephrasing $\blacktriangleleft$}
Ignoring shocks and setting $\psi_x = 0$, so the Taylor rule is just $i_t =  \psi_{\pi}\pi_t$, the two systems are
(throughout I'm using blue to denote negative values).
\begin{align*}
RE& \\
x_t &= \textcolor{blue}{- \sigma \psi_{\pi}} \pi_t+  \E_t x_{t+1}   + \sigma \E_t \pi_{t+1}  \\
\pi_t &= \kappa x_t +\beta \E_t \pi_{t+1} \\
Learning & \\
x_t &= \textcolor{blue}{ -\sigma \psi_{\pi}} \pi_t +\hat{\E}_t \sum_{T=t}^{\infty} \beta^{T-t }\big( (1-\beta) x_{T+1} + \textcolor{blue}{\sigma(1- \beta \psi_{\pi})}\pi_{T+1} \big)  \\
\pi_t &= \kappa x_t +\hat{\E}_t \sum_{T=t}^{\infty} (\alpha\beta)^{T-t }\big( \kappa \alpha \beta x_{T+1} + (1-\alpha)\beta \pi_{T+1} \big)
\end{align*}
Expressing $x, \pi$ as functions of expectations alone, this gives:
\begin{align*}
RE& \\
x_t &=  \textcolor{blue}{ \frac{\sigma(1-\beta\psi_{\pi})}{1+\sigma\psi_{\pi}\kappa}}\E_t \pi_{t+1}   +   \frac{1}{1+\sigma\psi_{\pi}\kappa}\E_t x_{t+1}  \\
\pi_t &=   \overbrace{\bigg( \textcolor{blue}{\frac{\kappa\sigma(1-\beta\psi_{\pi})}{1+\sigma\psi_{\pi}\kappa}}+\beta\bigg)}^{+}\E_t \pi_{t+1} +   \frac{\kappa}{1+\sigma\psi_{\pi}\kappa}\E_t x_{t+1} \\
Learning & \\
x_t & = \textcolor{blue}{\frac{-\sigma\psi_{\pi}}{w}} \begin{bmatrix} (1-\alpha)\beta&\kappa\alpha\beta & 0 \end{bmatrix} f_a + \frac{1}{w}\begin{bmatrix} \textcolor{blue}{\sigma(1-\beta\psi_{\pi})}&1-\beta & 0\end{bmatrix} f_b \\
\pi_t & = (1-\frac{\kappa\sigma\psi_{\pi}}{w}) \begin{bmatrix} (1-\alpha)\beta&\kappa\alpha\beta & 0 \end{bmatrix} f_a + \frac{\kappa}{w}\begin{bmatrix}  \textcolor{blue}{\sigma(1-\beta\psi_{\pi})}&1-\beta & 0\end{bmatrix} f_b
\end{align*}
This yields the stylized representation of how endogenous variables respond to expectations in the two formulations:

\vspace{-0.9cm}

\begin{multicols}{2}
\begin{align*}
& RE \\
x_t & = \overset{\textcolor{blue}{-}}{\E(\pi)} + \overset{+}{\E(x)} \\  
\pi_t & = \overset{+}{\E(\pi)} + \overset{+}{\E(x)} 
\end{align*}
\columnbreak

\vspace{0.1cm}

\begin{align*}
& Learning \\
x_t & = \overset{\textcolor{blue}{-}}{\E_a(\pi)}  + \overset{\textcolor{blue}{-}}{\E_b(\pi)}  
+ \overbrace{\overset{\textcolor{blue}{-}}{\E_a(x)}   + \overset{+}{\E_b(x)}}^{+ \; \text{since} \; f_a < f_b}   \\  
\pi_t & = \underbrace{\overset{+}{\E_a(\pi)} + \overset{\textcolor{blue}{-}}{\E_b(\pi)}}_{+ \; \text{since} \; \kappa \; \text{tiny}} 
+ \overset{+}{\E_a(x)} + \overset{+}{\E_b(x)}
\end{align*}
\end{multicols}

As a reminder, I restate the definition of the long-horizon expectations $f_a$ and $f_b$ from Equations (\ref{fafb}) and (\ref{fafb_analytical}):
\begin{align}
f_a(t)  \equiv  \hat{\E}_t\sum_{T=t}^{\infty} (\alpha\beta)^{T-t } \begin{bmatrix} \pi_{T+1} \\ x_{T+1} \\ i_{T+1} \end{bmatrix} \quad \quad \quad \quad f_b(t)  \equiv \hat{\E}_t\sum_{T=t}^{\infty} (\beta)^{T-t } \begin{bmatrix} \pi_{T+1} \\ x_{T+1} \\ i_{T+1} \end{bmatrix}
\end{align}
\begin{equation}
f_a(t) = \frac{1}{1-\alpha\beta}\begin{bmatrix} \bar{\pi}_t \\ 0 \\ 0 \end{bmatrix} + b(I_4 - \alpha\beta h_x)^{-1}s_t \quad \quad \quad f_b(t) = \frac{1}{1-\beta}\begin{bmatrix} \bar{\pi}_t \\ 0 \\ 0 \end{bmatrix}  + b(I_4 - \beta h_x)^{-1}s_t  \label{fafb_obvious}
\end{equation}

(And $b = g_x h_x$, where $h_x$ is the state transition matrix and $g_x$ is the observation matrix from the RE model solution.)

Looking at (\ref{fafb_obvious}), it's obvious that a) learning will show up in the intercept, as the second part will load the same on the shocks in each period; b) because of much heavier discounting of the intercept, $f_a$ will fluctuate a lot less than $f_b$. Therefore the effects coming from $f_b$ overweigh those from $f_a$, which is why the blue difference in the stylized representation goes away. 


%  A NOTE ON SHOCKS AND ANCHORING
%\newpage
%\section{A note on shocks and anchoring $\blacktriangleleft$} 
%\begin{figure}[h!]
%\includegraphics[scale = \myMediumFigScale]{\myFigPath materials8_gain_IRF_monpol_rho0_psi_pi_1_5}
%\caption{An IRF of the endogenous gain and drift for a monetary policy shock}
%\end{figure}
%
%Not just this figure, but also previous ones of the gain indicate that even if a shock knocks expectations out of being anchored, this doesn't lead to huge deviations in the mean path of the gain or the drift. Partly this happens because this particular shock only unanchors very few times (only once - increasing $\delta$ increases this number), which is why the mean becomes tiny and the CIs are so tight. Note also that it's tough to compute CIs for the inverse gain because when there's no difference between the gain with the shock $\delta$ and without (which is the vast majority of cases), you end up with a bunch of zeros. The inverse doesn't help either because it makes deviations smaller.



\end{document}





