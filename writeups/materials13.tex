\documentclass[11pt]{article}
\usepackage{amsmath, amsthm, amssymb,lscape, natbib}
\usepackage{mathtools}
\usepackage{subfigure}
\usepackage[font=footnotesize,labelfont=bf]{caption}
\usepackage{graphicx}
\usepackage{colortbl}
\usepackage{hhline}
\usepackage{multirow}
\usepackage{multicol}
\usepackage{setspace}
\usepackage[final]{pdfpages}
\usepackage[left=2.5cm,top=2.5cm,right=2.5cm, bottom=2.5cm]{geometry}
\usepackage{natbib} 
\usepackage{bibentry} 
\newcommand{\bibverse}[1]{\begin{verse} \bibentry{#1} \end{verse}}
\newcommand{\vs}{\vspace{.3in}}
\renewcommand{\ni}{\noindent}
\usepackage{xr-hyper}
\usepackage[]{hyperref}
\hypersetup{
    colorlinks=true,
    linkcolor=blue,
    filecolor=magenta,      
    urlcolor=cyan,
}
 
\urlstyle{same}
\usepackage[capposition=top]{floatrow}
\usepackage{amssymb}
\usepackage{relsize}
\usepackage[dvipsnames]{xcolor}
\usepackage{fancyhdr}
\usepackage{tikz}
 
\pagestyle{fancy} % customize header and footer
\fancyhf{} % clear initial header and footer
%\rhead{Overleaf}
\lhead{\centering \rightmark} % this adds subsection number and name
\lfoot{\centering \rightmark} 
\rfoot{\thepage} % put page number (the centering command puts it in the middle, don't matter if you put it in right or left footer)

\def \myFigPath {../figures/} 
% BE CAREFUL WITH FIGNAMES, IN LATEX THEY'RE NOT CASE SENSITIVE!!
\def \myTablePath {../tables/} 

%\definecolor{mygreen}{RGB}{0, 100, 0}
\definecolor{mygreen}{RGB}{0, 128, 0}

\definecolor{citec}{rgb}{0,0,.5}
\definecolor{linkc}{rgb}{0,0,.6}
\definecolor{bcolor}{rgb}{1,1,1}
\hypersetup{
%hidelinks = true
  colorlinks = true,
  urlcolor=linkc,
  linkcolor=linkc,
  citecolor = citec,
  filecolor = linkc,
  pdfauthor={Laura G\'ati},
}


\geometry{left=.83in,right=.89in,top=1in,
bottom=1in}
\linespread{1.5}
\renewcommand{\[}{\begin{equation}}
\renewcommand{\]}{\end{equation}}

% New Options
\newtheorem{prop}{Proposition}
\newtheorem{definition}{Definition}[section]
\newtheorem*{remark}{Remark}
\newtheorem{lemma}{Lemma}
\newtheorem{corollary}{Corollary}
\newtheorem{conjecture}{Conjecture}

%\newtheorem{theorem}{Theorem}[section] % the third argument specifies that their number will be adopted to the section
%\newtheorem{corollary}{Corollary}[theorem]
%\newtheorem{lemma}[theorem]{Lemma}
%\declaretheorem{proposition}
%\linespread{1.3}
%\raggedbottom
%\font\reali=msbm10 at 12pt

% New Commands
\newcommand{\real}{\hbox{\reali R}}
\newcommand{\realp}{\hbox{\reali R}_{\scriptscriptstyle +}}
\newcommand{\realpp}{\hbox{\reali R}_{\scriptscriptstyle ++}}
\newcommand{\R}{\mathbb{R}}
\DeclareMathOperator{\E}{\mathbb{E}}
\DeclareMathOperator{\argmin}{arg\,min}
\newcommand\w{3.0in}
\newcommand\wnum{3.0}
\def\myFigWidth{5.3in}
\def\mySmallerFigWidth{2.1in}
\def\myEvenBiggerFigScale{0.8}
\def\myPointSixFigScale{0.6}
\def\myBiggerFigScale{0.4}
\def\myFigScale{0.3}
\def\myMediumFigScale{0.25}
\def\mySmallFigScale{0.22}
\def\mySmallerFigScale{0.18}
\def\myTinyFigScale{0.16}
\def\myPointFourteenFigScale{0.14}
\def\myTinierFigScale{0.12}
\def\myAdjustableFigScale{0.14}
\newcommand\numberthis{\addtocounter{equation}{1}\tag{\theequation}} % this defines a command to make align only number this line
\newcommand{\code}[1]{\texttt{#1}} %code %

\renewcommand*\contentsname{Overview}
\setcounter{tocdepth}{2}

% define a command to make a huge question mark (it works in math mode)
\newcommand{\bigqm}[1][1]{\text{\larger[#1]{\textbf{?}}}}

\begin{document}

\linespread{1.0}

\title{Materials 13 - Still looking for a version of the model w/o overshooting}
\author{Laura G\'ati} 
\date{\today}
\maketitle

%%%%%%%%%%%%%%%%%%%%             DOCUMENT           %%%%%%%%%%%%%%%%%% 

\tableofcontents

%\listoffigures

%\newpage
\section{Model summary}
\begin{align}
x_t &=  -\sigma i_t +\hat{\E}_t \sum_{T=t}^{\infty} \beta^{T-t }\big( (1-\beta)x_{T+1} - \sigma(\beta i_{T+1} - \pi_{T+1}) +\sigma r_T^n \big)  \label{prestons18}  \\
\pi_t &= \kappa x_t +\hat{\E}_t \sum_{T=t}^{\infty} (\alpha\beta)^{T-t }\big( \kappa \alpha \beta x_{T+1} + (1-\alpha)\beta \pi_{T+1} + u_T\big) \label{prestons19}  \\
i_t &= \psi_{\pi}\pi_t + \psi_{x} x_t  + \bar{i}_t \label{TR}
\end{align}
\begin{equation}
\hat{\E}_t z_{t+h} =  \begin{bmatrix}\bar{\pi}_{t-1} \\ 0 \; \textcolor{red}{(\bar{x}_{t-1}) } \\ 0 \; \textcolor{red}{(\bar{i}_{t-1}) } \end{bmatrix}+ bh_x^{h-1}s_t  \quad \forall h\geq 1 \quad \quad b = g_x \; h_x \quad \quad \text{PLM} \label{PLM}
\end{equation}
\begin{equation}
\bar{\pi}_{t} = \bar{\pi}_{t-1} +k_t^{-1}\underbrace{\big(\pi_{t} -(\bar{\pi}_{t-1}+b_1s_{t-1}) \big)}_{\text{fcst error using (\ref{PLM})} } \quad \quad  \text{($b_1$ is the first row of $b$)}
\end{equation}
 \begin{align*}
k_t & = \begin{cases} k_{t-1}+1 \quad \text{for decreasing gain learning}  \\ \bar{g}^{-1}  \quad \text{for constant gain learning.}\numberthis
\end{cases} 
\end{align*}

\newpage
\section{To do}
\begin{enumerate}
\item Check $\psi_{\pi}$ above but close to 1 \\
$\rightarrow$ works but only quantitatively; qualitatively, the overshooting is still there, likely because this only cancels out one of the two channels through which $\E{\pi}$ affects $x_t$ negatively.
\item Fix shock for simulation \\
Indeed the issue was that for learning, I accidentally scaled down the shock by $\sigma_i < 1$, while for RE I had maintained $\sigma_i = 1$.
\item Interest rate smoothing as  $i_t = \rho i_{t-1} + (1-\rho)(\psi_{\pi}\pi_t + \psi_x x_t) + \bar{i}_t$ \\
Doesn't work either - it doesn't change the model except reduces $\psi_{\pi}$.
\item Indexation in NKPC
\begin{figure}[h!]
\includegraphics[scale=\mySmallFigScale]{\myFigPath command_IRFs_many_learning_RIR_LH_monpol_cgain_gbar_0_145_indexation_suboptimal_fcst_constant_only}
\caption{Baseline, indexation in NKPC with parameter $\gamma=0.128$, ``suboptimal forecasters'' info assumption}
\end{figure}\\
It works... \\
\begin{small}
\item Central bank's $\E\pi_{t+1}$ in TR?
\item Find optimal gain as analog to Kalman gain
\item Initialize beliefs away from RE somehow.
\item Davig \& Leeper-style switching Taylor rule where only generalized Taylor principle holds?

\end{small}
\item Learn $h_x$
\end{enumerate}

A quick question on projection facility: checking \texttt{eig(phi)} when $\phi$ isn't square?

\newpage
\section{Learning $h_x$}

%\begin{figure}[h!]
%\subfigure[Learning constant only]{\includegraphics[scale=\myAdjustableFigScale]{\myFigPath command_IRFs_many_learning_RIR_LH_monpol_cgain_gbar_0_145_true_baseline_dont_know_TR_constant_only}}
%\subfigure[Learning slope and constant]{\includegraphics[scale=\myAdjustableFigScale]{\myFigPath command_IRFs_many_learning_RIR_LH_monpol_cgain_gbar_0_145_true_baseline_dont_know_TR_slope_and_constant}}
%\caption{Baseline, agents don't know TR}
%\end{figure}

\end{document}





