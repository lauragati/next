\documentclass[11pt]{article}
\usepackage{amsmath, amsthm, amssymb,lscape, natbib}
\usepackage{mathtools}
\usepackage{subfigure}
\usepackage[font=footnotesize,labelfont=bf]{caption}
\usepackage{graphicx}
\usepackage{colortbl}
\usepackage{hhline}
\usepackage{multirow}
\usepackage{multicol}
\usepackage{setspace}
\usepackage[final]{pdfpages}
\usepackage[left=2.5cm,top=2.5cm,right=2.5cm, bottom=2.5cm]{geometry}
\usepackage{natbib} 
\usepackage{bibentry} 
\newcommand{\bibverse}[1]{\begin{verse} \bibentry{#1} \end{verse}}
\newcommand{\vs}{\vspace{.3in}}
\renewcommand{\ni}{\noindent}
\usepackage{xr-hyper}
\usepackage[]{hyperref}
\hypersetup{
    colorlinks=true,
    linkcolor=blue,
    filecolor=magenta,      
    urlcolor=cyan,
}
 
\urlstyle{same}
\usepackage[capposition=top]{floatrow}
\usepackage{amssymb}
\usepackage{relsize}
\usepackage[dvipsnames]{xcolor}
\usepackage{fancyhdr}
\usepackage{tikz}
 
\pagestyle{fancy} % customize header and footer
\fancyhf{} % clear initial header and footer
%\rhead{Overleaf}
\lhead{\centering \rightmark} % this adds subsection number and name
\lfoot{\centering \rightmark} 
\rfoot{\thepage} % put page number (the centering command puts it in the middle, don't matter if you put it in right or left footer)

\def \myFigPath {../figures/} 
% BE CAREFUL WITH FIGNAMES, IN LATEX THEY'RE NOT CASE SENSITIVE!!
\def \myTablePath {../tables/} 

%\definecolor{mygreen}{RGB}{0, 100, 0}
\definecolor{mygreen}{RGB}{0, 128, 0}

\definecolor{citec}{rgb}{0,0,.5}
\definecolor{linkc}{rgb}{0,0,.6}
\definecolor{bcolor}{rgb}{1,1,1}
\hypersetup{
%hidelinks = true
  colorlinks = true,
  urlcolor=linkc,
  linkcolor=linkc,
  citecolor = citec,
  filecolor = linkc,
  pdfauthor={Laura G\'ati},
}


\geometry{left=.83in,right=.89in,top=1in,
bottom=1in}
\linespread{1.5}
\renewcommand{\[}{\begin{equation}}
\renewcommand{\]}{\end{equation}}

% New Options
\newtheorem{prop}{Proposition}
\newtheorem{definition}{Definition}[section]
\newtheorem*{remark}{Remark}
\newtheorem{lemma}{Lemma}
\newtheorem{corollary}{Corollary}
\newtheorem{conjecture}{Conjecture}

%\newtheorem{theorem}{Theorem}[section] % the third argument specifies that their number will be adopted to the section
%\newtheorem{corollary}{Corollary}[theorem]
%\newtheorem{lemma}[theorem]{Lemma}
%\declaretheorem{proposition}
%\linespread{1.3}
%\raggedbottom
%\font\reali=msbm10 at 12pt

% New Commands
\newcommand{\real}{\hbox{\reali R}}
\newcommand{\realp}{\hbox{\reali R}_{\scriptscriptstyle +}}
\newcommand{\realpp}{\hbox{\reali R}_{\scriptscriptstyle ++}}
\newcommand{\R}{\mathbb{R}}
\DeclareMathOperator{\E}{\mathbb{E}}
\DeclareMathOperator{\argmin}{arg\,min}
\newcommand\w{3.0in}
\newcommand\wnum{3.0}
\def\myFigWidth{5.3in}
\def\mySmallerFigWidth{2.1in}
\def\myEvenBiggerFigScale{0.8}
\def\myPointSixFigScale{0.6}
\def\myBiggerFigScale{0.4}
\def\myFigScale{0.3}
\def\myMediumFigScale{0.25}
\def\mySmallFigScale{0.22}
\def\mySmallerFigScale{0.18}
\def\myTinyFigScale{0.16}
\def\myPointFourteenFigScale{0.14}
\def\myTinierFigScale{0.12}
\def\myAdjustableFigScale{0.14}
\newcommand\numberthis{\addtocounter{equation}{1}\tag{\theequation}} % this defines a command to make align only number this line
\newcommand{\code}[1]{\texttt{#1}} %code %

\renewcommand*\contentsname{Overview}
\setcounter{tocdepth}{2}

% define a command to make a huge question mark (it works in math mode)
\newcommand{\bigqm}[1][1]{\text{\larger[#1]{\textbf{?}}}}

\begin{document}

\linespread{1.0}

\title{Materials 17 - Analytical optimal monetary policy (a.k.a. making sense of Woodford and all those that cite him)}
\author{Laura G\'ati} 
\date{\today}
\maketitle

%%%%%%%%%%%%%%%%%%%%             DOCUMENT           %%%%%%%%%%%%%%%%%% 

\tableofcontents

%\listoffigures

\newpage
\section{Model summary}
\begin{align}
x_t &=  -\sigma i_t +\hat{\E}_t \sum_{T=t}^{\infty} \beta^{T-t }\big( (1-\beta)x_{T+1} - \sigma(\beta i_{T+1} - \pi_{T+1}) +\sigma r_T^n \big)  \label{prestons18}  \\
\pi_t &= \kappa x_t +\hat{\E}_t \sum_{T=t}^{\infty} (\alpha\beta)^{T-t }\big( \kappa \alpha \beta x_{T+1} + (1-\alpha)\beta \pi_{T+1} + u_T\big) \label{prestons19}  \\
i_t &= \psi_{\pi}\pi_t + \psi_{x} x_t  + \bar{i}_t \label{TR}
\end{align}
\begin{equation}
\hat{\E}_t z_{t+h} =  \bar{z}_{t-1} + bh_x^{h-1}s_t  \quad \forall h\geq 1 \quad \quad b = g_x\; h_x \quad \quad \text{PLM} \label{PLM}
\end{equation}
\begin{equation}
\bar{z}_{t} = \bar{z}_{t-1} +k_t^{-1}\underbrace{\big(z_{t} -(\bar{z}_{t-1}+bs_{t-1}) \big)}_{\text{fcst error using (\ref{PLM})} } 
\end{equation}
(Vector learning. For scalar learning, $\bar{z}= \begin{pmatrix} \bar{\pi} & 0 & 0\end{pmatrix}' $. I'm also not writing the case where the slope $b$ is also learned.)
 \begin{align*}
k_t & = \begin{cases} k_{t-1}+1 \quad \text{when} \quad \theta^{CEMP} < \bar{\theta} \quad \text{or}  \quad  \theta_t < \tilde{\theta}  \\ \bar{g}^{-1}  \quad \text{otherwise.}\numberthis
\end{cases} 
\end{align*}

\subsection{The CEMP vs. the CUSUM criterion}

CEMP's criterion  
\begin{equation}
\theta_t^{CEMP} = \max | \Sigma^{-1} ( \phi - \begin{bmatrix} F & G \end{bmatrix}) |
\end{equation}
where $\Sigma$ is the VC matrix of shocks, $\phi$ is the estimated matrix, $[F,G]$ is the ALM.

\noindent CUSUM-criterion
\begin{align}
\omega_t & =  \omega_{t-1} + \kappa k_{t-1}^{-1}(f_t f_t'  -\omega_{t-1})\\
\theta_t^{CUSUM} & =  \theta_{t-1} + \kappa k_{t-1}^{-1}(f_t'\omega_t^{-1}f_t -\theta_{t-1})
\end{align}

where $f$ is the most recent forecast error and $\omega$ is the estimated FEV. 

\newpage
\section{Optimal policy - dimensions of conceptual confusion}

\begin{enumerate}
\item Commitment vs. time-consistency \\
My understanding of Woodford is that commitment and time-consistency are two separate binaries. In words:
\begin{align*}
& \text{\emph{commitment}} = \begin{cases} \text{discretion:} \; \max \text{period utility, s.t. stuff taken as given}\\
\text{commitment:} \; \max \text{expected lifetime utility, s.t. model equations \& initial condition}
\end{cases} \\
& \text{\emph{time-consistency}} = \begin{cases} \text{time-inconsistent:} \; \text{initial condition is that multiplier = 0}\\
\text{time-consistent:} \; \max \text{initial condition is that endogenous stuff is at target initially}
\end{cases}
\end{align*}
This gives rise to 3 different optimality concepts:
\begin{align*}
& \begin{cases} \text{optimal discretionary policy} \; \rightarrow \text{time-inconsistent}\\
\text{optimal commitment ($t_0$-optimal)} \; \rightarrow \text{time-inconsistent} \\
\text{timelessly optimal commitment } \; \rightarrow \text{time-consistent} 
\end{cases}
\end{align*}
\item Constraints of the problem / endogenous variables of interest \\
Mainly model equations. But why doesn't Woodford take the IS-relation of the NK model as a constraint? He says: ``if there is no welfare loss resulting from nominal interest-rate variation, one may omit the constraint terms corresponding to the IS relation, as these constraints never bind." \\
$\rightarrow$ but what about fluctuations in the output gap? Gaspar et al 2011 at least claim to ignore the IS curve because they assume that the CB controls the output gap directly.\\
And what about expectations? It seems like Gaspar et al 2011 treat those the same as jumps. 
\item Optimal plan vs. implementation \\
My understanding is that solving the above policy problem just gives you the optimal plan: a time path for the endogenous variables that the policy maker wants to bring about. It's a separate question to ask how to implement that policy. 
\item Implementation of policy: ``simple rules," reaction functions and ``targeting rules" \\
For me it gets generally confusing at this point. 
\end{enumerate}


\end{document}





