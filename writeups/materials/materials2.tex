\documentclass[11pt]{article}
\usepackage{amsmath, amsthm, amssymb,lscape, natbib}
\usepackage{mathtools}
\usepackage{subfigure}
\usepackage[font=footnotesize,labelfont=bf]{caption}
\usepackage{graphicx}
\usepackage{colortbl}
\usepackage{hhline}
\usepackage{multirow}
\usepackage{multicol}
\usepackage{setspace}
\usepackage[final]{pdfpages}
\usepackage[left=2.5cm,top=2.5cm,right=2.5cm, bottom=2.5cm]{geometry}
\usepackage{natbib} 
\usepackage{bibentry} 
\newcommand{\bibverse}[1]{\begin{verse} \bibentry{#1} \end{verse}}
\newcommand{\vs}{\vspace{.3in}}
\renewcommand{\ni}{\noindent}
\usepackage{xr-hyper}
\usepackage[]{hyperref}
\usepackage[capposition=top]{floatrow}
\usepackage{amssymb}


\def \myFigPath {../figures/} 
% BE CAREFUL WITH FIGNAMES, IN LATEX THEY'RE NOT CASE SENSITIVE!!
\def \myTablePath {../tables/} 

\definecolor{citec}{rgb}{0,0,.5}
\definecolor{linkc}{rgb}{0,0,.6}
\definecolor{bcolor}{rgb}{1,1,1}
\hypersetup{
%hidelinks = true
  colorlinks = true,
  urlcolor=linkc,
  linkcolor=linkc,
  citecolor = citec,
  filecolor = linkc,
  pdfauthor={Laura G\'ati},
}


\geometry{left=.83in,right=.89in,top=1in,
bottom=1in}
\linespread{1.5}
\renewcommand{\[}{\begin{equation}}
\renewcommand{\]}{\end{equation}}

% New Options
\newtheorem{prop}{Proposition}
\newtheorem{definition}{Definition}[section]
\newtheorem*{remark}{Remark}
\newtheorem{lemma}{Lemma}
\newtheorem{corollary}{Corollary}
%\newtheorem{theorem}{Theorem}[section] % the third argument specifies that their number will be adopted to the section
%\newtheorem{corollary}{Corollary}[theorem]
%\newtheorem{lemma}[theorem]{Lemma}
%\declaretheorem{proposition}
%\linespread{1.3}
%\raggedbottom
%\font\reali=msbm10 at 12pt

% New Commands
\newcommand{\real}{\hbox{\reali R}}
\newcommand{\realp}{\hbox{\reali R}_{\scriptscriptstyle +}}
\newcommand{\realpp}{\hbox{\reali R}_{\scriptscriptstyle ++}}
\newcommand{\R}{\mathbb{R}}
\DeclareMathOperator{\E}{\mathbb{E}}
\DeclareMathOperator{\argmin}{arg\,min}
\newcommand\w{3.0in}
\newcommand\wnum{3.0}
\def\myFigWidth{5.3in}
\def\mySmallerFigWidth{2.1in}
\def\myEvenBiggerFigScale{0.8}
\def\myPointSixFigScale{0.6}
\def\myBiggerFigScale{0.4}
\def\myFigScale{0.3}
\def\mySmallFigScale{0.22}
\def\mySmallerFigScale{0.18}
\def\myTinyFigScale{0.16}
\def\myPointFourteenFigScale{0.14}
\def\myTinierFigScale{0.12}
\newcommand\numberthis{\addtocounter{equation}{1}\tag{\theequation}} % this defines a command to make align only number this line
\newcommand{\code}[1]{\texttt{#1}} %code %

\renewcommand*\contentsname{Overview}
\setcounter{tocdepth}{2}

\begin{document}

\linespread{1.0}

\title{Materials 2}
\author{Laura G\'ati} 
\date{\today}
\maketitle

%%%%%%%%%%%%%%%%%%%%             DOCUMENT           %%%%%%%%%%%%%%%%%% 

\tableofcontents

%\listoffigures


\section{A CEMP-Preston mix}

Suppose we have a NK model with LR forecasts being relevant, as in Preston (2005):
\begin{align}
x_t &=  -\sigma i_t +\hat{E}_t \sum_{T=t}^{\infty} \beta^{T-t }\big( (1-\beta)x_{T+1} - \sigma(\beta i_{T+1} - \pi_{T+1}) +\sigma r_T^n \big) \tag{Preston, eq. (18)} \\
\pi_t &= \kappa x_t +\hat{E}_t \sum_{T=t}^{\infty} (\alpha\beta)^{T-t }\big( \kappa \alpha \beta x_{T+1} + (1-\alpha)\beta \pi_{T+1} + u_T\big)\tag{Preston, eq. (19)} \\
i_t &= \psi_{\pi}\pi_t + \psi_{x} x_t + \bar{i}_t \tag{Preston, eq. (27)} 
\end{align}
 where I've 1) added $\sigma$ in front of $r_T^n,$ reflecting the derivation of the shock on the NKIS; 2) added $u_T$, a cost-push shock to the NKPC. 
 
 I'm assuming that the innovations can be summarized as:
 \begin{align}
 s_t & = P s_{t-1} + \epsilon_t \\
 \text{where} \quad 
 s_t & \equiv \begin{pmatrix} r_t^n \\ \bar{i}_t \\ u_t 
 \end{pmatrix} \quad 
 P  \equiv \begin{pmatrix} \rho_r & 0 & 0 \\ 0& \rho_i & 0 \\ 0&0& \rho_u 
 \end{pmatrix}  \quad 
 \epsilon_t \equiv \begin{pmatrix}\varepsilon_t^{r} \\ \varepsilon_t^{i}  \\ \varepsilon_t^{u} 
 \end{pmatrix}  \quad  \text{and } \quad \Sigma  =  \begin{pmatrix} \sigma_r & 0 & 0 \\ 0& \sigma_i & 0 \\ 0&0& \sigma_u 
 \end{pmatrix} 
 \end{align}
 Let $z_t$ summarize the endogenous variables as
 \begin{equation}
 z_t \equiv \begin{pmatrix} \pi_t \\ x_t \\ i_t
 \end{pmatrix}
 \end{equation}
 Then I can write system (1), (18), (19) and (27) compactly as
 \begin{align}
z_t & = A_1 f_a + A_2 f_b + A_3 s_t \label{RF} \\
s_t & = P s_{t-1} + \epsilon_t \label{exog}
\end{align}
 where $f_a$ and $f_b$ capture discounted long-run expectations of the endogenous states $z$, and matrices $A_1, A_2$ and $A_3$ gather coefficients as:
 \begin{align}
f_a & \equiv  \sum_{T=t}^{\infty} (\alpha\beta)^{T-t } z_{T+1}\\
f_b & \equiv \sum_{T=t}^{\infty} (\beta)^{T-t } z_{T+1}
\end{align}
 \begin{align}
A_1 & = \begin{pmatrix} g_{\pi a} \\ g_{x a} \\ \psi_{\pi}g_{\pi a} + \psi_xg_{x a}
\end{pmatrix}
\quad A_2 = \begin{pmatrix} g_{\pi b} \\ g_{x b} \\ \psi_{\pi}g_{\pi b} + \psi_xg_{x b}
\end{pmatrix}
 \quad A_3 = \begin{pmatrix} g_{\pi s} \\ g_{x s} \\ \psi_{\pi}g_{\pi s} + \psi_xg_{x s} + \begin{bmatrix} 0 & 1& 0\end{bmatrix}
\end{pmatrix} \\
g_{\pi a} & =(1-\frac{\kappa\sigma\psi_{\pi}}{w} )  \begin{bmatrix}(1-\alpha)\beta, \kappa\alpha\beta, 0 \end{bmatrix} \\
g_{x a} & =  \frac{-\sigma\psi_{\pi}}{w} \begin{bmatrix}(1-\alpha)\beta, \kappa\alpha\beta, 0 \end{bmatrix}\\
g_{\pi b} & = \frac{\kappa}{w} \begin{bmatrix}\sigma(1-\beta\psi_{\pi}), (1-\beta-\beta\sigma\psi_x, 0 \end{bmatrix}\\
g_{x b} & = \frac{1}{w} \begin{bmatrix}\sigma(1-\beta\psi_{\pi}), (1-\beta-\beta\sigma\psi_x, 0 \end{bmatrix} \\
g_{\pi s} & = (1-\frac{\kappa\sigma\psi_{\pi}}{w} )\begin{bmatrix} 0&0&1 \end{bmatrix} (I_3 - \alpha\beta P)^{-1} -\frac{\kappa\sigma}{w}\begin{bmatrix} -1&1&0 \end{bmatrix} (I_3 -\beta P)^{-1}\\
g_{x s} & =  \frac{-\sigma\psi_{\pi}}{w} \begin{bmatrix} 0&0&1 \end{bmatrix}(I_3 - \alpha\beta P)^{-1}  -\frac{\sigma}{w}\begin{bmatrix} -1&1&0 \end{bmatrix}(I_3 -\beta P)^{-1}\\
w & = 1+\sigma\psi_x +\kappa\sigma\psi_{\pi}
\end{align}


This is where the CEMP bit comes in: let agents form forecasts according to the relation
\begin{equation}
\hat{\E}_t z_{t+1} = \bar{z}_t + \underbrace{C}_{\text{TBD}}s_{t} + e_{t+1} \tag{PLM}
\end{equation}
where $\bar{z}_t$ is the LR expectation of all endogenous variables. CEMP would love if we called this the ``drift'' in beliefs. Let this drift evolve according to CEMP's criterion as:
\begin{align}
\bar{z}_t & = \bar{z}_{t-1} + k_t^{-1} f_{t-1}  \\
f_{t-1} & = z_{t-1} - \hat{\E}_{t-2}z_{t-1} \quad \text{(short-run forecast error)} \\
k_t & = \mathbb{I}(k_{t-1}+1) + (1-\mathbb{I})\bar{g}^{-1} \\
\mathbb{I} &= \begin{cases} 1 \quad \text{if} \quad \theta_t \leq \bar{\theta} \\
0 \quad \text{otherwise.}
\end{cases} \\
\text{where} \quad \theta_t & = |\hat{\E}_{t-1}z_t - \E_{t-1}z_t  | / (\sigma_r + \sigma_i + \sigma_u) \quad \text{(subjective - objective forecast)}
\end{align}
%$\theta_t$ measures the distance between subjective and objective, model-consistent forecasts, and is scaled by the variance of the noise in the model.
Anticipated utility: 
\begin{equation}
\hat{\E}_t \bar{z}_T = \bar{z}_t \; \forall \; T > t \Rightarrow \bar{z}_{T|t} = \bar{z}_t \; \forall \; T > t
\end{equation}


\subsection{Deriving the ALM}
To get the ALM, we need to write the expectations $f_a, f_b$ based on the PLM. Subbing in the PLM and using the anticipated utility assumption, I get
\begin{align}
f_a & = \frac{1}{1-\alpha\beta}\bar{z}_t + C(I_3-\alpha\beta P)^{-1}s_t \label{fa}\\
f_b & = \frac{1}{1-\beta}\bar{z}_t + C(I_3-\beta P)^{-1}s_t \label{fb}
\end{align}
Then the ALM is expression (\ref{RF}), with expectations evaluated using (\ref{fa}) and (\ref{fb}):
\begin{align}
z_t &= \underbrace{\bigg(A_1\frac{1}{1-\alpha\beta} +A_2\frac{1}{1-\beta}\bigg)}_{\equiv B_1}\bar{z}_t + \underbrace{\bigg(A_1C(I_3 - \alpha\beta P)^{-1} +A_2C(I_3 - \beta P)^{-1} +A_3\bigg)}_{\equiv B_2}s_t \tag{ALM} \\
z_t & = B_1 \bar{z}_t + B_2 s_t \label{ALM }
\end{align}
\subsection{SR forecast error and the criterion}
\begin{align*}
f_{t-1} & = z_{t-1} - \hat{\E}_{t-2}z_{t-1} \quad \text{(short-run forecast error: ALM - PLM)} \\
\theta_t & = |\hat{\E}_{t-1}z_t - \E_{t-1}z_t  | / (\sigma_r + \sigma_i + \sigma_u) \quad \text{(criterion: PLM - $\E_{t-1}$ALM)} \\
\Rightarrow \quad \quad \quad \quad \quad \quad  f_{t-1} & = B_1 \bar{z}_{t-1} - \bar{z}_{t-2} +B_2 s_{t-1} -Cs_{t-2}\\
(\sigma_r + \sigma_i + \sigma_u) \theta_t & = |(I_3 - B_1)\bar{z}_{t-1} + (I_3 - B_2P)s_{t-1}|
\end{align*}

\subsection{Model summary}
\begin{align}
z_t & = B_1 \bar{z}_t + B_2 s_t  \tag{ALM} \\
\bar{z}_t & = \bar{z}_{t-1} + k_t^{-1} f_{t-1} \tag{Drift LOM}\\
f_{t-1} & = B_1 \bar{z}_{t-1} - \bar{z}_{t-2} +B_2 s_{t-1} -Cs_{t-2} \tag{SR fcst error}\\ 
k_t & = \mathbf{f_k}(\bar{z}_{t-1}, k_{t-1}, s_{t-1})  \quad \text{where $\mathbf{f_k}$ evaluates the criterion $\theta_t$}\tag{Gain LOM} \\
(\sigma_r + \sigma_i + \sigma_u) \theta_t & = |(I_3 - B_1)\bar{z}_{t-1} + (I_3 - B_2P)s_{t-1}| \tag{criterion}\\
 \mathbf{f_k} &= \mathbb{I}_{\theta_t \leq \bar{\theta}}(k_{t-1}+1) + (1-\mathbb{I}_{\theta_t \leq \bar{\theta}})\bar{g}^{-1} \tag{anchoring}\\
s_t &= Ps_{t-1} +\epsilon_t \tag{exog. process}
\end{align}

%\newpage
\section{Two initial simulations}	
\begin{figure}[h!]
\subfigure[Observables 1]{
\includegraphics[scale = \mySmallerFigScale]{\myFigPath materials2_observables1}}
\subfigure[Nonlinear states 1]{
\includegraphics[scale = \mySmallerFigScale]{\myFigPath materials2_nonlin_states1}}
\subfigure[Observables 2]{
\includegraphics[scale = \mySmallerFigScale]{\myFigPath materials2_observables2}}
\subfigure[Nonlinear states 2]{
\includegraphics[scale = \mySmallerFigScale]{\myFigPath materials2_nonlin_states2}}
\end{figure}

Two (potentially connected) issues:
\begin{itemize}
\item $\bar{\theta}$ needs to be quite huge for gain to decrease (20 instead of CEMP's 0.029)
\item Nonlinear states? Based on CEMP's def of $\bar{\pi}$ being a nonlinear state because the gain is a nonlinear function of it, here $\bar{z}$ and $s$ are both nonlinear states.
\end{itemize}


\newpage





 
 
\end{document}



