\documentclass[11pt]{article}
\usepackage{amsmath, amsthm, amssymb,lscape, natbib}
\usepackage{mathtools}
\usepackage{subfigure}
\usepackage[font=footnotesize,labelfont=bf]{caption}
\usepackage{graphicx}
\usepackage{colortbl}
\usepackage{hhline}
\usepackage{multirow}
\usepackage{multicol}
\usepackage{setspace}
\usepackage[final]{pdfpages}
\usepackage[left=2.5cm,top=2.5cm,right=2.5cm, bottom=2.5cm]{geometry}
\usepackage{natbib} 
\usepackage{bibentry} 
\newcommand{\bibverse}[1]{\begin{verse} \bibentry{#1} \end{verse}}
\newcommand{\vs}{\vspace{.3in}}
\renewcommand{\ni}{\noindent}
\usepackage{xr-hyper}
\usepackage[]{hyperref}
\usepackage[capposition=top]{floatrow}
\usepackage{amssymb}
\usepackage{relsize}
\usepackage[dvipsnames]{xcolor}
\usepackage{fancyhdr}
\usepackage{tikz}
 
\pagestyle{fancy} % customize header and footer
\fancyhf{} % clear initial header and footer
%\rhead{Overleaf}
\lhead{\centering \rightmark} % this adds subsection number and name
\lfoot{\centering \rightmark} 
\rfoot{\thepage} % put page number (the centering command puts it in the middle, don't matter if you put it in right or left footer)

\def \myFigPath {../figures/} 
% BE CAREFUL WITH FIGNAMES, IN LATEX THEY'RE NOT CASE SENSITIVE!!
\def \myTablePath {../tables/} 

%\definecolor{mygreen}{RGB}{0, 100, 0}
\definecolor{mygreen}{RGB}{0, 128, 0}

\definecolor{citec}{rgb}{0,0,.5}
\definecolor{linkc}{rgb}{0,0,.6}
\definecolor{bcolor}{rgb}{1,1,1}
\hypersetup{
%hidelinks = true
  colorlinks = true,
  urlcolor=linkc,
  linkcolor=linkc,
  citecolor = citec,
  filecolor = linkc,
  pdfauthor={Laura G\'ati},
}


\geometry{left=.83in,right=.89in,top=1in,
bottom=1in}
\linespread{1.5}
\renewcommand{\[}{\begin{equation}}
\renewcommand{\]}{\end{equation}}

% New Options
\newtheorem{prop}{Proposition}
\newtheorem{definition}{Definition}[section]
\newtheorem*{remark}{Remark}
\newtheorem{lemma}{Lemma}
\newtheorem{corollary}{Corollary}
\newtheorem{conjecture}{Conjecture}

%\newtheorem{theorem}{Theorem}[section] % the third argument specifies that their number will be adopted to the section
%\newtheorem{corollary}{Corollary}[theorem]
%\newtheorem{lemma}[theorem]{Lemma}
%\declaretheorem{proposition}
%\linespread{1.3}
%\raggedbottom
%\font\reali=msbm10 at 12pt

% New Commands
\newcommand{\real}{\hbox{\reali R}}
\newcommand{\realp}{\hbox{\reali R}_{\scriptscriptstyle +}}
\newcommand{\realpp}{\hbox{\reali R}_{\scriptscriptstyle ++}}
\newcommand{\R}{\mathbb{R}}
\DeclareMathOperator{\E}{\mathbb{E}}
\DeclareMathOperator{\argmin}{arg\,min}
\newcommand\w{3.0in}
\newcommand\wnum{3.0}
\def\myFigWidth{5.3in}
\def\mySmallerFigWidth{2.1in}
\def\myEvenBiggerFigScale{0.8}
\def\myPointSixFigScale{0.6}
\def\myBiggerFigScale{0.4}
\def\myFigScale{0.3}
\def\mySmallFigScale{0.22}
\def\mySmallerFigScale{0.18}
\def\myTinyFigScale{0.16}
\def\myPointFourteenFigScale{0.14}
\def\myTinierFigScale{0.12}
\newcommand\numberthis{\addtocounter{equation}{1}\tag{\theequation}} % this defines a command to make align only number this line
\newcommand{\code}[1]{\texttt{#1}} %code %

\renewcommand*\contentsname{Overview}
\setcounter{tocdepth}{2}

% define a command to make a huge question mark (it works in math mode)
\newcommand{\bigqm}[1][1]{\text{\larger[#1]{\textbf{?}}}}

\begin{document}

\linespread{1.0}

\title{Materials 7 - IRFs and making sure RE is right}
\author{Laura G\'ati} 
\date{\today}
\maketitle

%%%%%%%%%%%%%%%%%%%%             DOCUMENT           %%%%%%%%%%%%%%%%%% 

\tableofcontents

%\listoffigures

%\newpage


\newpage
\section{Model summary with interest rate smoothing $\rho i_{t-1}$}
\begin{align}
x_t &=  -\sigma i_t +\hat{\E}_t \sum_{T=t}^{\infty} \beta^{T-t }\big( (1-\beta)x_{T+1} - \sigma(\beta i_{T+1} - \pi_{T+1}) +\sigma r_T^n \big)  \label{prestons18}  \\
\pi_t &= \kappa x_t +\hat{\E}_t \sum_{T=t}^{\infty} (\alpha\beta)^{T-t }\big( \kappa \alpha \beta x_{T+1} + (1-\alpha)\beta \pi_{T+1} + u_T\big) \label{prestons19}  \\
i_t &= \psi_{\pi}\pi_t + \psi_{x} x_t  + \textcolor{blue}{\rho i_{t-1}} + \bar{i}_t \label{TR}
\end{align}
\begin{equation}
\hat{\E}_t z_{t+h} =  \begin{bmatrix}\bar{\pi}_{t-1} \\ 0 \\ 0 \end{bmatrix}+ b\textcolor{blue}{hx}^{h-1}s_t  \quad \forall h\geq 1 \quad \quad b = gx \; hx \quad \quad \text{PLM} \label{PLM}
\end{equation}
\begin{equation}
\bar{\pi}_{t} = \bar{\pi}_{t-1} +k_t^{-1}\underbrace{\big(\pi_{t} -(\bar{\pi}_{t-1}+b_1s_{t-1}) \big)}_{\text{fcst error using (\ref{PLM})} } \quad \quad  \text{($b_1$ is the first row of $b$)}
\end{equation}
 \begin{align*}
k_t & = \mathbb{I}\times(k_{t-1}+1) + (1-\mathbb{I}) \times \bar{g}^{-1}  \label{gain} \numberthis\\
\mathbb{I} & = \begin{cases} 1 \quad \text{if} \; \theta_t \leq \bar{\theta}  \\ 0 \quad \text{otherwise.}\numberthis
\end{cases} \\
\theta_t & = |\hat{\E}_{t-1}\pi_t - \E_{t-1}\pi_t| / \sigma_s \quad \quad \text{CEMP criterion for the gain}\label{criterion}\numberthis
\end{align*}

The alternative criterion for the choice of gain is a recursive variant of the CUSUM-test (Brown, Durbin, Evans 1975):
\begin{enumerate}
\item Let $FE_t$ denote the short-run forecast error, and $\omega_t$ firms' estimate of the FE variance.
\item Let $\kappa \in (0,1)$ and $\tilde{\theta}$ be the new threshold value for the criterion.
\item Then for initial $(\omega_0, \theta_0)$, firms in every period estimate the criterion and the FEV as:
\begin{align}
 \omega_t & =  \omega_{t-1} + \kappa k_{t-1}^{-1}(FE_t^2 -\omega_{t-1})\\
\theta_t & =  \theta_{t-1} + \kappa k_{t-1}^{-1}(FE_t^2/\omega_t -\theta_{t-1})\\
k_t & = \mathbb{I}\times(k_{t-1}+1) + (1-\mathbb{I}) \times \bar{g}^{-1} \\
\mathbb{I} & = 1 \quad \text{if} \quad  \theta_t \leq \tilde{\theta}
\end{align}
\end{enumerate}

\newpage
\section{Compact notation - with lagged interest rate term in TR}
 \begin{align}
z_t & = A_p^{RE} \E_t z_{t+1} + A_s^{RE} s_t \label{LOM_RE} \\
z_t & = A_a^{LH} f_a(t) + A_b^{LH} f_b(t) + A_s^{LH} s_t \label{LOM_LH} \\
s_t & = P s_{t-1} + \epsilon_t \quad \quad \textcolor{blue}{\rightarrow} \quad s'_t  = hx \; s'_{t-1} + \epsilon'_t \label{exog} \\
 \quad \text{where} \quad 
 s'_t & \equiv \begin{pmatrix} r_t^n \\ \bar{i}_t \\ u_t \\ \textcolor{blue}{i_{t-1}}
 \end{pmatrix} \quad 
 hx  \equiv \begin{pmatrix} \rho_r & 0 & 0 & \textcolor{blue}{0} \\ 0& \rho_i & 0 & \textcolor{blue}{0} \\ 0&0& \rho_u & \textcolor{blue}{0}  \\ 
 \textcolor{blue}{gx_{3,1}}&\textcolor{blue}{gx_{3,2}}& \textcolor{blue}{gx_{3,3}} & \textcolor{blue}{gx_{3,4}}
 \end{pmatrix}  \quad 
 \epsilon'_t \equiv \begin{pmatrix}\varepsilon_t^{r} \\ \varepsilon_t^{i}  \\ \varepsilon_t^{u} \\ \textcolor{blue}{0} 
 \end{pmatrix}  \quad  \text{and } \quad \Sigma'  =  \begin{pmatrix} \sigma_r & 0 & 0 & \textcolor{blue}{0} \\ 0& \sigma_i & 0 & \textcolor{blue}{0}  \\ 0&0& \sigma_u & \textcolor{blue}{0}  \\ \textcolor{blue}{0}  & \textcolor{blue}{0} & \textcolor{blue}{0} & \textcolor{blue}{0} &
 \end{pmatrix} 
\end{align}

$i_{t-1}$ is an endogenous state and breaks the link that previously had $P = hx$; now this is no longer true. In particular, using Matlabby notation, $P = hx(1:3,1:3)$. 

And the $A^{RE}_s$ and $A^{LH}_s$ are given by:
\begin{align}
A_s^{RE} &= \begin{pmatrix}   \frac{\kappa\sigma}{w}  &-\frac{\kappa\sigma}{w}  & 1-\frac{\kappa\sigma\psi_{\pi}}{w} & \textcolor{blue}{0}\\
 \frac{ \sigma}{w} &  -\frac{\sigma}{w} & -\frac{\sigma\psi_{\pi}}{w} & \textcolor{blue}{0}\\ 
 \psi_x( \frac{\sigma}{w}) + \psi_{\pi}( \frac{\kappa\sigma}{w}) & \psi_x(- \frac{\sigma}{w}) + \psi_{\pi}(- \frac{\kappa\sigma}{w}) +1 &  \psi_x(-\frac{\sigma\psi_{\pi}}{w}) + \psi_{\pi}( 1-\frac{\kappa\sigma\psi_{\pi}}{w}) & \textcolor{blue}{\rho}\end{pmatrix}  
\\
 A_s^{LH} & = \begin{pmatrix} g_{\pi s} \\ g_{x s} \\ \psi_{\pi}g_{\pi s} + \psi_xg_{x s} + \begin{bmatrix} 0 & 1& 0 & \textcolor{blue}{\rho}\end{bmatrix}
\end{pmatrix} \\
g_{\pi s} & = (1-\frac{\kappa\sigma\psi_{\pi}}{w} )\begin{bmatrix} 0&0&1 & \textcolor{blue}{0}\end{bmatrix} (I_{\textcolor{blue}{4}} - \alpha\beta \textcolor{blue}{hx})^{-1} -\frac{\kappa\sigma}{w}\begin{bmatrix} -1&1&0 & \textcolor{blue}{\rho} \end{bmatrix} (I_{\textcolor{blue}{4}} -\beta \textcolor{blue}{hx})^{-1}\\
g_{x s} & =  \frac{-\sigma\psi_{\pi}}{w} \begin{bmatrix} 0&0&1& \textcolor{blue}{0} \end{bmatrix}(I_{\textcolor{blue}{4}} - \alpha\beta \textcolor{blue}{hx})^{-1}  -\frac{\sigma}{w}\begin{bmatrix} -1&1&0 & \textcolor{blue}{\rho}\end{bmatrix}(I_{\textcolor{blue}{4}} -\beta \textcolor{blue}{hx})^{-1}
\end{align}

%%%%%%%%%%%  IRFs
\newpage
\section{IRFs splitting by decreasing, constant and endogenous gain learning}

\subsection{Natural rate shock}	
\begin{figure}[h!]
\subfigure[IRFs, \colorbox{yellow}{$\rho$ = 0}]{\includegraphics[scale = \mySmallFigScale]{\myFigPath materials7_IRFs_natrate_rho0_psi_pi_1_5}} 
\subfigure[Gain, \colorbox{yellow}{$\rho$ = 0}]{\includegraphics[scale = \mySmallerFigScale]{\myFigPath materials7_gain_natrate_rho0_psi_pi_1_5}} 
\subfigure[IRFs, \colorbox{yellow}{$\rho$ = 0.3}]{\includegraphics[scale = \mySmallFigScale]{\myFigPath materials7_IRFs_natrate_rho0_3_psi_pi_1_5} }
\subfigure[Gain, \colorbox{yellow}{$\rho$ = 0.3}]{\includegraphics[scale = \mySmallerFigScale]{\myFigPath materials7_gain_natrate_rho0_3_psi_pi_1_5} }
\subfigure[IRFs, \colorbox{yellow}{$\rho$ = 0.6}]{\includegraphics[scale = \mySmallFigScale]{\myFigPath materials7_IRFs_natrate_rho0_6_psi_pi_1_5} }
\subfigure[Gain, \colorbox{yellow}{$\rho$ = 0.6}]{\includegraphics[scale = \mySmallerFigScale]{\myFigPath materials7_gain_natrate_rho0_6_psi_pi_1_5} }
\subfigure[IRFs, \colorbox{yellow}{$\rho$ = 0.9}]{\includegraphics[scale = \mySmallFigScale]{\myFigPath materials7_IRFs_natrate_rho0_9_psi_pi_1_5} }
\subfigure[Gain, \colorbox{yellow}{$\rho$ = 0.9}]{\includegraphics[scale = \mySmallerFigScale]{\myFigPath materials7_gain_natrate_rho0_9_psi_pi_1_5} }
\caption{Moving $\rho$}
\end{figure}

\begin{figure}[h!]
\subfigure[IRFs, \colorbox{yellow}{$\psi_{\pi}$ = 1.1}]{\includegraphics[scale = \mySmallFigScale]{\myFigPath materials7_IRFs_natrate_rho0_psi_pi_1_1}} 
\subfigure[Gain, \colorbox{yellow}{$\psi_{\pi}$ = 1.1}]{\includegraphics[scale = \mySmallerFigScale]{\myFigPath materials7_gain_natrate_rho0_psi_pi_1_1}} 
\subfigure[IRFs, \colorbox{yellow}{$\psi_{\pi}$ = 1.5}]{\includegraphics[scale = \mySmallFigScale]{\myFigPath materials7_IRFs_natrate_rho0_psi_pi_1_5} }
\subfigure[Gain, \colorbox{yellow}{$\psi_{\pi}$ = 1.5}]{\includegraphics[scale = \mySmallerFigScale]{\myFigPath materials7_gain_natrate_rho0_psi_pi_1_5} }
\subfigure[IRFs, \colorbox{yellow}{$\psi_{\pi}$ = 2}]{\includegraphics[scale = \mySmallFigScale]{\myFigPath materials7_IRFs_natrate_rho0_psi_pi_2} }
\subfigure[Gain, \colorbox{yellow}{$\psi_{\pi}$ =2}]{\includegraphics[scale = \mySmallerFigScale]{\myFigPath materials7_gain_natrate_rho0_psi_pi_2} }
\caption{Moving $\psi_{\pi}$}
\end{figure}
\clearpage


\subsection{Monetary policy shock}	
\begin{figure}[h!]
\subfigure[IRFs, \colorbox{yellow}{$\rho$ = 0}]{\includegraphics[scale = \mySmallFigScale]{\myFigPath materials7_IRFs_monpol_rho0_psi_pi_1_5}} 
\subfigure[Gain, \colorbox{yellow}{$\rho$ = 0}]{\includegraphics[scale = \mySmallerFigScale]{\myFigPath materials7_gain_monpol_rho0_psi_pi_1_5}} 
\subfigure[IRFs, \colorbox{yellow}{$\rho$ = 0.3}]{\includegraphics[scale = \mySmallFigScale]{\myFigPath materials7_IRFs_monpol_rho0_3_psi_pi_1_5} }
\subfigure[Gain, \colorbox{yellow}{$\rho$ = 0.3}]{\includegraphics[scale = \mySmallerFigScale]{\myFigPath materials7_gain_monpol_rho0_3_psi_pi_1_5} }
\subfigure[IRFs, \colorbox{yellow}{$\rho$ = 0.6}]{\includegraphics[scale = \mySmallFigScale]{\myFigPath materials7_IRFs_monpol_rho0_6_psi_pi_1_5} }
\subfigure[Gain, \colorbox{yellow}{$\rho$ = 0.6}]{\includegraphics[scale = \mySmallerFigScale]{\myFigPath materials7_gain_monpol_rho0_6_psi_pi_1_5} }
\subfigure[IRFs, \colorbox{yellow}{$\rho$ = 0.9}]{\includegraphics[scale = \mySmallFigScale]{\myFigPath materials7_IRFs_monpol_rho0_9_psi_pi_1_5} }
\subfigure[Gain, \colorbox{yellow}{$\rho$ = 0.9}]{\includegraphics[scale = \mySmallerFigScale]{\myFigPath materials7_gain_monpol_rho0_9_psi_pi_1_5} }
\caption{Moving $\rho$}
\end{figure}

\begin{figure}[h!]
\subfigure[IRFs, \colorbox{yellow}{$\psi_{\pi}$ = 1.1}]{\includegraphics[scale = \mySmallFigScale]{\myFigPath materials7_IRFs_monpol_rho0_psi_pi_1_1}} 
\subfigure[Gain, \colorbox{yellow}{$\psi_{\pi}$ = 1.1}]{\includegraphics[scale = \mySmallerFigScale]{\myFigPath materials7_gain_monpol_rho0_psi_pi_1_1}} 
\subfigure[IRFs, \colorbox{yellow}{$\psi_{\pi}$ = 1.5}]{\includegraphics[scale = \mySmallFigScale]{\myFigPath materials7_IRFs_monpol_rho0_psi_pi_1_5} }
\subfigure[Gain, \colorbox{yellow}{$\psi_{\pi}$ = 1.5}]{\includegraphics[scale = \mySmallerFigScale]{\myFigPath materials7_gain_monpol_rho0_psi_pi_1_5} }
\subfigure[IRFs, \colorbox{yellow}{$\psi_{\pi}$ = 2}]{\includegraphics[scale = \mySmallFigScale]{\myFigPath materials7_IRFs_monpol_rho0_psi_pi_2} }
\subfigure[Gain, \colorbox{yellow}{$\psi_{\pi}$ =2}]{\includegraphics[scale = \mySmallerFigScale]{\myFigPath materials7_gain_monpol_rho0_psi_pi_2} }
\caption{Moving $\psi_{\pi}$}
\end{figure}
\clearpage



\subsection{Cost-push shock}				
\begin{figure}[h!]
\subfigure[IRFs, \colorbox{yellow}{$\rho$ = 0}]{\includegraphics[scale = \mySmallFigScale]{\myFigPath materials7_IRFs_costpush_rho0_psi_pi_1_5}} 
\subfigure[Gain, \colorbox{yellow}{$\rho$ = 0}]{\includegraphics[scale = \mySmallerFigScale]{\myFigPath materials7_gain_costpush_rho0_psi_pi_1_5}} 
\subfigure[IRFs, \colorbox{yellow}{$\rho$ = 0.3}]{\includegraphics[scale = \mySmallFigScale]{\myFigPath materials7_IRFs_costpush_rho0_3_psi_pi_1_5} }
\subfigure[Gain, \colorbox{yellow}{$\rho$ = 0.3}]{\includegraphics[scale = \mySmallerFigScale]{\myFigPath materials7_gain_costpush_rho0_3_psi_pi_1_5} }
\subfigure[IRFs, \colorbox{yellow}{$\rho$ = 0.6}]{\includegraphics[scale = \mySmallFigScale]{\myFigPath materials7_IRFs_costpush_rho0_6_psi_pi_1_5} }
\subfigure[Gain, \colorbox{yellow}{$\rho$ = 0.6}]{\includegraphics[scale = \mySmallerFigScale]{\myFigPath materials7_gain_costpush_rho0_6_psi_pi_1_5} }
\subfigure[IRFs, \colorbox{yellow}{$\rho$ = 0.9}]{\includegraphics[scale = \mySmallFigScale]{\myFigPath materials7_IRFs_costpush_rho0_9_psi_pi_1_5} }
\subfigure[Gain, \colorbox{yellow}{$\rho$ = 0.9}]{\includegraphics[scale = \mySmallerFigScale]{\myFigPath materials7_gain_costpush_rho0_9_psi_pi_1_5} }
\caption{Moving $\rho$}
\end{figure}

\begin{figure}[h!]
\subfigure[IRFs, \colorbox{yellow}{$\psi_{\pi}$ = 1.1}]{\includegraphics[scale = \mySmallFigScale]{\myFigPath materials7_IRFs_costpush_rho0_psi_pi_1_1}} 
\subfigure[Gain, \colorbox{yellow}{$\psi_{\pi}$ = 1.1}]{\includegraphics[scale = \mySmallerFigScale]{\myFigPath materials7_gain_costpush_rho0_psi_pi_1_1}} 
\subfigure[IRFs, \colorbox{yellow}{$\psi_{\pi}$ = 1.5}]{\includegraphics[scale = \mySmallFigScale]{\myFigPath materials7_IRFs_costpush_rho0_psi_pi_1_5} }
\subfigure[Gain, \colorbox{yellow}{$\psi_{\pi}$ = 1.5}]{\includegraphics[scale = \mySmallerFigScale]{\myFigPath materials7_gain_costpush_rho0_psi_pi_1_5} }
\subfigure[IRFs, \colorbox{yellow}{$\psi_{\pi}$ = 2}]{\includegraphics[scale = \mySmallFigScale]{\myFigPath materials7_IRFs_costpush_rho0_psi_pi_2} }
\subfigure[Gain, \colorbox{yellow}{$\psi_{\pi}$ =2}]{\includegraphics[scale = \mySmallerFigScale]{\myFigPath materials7_gain_costpush_rho0_psi_pi_2} }
\caption{Moving $\psi_{\pi}$}
\end{figure}

\clearpage
\section{Current set of baseline parameters}
%\begin{center}
\begin{tabular}{ c | c  | l | l}
 $\beta$ & 0.99 & stochastic discount factor & standard (Woodford 2003/2011) \\  \hline
 $\sigma$ & 1  & IES & consistent with long-run growth \\  \hline
 $\alpha$ & 0.5 &  Calvo probability of not adjusting  & match 6-month duration of prices (can increase to 0.75)\\\hline
 $\psi_{\pi} $& 1.5  & coefficient of inflation in Taylor rule & Taylor \\\hline
 $\psi_x$ & 0   & coefficient of output gap in Taylor rule  & focus on $\pi$\\\hline
 $\bar{g}$ & $0.145$  & value of the constant gain & CEMP \\\hline
& & \\ [-1em] % this adds an extra empty row, and decreases its size, so it looks as if thetbar's row was higher
 $\bar{\theta}$ &  1  & threshold deviation between $\hat{\E}$ \& $\E$ & CEMP: 0.029\\ \hline
    $\rho_r$ & 0 &   persistence of natural rate shock & n.a. \\ \hline
    $\rho_i$ & 0.6 &  persistence of monetary policy shock & CEMP: 0.877  (can increase to 0.7 if $\alpha=0.75$) \\ \hline
    $\rho_u$ & 0  &  persistence of cost-push shock  & CEMP\\ \hline
    $\sigma_r$ & 0.1 & standard deviation of natural rate shock & n.a. \\ \hline
    $\sigma_i$ &  0.359  &standard deviation of mon. policy shock & CEMP \\ \hline
    $\sigma_u$ & 0.277 & standard deviation of cost-push shock & CEMP  \\ \hline
    $\theta$ & 10 & price elasticity of demand &Woodford 2003/2011, Chari, Kehoe \& McGrattan 2000  \\ \hline
    $\omega$ & 1.25 & elasticity of marginal cost to output &Woodford 2003/2011,  Chari, Kehoe \& McGrattan 2000  \\ \hline
\end{tabular}
%\end{center}




\end{document}





