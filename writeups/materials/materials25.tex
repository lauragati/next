\documentclass[11pt]{article}
\usepackage{amsmath, amsthm, amssymb,lscape, natbib}
\usepackage{mathtools}
\usepackage{subfigure}
\usepackage[font=footnotesize,labelfont=bf]{caption}
\usepackage{graphicx}
\usepackage{colortbl}
\usepackage{hhline}
\usepackage{multirow}
\usepackage{multicol}
\usepackage{setspace}
\usepackage[final]{pdfpages}
\usepackage[left=2.5cm,top=2.5cm,right=2.5cm, bottom=2.5cm]{geometry}
\usepackage{natbib} 
\usepackage{bibentry} 
\newcommand{\bibverse}[1]{\begin{verse} \bibentry{#1} \end{verse}}
\newcommand{\vs}{\vspace{.3in}}
\renewcommand{\ni}{\noindent}
\usepackage{xr-hyper}
\usepackage[]{hyperref}
\hypersetup{
    colorlinks=true,
    linkcolor=blue,
    filecolor=magenta,      
    urlcolor=cyan,
}
 
\urlstyle{same}
\usepackage[capposition=top]{floatrow}
\usepackage{amssymb}
\usepackage{relsize}
\usepackage[dvipsnames]{xcolor}
\usepackage{fancyhdr}
\usepackage{tikz}
 
\pagestyle{fancy} % customize header and footer
\fancyhf{} % clear initial header and footer
%\rhead{Overleaf}
\lhead{\centering \rightmark} % this adds subsection number and name
\lfoot{\centering \rightmark} 
\rfoot{\thepage} % put page number (the centering command puts it in the middle, don't matter if you put it in right or left footer)

\def \myFigPath {../../figures/} 
% BE CAREFUL WITH FIGNAMES, IN LATEX THEY'RE NOT CASE SENSITIVE!!
\def \myTablePath {../../tables/} 

%\definecolor{mygreen}{RGB}{0, 100, 0}
\definecolor{mygreen}{RGB}{0, 128, 0}

\definecolor{citec}{rgb}{0,0,.5}
\definecolor{linkc}{rgb}{0,0,.6}
\definecolor{bcolor}{rgb}{1,1,1}
\hypersetup{
%hidelinks = true
  colorlinks = true,
  urlcolor=linkc,
  linkcolor=linkc,
  citecolor = citec,
  filecolor = linkc,
  pdfauthor={Laura G\'ati},
}


\geometry{left=.83in,right=.89in,top=1in,
bottom=1in}
\linespread{1.5}
\renewcommand{\[}{\begin{equation}}
\renewcommand{\]}{\end{equation}}

% New Options
\newtheorem{prop}{Proposition}
\newtheorem{definition}{Definition}[section]
\newtheorem*{remark}{Remark}
\newtheorem{lemma}{Lemma}
\newtheorem{corollary}{Corollary}
\newtheorem{conjecture}{Conjecture}

%\newtheorem{theorem}{Theorem}[section] % the third argument specifies that their number will be adopted to the section
%\newtheorem{corollary}{Corollary}[theorem]
%\newtheorem{lemma}[theorem]{Lemma}
%\declaretheorem{proposition}
%\linespread{1.3}
%\raggedbottom
%\font\reali=msbm10 at 12pt

% New Commands
\newcommand{\real}{\hbox{\reali R}}
\newcommand{\realp}{\hbox{\reali R}_{\scriptscriptstyle +}}
\newcommand{\realpp}{\hbox{\reali R}_{\scriptscriptstyle ++}}
\newcommand{\R}{\mathbb{R}}
\DeclareMathOperator{\E}{\mathbb{E}}
\DeclareMathOperator{\argmin}{arg\,min}
\newcommand\w{3.0in}
\newcommand\wnum{3.0}
\def\myFigWidth{5.3in}
\def\mySmallerFigWidth{2.1in}
\def\myEvenBiggerFigScale{0.8}
\def\myPointSixFigScale{0.6}
\def\myBiggerFigScale{0.4}
\def\myFigScale{0.3}
\def\myMediumFigScale{0.25}
\def\mySmallFigScale{0.22}
\def\mySmallerFigScale{0.18}
\def\myTinyFigScale{0.16}
\def\myPointFourteenFigScale{0.14}
\def\myTinierFigScale{0.12}
\def\myAdjustableFigScale{0.16}
\newcommand\numberthis{\addtocounter{equation}{1}\tag{\theequation}} % this defines a command to make align only number this line
\newcommand{\code}[1]{\texttt{#1}} %code %

\renewcommand*\contentsname{Overview}
\setcounter{tocdepth}{2}

% define a command to make a huge question mark (it works in math mode)
\newcommand{\bigqm}[1][1]{\text{\larger[#1]{\textbf{?}}}}

\begin{document}

\linespread{1.0}

\title{Materials 25 - Preparing macro lunch}
\author{Laura G\'ati} 
\date{\today}
\maketitle

%%%%%%%%%%%%%%%%%%%%             DOCUMENT           %%%%%%%%%%%%%%%%%% 

\tableofcontents

%\listoffigures


\begin{align*}
A_a & = \left(
\begin{array}{ccc}
 -\frac{(\alpha -1) \beta  (\delta  \sigma +1)}{\delta  \sigma +\kappa  \psi  \sigma +1} & \frac{\alpha  \beta  \kappa  (\delta  \sigma +1)}{\delta  \sigma +\kappa  \psi  \sigma +1} & 0 \\
 \frac{(\alpha -1) \beta  \sigma  \psi }{\delta  \sigma +\kappa  \psi  \sigma +1} & -\frac{\alpha  \beta  \kappa  \sigma  \psi }{\delta  \sigma +\kappa  \psi  \sigma +1} & 0 \\
 -\frac{(\alpha -1) \beta  \psi }{\delta  \sigma +\kappa  \psi  \sigma +1} & \frac{\alpha  \beta  \kappa  \psi }{\delta  \sigma +\kappa  \psi  \sigma +1} & 0 \\
\end{array}
\right) \\
A_b & = \left(
\begin{array}{ccc}
 \frac{\kappa  \sigma  (1-\beta  \psi )}{\delta  \sigma +\kappa  \psi  \sigma +1} & -\frac{\kappa  (\delta  \sigma  \beta +\beta -1)}{\delta  \sigma +\kappa  \psi  \sigma +1} & 0 \\
 \frac{\sigma -\beta  \sigma  \psi }{\delta  \sigma +\kappa  \psi  \sigma +1} & -\frac{\delta  \sigma  \beta +\beta -1}{\delta  \sigma +\kappa  \psi  \sigma +1} & 0 \\
 -\frac{\sigma  (\beta  \psi -1) (\delta +\kappa  \psi )}{\delta  \sigma +\kappa  \psi  \sigma +1} & -\frac{(\delta  \sigma  \beta +\beta -1) (\delta +\kappa  \psi )}{\delta  \sigma +\kappa  \psi  \sigma +1} & 0 \\
\end{array}
\right) \\
A_s & = \left(
\begin{array}{ccc}
 \frac{(\delta  \sigma +1) \text{ia}(3,1)+\kappa  \sigma  (\text{ib}(1,1)-\text{ib}(2,1))}{\delta  \sigma +\kappa  \psi  \sigma +1} & \frac{(\delta  \sigma +1) \text{ia}(3,2)+\kappa  \sigma  (\text{ib}(1,2)-\text{ib}(2,2))}{\delta  \sigma +\kappa  \psi  \sigma +1} & \frac{(\delta  \sigma +1) \text{ia}(3,3)+\kappa  \sigma  (\text{ib}(1,3)-\text{ib}(2,3))}{\delta  \sigma +\kappa  \psi  \sigma +1} \\
 -\frac{\sigma  (\psi  \text{ia}(3,1)-\text{ib}(1,1)+\text{ib}(2,1))}{\delta  \sigma +\kappa  \psi  \sigma +1} & -\frac{\sigma  (\psi  \text{ia}(3,2)-\text{ib}(1,2)+\text{ib}(2,2))}{\delta  \sigma +\kappa  \psi  \sigma +1} & -\frac{\sigma  (\psi  \text{ia}(3,3)-\text{ib}(1,3)+\text{ib}(2,3))}{\delta  \sigma +\kappa  \psi  \sigma +1} \\
 \frac{\psi  \text{ia}(3,1)+\delta  \sigma  \text{ib}(1,1)+\kappa  \sigma  \psi  \text{ib}(1,1)-\sigma  (\delta +\kappa  \psi ) \text{ib}(2,1)}{\delta  \sigma +\kappa  \psi  \sigma +1} & \frac{\psi  \text{ia}(3,2)+\delta  \sigma  (\text{ib}(1,2)-\text{ib}(2,2)+1)+\kappa  \sigma  \psi  (\text{ib}(1,2)-\text{ib}(2,2)+1)+1}{\delta  \sigma +\kappa  \psi  \sigma +1} & \frac{\psi  \text{ia}(3,3)+\delta  \sigma  \text{ib}(1,3)+\kappa  \sigma  \psi  \text{ib}(1,3)-\sigma  (\delta +\kappa  \psi ) \text{ib}(2,3)}{\delta  \sigma +\kappa  \psi  \sigma +1} \\
\end{array}
\right) \\
& \text{If shocks are iid, ia and ib are identity matrices, and so As becomes:}\\
A_s & = \left(
\begin{array}{ccc}
 \frac{\kappa  \sigma }{\delta  \sigma +\kappa  \psi  \sigma +1} & -\frac{\kappa  \sigma }{\delta  \sigma +\kappa  \psi  \sigma +1} & \frac{\delta  \sigma +1}{\delta  \sigma +\kappa  \psi  \sigma +1} \\
 \frac{\sigma }{\delta  \sigma +\kappa  \psi  \sigma +1} & -\frac{\sigma }{\delta  \sigma +\kappa  \psi  \sigma +1} & -\frac{\sigma  \psi }{\delta  \sigma +\kappa  \psi  \sigma +1} \\
 \frac{\kappa  \psi  \sigma +\sigma\delta }{\delta  \sigma +\kappa  \psi  \sigma +1} & \frac{1}{\delta  \sigma +\kappa  \psi  \sigma +1} & \frac{\psi }{\delta  \sigma +\kappa  \psi  \sigma +1} \\
\end{array}
\right)
\end{align*}




%%%%%%%%%%   VFI   %%%%%%%%%%%%%%%%%%%%%%%%%%%%%%%%%%%%

\newpage
\section{A value function iteration attempt at finding the optimal interest-rate-sequence}
The planner chooses $\{\pi_t, x_t, i_t, f_{a,t},  f_{b,t}, \bar{\pi}_t, k_t^{-1}\}_{t=t_0}^{\infty}$ to minimize
\begin{align}
V(\mathbf{x}_t,t)& = \max -\bigg\{ (\pi_t^2 + \lambda_x x_t^2) + \beta \E_t V(\mathbf{x}_{t+1},t+1) \bigg\} \\
& \text{s.t. to model equations}
\end{align}
Model equations are:
 \begin{align}
 \pi_t & = \kappa x_t +(1-\alpha)\beta f_a(t) +\kappa\alpha\beta b_2 (I_3 - \alpha\beta h_x)^{-1}s_t +e_3(I_3 - \alpha\beta h_x)^{-1}s_t  \label{midsimple_first}\\
 x_t & = -\sigma i_t +\sigma f_b(t)  +  (1-\beta)b_2 (I_3 - \beta h_x)^{-1}s_t - \sigma\beta b_3 (I_3 - \beta h_x)^{-1}s_t +\sigma e_1(I_3 - \beta h_x)^{-1}s_t  \big) \\
 f_a(t) &= \frac{1}{1-\alpha\beta}\bar{\pi}_{t-1}  + b_1(I_3 - \alpha\beta h_x)^{-1}s_t  \\
 f_b(t) & = \frac{1}{1-\beta}\bar{\pi}_{t-1}  + b_1(I_3 - \beta h_x)^{-1}s_t  \\
 \bar{\pi}_{t} & = \bar{\pi}_{t-1} + k_t^{-1}\big(\pi_{t} -(\bar{\pi}_{t-1}+b_1 s_{t-1}) \big)     \\
 k_t^{-1} & = k_{t-1}^{-1}+ \mathbf{g}(\pi_t - \bar{\pi}_{t-1}-b_1 s_{t-1})  
 \label{midsimple_last}
\end{align}

\newpage
Let's substitute out $x_t, f_{a,t}$ and $f_{b,t}$, so that the state vector is simply $\mathbf{x}_t = (\bar{\pi}_{t-1}, k_t^{-1},r_t^n,u_t)'$.

The problem becomes to choose $\{\pi_t, i_t, \bar{\pi}_t, k_t^{-1}\}_{t=t_0}^{\infty}$ to minimize
\begin{align*}
V(\mathbf{x}_t,t)& = \max -\bigg\{ \pi_t^2 +  \lambda_x\sigma^2 i_t^2 +\lambda_x\frac{\sigma^2}{(1-\beta)^2}\bar{\pi}_{t-1}^2 -\lambda_x\frac{\sigma^2}{1-\beta}i_t \bar{\pi}_{t-1} \\
&-\lambda_x\sigma\bigg(\sigma b_1(I_3 - \beta h_x)^{-1}  +  (1-\beta)b_2 (I_3 - \beta h_x)^{-1} - \sigma\beta b_3 (I_3 - \beta h_x)^{-1} +\sigma e_1(I_3 - \beta h_x)^{-1}\bigg)i_ts_t\\
& + \lambda_x \frac{\sigma}{1-\beta}\bigg(\sigma b_1(I_3 - \beta h_x)^{-1}  +  (1-\beta)b_2 (I_3 - \beta h_x)^{-1} - \sigma\beta b_3 (I_3 - \beta h_x)^{-1} +\sigma e_1(I_3 - \beta h_x)^{-1}\bigg)\bar{\pi}_{t-1}s_t\\
& +\lambda_x\bigg(\sigma b_1(I_3 - \beta h_x)^{-1}  +  (1-\beta)b_2 (I_3 - \beta h_x)^{-1} - \sigma\beta b_3 (I_3 - \beta h_x)^{-1} +\sigma e_1(I_3 - \beta h_x)^{-1}\bigg)^2s_t \\
& + \beta \E_t V(\mathbf{x}_{t+1},t+1) \bigg\} \numberthis \\
& \text{s.t. to model equations}
\end{align*}
 \begin{align*}
 \pi_t & = -\kappa\sigma i_t +\bigg(\kappa\sigma \frac{1}{1-\beta} +  \frac{(1-\alpha)\beta}{1-\alpha\beta}\bigg)\bar{\pi}_{t-1}  \\
 & +\bigg(\kappa\sigma b_1(I_3 - \beta h_x)^{-1}   +  \kappa(1-\beta)b_2 (I_3 - \beta h_x)^{-1} - \kappa\sigma\beta b_3 (I_3 - \beta h_x)^{-1}+\kappa\sigma e_1(I_3 - \beta h_x)^{-1}   \\
 &  + (1-\alpha)\beta b_1(I_3 - \alpha\beta h_x)^{-1}  +\kappa\alpha\beta b_2 (I_3 - \alpha\beta h_x)^{-1} +e_3(I_3 - \alpha\beta h_x)^{-1}\bigg)s_t \numberthis \\
 \bar{\pi}_{t} & = \bar{\pi}_{t-1} + k_t^{-1}\big(\pi_{t} -(\bar{\pi}_{t-1}+b_1 s_{t-1}) \big)   \numberthis  \\
 k_t^{-1} & = k_{t-1}^{-1}+ \mathbf{g}(\pi_t - \bar{\pi}_{t-1}-b_1 s_{t-1})  \numberthis
\end{align*}

Let's simplify further. Let
\begin{align*}
\Omega_1 &\equiv -\lambda_x\sigma\bigg(\sigma b_1(I_3 - \beta h_x)^{-1}  +  (1-\beta)b_2 (I_3 - \beta h_x)^{-1} - \sigma\beta b_3 (I_3 - \beta h_x)^{-1} +\sigma e_1(I_3 - \beta h_x)^{-1}\bigg) \numberthis \\
\Omega_2 &\equiv \lambda_x \frac{\sigma}{1-\beta}\bigg(\sigma b_1(I_3 - \beta h_x)^{-1}  +  (1-\beta)b_2 (I_3 - \beta h_x)^{-1} - \sigma\beta b_3 (I_3 - \beta h_x)^{-1} +\sigma e_1(I_3 - \beta h_x)^{-1}\bigg) \numberthis\\
\Omega_3 &\equiv \lambda_x\bigg(\sigma b_1(I_3 - \beta h_x)^{-1}  +  (1-\beta)b_2 (I_3 - \beta h_x)^{-1} - \sigma\beta b_3 (I_3 - \beta h_x)^{-1} +\sigma e_1(I_3 - \beta h_x)^{-1}\bigg)^2 \numberthis\\
\Omega_4 & \equiv \bigg(\kappa\sigma \frac{1}{1-\beta} +  \frac{(1-\alpha)\beta}{1-\alpha\beta}\bigg) \numberthis \\
\Omega_5 &\equiv \bigg(\kappa\sigma b_1(I_3 - \beta h_x)^{-1}   +  \kappa(1-\beta)b_2 (I_3 - \beta h_x)^{-1} - \kappa\sigma\beta b_3 (I_3 - \beta h_x)^{-1}+\kappa\sigma e_1(I_3 - \beta h_x)^{-1}   \\
 &  + (1-\alpha)\beta b_1(I_3 - \alpha\beta h_x)^{-1}  +\kappa\alpha\beta b_2 (I_3 - \alpha\beta h_x)^{-1} +e_3(I_3 - \alpha\beta h_x)^{-1}\bigg)\numberthis
\end{align*}

Then I can rewrite the problem as
\begin{align*}
V(\mathbf{x}_t,t)& = \max -\bigg\{ \pi_t^2 +  \lambda_x\sigma^2 i_t^2 +\lambda_x\frac{\sigma^2}{(1-\beta)^2}\bar{\pi}_{t-1}^2 -\lambda_x\frac{\sigma^2}{1-\beta}i_t \bar{\pi}_{t-1} \\
&+\Omega_1i_ts_t + \Omega_2\bar{\pi}_{t-1}s_t +\Omega_3s_t + \beta \E_t V(\mathbf{x}_{t+1},t+1) \bigg\} \numberthis \\
& \text{s.t.} \\
 \pi_t & = -\kappa\sigma i_t +\Omega_4\bar{\pi}_{t-1}  +\Omega_5 s_t \numberthis \\
 \bar{\pi}_{t} & = \bar{\pi}_{t-1} + k_t^{-1}\big(\pi_{t} -(\bar{\pi}_{t-1}+b_1 s_{t-1}) \big)   \numberthis  \\
 k_t^{-1} & = k_{t-1}^{-1}+ \mathbf{g}(\pi_t - \bar{\pi}_{t-1}-b_1 s_{t-1})  \numberthis
\end{align*}

to sub out $\pi_t$ and get
\begin{align*}
V(\mathbf{x}_t,t)& = \max -\bigg\{ (\lambda_x\sigma^2 +(\kappa\sigma)^2)i_t^2 +(\lambda_x\frac{\sigma^2}{(1-\beta)^2} + \Omega_4^2)\bar{\pi}_{t-1}^2 +(-\lambda_x\frac{\sigma^2}{1-\beta} -\kappa\sigma\Omega_4 )i_t \bar{\pi}_{t-1} \\
&+(\Omega_1-\kappa\sigma\Omega_5)i_ts_t + (\Omega_2+\Omega_4\Omega_5)\bar{\pi}_{t-1}s_t +(\Omega_3+\Omega_5^2)s_t + \beta \E_t V(\mathbf{x}_{t+1},t+1) \bigg\} \numberthis \\
& \text{s.t.} \\
 \bar{\pi}_{t} & = \bar{\pi}_{t-1} + k_t^{-1}\big(-\kappa\sigma i_t +(\Omega_4-1)\bar{\pi}_{t-1}  +\Omega_5 s_t-b_1 s_{t-1}) \big)   \numberthis  \\
 k_t^{-1} & = k_{t-1}^{-1}+ \mathbf{g}(-\kappa\sigma i_t +(\Omega_4-1)\bar{\pi}_{t-1}  +\Omega_5 s_t-b_1 s_{t-1})  \numberthis
\end{align*}

Let's introduce more $\Omega$s. Let
\begin{align}
\Omega_6 & \equiv (\lambda_x\sigma^2 +(\kappa\sigma)^2) \\
\Omega_7 & \equiv (\lambda_x\frac{\sigma^2}{(1-\beta)^2} + \Omega_4^2) \\
\Omega_8 & \equiv (-\lambda_x\frac{\sigma^2}{1-\beta} -\kappa\sigma\Omega_4 )\\
\Omega_9 &\equiv  (\Omega_1-\kappa\sigma\Omega_5) \\
\Omega_{10} &\equiv  (\Omega_2+\Omega_4\Omega_5) \\
\Omega_{11} &     \equiv (\Omega_3+\Omega_5^2) \\
\Omega_{12} &     \equiv \Omega_4-1
\end{align}
Then the problem takes its final form:
\begin{align*}
V(\mathbf{x}_t,t)& = \max_{i_t} -\bigg\{ \Omega_6 i_t^2 +\Omega_7\bar{\pi}_{t-1}^2 +\Omega_8i_t \bar{\pi}_{t-1} +\Omega_9i_ts_t + \Omega_{10}\bar{\pi}_{t-1}s_t +\Omega_{11}s_t + \beta \E_t V(\mathbf{x}_{t+1},t+1) \bigg\} \numberthis \\
& \text{s.t.} \\
 \bar{\pi}_{t} & = \bar{\pi}_{t-1} + k_t^{-1}\big(-\kappa\sigma i_t +\Omega_{12}\bar{\pi}_{t-1}  +\Omega_5 s_t-b_1 s_{t-1}) \big)   \numberthis  \\
 k_t^{-1} & = k_{t-1}^{-1}+ \mathbf{g}(-\kappa\sigma i_t +\Omega_{12}\bar{\pi}_{t-1}  +\Omega_5 s_t-b_1 s_{t-1})  \numberthis
\end{align*}


\newpage
\appendix
% the following command makes equation numbering include the section first, but just for what follows
\numberwithin{equation}{section}
\section{Model summary}

\vspace{-0.5cm}

\begin{align}
x_t &=  -\sigma i_t +\hat{\E}_t \sum_{T=t}^{\infty} \beta^{T-t }\big( (1-\beta)x_{T+1} - \sigma(\beta i_{T+1} - \pi_{T+1}) +\sigma r_T^n \big)  \label{prestons18}  \\
\pi_t &= \kappa x_t +\hat{\E}_t \sum_{T=t}^{\infty} (\alpha\beta)^{T-t }\big( \kappa \alpha \beta x_{T+1} + (1-\alpha)\beta \pi_{T+1} + u_T\big) \label{prestons19}  \\
i_t &= \psi_{\pi}\pi_t + \psi_{x} x_t  + \bar{i}_t \label{TR} \quad \quad (\text{if imposed})
\end{align}

\vspace{-1.2cm}

\begin{align}
\text{PLM:} \quad \quad & \hat{\E}_t z_{t+h}  =  a_{t-1} + bh_x^{h-1}s_t  \quad \forall h\geq 1 \quad \quad b = g_x\; h_x \quad \quad  \label{PLM} \\
\text{Updating:} \quad \quad & a_{t}  =a_{t-1} +k_t^{-1}\big(z_{t} -(a_{t-1}+b s_{t-1}) \big)  \\
\text{Anchoring function:} \quad \quad & k_t  = k_{t-1} + \mathbf{g}(fe_{t-1}^2) \\
\text{Forecast error:} \quad \quad & fe_{t-1}  = z_t - (a_{t-1}+b s_{t-1})\\
\text{LH expectations:} \quad \quad & f_a(t) = \frac{1}{1-\alpha\beta}a_{t-1}  + b(\mathbb{I}_{nx} - \alpha\beta h)^{-1}s_t \quad \quad  f_b(t) = \frac{1}{1-\beta}a_{t-1}  + b(\mathbb{I}_{nx} - \beta h)^{-1}s_t  \label{fafb_anal}
\end{align}

\vspace{-0.5cm}

This notation captures vector learning ($z$ learned) for intercept only. For scalar learning, $a_t= \begin{pmatrix} \bar{\pi}_t & 0 & 0\end{pmatrix}' $ and $b_1$ designates the first row of $b$. The observables $(\pi, x)$ are determined as:
\begin{align}
x_t &=  -\sigma i_t + \begin{bmatrix} \sigma & 1-\beta & -\sigma\beta \end{bmatrix} f_b + \sigma \begin{bmatrix} 1 & 0 & 0 \end{bmatrix} (\mathbb{I}_{nx} - \beta h_x)^{-1} s_t \label{A9} \\
\pi_t &= \kappa x_t  + \begin{bmatrix} (1-\alpha)\beta & \kappa\alpha\beta & 0 \end{bmatrix}  f_a + \begin{bmatrix} 0 & 0 & 1 \end{bmatrix}  (\mathbb{I}_{nx} - \alpha \beta h_x)^{-1}  s_t \label{A10}
\end{align}

\section{Target criterion}\label{target_crit_levels}
The target criterion in the simplified model (scalar learning of inflation intercept only, $k_t^{-1} = \mathbf{g}(fe_{t-1})$):
\begin{align*}
\pi_t  = -\frac{\lambda_x}{\kappa}\bigg\{x_t - \frac{(1-\alpha)\beta}{1-\alpha\beta} \bigg(k_t^{-1}+((\pi_t - \bar{\pi}_{t-1}-b_1 s_{t-1}))\mathbf{g}_{\pi}(t) \bigg) \\
\bigg(\E_t\sum_{i=1}^{\infty}x_{t+i}\prod_{j=0}^{i-1}(1-k_{t+1+j}^{-1} - (\pi_{t+1+j} - \bar{\pi}_{t+j}-b_1 s_{t+j})\mathbf{g_{\bar{\pi}}}(t+j)) \bigg)
\bigg\} \numberthis \label{target}
\end{align*}
where I'm using the notation that $\prod_{j=0}^{0} \equiv 1$. For interpretation purposes, let me rewrite this as follows:
\begin{align*}
\pi_t  = & \; \textcolor{red}{-\frac{\lambda_x}{\kappa} x_t} \textcolor{blue}{ \; + \frac{\lambda_x}{\kappa} \frac{(1-\alpha)\beta}{1-\alpha\beta} \bigg(k_t^{-1}+ fe^{eve}_{t|t-1}\mathbf{g}_{\pi}(t) \bigg)\E_t\sum_{i=1}^{\infty}x_{t+i}}  \\
& \textcolor{mygreen}{- \frac{\lambda_x}{\kappa} \frac{(1-\alpha)\beta}{1-\alpha\beta} \bigg(k_t^{-1}+ fe^{eve}_{t|t-1}\mathbf{g}_{\pi}(t) \bigg) \bigg(\E_t\sum_{i=1}^{\infty}x_{t+i}\prod_{j=0}^{i-1}(k_{t+1+j}^{-1} + fe^{eve}_{t+1+j|t+j})\mathbf{g_{\bar{\pi}}}(t+j) \bigg)}
\numberthis \label{target_interpretation}
\end{align*}
Interpretation: \textcolor{red}{tradeoffs from discretion in RE} + \textcolor{blue}{effect of current level and change of the gain on future tradeoffs} + \textcolor{mygreen}{effect of future expected levels and changes of the gain on future tradeoffs}

\section{A target criterion system for an anchoring function specified for gain changes}\label{target_crit_changes}
\begin{equation}
k_t = k_{t-1} + \mathbf{g}(fe_{t|t-1})
\end{equation}
Turns out the $k_{t-1}$ adds one $\varphi_{6,t+1}$ too many which makes the target criterion unwieldy. The FOCs of the Ramsey problem are
\begin{align}
& 2\pi_t + 2\frac{\lambda}{\kappa}x_t -k_t^{-1} \varphi_{5,t} - \mathbf{g}_{\pi}(t)\varphi_{6,t}  = 0 \label{gaspar22}\\
& c x_{t+1} + \varphi_{5,t} -(1-k_t^{-1})\varphi_{5,t+1} +\mathbf{g}_{\bar{\pi}}(t)\varphi_{6,t+1} = 0 \label{gaspar21}\\
& \varphi_{6,t} \; \textcolor{red}{+\; \varphi_{6,t+1}} = fe_t \varphi_{5,t} \label{constraints}
\end{align}
where the red multiplier is the new element vis-a-vis the case where the anchoring function is specified in levels ($k_t^{-1} = \mathbf{g}(fe_{t-1})$, as in App. \ref{target_crit_levels}), and I'm using the shorthand notation
\begin{align}
c & = -\frac{2(1-\alpha)\beta}{1-\alpha\beta}\frac{\lambda}{\kappa} \\ 
fe_t & = \pi_t - \bar{\pi}_{t-1}-b s_{t-1}
\end{align}
(\ref{gaspar22}) says that in anchoring, the discretion tradeoff is complemented with tradeoffs coming from learning ($\varphi_{5,t}$), which are more binding when expectations are unanchored ($k_{t}^{-1}$ high). Moreover, the change in the anchoring of expectations imposes an additional constraint ($\varphi_{6,t}$), which is more strongly binding if the gain responds strongly to inflation ($\mathbf{g}_{\pi}(t)$).
One can simplify this three-equation-system to:
\begin{align}
\varphi_{6,t} & = -c fe_t x_{t+1} + \bigg(1+ \frac{fe_t}{fe_{t+1}}(1-k_{t+1}^{-1}) -fe_t \mathbf{g}_{\bar{\pi}}(t) \bigg) \varphi_{6,t+1} -\frac{fe_t}{fe_{t+1}}(1-k_{t+1}^{-1})\varphi_{6,t+2}\label{6'} \\
0 & = 2\pi_t + 2\frac{\lambda}{\kappa}x_t   - \bigg( \frac{k_t^{-1}}{fe_t} + \mathbf{g}_{\pi}(t)\bigg)\varphi_{6,t} + \frac{k_t^{-1}}{fe_t}\varphi_{6,t+1}\label{1'}
\end{align}
Unfortunately, I haven't been able to solve (\ref{6'}) for $\varphi_{6,t}$ and therefore I can't express the target criterion so nicely as before. The only thing I can say is to direct the targeting rule-following central bank to compute $\varphi_{6,t}$ as the solution to (\ref{1'}), and then evaluate (\ref{6'}) as a target criterion. The solution to (\ref{1'}) is given by:
\begin{equation}
\varphi_{6,t} = -2\E_t\sum_{i=0}^{\infty}(\pi_{t+i}+\frac{\lambda_x}{\kappa}x_{t+i})\prod_{j=0}^{i-1}\frac{\frac{k_{t+j}^{-1}}{fe_{t+j}}}{\frac{k_{t+j}^{-1}}{fe_{t+j}} + \mathbf{g}_{\pi}(t+j)} \label{sol1'}
\end{equation}
Interpretation: the anchoring constraint is not binding ($\varphi_{6,t}=0$) if the CB always hits the target (
$\pi_{t+i}+\frac{\lambda_x}{\kappa}x_{t+i} = 0 \quad \forall i$); or expectations are always anchored ($k_{t+j}^{-1}=0 \quad \forall j$). 


\end{document}

%%%%%%%%%%%%%    SUBFIGURE  %%%%%%%%%%%
%\begin{figure}[h!]
%\subfigure[Hodrick-Prescott, $\lambda=1600$]{\includegraphics[scale=\myAdjustableFigScale]{\myFigPath materials22_gain_dhat_HP}}
%\hfill % this is great to intro dpace between subfigures
%\subfigure[Hamilton, 4 lags, $h=8$]{\includegraphics[scale=\myAdjustableFigScale]{\myFigPath materials22_gain_dhat_Hamilton}}
%\subfigure[Baxter-King, $(6,32)$ quarters, truncation at 12 lags]{\includegraphics[scale=\myAdjustableFigScale]{\myFigPath materials22_gain_dhat_BK}}
%\caption{Inverse gain for $\hat{d}$ for the different filters}
%\end{figure}

%%%%%%%%%%%%%    TABLE  %%%%%%%%%%%
%\begin{center}
%\begin{table}[h!]
%\caption{$\hat{d}$}
%\begin{tabular}{ c |c |c }
%  & $W = I$ & $W = \text{diag}(\hat{\sigma}_{ac(0)}, \dots, \hat{\sigma}_{ac(K)})$ \\ 
%  \hline
% HP & 77.7899 & 10 \\  
% \hline
% Hamilton & 32.1649 & 10 \\  
% \hline
% BK & 90.3929 & 10    
%\end{tabular}
%\end{table}
%\end{center}





