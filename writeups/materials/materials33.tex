\documentclass[11pt]{article}
\usepackage{amsmath, amsthm, amssymb,lscape, natbib}
\usepackage{mathtools}
\usepackage{subfigure}
\usepackage[font=footnotesize,labelfont=bf]{caption}
\usepackage{graphicx}
\usepackage{colortbl}
\usepackage{hhline}
\usepackage{multirow}
\usepackage{multicol}
\usepackage{setspace}
\usepackage[final]{pdfpages}
\usepackage[left=2.5cm,top=2.5cm,right=2.5cm, bottom=2.5cm]{geometry}
\usepackage{natbib} 
\usepackage{bibentry} 
\newcommand{\bibverse}[1]{\begin{verse} \bibentry{#1} \end{verse}}
\newcommand{\vs}{\vspace{.3in}}
\renewcommand{\ni}{\noindent}
\usepackage{xr-hyper}
\usepackage[]{hyperref}
\hypersetup{
    colorlinks=true,
    linkcolor=blue,
    filecolor=magenta,      
    urlcolor=cyan,
}
 
\urlstyle{same}
\usepackage[capposition=top]{floatrow}
\usepackage{amssymb}
\usepackage{relsize}
\usepackage[dvipsnames]{xcolor}
\usepackage{fancyhdr}
\usepackage{tikz}
 
\pagestyle{fancy} % customize header and footer
\fancyhf{} % clear initial header and footer
%\rhead{Overleaf}
\lhead{\centering \rightmark} % this adds subsection number and name
\lfoot{\centering \rightmark} 
\rfoot{\thepage} % put page number (the centering command puts it in the middle, don't matter if you put it in right or left footer)

\def \myFigPath {../../figures/} 
% BE CAREFUL WITH FIGNAMES, IN LATEX THEY'RE NOT CASE SENSITIVE!!
\def \myTablePath {../../tables/} 

%\definecolor{mygreen}{RGB}{0, 100, 0}
\definecolor{mygreen}{RGB}{0, 128, 0}

\definecolor{citec}{rgb}{0,0,.5}
\definecolor{linkc}{rgb}{0,0,.6}
\definecolor{bcolor}{rgb}{1,1,1}
\hypersetup{
%hidelinks = true
  colorlinks = true,
  urlcolor=linkc,
  linkcolor=linkc,
  citecolor = citec,
  filecolor = linkc,
  pdfauthor={Laura G\'ati},
}


\geometry{left=.83in,right=.89in,top=1in,
bottom=1in}
\linespread{1.5}
\renewcommand{\[}{\begin{equation}}
\renewcommand{\]}{\end{equation}}

% New Options
\newtheorem{prop}{Proposition}
\newtheorem{definition}{Definition}[section]
\newtheorem*{remark}{Remark}
\newtheorem{lemma}{Lemma}
\newtheorem{corollary}{Corollary}
\newtheorem{conjecture}{Conjecture}

%\newtheorem{theorem}{Theorem}[section] % the third argument specifies that their number will be adopted to the section
%\newtheorem{corollary}{Corollary}[theorem]
%\newtheorem{lemma}[theorem]{Lemma}
%\declaretheorem{proposition}
%\linespread{1.3}
%\raggedbottom
%\font\reali=msbm10 at 12pt

% New Commands
\newcommand{\real}{\hbox{\reali R}}
\newcommand{\realp}{\hbox{\reali R}_{\scriptscriptstyle +}}
\newcommand{\realpp}{\hbox{\reali R}_{\scriptscriptstyle ++}}
\newcommand{\R}{\mathbb{R}}
\DeclareMathOperator{\E}{\mathbb{E}}
\DeclareMathOperator{\argmin}{arg\,min}
\newcommand\w{3.0in}
\newcommand\wnum{3.0}
\def\myFigWidth{5.3in}
\def\mySmallerFigWidth{2.1in}
\def\myEvenBiggerFigScale{0.8}
\def\myPointSixFigScale{0.6}
\def\myBiggerFigScale{0.4}
\def\myFigScale{0.3}
\def\myMediumFigScale{0.25}
\def\mySmallFigScale{0.22}
\def\mySmallerFigScale{0.18}
\def\myTinyFigScale{0.16}
\def\myPointFourteenFigScale{0.14}
\def\myTinierFigScale{0.12}
\def\myAdjustableFigScale{0.20}
\newcommand\numberthis{\addtocounter{equation}{1}\tag{\theequation}} % this defines a command to make align only number this line
\newcommand{\code}[1]{\texttt{#1}} %code %

\renewcommand*\contentsname{Overview}
\setcounter{tocdepth}{2}

% define a command to make a huge question mark (it works in math mode)
\newcommand{\bigqm}[1][1]{\text{\larger[#1]{\textbf{?}}}}

\begin{document}

\linespread{1.0}

\title{Materials 33 - Estimating the anchoring function}
\author{Laura G\'ati} 
\date{\today}
\maketitle

%%%%%%%%%%%%%%%%%%%%             DOCUMENT           %%%%%%%%%%%%%%%%%% 

%\tableofcontents

%\listoffigures

\section{Estimation procedure}
Instead of the AR(1) anchoring function used so far (Equation \ref{A6}), I use the following equation
\begin{equation}
k_t^{-1} = \alpha s(X)
\end{equation}
where $X = (k^{-1}_{t-1}, fe_{t|t-1})$ and I use piecewise linear interpolation. Now I initialize $\alpha_0$ by specifying a grid for $X$, passing the grid through Equation (\ref{A6}) to generate $k^{-1}_t$-values, and approximating by fitting the grid to the $k^{-1}_t$-values. This means that if the functional relationship in Fig. \ref{fig_initial_anchor_fct} is the right one, then the AR(1) specification of Equation (\ref{A6}) isn't a bad description of the anchoring function.

Then I estimate $\alpha$ using GMM, targeting the autocovariance structure of inflation, the output gap and the nominal interest rate (federal funds rate) in the data. 

\begin{figure}[h!]
\includegraphics[scale=0.17]{\myFigPath materials33_initial_anchor_fct__date_12_Jun_2020}
\caption{Initialization via Equation (\ref{A6}) implies this functional relationship}
\label{fig_initial_anchor_fct}
\end{figure}

$T=233$ before BK-filtering, $T=209$ after BK-filtering. Using the ``constant-only, inflation-only'' learning PLM. I drop the $ndrop=5$ initial values.

\section{Estimation issues}
\begin{enumerate}
%\item To estimate the autocorrelations in the data, I estimate reduced-form VAR. Matlab complains that my regressor matrix $X'X$ is singular. --> reintroduced Taylor rule shock to solve stochastic singularity
\item Different bounds on  $\alpha$ yield different estimates, see below.
\item The estimation is also very sensitive to the choice of the grids for the gain and for forecast errors. For example, choosing $n_{grid} = 11$ instead of 10 leads to massively different estimates. In particular, odd vs. even seems to matter.
\item Given $\hat{\alpha}$, PEA has a negative gain problem. I don't understand how!
\end{enumerate}

\begin{figure}[h!]
\subfigure[Coefficients unrestricted, $n_{grid} = 10$]{\includegraphics[scale=\myPointFourteenFigScale]{\myFigPath materials33_estimated_alphas_ndrop_5_ng_10_alpha_constraint_unrestricted_16_Jun_2020}}
\hfill 
\subfigure[Coefficients restricted $> 0$, $n_{grid} = 10$]{\includegraphics[scale=\myPointFourteenFigScale]{\myFigPath materials33_estimated_alphas_ndrop_5_ng_10_alpha_constraint_positive_16_Jun_2020}}
\subfigure[Coefficients unrestricted, $n_{grid} = 11$]{\includegraphics[scale=\myPointFourteenFigScale]{\myFigPath materials33_estimated_alphas_ndrop_5_ng_11_alpha_constraint_unrestricted_16_Jun_2020}}
\hfill
\subfigure[Coefficients restricted $> 0$, $n_{grid} = 11$]{\includegraphics[scale=\myPointFourteenFigScale]{\myFigPath materials33_estimated_alphas_ndrop_5_ng_11_alpha_constraint_positive_16_Jun_2020}}
\subfigure[Coefficients unrestricted, $n_{grid} = 20$. This actually goes negative!]{\includegraphics[scale=\myPointFourteenFigScale]{\myFigPath materials33_estimated_alphas_ndrop_5_ng_20_alpha_constraint_unrestricted_16_Jun_2020}}
\hfill
\subfigure[Coefficients restricted $> 0$, $n_{grid} = 20$]{\includegraphics[scale=\myPointFourteenFigScale]{\myFigPath materials33_estimated_alphas_ndrop_5_ng_20_alpha_constraint_positive_16_Jun_2020}}
\caption{$k^{-1}_t$ as a function of $k^{-1}_{t-1}$ and $fe_{t|t-1}$ given $\hat{\alpha}^{GMM}$}
\label{fig_estmtd_anchor_fct}
\end{figure}

\clearpage

I thought of something: I have $ngrid^2$ parameters and 45 moments. This means my grid can at most have 6 elements (or I need more autocovariance lags to have more moments).
\begin{figure}[h!]
\subfigure[Coefficients unrestricted, $n_{grid} = 6$]{\includegraphics[scale=\myPointFourteenFigScale]{\myFigPath materials33_estimated_alphas_ndrop_5_ng_6_alpha_constraint_unrestricted_16_Jun_2020}}
\hfill 
\subfigure[Coefficients restricted $> 0$, $n_{grid} = 6$]{\includegraphics[scale=\myPointFourteenFigScale]{\myFigPath materials33_estimated_alphas_ndrop_5_ng_6_alpha_constraint_positive_16_Jun_2020}}
\subfigure[Coefficients unrestricted, $n_{grid} = 5$]{\includegraphics[scale=\myPointFourteenFigScale]{\myFigPath materials33_estimated_alphas_ndrop_5_ng_5_alpha_constraint_unrestricted_16_Jun_2020}}
\hfill
\subfigure[Coefficients restricted $> 0$, $n_{grid} = 5$]{\includegraphics[scale=\myPointFourteenFigScale]{\myFigPath materials33_estimated_alphas_ndrop_5_ng_5_alpha_constraint_positive_16_Jun_2020}}
\subfigure[Coefficients unrestricted, $n_{grid} = 4$.]{\includegraphics[scale=\myPointFourteenFigScale]{\myFigPath materials33_estimated_alphas_ndrop_5_ng_4_alpha_constraint_unrestricted_16_Jun_2020}}
\hfill
\subfigure[Coefficients restricted $> 0$, $n_{grid} = 4$]{\includegraphics[scale=\myPointFourteenFigScale]{\myFigPath materials33_estimated_alphas_ndrop_5_ng_4_alpha_constraint_positive_16_Jun_2020}}
\caption{$k^{-1}_t$ as a function of $k^{-1}_{t-1}$ and $fe_{t|t-1}$ given $\hat{\alpha}^{GMM}$}
\label{fig_estmtd_anchor_fct}
\end{figure}

\clearpage
\section{Older plots for me}
% constant only, pi only PLM
% beta restricted to be positive
\begin{figure}[h!]
\subfigure[$N_{burn}=0$]{\includegraphics[scale=\myAdjustableFigScale]{\myFigPath materials33_estmtd_anchor_fct_nburn_0_lb_pos_k1min_1e-05_date_15_Jun_2020}}
\hfill % this is great to intro dpace between subfigures
\subfigure[$N_{burn}=25$]{\includegraphics[scale=\myAdjustableFigScale]{\myFigPath materials33_estmtd_anchor_fct_nburn_25_lb_pos_k1min_1e-05_date_15_Jun_2020}}
\subfigure[$N_{burn}=50$]{\includegraphics[scale=\myAdjustableFigScale]{\myFigPath materials33_estmtd_anchor_fct_nburn_50_lb_pos_k1min_1e-05_date_15_Jun_2020}}
\caption{$k^{-1}_t$ as a function of $k^{-1}_{t-1}$ and $fe_{t|t-1}$ given $\hat{\alpha}^{GMM} > 0$}
\label{fig_estmtd_anchor_fct_restricted}
\end{figure}

% beta not restricted to be positive
\begin{figure}[h!]
\subfigure[$N_{burn}=0$]{\includegraphics[scale=\myAdjustableFigScale]{\myFigPath materials33_estmtd_anchor_fct_nburn_0_lb_neg_k1min_1e-05_date_15_Jun_2020}}
\hfill % this is great to intro dpace between subfigures
\subfigure[$N_{burn}=25$]{\includegraphics[scale=\myAdjustableFigScale]{\myFigPath materials33_estmtd_anchor_fct_nburn_25_lb_neg_k1min_1e-05_date_15_Jun_2020}}
\subfigure[$N_{burn}=50$]{\includegraphics[scale=\myAdjustableFigScale]{\myFigPath materials33_estmtd_anchor_fct_nburn_50_lb_neg_k1min_1e-05_date_15_Jun_2020}}
\caption{$k^{-1}_t$ as a function of $k^{-1}_{t-1}$ and $fe_{t|t-1}$ given $\hat{\alpha}^{GMM}$ (not restricted $>0$)}
\label{fig_estmtd_anchor_fct_unrestricted}
\end{figure}

% constant only PLM
%% beta restricted to be positive
%\begin{figure}[h!]
%\subfigure[$N_{burn}=0$]{\includegraphics[scale=\myAdjustableFigScale]{\myFigPath materials33_estmtd_anchor_fct_nburn_0_lb_pos_date_12_Jun_2020}}
%\hfill % this is great to intro dpace between subfigures
%\subfigure[$N_{burn}=25$]{\includegraphics[scale=\myAdjustableFigScale]{\myFigPath materials33_estmtd_anchor_fct_nburn_25_lb_pos_date_12_Jun_2020}}
%\subfigure[$N_{burn}=50$]{\includegraphics[scale=\myAdjustableFigScale]{\myFigPath materials33_estmtd_anchor_fct_nburn_50_lb_pos_date_12_Jun_2020}}
%\caption{$k^{-1}_t$ as a function of $k^{-1}_{t-1}$ and $fe_{t|t-1}$ given $\hat{\alpha}^{GMM} > 0$}
%\label{fig_estmtd_anchor_fct}
%\end{figure}
%
%% beta not restricted to be positive
%\begin{figure}[h!]
%\subfigure[$N_{burn}=0$]{\includegraphics[scale=\myAdjustableFigScale]{\myFigPath materials33_estmtd_anchor_fct_nburn_0_lb_neg_date_12_Jun_2020}}
%\hfill % this is great to intro dpace between subfigures
%\subfigure[$N_{burn}=25$]{\includegraphics[scale=\myAdjustableFigScale]{\myFigPath materials33_estmtd_anchor_fct_nburn_25_lb_neg_date_12_Jun_2020}}
%\subfigure[$N_{burn}=50$]{\includegraphics[scale=\myAdjustableFigScale]{\myFigPath materials33_estmtd_anchor_fct_nburn_50_lb_neg_date_12_Jun_2020}}
%\caption{$k^{-1}_t$ as a function of $k^{-1}_{t-1}$ and $fe_{t|t-1}$ given $\hat{\alpha}^{GMM}$ (not restricted $>0$)}
%\label{fig_estmtd_anchor_fct_unrestricted}
%\end{figure}

\clearpage


\begin{figure}[h!]
\includegraphics[scale=0.18]{\myFigPath materials33_estmtd_anchor_fct_nburn_50_lb_pos_k1min_0_date_15_Jun_2020}
\caption{$min(k^{-1}) = 0$ instead of 0.00001}
\floatfoot{$N_{burn} = 50$, PLM = constant only, inflation only}
\label{mink1_0}
\end{figure}


\begin{figure}[h!]
\includegraphics[scale=0.18]{\myFigPath materials33_estmtd_anchor_fct_nburn_50_lb_pos_k1min_0_ng_11_ngfine_100_date_15_Jun_2020}
\caption{$n_{grid}$=11}
\floatfoot{$min(k^{-1}) = 0, \alpha >0, N_{burn}=50$, constant-only, inflation-only PLM}
\label{ng11}
\end{figure}


%\clearpage
\section{Is the estimated LOM gain in the nonpermissible range for PEA?}
If I fit an AR(1) (Equation \ref{A6}) to the estimated gain evolution, I obtain $(\rho_k, \gamma_k)=(1, 0.0123)$ which results in the following relationship:
\begin{figure}[h!]
\includegraphics[scale=0.18]{\myFigPath analyze_estim_LOMgain_fitAR_15_Jun_2020}
\caption{AR(1) approximation to the estimated functional form of the gain}
\label{fitAR1}
\end{figure}

But these are very high values; the model isn't E-stable for such high values. Rerunning PEA with the AR(1) specification and the values (1, 0.0123) is able to solve, but involves a gain that's increasing over time!


\clearpage
%\newpage
\section{PEA plots with constant only, $\pi$ only learning}
\begin{figure}[h!]
\subfigure[Taylor rule, observables]{\includegraphics[scale=0.14]{\myFigPath command_pea_implement_anchTC_obs_TR_constant_only_pi_only_14_Jun_2020}}
\hfill % this is great to intro dpace between subfigures
\subfigure[Taylor rule, inverse gain]{\includegraphics[scale=0.14]{\myFigPath command_pea_implement_anchTC_invgain_TR_constant_only_pi_only_14_Jun_2020}}
\subfigure[Optimal sequence, observables]{\includegraphics[scale=0.14]{\myFigPath command_pea_implement_anchTC_obs_anchTC_constant_only_pi_only_14_Jun_2020}}
\hfill
\subfigure[Optimal sequence, inverse gain]{\includegraphics[scale=0.14]{\myFigPath command_pea_implement_anchTC_invgain_anchTC_constant_only_pi_only_14_Jun_2020}}
\caption{}
\label{updated_PEA_constant_only_pi_only}
\end{figure}

\begin{figure}[h!]
\includegraphics[scale=0.18]{\myFigPath compare_value_pea_results_value_outputs_server02_Jun_2020_18_21_35_pea_outputs_14_Jun_2020_09_56_33_pretty_14_Jun_2020}
\caption{VFI-PEA comparison using the ``constant only, $\pi$ only'' PLM}
\label{updated_PEA_VFI_comparison_constant_only_pi_only}
\end{figure}


%%%%%%%%%%%%%%%%%%%%%%%%%%%%%%%%%%%%%%%%%%%%%%%%%%%%%%%%%%%%%                                              APPENDIX
%%%%%%%%%%%%%%%%%%%%%%%%%%%%%%%%%%%%%%%%%%%%%%%%%%%%%%%%%%%%
    \clearpage
%    \newpage
\appendix
% the following command makes equation numbering include the section first, but just for what follows
\numberwithin{equation}{section}
\section{Model summary}

\vspace{-0.5cm}

\begin{align}
x_t &=  -\sigma i_t +\hat{\E}_t \sum_{T=t}^{\infty} \beta^{T-t }\big( (1-\beta)x_{T+1} - \sigma(\beta i_{T+1} - \pi_{T+1}) +\sigma r_T^n \big)  \label{A1}  \\
\pi_t &= \kappa x_t +\hat{\E}_t \sum_{T=t}^{\infty} (\alpha\beta)^{T-t }\big( \kappa \alpha \beta x_{T+1} + (1-\alpha)\beta \pi_{T+1} + u_T\big) \label{A2}  \\
i_t &= \psi_{\pi}\pi_t + \psi_{x} x_t  + \bar{i}_t \label{TR} \quad \quad (\text{if imposed})
\end{align}

\vspace{-1.2cm}

\begin{align}
\text{PLM:} \quad \quad & \hat{\E}_t z_{t+h}  =  a_{t-1} + bh_x^{h-1}s_t  \quad \forall h\geq 1 \quad \quad b = g_x\; h_x \quad \quad  \label{PLM} \\
\text{Updating:} \quad \quad & a_{t}  =a_{t-1} +k_t^{-1}\big(z_{t} -(a_{t-1}+b s_{t-1}) \big)  \label{A5} \\
\text{Anchoring function:} \quad \quad & k^{-1}_t  = \rho_k k^{-1}_{t-1} + \gamma_k fe_{t-1}^2 \label{A6}\\
\text{Forecast error:} \quad \quad & fe_{t-1}  = z_t - (a_{t-1}+b s_{t-1}) \label{A7} \\
\text{LH expectations:} \quad \quad & f_a(t) = \frac{1}{1-\alpha\beta}a_{t-1}  + b(\mathbb{I}_{nx} - \alpha\beta h)^{-1}s_t \quad \quad  f_b(t) = \frac{1}{1-\beta}a_{t-1}  + b(\mathbb{I}_{nx} - \beta h)^{-1}s_t  \label{A8}
\end{align}

\vspace{-0.5cm}

This notation captures vector learning ($z$ learned) for intercept only. For scalar learning, $a_t= \begin{pmatrix} \bar{\pi}_t & 0 & 0\end{pmatrix}' $ and $b_1$ designates the first row of $b$. The observables $(\pi, x)$ are determined as:
\begin{align}
x_t &=  -\sigma i_t + \begin{bmatrix} \sigma & 1-\beta & -\sigma\beta \end{bmatrix} f_b + \sigma \begin{bmatrix} 1 & 0 & 0 \end{bmatrix} (\mathbb{I}_{nx} - \beta h_x)^{-1} s_t \label{A9} \\
\pi_t &= \kappa x_t  + \begin{bmatrix} (1-\alpha)\beta & \kappa\alpha\beta & 0 \end{bmatrix}  f_a + \begin{bmatrix} 0 & 0 & 1 \end{bmatrix}  (\mathbb{I}_{nx} - \alpha \beta h_x)^{-1}  s_t \label{A10}
\end{align}

\section{Target criterion}\label{target_crit_levels}
The target criterion in the simplified model (scalar learning of inflation intercept only, $k_t^{-1} = \mathbf{g}(fe_{t-1})$):
\begin{align*}
\pi_t  = -\frac{\lambda_x}{\kappa}\bigg\{x_t - \frac{(1-\alpha)\beta}{1-\alpha\beta} \bigg(k_t^{-1}+((\pi_t - \bar{\pi}_{t-1}-b_1 s_{t-1}))\mathbf{g}_{\pi}(t) \bigg) \\
\bigg(\E_t\sum_{i=1}^{\infty}x_{t+i}\prod_{j=0}^{i-1}(1-k_{t+1+j}^{-1} - (\pi_{t+1+j} - \bar{\pi}_{t+j}-b_1 s_{t+j})\mathbf{g_{\bar{\pi}}}(t+j)) \bigg)
\bigg\} \numberthis \label{target}
\end{align*}
where I'm using the notation that $\prod_{j=0}^{0} \equiv 1$. For interpretation purposes, let me rewrite this as follows:
\begin{align*}
\pi_t  = & \; \textcolor{red}{-\frac{\lambda_x}{\kappa} x_t} \textcolor{blue}{ \; + \frac{\lambda_x}{\kappa} \frac{(1-\alpha)\beta}{1-\alpha\beta} \bigg(k_t^{-1}+ fe^{eve}_{t|t-1}\mathbf{g}_{\pi}(t) \bigg)\E_t\sum_{i=1}^{\infty}x_{t+i}}  \\
& \textcolor{mygreen}{- \frac{\lambda_x}{\kappa} \frac{(1-\alpha)\beta}{1-\alpha\beta} \bigg(k_t^{-1}+ fe^{eve}_{t|t-1}\mathbf{g}_{\pi}(t) \bigg) \bigg(\E_t\sum_{i=1}^{\infty}x_{t+i}\prod_{j=0}^{i-1}(k_{t+1+j}^{-1} + fe^{eve}_{t+1+j|t+j})\mathbf{g_{\bar{\pi}}}(t+j) \bigg)}
\numberthis \label{target_interpretation}
\end{align*}
Interpretation: \textcolor{red}{tradeoffs from discretion in RE} + \textcolor{blue}{effect of current level and change of the gain on future tradeoffs} + \textcolor{mygreen}{effect of future expected levels and changes of the gain on future tradeoffs}

%\section{A target criterion system for an anchoring function specified for gain changes}\label{target_crit_changes}
%\begin{equation}
%k_t = k_{t-1} + \mathbf{g}(fe_{t|t-1})
%\end{equation}
%Turns out the $k_{t-1}$ adds one $\varphi_{6,t+1}$ too many which makes the target criterion unwieldy. The FOCs of the Ramsey problem are
%\begin{align}
%& 2\pi_t + 2\frac{\lambda}{\kappa}x_t -k_t^{-1} \varphi_{5,t} - \mathbf{g}_{\pi}(t)\varphi_{6,t}  = 0 \label{gaspar22}\\
%& c x_{t+1} + \varphi_{5,t} -(1-k_t^{-1})\varphi_{5,t+1} +\mathbf{g}_{\bar{\pi}}(t)\varphi_{6,t+1} = 0 \label{gaspar21}\\
%& \varphi_{6,t} \; \textcolor{red}{+\; \varphi_{6,t+1}} = fe_t \varphi_{5,t} \label{constraints}
%\end{align}
%where the red multiplier is the new element vis-a-vis the case where the anchoring function is specified in levels ($k_t^{-1} = \mathbf{g}(fe_{t-1})$, as in App. \ref{target_crit_levels}), and I'm using the shorthand notation
%\begin{align}
%c & = -\frac{2(1-\alpha)\beta}{1-\alpha\beta}\frac{\lambda}{\kappa} \\ 
%fe_t & = \pi_t - \bar{\pi}_{t-1}-b s_{t-1}
%\end{align}
%(\ref{gaspar22}) says that in anchoring, the discretion tradeoff is complemented with tradeoffs coming from learning ($\varphi_{5,t}$), which are more binding when expectations are unanchored ($k_{t}^{-1}$ high). Moreover, the change in the anchoring of expectations imposes an additional constraint ($\varphi_{6,t}$), which is more strongly binding if the gain responds strongly to inflation ($\mathbf{g}_{\pi}(t)$).
%One can simplify this three-equation-system to:
%\begin{align}
%\varphi_{6,t} & = -c fe_t x_{t+1} + \bigg(1+ \frac{fe_t}{fe_{t+1}}(1-k_{t+1}^{-1}) -fe_t \mathbf{g}_{\bar{\pi}}(t) \bigg) \varphi_{6,t+1} -\frac{fe_t}{fe_{t+1}}(1-k_{t+1}^{-1})\varphi_{6,t+2}\label{6'} \\
%0 & = 2\pi_t + 2\frac{\lambda}{\kappa}x_t   - \bigg( \frac{k_t^{-1}}{fe_t} + \mathbf{g}_{\pi}(t)\bigg)\varphi_{6,t} + \frac{k_t^{-1}}{fe_t}\varphi_{6,t+1}\label{1'}
%\end{align}
%Unfortunately, I haven't been able to solve (\ref{6'}) for $\varphi_{6,t}$ and therefore I can't express the target criterion so nicely as before. The only thing I can say is to direct the targeting rule-following central bank to compute $\varphi_{6,t}$ as the solution to (\ref{1'}), and then evaluate (\ref{6'}) as a target criterion. The solution to (\ref{1'}) is given by:
%\begin{equation}
%\varphi_{6,t} = -2\E_t\sum_{i=0}^{\infty}(\pi_{t+i}+\frac{\lambda_x}{\kappa}x_{t+i})\prod_{j=0}^{i-1}\frac{\frac{k_{t+j}^{-1}}{fe_{t+j}}}{\frac{k_{t+j}^{-1}}{fe_{t+j}} + \mathbf{g}_{\pi}(t+j)} \label{sol1'}
%\end{equation}
%Interpretation: the anchoring constraint is not binding ($\varphi_{6,t}=0$) if the CB always hits the target (
%$\pi_{t+i}+\frac{\lambda_x}{\kappa}x_{t+i} = 0 \quad \forall i$); or expectations are always anchored ($k_{t+j}^{-1}=0 \quad \forall j$). 


\end{document}

%%%%%%%%%%%%%    SUBFIGURE  %%%%%%%%%%%
%\begin{figure}[h!]
%\subfigure[Hodrick-Prescott, $\lambda=1600$]{\includegraphics[scale=\myAdjustableFigScale]{\myFigPath materials22_gain_dhat_HP}}
%\hfill % this is great to intro dpace between subfigures
%\subfigure[Hamilton, 4 lags, $h=8$]{\includegraphics[scale=\myAdjustableFigScale]{\myFigPath materials22_gain_dhat_Hamilton}}
%\subfigure[Baxter-King, $(6,32)$ quarters, truncation at 12 lags]{\includegraphics[scale=\myAdjustableFigScale]{\myFigPath materials22_gain_dhat_BK}}
%\caption{Inverse gain for $\hat{d}$ for the different filters}
%\end{figure}

%%%%%%%%%%%%%    TABLE  %%%%%%%%%%%
%\begin{center}
%\begin{table}[h!]
%\caption{$\hat{d}$}
%\begin{tabular}{ c |c |c }
%  & $W = I$ & $W = \text{diag}(\hat{\sigma}_{ac(0)}, \dots, \hat{\sigma}_{ac(K)})$ \\ 
%  \hline
% HP & 77.7899 & 10 \\  
% \hline
% Hamilton & 32.1649 & 10 \\  
% \hline
% BK & 90.3929 & 10    
%\end{tabular}
%\end{table}
%\end{center}





