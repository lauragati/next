\documentclass[11pt]{article}
\usepackage{amsmath, amsthm, amssymb,lscape, natbib}
\usepackage{mathtools}
\usepackage{subfigure}
\usepackage[font=footnotesize,labelfont=bf]{caption}
\usepackage{graphicx}
\usepackage{colortbl}
\usepackage{hhline}
\usepackage{multirow}
\usepackage{multicol}
\usepackage{setspace}
\usepackage[final]{pdfpages}
\usepackage[left=2.5cm,top=2.5cm,right=2.5cm, bottom=2.5cm]{geometry}
\usepackage{natbib} 
\usepackage{bibentry} 
\newcommand{\bibverse}[1]{\begin{verse} \bibentry{#1} \end{verse}}
\newcommand{\vs}{\vspace{.3in}}
\renewcommand{\ni}{\noindent}
\usepackage{xr-hyper}
\usepackage[]{hyperref}
\hypersetup{
    colorlinks=true,
    linkcolor=blue,
    filecolor=magenta,      
    urlcolor=cyan,
}
 
\urlstyle{same}
\usepackage[capposition=top]{floatrow}
\usepackage{amssymb}
\usepackage{relsize}
\usepackage[dvipsnames]{xcolor}
\usepackage{fancyhdr}
\usepackage{tikz}
 
%\pagestyle{fancy} % customize header and footer
%\fancyhf{} % clear initial header and footer
%%\rhead{Overleaf}
%\lhead{\centering \rightmark} % this adds subsection number and name
%\lfoot{\centering \rightmark} 
%\rfoot{\thepage} % put page number (the centering command puts it in the middle, don't matter if you put it in right or left footer)

\def \myFigPath {../../figures/} 
% BE CAREFUL WITH FIGNAMES, IN LATEX THEY'RE NOT CASE SENSITIVE!!
\def \myTablePath {../../tables/} 

%\definecolor{mygreen}{RGB}{0, 100, 0}
\definecolor{mygreen}{RGB}{0, 128, 0}

\definecolor{citec}{rgb}{0,0,.5}
\definecolor{linkc}{rgb}{0,0,.6}
\definecolor{bcolor}{rgb}{1,1,1}
\hypersetup{
%hidelinks = true
  colorlinks = true,
  urlcolor=linkc,
  linkcolor=linkc,
  citecolor = citec,
  filecolor = linkc,
  pdfauthor={Laura G\'ati},
}


\geometry{left=.5in,right=.5in,top=0.5in,
bottom=0.5in}
\linespread{1.5}
\renewcommand{\[}{\begin{equation}}
\renewcommand{\]}{\end{equation}}

% New Options
\newtheorem{prop}{Proposition}
\newtheorem{definition}{Definition}[section]
\newtheorem*{remark}{Remark}
\newtheorem{lemma}{Lemma}
\newtheorem{corollary}{Corollary}
\newtheorem{conjecture}{Conjecture}

%\newtheorem{theorem}{Theorem}[section] % the third argument specifies that their number will be adopted to the section
%\newtheorem{corollary}{Corollary}[theorem]
%\newtheorem{lemma}[theorem]{Lemma}
%\declaretheorem{proposition}
%\linespread{1.3}
%\raggedbottom
%\font\reali=msbm10 at 12pt

% New Commands
\newcommand{\real}{\hbox{\reali R}}
\newcommand{\realp}{\hbox{\reali R}_{\scriptscriptstyle +}}
\newcommand{\realpp}{\hbox{\reali R}_{\scriptscriptstyle ++}}
\newcommand{\R}{\mathbb{R}}
\DeclareMathOperator{\E}{\mathbb{E}}
\DeclareMathOperator{\argmin}{arg\,min}
\newcommand\w{3.0in}
\newcommand\wnum{3.0}
\def\myFigWidth{5.3in}
\def\mySmallerFigWidth{2.1in}
\def\myEvenBiggerFigScale{0.8}
\def\myPointSixFigScale{0.6}
\def\myBiggerFigScale{0.4}
\def\myFigScale{0.3}
\def\myMediumFigScale{0.25}
\def\mySmallFigScale{0.22}
\def\mySmallerFigScale{0.18}
\def\myTinyFigScale{0.16}
\def\myPointFourteenFigScale{0.14}
\def\myTinierFigScale{0.12}
\def\myAdjustableFigScale{0.16}
\newcommand\numberthis{\addtocounter{equation}{1}\tag{\theequation}} % this defines a command to make align only number this line
\newcommand{\code}[1]{\texttt{#1}} %code %

\renewcommand*\contentsname{Overview}
\setcounter{tocdepth}{2}

% define a command to make a huge question mark (it works in math mode)
\newcommand{\bigqm}[1][1]{\text{\larger[#1]{\textbf{?}}}}

% use package and define command to add blank page
\usepackage{afterpage}
\newcommand\blankpage{%
    \null
    \thispagestyle{empty}%
    \addtocounter{page}{-1}%
    \newpage}

\begin{document}

\linespread{1.0}

\title{Materials 44 - I think I've got the estimation}
\author{Laura G\'ati} 
\date{\today}
\maketitle

%%%%%%%%%%%%%%%%%%%%             DOCUMENT           %%%%%%%%%%%%%%%%%% 

\tableofcontents

%\listoffigures

\newpage



\clearpage
\section{Inconsistency was: didn't annualize expectations in true data}

\begin{figure}[h!]
\subfigure[]{\includegraphics[scale=0.34]{\myFigPath alphas_N_100_nfe_5_gridspacing_manual_Wdiffs2_100000_Wmid_1000_Nsimulations_command_GMM_LOMgain_univariate_31_Aug_2020_14_31_06}}
\subfigure[]{\includegraphics[scale=0.34]{\myFigPath autocovariogram_sim_N_100_nfe_5_gridspacing_manual_Wdiffs2_100000_Wmid_1000_Nsimulations_command_GMM_LOMgain_univariate_31_Aug_2020_14_31_06}}
\caption{Calibration C, use expectations, ridge tuning = 0.01, initialize at truth}
\end{figure}

\clearpage
\begin{figure}[h!]
\subfigure[]{\includegraphics[scale=0.34]{\myFigPath alphas_N_100_nfe_5_gridspacing_manual_Wdiffs2_100000_Wmid_1000_Nsimulations_command_GMM_LOMgain_univariate_09_Sep_2020_16_03_25}}
\subfigure[]{\includegraphics[scale=0.34]{\myFigPath autocovariogram_sim_N_100_nfe_5_gridspacing_manual_Wdiffs2_100000_Wmid_1000_Nsimulations_command_GMM_LOMgain_univariate_09_Sep_2020_16_03_25}}
\caption{Calibration C, use expectations, ridge tuning = 0.01, \colorbox{yellow}{initialize above truth, annualize expectations in true data}}
\end{figure}

I think that's success!

\clearpage
\subsection{Check with same seed as true data and $N=1$}
\begin{figure}[h!]
\subfigure[]{\includegraphics[scale=0.34]{\myFigPath alphas_N_1_nfe_5_gridspacing_manual_Wdiffs2_100000_Wmid_1000_Nsimulations_command_GMM_LOMgain_univariate_09_Sep_2020_16_19_05}}
\subfigure[]{\includegraphics[scale=0.34]{\myFigPath autocovariogram_sim_N_1_nfe_5_gridspacing_manual_Wdiffs2_100000_Wmid_1000_Nsimulations_command_GMM_LOMgain_univariate_09_Sep_2020_16_19_05}}
\caption{Calibration C, use expectations, ridge tuning = 0.01, \colorbox{yellow}{initialize at truth}, annualize expectations in true data}
\end{figure}

Yes!

\clearpage
\section{Real data}

\begin{figure}[h!]
\subfigure[]{\includegraphics[scale=0.34]{\myFigPath alph_opt_N_100_nfe_5_gridspacing_manual_Wdiffs2_100000_Wmid_1000_Nsimulations_coax_solver_server_11_Sep_2020_10_10_18}}
\subfigure[]{\includegraphics[scale=0.34]{\myFigPath autocovariogram_N_100_nfe_5_gridspacing_manual_Wdiffs2_100000_Wmid_1000_Nsimulations_coax_solver_server_11_Sep_2020_10_10_18}}
\caption{Settings from Fig 2,  \colorbox{yellow}{10 different starting points, showing top 3 (blue is best), $N=100$}}
\end{figure}

Darn. It doesn't hit the moments I want it to. So let's manually re-weight the own-autocovariances.


\clearpage
\begin{figure}[h!]
\subfigure[]{\includegraphics[scale=0.34]{\myFigPath alph_opt_N_100_nfe_5_gridspacing_manual_Wdiffs2_100000_Wmid_1000_Nsimulations_coax_solver_server_11_Sep_2020_13_57_47}}
\subfigure[]{\includegraphics[scale=0.34]{\myFigPath autocovariogram_N_100_nfe_5_gridspacing_manual_Wdiffs2_100000_Wmid_1000_Nsimulations_coax_solver_server_11_Sep_2020_13_57_47}}
\caption{Settings from Fig 2,  5 different starting points, showing top 3 (blue is best), $N=100$, \colorbox{yellow}{manually putting more weight on own autocovariances}}
\end{figure}

\clearpage
\begin{figure}[h!]
\subfigure[]{\includegraphics[scale=0.34]{\myFigPath alph_opt_N_500_nfe_5_gridspacing_manual_Wdiffs2_100000_Wmid_1000_Nsimulations_coax_solver_server_11_Sep_2020_15_46_40}}
\subfigure[]{\includegraphics[scale=0.34]{\myFigPath autocovariogram_N_500_nfe_5_gridspacing_manual_Wdiffs2_100000_Wmid_1000_Nsimulations_coax_solver_server_11_Sep_2020_15_46_40}}
\caption{Settings from Fig 2,  5 different starting points, showing top 3 (blue is best),  \colorbox{yellow}{$N=500$},manually putting more weight on own autocovariances}
\end{figure}

This took 5h:12min.

$\hat{\alpha}_i = (0.3346;    0.2513;    0.001;    0.2399;    0.3277)$

Loss(PEA), estimate = 4.7524 ($N=100, T=100$)


\clearpage
\begin{figure}[h!]
\subfigure[]{\includegraphics[scale=0.34]{\myFigPath alph_opt_aloneN_500_nfe_5_gridspacing_manual_Wdiffs2_100000_Wmid_1000_Nsimulations_coax_solver_server_11_Sep_2020_15_46_40}}
\subfigure[]{\includegraphics[scale=0.34]{\myFigPath autocovariogram_alone_N_500_nfe_5_gridspacing_manual_Wdiffs2_100000_Wmid_1000_Nsimulations_coax_solver_server_11_Sep_2020_15_46_40}}
\caption{Same as the previous, just the best candidate alone}
\end{figure}

\clearpage
\begin{figure}[h!]
\subfigure[]{\includegraphics[scale=0.34]{\myFigPath alph_opt_N_100_nfe_5_gridspacing_manual_Wdiffs2_100000_Wmid_1000_Nsimulations_coax_solver_server_15_Sep_2020_14_25_06}}
\subfigure[]{\includegraphics[scale=0.34]{\myFigPath autocovariogram_N_100_nfe_5_gridspacing_manual_Wdiffs2_100000_Wmid_1000_Nsimulations_coax_solver_server_15_Sep_2020_14_25_06}}
\caption{Settings from Fig 2,  \colorbox{yellow}{20} different starting points, showing top 3 (blue is best),  \colorbox{yellow}{$N=100$}, manually putting more weight on own autocovariances}
\end{figure}


$\hat{\alpha}_i = (0.2621;    0.1847;    0;    0.2115;    0.2817)$

\clearpage
\begin{figure}[h!]
\subfigure[]{\includegraphics[scale=0.34]{\myFigPath alph_opt_N_1000_nfe_5_gridspacing_manual_Wdiffs2_100000_Wmid_1000_Nsimulations_coax_solver_server_15_Sep_2020_16_14_00}}
\subfigure[]{\includegraphics[scale=0.34]{\myFigPath autocovariogram_N_1000_nfe_5_gridspacing_manual_Wdiffs2_100000_Wmid_1000_Nsimulations_coax_solver_server_15_Sep_2020_16_14_00}}
\caption{Settings from Fig 2,  \colorbox{yellow}{10} different starting points, showing top 3 (blue is best),  \colorbox{yellow}{$N=1000$}, manually putting more weight on own autocovariances}
\end{figure}


$\hat{\alpha}_i = (0.8161;    0.6133    0;    0.3342;    0.4452)$

(Saved as \texttt{estim\_LOMgain\_outputs\_univariate\_coax15\_Sep\_2020\_16\_14\_00.mat}, going to call this the ``complete Materials 44 candidate'' to differentiate from the older ``Materials 44 candidate''.)


\clearpage
\section{Policy isn't a function of $k^{-1}_t$}
%%%%%%%%%%%%%%%%%%%%%%
The anchoring function is (\ref{A6}): $k^{-1}_t  = \sum_i \alpha_i b_i(fe_{t|t-1})$. This essentially eliminates $k$ as a state variable. 



%%%%%%%%%%%               BLANK   %%%%%%%%%%%%%%%%%%%%%%%%%%%%%%
% adds blank page
\afterpage{\blankpage}


%%%%%%%%%%%%%%%%%%%%%%%%%%%%%%%%%%%%%%%%%%%%%%%%%%%%%%%%%%%%%                                              APPENDIX
%%%%%%%%%%%%%%%%%%%%%%%%%%%%%%%%%%%%%%%%%%%%%%%%%%%%%%%%%%%%
    \clearpage
%    \newpage
\appendix
% the following command makes equation numbering include the section first, but just for what follows
\numberwithin{equation}{section}
\section{Model summary}

\vspace{-0.5cm}

\begin{align}
x_t &=  -\sigma i_t +\hat{\E}_t \sum_{T=t}^{\infty} \beta^{T-t }\big( (1-\beta)x_{T+1} - \sigma(\beta i_{T+1} - \pi_{T+1}) +\sigma r_T^n \big)  \label{A1}  \\
\pi_t &= \kappa x_t +\hat{\E}_t \sum_{T=t}^{\infty} (\alpha\beta)^{T-t }\big( \kappa \alpha \beta x_{T+1} + (1-\alpha)\beta \pi_{T+1} + u_T\big) \label{A2}  \\
i_t &= \psi_{\pi}\pi_t + \psi_{x} x_t  + \bar{i}_t \label{TR} \quad \quad (\text{if imposed})
\end{align}

\vspace{-1.2cm}

\begin{align}
\text{PLM:} \quad \quad & \hat{\E}_t z_{t+h}  =  a_{t-1} + bh_x^{h-1}s_t  \quad \forall h\geq 1 \quad \quad b = g_x\; h_x \quad \quad  \label{PLM} \\
\text{Updating:} \quad \quad & a_{t}  =a_{t-1} +k_t^{-1}\big(z_{t} -(a_{t-1}+b s_{t-1}) \big)  \label{A5} \\
\text{Anchoring function:} \quad \quad & k^{-1}_t  = \sum_i \alpha_i b_i(fe_{t|t-1}) \label{A6}\\
\text{Forecast error:} \quad \quad & fe_{t-1}  = z_t - (a_{t-1}+b s_{t-1}) \label{A7} \\
\text{LH expectations:} \quad \quad & f_a(t) = \frac{1}{1-\alpha\beta}a_{t-1}  + b(\mathbb{I}_{nx} - \alpha\beta h)^{-1}s_t \quad \quad  f_b(t) = \frac{1}{1-\beta}a_{t-1}  + b(\mathbb{I}_{nx} - \beta h)^{-1}s_t  \label{A8}
\end{align}

\vspace{-0.5cm}

This notation captures vector learning ($z$ learned) for intercept only. For scalar learning, $a_t= \begin{pmatrix} \bar{\pi}_t & 0 & 0\end{pmatrix}' $ and $b_1$ designates the first row of $b$. The observables $(\pi, x)$ are determined as:
\begin{align}
x_t &=  -\sigma i_t + \begin{bmatrix} \sigma & 1-\beta & -\sigma\beta \end{bmatrix} f_b + \sigma \begin{bmatrix} 1 & 0 & 0 \end{bmatrix} (\mathbb{I}_{nx} - \beta h_x)^{-1} s_t \label{A9} \\
\pi_t &= \kappa x_t  + \begin{bmatrix} (1-\alpha)\beta & \kappa\alpha\beta & 0 \end{bmatrix}  f_a + \begin{bmatrix} 0 & 0 & 1 \end{bmatrix}  (\mathbb{I}_{nx} - \alpha \beta h_x)^{-1}  s_t \label{A10}
\end{align}

\section{Target criterion}\label{target_crit_levels}
The target criterion in the simplified model (scalar learning of inflation intercept only, $k_t^{-1} = \mathbf{g}(fe_{t-1})$):
\begin{align*}
\pi_t  = -\frac{\lambda_x}{\kappa}\bigg\{x_t - \frac{(1-\alpha)\beta}{1-\alpha\beta} \bigg(k_t^{-1}+((\pi_t - \bar{\pi}_{t-1}-b_1 s_{t-1}))\mathbf{g}_{\pi}(t) \bigg) \\
\bigg(\E_t\sum_{i=1}^{\infty}x_{t+i}\prod_{j=0}^{i-1}(1-k_{t+1+j}^{-1} - (\pi_{t+1+j} - \bar{\pi}_{t+j}-b_1 s_{t+j})\mathbf{g_{\bar{\pi}}}(t+j)) \bigg)
\bigg\} \numberthis \label{target}
\end{align*}
where I'm using the notation that $\prod_{j=0}^{0} \equiv 1$. For interpretation purposes, let me rewrite this as follows:
\begin{align*}
\pi_t  = & \; \textcolor{red}{-\frac{\lambda_x}{\kappa} x_t} \textcolor{blue}{ \; + \frac{\lambda_x}{\kappa} \frac{(1-\alpha)\beta}{1-\alpha\beta} \bigg(k_t^{-1}+ fe^{eve}_{t|t-1}\mathbf{g}_{\pi}(t) \bigg)\E_t\sum_{i=1}^{\infty}x_{t+i}}  \\
& \textcolor{mygreen}{- \frac{\lambda_x}{\kappa} \frac{(1-\alpha)\beta}{1-\alpha\beta} \bigg(k_t^{-1}+ fe^{eve}_{t|t-1}\mathbf{g}_{\pi}(t) \bigg) \bigg(\E_t\sum_{i=1}^{\infty}x_{t+i}\prod_{j=0}^{i-1}(k_{t+1+j}^{-1} + fe^{eve}_{t+1+j|t+j})\mathbf{g_{\bar{\pi}}}(t+j) \bigg)}
\numberthis \label{target_interpretation}
\end{align*}
Interpretation: \textcolor{red}{tradeoffs from discretion in RE} + \textcolor{blue}{effect of current level and change of the gain on future tradeoffs} + \textcolor{mygreen}{effect of future expected levels and changes of the gain on future tradeoffs}







\end{document}





