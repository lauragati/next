\documentclass[11pt]{article}
\usepackage{amsmath, amsthm, amssymb,lscape, natbib}
\usepackage{mathtools}
\usepackage{subfigure}
\usepackage[font=footnotesize,labelfont=bf]{caption}
\usepackage{graphicx}
\usepackage{colortbl}
\usepackage{hhline}
\usepackage{multirow}
\usepackage{multicol}
\usepackage{setspace}
\usepackage[final]{pdfpages}
\usepackage[left=2.5cm,top=2.5cm,right=2.5cm, bottom=2.5cm]{geometry}
\usepackage{natbib} 
\usepackage{bibentry} 
\newcommand{\bibverse}[1]{\begin{verse} \bibentry{#1} \end{verse}}
\newcommand{\vs}{\vspace{.3in}}
\renewcommand{\ni}{\noindent}
\usepackage{xr-hyper}
\usepackage[]{hyperref}
\hypersetup{
    colorlinks=true,
    linkcolor=blue,
    filecolor=magenta,      
    urlcolor=cyan,
}
 
\urlstyle{same}
\usepackage[capposition=top]{floatrow}
\usepackage{amssymb}
\usepackage{relsize}
\usepackage[dvipsnames]{xcolor}
\usepackage{fancyhdr}
\usepackage{tikz}
 
\pagestyle{fancy} % customize header and footer
\fancyhf{} % clear initial header and footer
%\rhead{Overleaf}
\lhead{\centering \rightmark} % this adds subsection number and name
\lfoot{\centering \rightmark} 
\rfoot{\thepage} % put page number (the centering command puts it in the middle, don't matter if you put it in right or left footer)

\def \myFigPath {../figures/} 
% BE CAREFUL WITH FIGNAMES, IN LATEX THEY'RE NOT CASE SENSITIVE!!
\def \myTablePath {../tables/} 

%\definecolor{mygreen}{RGB}{0, 100, 0}
\definecolor{mygreen}{RGB}{0, 128, 0}

\definecolor{citec}{rgb}{0,0,.5}
\definecolor{linkc}{rgb}{0,0,.6}
\definecolor{bcolor}{rgb}{1,1,1}
\hypersetup{
%hidelinks = true
  colorlinks = true,
  urlcolor=linkc,
  linkcolor=linkc,
  citecolor = citec,
  filecolor = linkc,
  pdfauthor={Laura G\'ati},
}


\geometry{left=.83in,right=.89in,top=1in,
bottom=1in}
\linespread{1.5}
\renewcommand{\[}{\begin{equation}}
\renewcommand{\]}{\end{equation}}

% New Options
\newtheorem{prop}{Proposition}
\newtheorem{definition}{Definition}[section]
\newtheorem*{remark}{Remark}
\newtheorem{lemma}{Lemma}
\newtheorem{corollary}{Corollary}
\newtheorem{conjecture}{Conjecture}

%\newtheorem{theorem}{Theorem}[section] % the third argument specifies that their number will be adopted to the section
%\newtheorem{corollary}{Corollary}[theorem]
%\newtheorem{lemma}[theorem]{Lemma}
%\declaretheorem{proposition}
%\linespread{1.3}
%\raggedbottom
%\font\reali=msbm10 at 12pt

% New Commands
\newcommand{\real}{\hbox{\reali R}}
\newcommand{\realp}{\hbox{\reali R}_{\scriptscriptstyle +}}
\newcommand{\realpp}{\hbox{\reali R}_{\scriptscriptstyle ++}}
\newcommand{\R}{\mathbb{R}}
\DeclareMathOperator{\E}{\mathbb{E}}
\DeclareMathOperator{\argmin}{arg\,min}
\newcommand\w{3.0in}
\newcommand\wnum{3.0}
\def\myFigWidth{5.3in}
\def\mySmallerFigWidth{2.1in}
\def\myEvenBiggerFigScale{0.8}
\def\myPointSixFigScale{0.6}
\def\myBiggerFigScale{0.4}
\def\myFigScale{0.3}
\def\myMediumFigScale{0.25}
\def\mySmallFigScale{0.22}
\def\mySmallerFigScale{0.18}
\def\myTinyFigScale{0.16}
\def\myPointFourteenFigScale{0.14}
\def\myTinierFigScale{0.12}
\def\myAdjustableFigScale{0.14}
\newcommand\numberthis{\addtocounter{equation}{1}\tag{\theequation}} % this defines a command to make align only number this line
\newcommand{\code}[1]{\texttt{#1}} %code %

\renewcommand*\contentsname{Overview}
\setcounter{tocdepth}{2}

% define a command to make a huge question mark (it works in math mode)
\newcommand{\bigqm}[1][1]{\text{\larger[#1]{\textbf{?}}}}

\begin{document}

\linespread{1.0}

\title{Materials 22 for Peter - Results and impossibilities so far}
\author{Laura G\'ati} 
\date{\today}
\maketitle

%%%%%%%%%%%%%%%%%%%%             DOCUMENT           %%%%%%%%%%%%%%%%%% 

% \tableofcontents

%\listoffigures

\section{Results and impossibilities}
\begin{enumerate}
\item[] Analytical
\item \underline{Commitment solution doesn't exist under learning (we already saw)}
\item \underline{Cannot solve for Ramsey policy because an endogenous gain makes the problem nonlinear} \\
Let $f_t$ be the expectation of the private sector. Then with an endogenous gain, one of the model equations is
\begin{equation}
f_t = f_{t-1} + k_t^{-1}(\pi_t - f_{t-1}) \label{simpleRLS}
\end{equation}
\item \underline{For Ramsey policy, can solve for target criterion}
\\ (Woodford's book, Svensson). For a simplified version of the model where agents only learn the mean of inflation, it is
\begin{align*}
\pi_t  = -\frac{\lambda_x}{\kappa}\bigg\{x_t - \frac{(1-\alpha)\beta}{1-\alpha\beta} \bigg(k_t^{-1}+((\pi_t - \bar{\pi}_{t-1}-b_1 s_{t-1}))\mathbf{g}_{\pi}(t) \bigg) \\
\bigg(\E_t\sum_{i=1}^{\infty}x_{t+i}\prod_{j=1}^{i-1}(1-k_{t+j}^{-1}(\pi_{t+1+j} - \bar{\pi}_{t+j}-b_1 s_{t+j})) \bigg)
\bigg\} \numberthis \label{target}
\end{align*}
Here $\mathbf{g}$ is an unspecified anchoring function that determines how the gain changes as a function of the forecast error. $\mathbf{g_{\pi}}$ is its derivative wrt. $\pi$.
\item (No commitment would imply renewed interest in Taylor-type rules, but cannot solve for optimal Taylor-rule coefficients because 1) cannot solve for optimal time paths of observables 2) ``noninertial plan" \`a la Woodford is too restrictive for a learning model.)
\item [] Numerical
\item \underline{With anchoring, central bank's loss function is U-shaped in $\psi_{\pi}$ (Taylor-rule coefficient on inflation)} \\
Central bank optimally less aggressive on inflation under anchoring than under RE.
\end{enumerate}

\section{Where to go from here}
\begin{enumerate}
\item I know that optimal policy is characterized by the target criterion (\ref{target}). Can check how close a Taylor-rule (or alternative rules) can come to implementing it. 
\item From problem to possibility: estimating an anchoring function\\
The anchoring criterion I used so far was the CUSUM-inspired discrete criterion $\rightarrow$ analytical work required replacement by a smooth alternative, $\mathbf{g}$. Now I can try to learn what that object looks like. Right now I am estimating via GMM a very simplistic function of the form:
\begin{equation}
k_t = k_{t+1} + \frac{1}{(d \; fe_{t-1})^2}
\end{equation}
where $fe_{t-1}$ is the most recent forecast error and $d$ is the only parameter I'm estimating.
\end{enumerate}




\end{document}





