\documentclass[11pt]{article}
\usepackage{amsmath, amsthm, amssymb,lscape, natbib}
\usepackage{mathtools}
\usepackage{subfigure}
\usepackage[font=footnotesize,labelfont=bf]{caption}
\usepackage{graphicx}
\usepackage{colortbl}
\usepackage{hhline}
\usepackage{multirow}
\usepackage{multicol}
\usepackage{setspace}
\usepackage[final]{pdfpages}
\usepackage[left=2.5cm,top=2.5cm,right=2.5cm, bottom=2.5cm]{geometry}
\usepackage{natbib} 
\usepackage{bibentry} 
\newcommand{\bibverse}[1]{\begin{verse} \bibentry{#1} \end{verse}}
\newcommand{\vs}{\vspace{.3in}}
\renewcommand{\ni}{\noindent}
\usepackage{xr-hyper}
\usepackage[]{hyperref}
\hypersetup{
    colorlinks=true,
    linkcolor=blue,
    filecolor=magenta,      
    urlcolor=cyan,
}
 
\urlstyle{same}
\usepackage[capposition=top]{floatrow}
\usepackage{amssymb}
\usepackage{relsize}
\usepackage[dvipsnames]{xcolor}
\usepackage{fancyhdr}
\usepackage{tikz}
 
%\pagestyle{fancy} % customize header and footer
%\fancyhf{} % clear initial header and footer
%%\rhead{Overleaf}
%\lhead{\centering \rightmark} % this adds subsection number and name
%\lfoot{\centering \rightmark} 
%\rfoot{\thepage} % put page number (the centering command puts it in the middle, don't matter if you put it in right or left footer)

\def \myFigPath {../../figures/} 
% BE CAREFUL WITH FIGNAMES, IN LATEX THEY'RE NOT CASE SENSITIVE!!
\def \myTablePath {../../tables/} 

%\definecolor{mygreen}{RGB}{0, 100, 0}
\definecolor{mygreen}{RGB}{0, 128, 0}

\definecolor{citec}{rgb}{0,0,.5}
\definecolor{linkc}{rgb}{0,0,.6}
\definecolor{bcolor}{rgb}{1,1,1}
\hypersetup{
%hidelinks = true
  colorlinks = true,
  urlcolor=linkc,
  linkcolor=linkc,
  citecolor = citec,
  filecolor = linkc,
  pdfauthor={Laura G\'ati},
}


\geometry{left=.5in,right=.5in,top=0.5in,
bottom=0.5in}
\linespread{1.5}
\renewcommand{\[}{\begin{equation}}
\renewcommand{\]}{\end{equation}}

% New Options
\newtheorem{prop}{Proposition}
\newtheorem{definition}{Definition}[section]
\newtheorem*{remark}{Remark}
\newtheorem{lemma}{Lemma}
\newtheorem{corollary}{Corollary}
\newtheorem{conjecture}{Conjecture}

%\newtheorem{theorem}{Theorem}[section] % the third argument specifies that their number will be adopted to the section
%\newtheorem{corollary}{Corollary}[theorem]
%\newtheorem{lemma}[theorem]{Lemma}
%\declaretheorem{proposition}
%\linespread{1.3}
%\raggedbottom
%\font\reali=msbm10 at 12pt

% New Commands
\newcommand{\real}{\hbox{\reali R}}
\newcommand{\realp}{\hbox{\reali R}_{\scriptscriptstyle +}}
\newcommand{\realpp}{\hbox{\reali R}_{\scriptscriptstyle ++}}
\newcommand{\R}{\mathbb{R}}
\DeclareMathOperator{\E}{\mathbb{E}}
\DeclareMathOperator{\argmin}{arg\,min}
\newcommand\w{3.0in}
\newcommand\wnum{3.0}
\def\myFigWidth{5.3in}
\def\mySmallerFigWidth{2.1in}
\def\myEvenBiggerFigScale{0.8}
\def\myPointSixFigScale{0.6}
\def\myBiggerFigScale{0.4}
\def\myFigScale{0.3}
\def\myMediumFigScale{0.25}
\def\mySmallFigScale{0.22}
\def\mySmallerFigScale{0.18}
\def\myTinyFigScale{0.16}
\def\myPointFourteenFigScale{0.14}
\def\myTinierFigScale{0.12}
\def\myAdjustableFigScale{0.16}
\newcommand\numberthis{\addtocounter{equation}{1}\tag{\theequation}} % this defines a command to make align only number this line
\newcommand{\code}[1]{\texttt{#1}} %code %

\renewcommand*\contentsname{Overview}
\setcounter{tocdepth}{2}

% define a command to make a huge question mark (it works in math mode)
\newcommand{\bigqm}[1][1]{\text{\larger[#1]{\textbf{?}}}}

% use package and define command to add blank page
\usepackage{afterpage}
\newcommand\blankpage{%
    \null
    \thispagestyle{empty}%
    \addtocounter{page}{-1}%
    \newpage}

\begin{document}

\linespread{1.0}

\title{Materials 40 - Still trying to understand why not identified}
\author{Laura G\'ati} 
\date{\today}
\maketitle

%%%%%%%%%%%%%%%%%%%%             DOCUMENT           %%%%%%%%%%%%%%%%%% 

\tableofcontents

%\listoffigures

\newpage

\section{Loss when varying one parameter, more details}

\begin{figure}[h!]
\subfigure[\colorbox{yellow}{Reference figure from Materials 39: increment around true value 0.02}]{\includegraphics[scale=0.32]{\myFigPath loss_for_indi_alphas_others_at_true_dontrescale_constant_only_pi_only_N_100_nfe_5femax_2_loss_0_gridspacing_uniform_Wdiffs2_100000_Wmid_0_Nsimulations_scaleW_0_use_expectations_0_use_meas_error_0_command_GMM_LOMgain_univariate_04_Aug_2020}}
\subfigure[\colorbox{yellow}{Minima: $(0.041; 0.0187; 0 ; 0.0218 ;0.041)$} ]{\includegraphics[scale=0.32]{\myFigPath loss_indi_constant_only_pi_only_N_100_nfe_5femax_2_loss_944_gridspacing_uniform_Wdiffs2_100000_Wmid_0_Nsimulations_scaleW_0_use_expectations_0_use_meas_error_0_command_GMM_LOMgain_univariate_07_Aug_2020}}

\caption{Loss for $N=100$, \colorbox{yellow}{NOT} using 1-step ahead forecasts of inflation, estimate mean moments once, imposing convexity with weight 100K, w/o measurement error, truth with $nfe=5, fe \in(-2,2)$}
\end{figure}

\clearpage
\subsection{Convexity and 0 at 0 restrictions}
\begin{figure}[h!]
\subfigure[\colorbox{yellow}{Shutting of the convexity moment (weight 0)}]{\includegraphics[scale=0.32]{\myFigPath loss_indi_nrange20_constant_only_pi_only_N_100_nfe_5femax_2_loss_944_gridspacing_uniform_Wdiffs2_0_Wmid_0_Nsimulations_scaleW_0_use_expectations_0_use_meas_error_0_command_GMM_LOMgain_univariate_07_Aug_2020}}
\subfigure[\colorbox{yellow}{Adding the 0 at 0 restriction (weight 1000)} ]{\includegraphics[scale=0.32]{\myFigPath loss_indi_nrange10_constant_only_pi_only_N_100_nfe_5femax_2_loss_944_gridspacing_uniform_Wdiffs2_100000_Wmid_1000_Nsimulations_scaleW_0_use_expectations_0_use_meas_error_0_command_GMM_LOMgain_univariate_07_Aug_2020}}
%\subfigure[\colorbox{yellow}{Shutting of the convexity moment and adding the 0 at 0 restriction} ]{\includegraphics[scale=0.26]{\myFigPath loss_indi_nrange10_constant_only_pi_only_N_100_nfe_5femax_2_loss_944_gridspacing_uniform_Wdiffs2_0_Wmid_1000_Nsimulations_scaleW_0_use_expectations_0_use_meas_error_0_command_GMM_LOMgain_univariate_07_Aug_2020}}
\caption{Variations I}
\end{figure}

\clearpage
\subsection{Measurement error and rescale W}

\begin{figure}[h!]
\subfigure[\colorbox{yellow}{Add measurement error}]{\includegraphics[scale=0.32]{\myFigPath loss_indi_nrange20_constant_only_pi_only_N_100_nfe_5femax_2_loss_944_gridspacing_uniform_Wdiffs2_100000_Wmid_0_Nsimulations_scaleW_0_use_expectations_0_use_meas_error_1_command_GMM_LOMgain_univariate_07_Aug_2020}}
\subfigure[\colorbox{yellow}{Rescale W} ]{\includegraphics[scale=0.34]{\myFigPath loss_indi_nrange10_constant_only_pi_only_N_100_nfe_5femax_2_loss_944_gridspacing_uniform_Wdiffs2_100000_Wmid_0_Nsimulations_scaleW_1_use_expectations_0_use_meas_error_0_command_GMM_LOMgain_univariate_07_Aug_2020}}
\caption{Variations II}
\end{figure}

\clearpage
\subsection{Add expectations w/ and w/o rescaling W}
\begin{figure}[h!]
\subfigure[\colorbox{yellow}{A shorter dataset w/o expectations} ]{\includegraphics[scale=0.22]{\myFigPath loss_indi_nrange10_constant_only_pi_only_N_100_nfe_5femax_2_loss_944_gridspacing_uniform_Wdiffs2_100000_Wmid_0_Nsimulations_scaleW_0_use_expectations_0_use_meas_error_0_command_GMM_LOMgain_univariate_07_Aug_2020}}
\subfigure[\colorbox{yellow}{A shorter dataset adding expectations} ]{\includegraphics[scale=0.24]{\myFigPath loss_indi_nrange20_constant_only_pi_only_N_100_nfe_5femax_2_loss_944_gridspacing_uniform_Wdiffs2_100000_Wmid_0_Nsimulations_scaleW_0_use_expectations_1_use_meas_error_0_command_GMM_LOMgain_univariate_07_Aug_2020}}
\subfigure[\colorbox{yellow}{A shorter dataset adding expectations, rescaling W} ]{\includegraphics[scale=0.24]{\myFigPath loss_indi_nrange10_constant_only_pi_only_N_100_nfe_5femax_2_loss_0_gridspacing_uniform_Wdiffs2_100000_Wmid_0_Nsimulations_scaleW_1_use_expectations_1_use_meas_error_0_command_GMM_LOMgain_univariate_09_Aug_2020}}
\caption{Variations III}
\end{figure}

\clearpage
\subsection{Loss w/o expectations but $N=1000$}
\begin{figure}[h!]
\subfigure[\colorbox{yellow}{Reproduces Fig 1, Panel b} ]{\includegraphics[scale=0.32]{\myFigPath loss_indi_constant_only_pi_only_N_100_nfe_5femax_2_loss_944_gridspacing_uniform_Wdiffs2_100000_Wmid_0_Nsimulations_scaleW_0_use_expectations_0_use_meas_error_0_command_GMM_LOMgain_univariate_07_Aug_2020}}
\subfigure[\colorbox{yellow}{N=1000} ]{\includegraphics[scale=0.32]{\myFigPath loss_indi_nrange10_constant_only_pi_only_N_1000_nfe_5femax_2_loss_944_gridspacing_uniform_Wdiffs2_100000_Wmid_0_Nsimulations_scaleW_0_use_expectations_0_use_meas_error_0_command_GMM_LOMgain_univariate_07_Aug_2020}}
\caption{Variations IV}
\end{figure}

\clearpage
\section{Take a deep breath: what have I learned?}
\begin{enumerate}
\item Some indication that the measurement error is screwed up, but I can bypass it, so ignore for now.
\item Rescaling might exit too soon. Main problem is it shouldn't change the \emph{shape} of the loss function, but does. Yet no indication of numerical matrix inversion problems. I don't understand.
\item Indication that something is screwed up with the expectations, potentially connected with the rescaling. Ridge didn't really help either. 
\item Loss function indicates that the parameters \emph{are} identified. However, since loss is greater at true values than at estimated ones, it seems that the truth is a local, not a global min. I need to i) use some tricks to find this min ii) understand why this min isn't the global. I have a hypothesis: \\
I think expectations in the true data aren't very large, and thus also aren't fluctuating enough. This screws up the moments somehow, but it also means that the estimation wants to set the $\alpha$s corresponding to large forecast errors to a low value, b/c otherwise it would cause fluctuations that aren't there in the data. Combined with the zero-neighborhood problem, the flat estimate is the result.
\end{enumerate}

\clearpage
\section{Truth with more action in expectations}

\begin{figure}[h!]
\subfigure[Estimated parameters ]{\includegraphics[scale=0.22]{\myFigPath alphas_constant_only_pi_only_N_100_nfe_5femax_2_loss_4_gridspacing_uniform_Wdiffs2_100000_Wmid_0_Nsimulations_scaleW_0_use_expectations_0_use_meas_error_0_command_GMM_LOMgain_univariate_10_Aug_2020}}
\subfigure[Autocovariogram ]{\includegraphics[scale=0.24]{\myFigPath autocovariogram_sim_constant_only_pi_only_N_100_nfe_5femax_2_loss_4_gridspacing_uniform_Wdiffs2_100000_Wmid_0_Nsimulations_scaleW_0_use_expectations_0_use_meas_error_0_command_GMM_LOMgain_univariate_10_Aug_2020}}
\subfigure[Loss]{\includegraphics[scale=0.22]{\myFigPath loss_increment_0_01_constant_only_pi_only_N_100_nfe_5femax_2_loss_3_gridspac_uniform_Wdiffs2_100000_Wmid_0_Nsimulations_scaleW_0_use_exp_0_use_meas_err_0_command_GMM_LOMgain_univariate_10_Aug_2020}}
\subfigure[Loss, zooming in]{\includegraphics[scale=0.24]{\myFigPath loss_indi_nrange10_constant_only_pi_only_N_100_nfe_5femax_2_loss_3_gridspacing_uniform_Wdiffs2_100000_Wmid_0_Nsimulations_scaleW_0_use_expectations_0_use_meas_error_0_command_GMM_LOMgain_univariate_10_Aug_2020}}
\caption{Not using 1-step ahead forecasts of inflation, estimate mean moments once, imposing convexity with weight 100K, w/o measurement error, truth with $nfe=5, fe \in(-2,2)$ \colorbox{yellow}{true $\alpha$s scaled up by 4}}
\end{figure}

For every evaluation of the loss function, very many simulations explode.


\clearpage
\section{Look into behavior of simulation as a function of shocks and $\alpha$s}

\subsection{Cross-section for $\alpha^{true}$}
\begin{figure}[h!]
\subfigure[Distributions, $N=100$]{\includegraphics[scale=0.22]{\myFigPath distribs_constant_only_pi_only_N_100_nfe_5_femax_2_command_check_simulation_approx_11_Aug_2020}}
\subfigure[Means, $N=100$]{\includegraphics[scale=0.22]{\myFigPath means_constant_only_pi_only_N_100_nfe_5_femax_2_command_check_simulation_approx_11_Aug_2020}}
\subfigure[Distributions, $N=1000$]{\includegraphics[scale=0.22]{\myFigPath distribs_constant_only_pi_only_N_1000_nfe_5_femax_2_command_check_simulation_approx_11_Aug_2020}}
\subfigure[Means, $N=1000$]{\includegraphics[scale=0.22]{\myFigPath means_constant_only_pi_only_N_1000_nfe_5_femax_2_command_check_simulation_approx_11_Aug_2020}}
\caption{Summary statistics of simulation in a cross-section, $\alpha=(0.05,0.025,0,0.025,0.05)$, $nfe=5, fe \in(-2,2)$, \texttt{rng(1)}}
\end{figure}

\clearpage
\subsection{Cross-sectional simulations moving particular elements of $\alpha$ at a time}
\begin{figure}[h!]
\subfigure[]{\includegraphics[height=10cm, width=15cm]{\myFigPath k1_scenario_1__constant_only_pi_only_N_100_nfe_5_femax_2_command_check_simulation_approx_11_Aug_2020}}
\subfigure[]{\includegraphics[scale=0.24]{\myFigPath pibar_scenario_1__constant_only_pi_only_N_100_nfe_5_femax_2_command_check_simulation_approx_11_Aug_2020}}
\subfigure[]{\includegraphics[scale=0.24]{\myFigPath fe_scenario_1__constant_only_pi_only_N_100_nfe_5_femax_2_command_check_simulation_approx_11_Aug_2020}}
\caption{Cross-sectional means when varying $\alpha_{1,5}, N=100$, $\alpha=(\colorbox{yellow}{0.05},0.025,0,0.025,\colorbox{yellow}{0.05})$, $nfe=5, fe \in(-2,2)$, \texttt{rng(1)}}
\end{figure}

\begin{figure}[h!]
\subfigure[]{\includegraphics[scale=0.42]{\myFigPath k1_scenario_2__constant_only_pi_only_N_100_nfe_5_femax_2_command_check_simulation_approx_11_Aug_2020}}
\subfigure[]{\includegraphics[scale=0.24]{\myFigPath pibar_scenario_2__constant_only_pi_only_N_100_nfe_5_femax_2_command_check_simulation_approx_11_Aug_2020}}
\subfigure[]{\includegraphics[scale=0.24]{\myFigPath fe_scenario_2__constant_only_pi_only_N_100_nfe_5_femax_2_command_check_simulation_approx_11_Aug_2020}}
\caption{Cross-sectional means when varying $\alpha_{2,4}, N=100$, $\alpha=(0.05,\colorbox{yellow}{0.025},0,\colorbox{yellow}{0.025},0.05)$, $nfe=5, fe \in(-2,2)$, \texttt{rng(1)}}
\end{figure}

\begin{figure}[h!]
\subfigure[]{\includegraphics[scale=0.42]{\myFigPath k1_scenario_3__constant_only_pi_only_N_100_nfe_5_femax_2_command_check_simulation_approx_11_Aug_2020}}
\subfigure[]{\includegraphics[scale=0.24]{\myFigPath pibar_scenario_3__constant_only_pi_only_N_100_nfe_5_femax_2_command_check_simulation_approx_11_Aug_2020}}
\subfigure[]{\includegraphics[scale=0.24]{\myFigPath fe_scenario_3__constant_only_pi_only_N_100_nfe_5_femax_2_command_check_simulation_approx_11_Aug_2020}}
\caption{Cross-sectional means when varying $\alpha_{3}, N=100$, $\alpha=(0.05,0.025,\colorbox{yellow}{0},0.025,0.05)$, $nfe=5, fe \in(-2,2)$, \texttt{rng(1)}}
\end{figure}

\clearpage
\section{Only $\alpha$ corresponding to large forecast errors}


\begin{figure}[h!]
\subfigure[Estimated parameters ]{\includegraphics[scale=0.32]{\myFigPath alphas_constant_only_pi_only_N_100_nfe_5femax_2_loss_944_gridspacing_manual_Wdiffs2_100000_Wmid_0_Nsimulations_scaleW_0_use_expectations_0_use_meas_error_0_command_GMM_LOMgain_univariate_12_Aug_2020_08_05_15}}
\subfigure[Autocovariogram ]{\includegraphics[scale=0.32]{\myFigPath autocovariogram_sim_constant_only_pi_only_N_100_nfe_5femax_2_loss_944_gridspacing_manual_Wdiffs2_100000_Wmid_0_Nsimulations_scaleW_0_use_expectations_0_use_meas_error_0_command_GMM_LOMgain_univariate_12_Aug_2020_08_05_15}}
\caption{Not using 1-step ahead forecasts of inflation, estimate mean moments once, imposing convexity with weight 100K, w/o measurement error, truth with $nfe=5, fe \in(-2,2)$, \colorbox{yellow}{gridpoints $= [-2,-1.5,0,1.5,2]$}}
\end{figure}

\begin{figure}[h!]
\subfigure[Estimated parameters ]{\includegraphics[scale=0.32]{\myFigPath alphas_constant_only_pi_only_N_100_nfe_5femax_2_loss_938_gridspacing_manual_Wdiffs2_100000_Wmid_0_Nsimulations_scaleW_0_use_expectations_0_use_meas_error_0_command_GMM_LOMgain_univariate_12_Aug_2020_08_13_33}}
\subfigure[Autocovariogram ]{\includegraphics[scale=0.32]{\myFigPath autocovariogram_sim_constant_only_pi_only_N_100_nfe_5femax_2_loss_938_gridspacing_manual_Wdiffs2_100000_Wmid_0_Nsimulations_scaleW_0_use_expectations_0_use_meas_error_0_command_GMM_LOMgain_univariate_12_Aug_2020_08_13_33}}
\caption{Not using 1-step ahead forecasts of inflation, estimate mean moments once, imposing convexity with weight 100K, w/o measurement error, truth with $nfe=5, fe \in(-2,2)$, \colorbox{yellow}{gridpoints $= [-4,-3,0,3,4]$}}
\end{figure}

\begin{figure}[h!]
\subfigure[Estimated parameters ]{\includegraphics[scale=0.32]{\myFigPath alphas_constant_only_pi_only_N_100_nfe_4femax_2_loss_947_gridspacing_manual_Wdiffs2_100000_Wmid_0_Nsimulations_scaleW_0_use_expectations_0_use_meas_error_0_command_GMM_LOMgain_univariate_12_Aug_2020_08_26_44}}
\subfigure[Autocovariogram ]{\includegraphics[scale=0.32]{\myFigPath autocovariogram_sim_constant_only_pi_only_N_100_nfe_4femax_2_loss_947_gridspacing_manual_Wdiffs2_100000_Wmid_0_Nsimulations_scaleW_0_use_expectations_0_use_meas_error_0_command_GMM_LOMgain_univariate_12_Aug_2020_08_26_44}}
\caption{Not using 1-step ahead forecasts of inflation, estimate mean moments once, imposing convexity with weight 100K, w/o measurement error, truth with $nfe=5, fe \in(-2,2)$, \colorbox{yellow}{gridpoints $= [-4,-3,3,4]$ (no zero-point)}}
\end{figure}

\begin{figure}[h!]
\subfigure[Estimated parameters ]{\includegraphics[scale=0.32]{\myFigPath alphas_constant_only_pi_only_N_100_nfe_5femax_2_loss_1040_gridspacing_manual_Wdiffs2_100000_Wmid_1000_Nsimulations_scaleW_0_use_expectations_0_use_meas_error_0_command_GMM_LOMgain_univariate_12_Aug_2020_08_31_07}}
\subfigure[Autocovariogram ]{\includegraphics[scale=0.32]{\myFigPath autocovariogram_sim_constant_only_pi_only_N_100_nfe_5femax_2_loss_1040_gridspacing_manual_Wdiffs2_100000_Wmid_1000_Nsimulations_scaleW_0_use_expectations_0_use_meas_error_0_command_GMM_LOMgain_univariate_12_Aug_2020_08_31_07}}
\caption{Not using 1-step ahead forecasts of inflation, estimate mean moments once, imposing convexity with weight 100K, w/o measurement error, truth with $nfe=5, fe \in(-2,2)$, \colorbox{yellow}{gridpoints $= [-4,-3,0,3,4]$ with 0 at 0 imposed with weight 1000}}
\end{figure}


\begin{figure}[h!]
\subfigure[Estimated parameters ]{\includegraphics[scale=0.32]{\myFigPath alphas_constant_only_pi_only_N_100_nfe_5femax_2_loss_0_gridspacing_manual_Wdiffs2_100000_Wmid_1000_Nsimulations_scaleW_0_use_expectations_0_use_meas_error_0_command_GMM_LOMgain_univariate_12_Aug_2020_08_48_50}}
\subfigure[Autocovariogram ]{\includegraphics[scale=0.32]{\myFigPath autocovariogram_sim_constant_only_pi_only_N_100_nfe_5femax_2_loss_0_gridspacing_manual_Wdiffs2_100000_Wmid_1000_Nsimulations_scaleW_0_use_expectations_0_use_meas_error_0_command_GMM_LOMgain_univariate_12_Aug_2020_08_48_50}}
\caption{Not using 1-step ahead forecasts of inflation, estimate mean moments once, imposing convexity with weight 100K, w/o measurement error, truth with $nfe=5, fe \in(-2,2)$, \colorbox{yellow}{gridpoints $= [-4,-3,0,3,4]$ with 0 at 0 imposed with weight 1000,} \colorbox{yellow}{true parameters scaled up by 4}}
\end{figure}


\clearpage
\section{Notes on the GMM weighting matrix}

The loss function takes the form:
\begin{align}
L = (\Omega^{data} - \Omega^{model}) W (\Omega^{data} - \Omega^{model})' 
\end{align}
where $\Omega$ are the moments and $W$ is the weighting matrix. Let $\Sigma \equiv Var(\Omega^{data, bootstrap})$. Then $W = \Sigma^{-1}$.

For \texttt{lsqnonlin}, I write the residuals of the objective function as
\begin{align}
R = (vec(\Omega^{data}) - vec(\Omega^{model})) diag(W)
\end{align}

I implement additional restrictions by adding elements to the residuals vector $R$ with a manually specified weight. For example, the convexity restriction is added as follows:
\begin{itemize}
\item For each guess $\alpha$, compute $g''_i$, numerical second derivatives of the anchoring function. There are always \texttt{numel(alpha)-2} of those, so if I estimate 5 knots, there are 3 second derivatives, $i=1,2,3$. 
\item For $i=1,2,3$, add the following elements to the residuals vector $R$: 
\begin{equation}
\begin{cases} (g''_i)^2 W^{convexity} \quad \quad \text{if} \quad \quad g''_i < 0 \\
0  \quad \quad \text{otherwise} \end{cases}
\end{equation}
where $W^{convexity}$ is a manually specified weight.
\end{itemize}

Rescaling: when some elements of $\Sigma$ are very small, $W$ becomes large. This shouldn't be a problem if the fact that some elements of $\Sigma$ are close to zero doesn't cause numerical issues during inversion. Therefore what I do:
\begin{itemize}
\item Find smallest element of $\Sigma$, and write it as some number times $10^s, s<0$. 
\item $W^{rescaled} = (\Sigma \times 10^{|s|})^{-1}$
\end{itemize}


I've checked:
\begin{itemize}
\item $W W^{-1} = I$ and $W^{rescaled} (W^{rescaled})^{-1} = I$
\item $L(W) = L(W^{rescaled}) \times (10^{|s|})^2$
\item What does cause problems in Matlab: $(a\Sigma)^{-1} \neq \frac{1}{a}\Sigma^{-1}$ if $a$ close to zero.
\end{itemize}









%%%%%%%%%%%%%%%%%%%%%%%%%   FOR ME
\clearpage
\section{Some additional estimation exercises for the N simulations strategy with settings preceding those of the previous section}

\subsection{Real data with $\alpha$s out in the edges, as above}
\begin{figure}[h!]
\subfigure[Estimated parameters ]{\includegraphics[scale=0.32]{\myFigPath alph_opt_constant_only_pi_only_N_100_nfe_5femax_2_loss_382_gridspacing_manual_Wdiffs2_100000_Wmid_1000_Nsimulations_scaleW_0_use_expectations_0_use_meas_error_0_command_GMM_LOMgain_univariate_12_Aug_2020_09_16_29}}
\subfigure[Autocovariogram ]{\includegraphics[scale=0.32]{\myFigPath autocovariogram_constant_only_pi_only_N_100_nfe_5femax_2_loss_382_gridspacing_manual_Wdiffs2_100000_Wmid_1000_Nsimulations_scaleW_0_use_expectations_0_use_meas_error_0_command_GMM_LOMgain_univariate_12_Aug_2020_09_16_29}}
\caption{Not using 1-step ahead forecasts of inflation, estimate mean moments once, imposing convexity with weight 100K, w/o measurement error, real data, \colorbox{yellow}{gridpoints $= [-4,-3,0,3,4]$ with 0 at 0 imposed with weight 1000}, $N=100$}
\end{figure}

\clearpage
\subsection{$N=1000$}
\begin{figure}[h!]
\subfigure[Estimated parameters ]{\includegraphics[scale=0.32]{\myFigPath alphas_constant_only_pi_only_N_1000_nfe_5femax_2_loss_1196_gridspacing_uniform_Wdiffs2_100000_Wmid_0_Nsimulations_scaleW_0_use_expectations_0_use_meas_error_0_command_GMM_LOMgain_univariate_07_Aug_2020}}
\subfigure[Autocovariogram ]{\includegraphics[scale=0.32]{\myFigPath autocovariogram_sim_constant_only_pi_only_N_1000_nfe_5femax_2_loss_1196_gridspacing_uniform_Wdiffs2_100000_Wmid_0_Nsimulations_scaleW_0_use_expectations_0_use_meas_error_0_command_GMM_LOMgain_univariate_07_Aug_2020}}
\caption{\colorbox{yellow}{NOT} using 1-step ahead forecasts of inflation, estimate mean moments once, imposing convexity with weight 100K, w/o measurement error, truth with $nfe=5, fe \in(-2,2)$}
\end{figure}

\clearpage
\subsection{Expectations}
\begin{figure}[h!]
\subfigure[Estimated parameters ]{\includegraphics[scale=0.3]{\myFigPath alphas_constant_only_pi_only_N_100_nfe_5femax_2_loss_5243_gridspacing_uniform_Wdiffs2_100000_Wmid_0_Nsimulations_scaleW_0_use_expectations_1_use_meas_error_0_command_GMM_LOMgain_univariate_09_Aug_2020}}
\subfigure[Autocovariogram ]{\includegraphics[scale=0.44]{\myFigPath autocovariogram_sim_constant_only_pi_only_N_100_nfe_5femax_2_loss_5243_gridspacing_uniform_Wdiffs2_100000_Wmid_0_Nsimulations_scaleW_0_use_expectations_1_use_meas_error_0_command_GMM_LOMgain_univariate_09_Aug_2020}}
\caption{\colorbox{yellow}{using 1-step ahead forecasts of inflation}, estimate mean moments once, imposing convexity with weight 100K, w/o measurement error, truth with $nfe=5, fe \in(-2,2)$}
\end{figure}


\clearpage
\subsection{Expectations and rescaling}
\begin{figure}[h!]
\subfigure[Estimated parameters ]{\includegraphics[scale=0.3]{\myFigPath alphas_constant_only_pi_only_N_100_nfe_5femax_2_loss_0_gridspacing_uniform_Wdiffs2_100000_Wmid_0_Nsimulations_scaleW_1_use_expectations_1_use_meas_error_0_command_GMM_LOMgain_univariate_09_Aug_2020}}
\subfigure[Autocovariogram ]{\includegraphics[scale=0.44]{\myFigPath autocovariogram_sim_constant_only_pi_only_N_100_nfe_5femax_2_loss_0_gridspacing_uniform_Wdiffs2_100000_Wmid_0_Nsimulations_scaleW_1_use_expectations_1_use_meas_error_0_command_GMM_LOMgain_univariate_09_Aug_2020}}
\caption{\colorbox{yellow}{using 1-step ahead forecasts of inflation, rescaling W}, estimate mean moments once, imposing convexity with weight 100K, w/o measurement error, truth with $nfe=5, fe \in(-2,2)$}
\end{figure}

%\clearpage
%\subsection{Expectations and ridge}
%\begin{figure}[h!]
%\subfigure[Estimated parameters ]{\includegraphics[scale=0.3]{\myFigPath alphas_constant_only_pi_only_N_100_nfe_5femax_2_loss_17370525_gridspacing_uniform_Wdiffs2_100000_Wmid_0_Nsimulations_scaleW_0_use_expectations_1_use_meas_error_0_command_GMM_LOMgain_univariate_09_Aug_2020}}
%\subfigure[Autocovariogram ]{\includegraphics[scale=0.44]{\myFigPath autocovariogram_sim_constant_only_pi_only_N_100_nfe_5femax_2_loss_17370525_gridspacing_uniform_Wdiffs2_100000_Wmid_0_Nsimulations_scaleW_0_use_expectations_1_use_meas_error_0_command_GMM_LOMgain_univariate_09_Aug_2020}}
%\caption{\colorbox{yellow}{using 1-step ahead forecasts of inflation, ridge regression for data generation and estimation}, estimate mean moments once, imposing convexity with weight 100K, w/o measurement error, truth with $nfe=5, fe \in(-2,2)$}
%\end{figure}


\clearpage
\subsection{0 at 0}
\begin{figure}[h!]
\subfigure[Estimated parameters ]{\includegraphics[scale=0.3]{\myFigPath alphas_constant_only_pi_only_N_100_nfe_5femax_2_loss_1031_gridspacing_uniform_Wdiffs2_100000_Wmid_1000_Nsimulations_scaleW_0_use_expectations_0_use_meas_error_0_command_GMM_LOMgain_univariate_09_Aug_2020}}
\subfigure[Autocovariogram ]{\includegraphics[scale=0.44]{\myFigPath autocovariogram_sim_constant_only_pi_only_N_100_nfe_5femax_2_loss_1031_gridspacing_uniform_Wdiffs2_100000_Wmid_1000_Nsimulations_scaleW_0_use_expectations_0_use_meas_error_0_command_GMM_LOMgain_univariate_09_Aug_2020}}
\caption{\colorbox{yellow}{0 at 0 imposed with weight 1000} not using 1-step ahead forecasts of inflation, not rescaling W, estimate mean moments once, imposing convexity with weight 100K, w/o measurement error, truth with $nfe=5, fe \in(-2,2)$}
\end{figure}

\clearpage
\subsection{0 at 0, more convexity}
\begin{figure}[h!]
\subfigure[Estimated parameters ]{\includegraphics[scale=0.3]{\myFigPath alphas_constant_only_pi_only_N_100_nfe_5femax_2_loss_1038_gridspacing_uniform_Wdiffs2_1000000_Wmid_1000_Nsimulations_scaleW_0_use_expectations_0_use_meas_error_0_command_GMM_LOMgain_univariate_09_Aug_2020}}
\subfigure[Autocovariogram ]{\includegraphics[scale=0.44]{\myFigPath autocovariogram_sim_constant_only_pi_only_N_100_nfe_5femax_2_loss_1038_gridspacing_uniform_Wdiffs2_1000000_Wmid_1000_Nsimulations_scaleW_0_use_expectations_0_use_meas_error_0_command_GMM_LOMgain_univariate_09_Aug_2020}}
\caption{\colorbox{yellow}{0 at 0 imposed with weight 1000} not using 1-step ahead forecasts of inflation, not rescaling W, estimate mean moments once, imposing convexity with weight 1000K, w/o measurement error, truth with $nfe=5, fe \in(-2,2)$}
\end{figure}

\clearpage
\subsection{Identity weighting matrix}
\begin{figure}[h!]
\subfigure[Estimated parameters ]{\includegraphics[scale=0.3]{\myFigPath alphas_constant_only_pi_only_N_100_nfe_5femax_2_loss_0_gridspacing_uniform_Wdiffs2_100000_Wmid_0_Nsimulations_scaleW_0_use_expectations_0_use_meas_error_0_command_GMM_LOMgain_univariate_09_Aug_2020}}
\subfigure[Autocovariogram ]{\includegraphics[scale=0.44]{\myFigPath autocovariogram_sim_constant_only_pi_only_N_100_nfe_5femax_2_loss_0_gridspacing_uniform_Wdiffs2_100000_Wmid_0_Nsimulations_scaleW_0_use_expectations_0_use_meas_error_0_command_GMM_LOMgain_univariate_09_Aug_2020}}
\caption{\colorbox{yellow}{identity weighting matrix}, not using 1-step ahead forecasts of inflation, not rescaling W, estimate mean moments once, imposing convexity with weight 1000K, w/o measurement error, truth with $nfe=5, fe \in(-2,2)$}
\end{figure}

%%%%%%%%%%%               BLANK   %%%%%%%%%%%%%%%%%%%%%%%%%%%%%%
% adds blank page
\afterpage{\blankpage}


%%%%%%%%%%%%%%%%%%%%%%%%%%%%%%%%%%%%%%%%%%%%%%%%%%%%%%%%%%%%%                                              APPENDIX
%%%%%%%%%%%%%%%%%%%%%%%%%%%%%%%%%%%%%%%%%%%%%%%%%%%%%%%%%%%%
    \clearpage
%    \newpage
\appendix
% the following command makes equation numbering include the section first, but just for what follows
\numberwithin{equation}{section}
\section{Model summary}

\vspace{-0.5cm}

\begin{align}
x_t &=  -\sigma i_t +\hat{\E}_t \sum_{T=t}^{\infty} \beta^{T-t }\big( (1-\beta)x_{T+1} - \sigma(\beta i_{T+1} - \pi_{T+1}) +\sigma r_T^n \big)  \label{A1}  \\
\pi_t &= \kappa x_t +\hat{\E}_t \sum_{T=t}^{\infty} (\alpha\beta)^{T-t }\big( \kappa \alpha \beta x_{T+1} + (1-\alpha)\beta \pi_{T+1} + u_T\big) \label{A2}  \\
i_t &= \psi_{\pi}\pi_t + \psi_{x} x_t  + \bar{i}_t \label{TR} \quad \quad (\text{if imposed})
\end{align}

\vspace{-1.2cm}

\begin{align}
\text{PLM:} \quad \quad & \hat{\E}_t z_{t+h}  =  a_{t-1} + bh_x^{h-1}s_t  \quad \forall h\geq 1 \quad \quad b = g_x\; h_x \quad \quad  \label{PLM} \\
\text{Updating:} \quad \quad & a_{t}  =a_{t-1} +k_t^{-1}\big(z_{t} -(a_{t-1}+b s_{t-1}) \big)  \label{A5} \\
\text{Anchoring function:} \quad \quad & k^{-1}_t  = \rho_k k^{-1}_{t-1} + \gamma_k fe_{t-1}^2 \label{A6}\\
\text{Forecast error:} \quad \quad & fe_{t-1}  = z_t - (a_{t-1}+b s_{t-1}) \label{A7} \\
\text{LH expectations:} \quad \quad & f_a(t) = \frac{1}{1-\alpha\beta}a_{t-1}  + b(\mathbb{I}_{nx} - \alpha\beta h)^{-1}s_t \quad \quad  f_b(t) = \frac{1}{1-\beta}a_{t-1}  + b(\mathbb{I}_{nx} - \beta h)^{-1}s_t  \label{A8}
\end{align}

\vspace{-0.5cm}

This notation captures vector learning ($z$ learned) for intercept only. For scalar learning, $a_t= \begin{pmatrix} \bar{\pi}_t & 0 & 0\end{pmatrix}' $ and $b_1$ designates the first row of $b$. The observables $(\pi, x)$ are determined as:
\begin{align}
x_t &=  -\sigma i_t + \begin{bmatrix} \sigma & 1-\beta & -\sigma\beta \end{bmatrix} f_b + \sigma \begin{bmatrix} 1 & 0 & 0 \end{bmatrix} (\mathbb{I}_{nx} - \beta h_x)^{-1} s_t \label{A9} \\
\pi_t &= \kappa x_t  + \begin{bmatrix} (1-\alpha)\beta & \kappa\alpha\beta & 0 \end{bmatrix}  f_a + \begin{bmatrix} 0 & 0 & 1 \end{bmatrix}  (\mathbb{I}_{nx} - \alpha \beta h_x)^{-1}  s_t \label{A10}
\end{align}

\section{Target criterion}\label{target_crit_levels}
The target criterion in the simplified model (scalar learning of inflation intercept only, $k_t^{-1} = \mathbf{g}(fe_{t-1})$):
\begin{align*}
\pi_t  = -\frac{\lambda_x}{\kappa}\bigg\{x_t - \frac{(1-\alpha)\beta}{1-\alpha\beta} \bigg(k_t^{-1}+((\pi_t - \bar{\pi}_{t-1}-b_1 s_{t-1}))\mathbf{g}_{\pi}(t) \bigg) \\
\bigg(\E_t\sum_{i=1}^{\infty}x_{t+i}\prod_{j=0}^{i-1}(1-k_{t+1+j}^{-1} - (\pi_{t+1+j} - \bar{\pi}_{t+j}-b_1 s_{t+j})\mathbf{g_{\bar{\pi}}}(t+j)) \bigg)
\bigg\} \numberthis \label{target}
\end{align*}
where I'm using the notation that $\prod_{j=0}^{0} \equiv 1$. For interpretation purposes, let me rewrite this as follows:
\begin{align*}
\pi_t  = & \; \textcolor{red}{-\frac{\lambda_x}{\kappa} x_t} \textcolor{blue}{ \; + \frac{\lambda_x}{\kappa} \frac{(1-\alpha)\beta}{1-\alpha\beta} \bigg(k_t^{-1}+ fe^{eve}_{t|t-1}\mathbf{g}_{\pi}(t) \bigg)\E_t\sum_{i=1}^{\infty}x_{t+i}}  \\
& \textcolor{mygreen}{- \frac{\lambda_x}{\kappa} \frac{(1-\alpha)\beta}{1-\alpha\beta} \bigg(k_t^{-1}+ fe^{eve}_{t|t-1}\mathbf{g}_{\pi}(t) \bigg) \bigg(\E_t\sum_{i=1}^{\infty}x_{t+i}\prod_{j=0}^{i-1}(k_{t+1+j}^{-1} + fe^{eve}_{t+1+j|t+j})\mathbf{g_{\bar{\pi}}}(t+j) \bigg)}
\numberthis \label{target_interpretation}
\end{align*}
Interpretation: \textcolor{red}{tradeoffs from discretion in RE} + \textcolor{blue}{effect of current level and change of the gain on future tradeoffs} + \textcolor{mygreen}{effect of future expected levels and changes of the gain on future tradeoffs}

%\section{A target criterion system for an anchoring function specified for gain changes}\label{target_crit_changes}
%\begin{equation}
%k_t = k_{t-1} + \mathbf{g}(fe_{t|t-1})
%\end{equation}
%Turns out the $k_{t-1}$ adds one $\varphi_{6,t+1}$ too many which makes the target criterion unwieldy. The FOCs of the Ramsey problem are
%\begin{align}
%& 2\pi_t + 2\frac{\lambda}{\kappa}x_t -k_t^{-1} \varphi_{5,t} - \mathbf{g}_{\pi}(t)\varphi_{6,t}  = 0 \label{gaspar22}\\
%& c x_{t+1} + \varphi_{5,t} -(1-k_t^{-1})\varphi_{5,t+1} +\mathbf{g}_{\bar{\pi}}(t)\varphi_{6,t+1} = 0 \label{gaspar21}\\
%& \varphi_{6,t} \; \textcolor{red}{+\; \varphi_{6,t+1}} = fe_t \varphi_{5,t} \label{constraints}
%\end{align}
%where the red multiplier is the new element vis-a-vis the case where the anchoring function is specified in levels ($k_t^{-1} = \mathbf{g}(fe_{t-1})$, as in App. \ref{target_crit_levels}), and I'm using the shorthand notation
%\begin{align}
%c & = -\frac{2(1-\alpha)\beta}{1-\alpha\beta}\frac{\lambda}{\kappa} \\ 
%fe_t & = \pi_t - \bar{\pi}_{t-1}-b s_{t-1}
%\end{align}
%(\ref{gaspar22}) says that in anchoring, the discretion tradeoff is complemented with tradeoffs coming from learning ($\varphi_{5,t}$), which are more binding when expectations are unanchored ($k_{t}^{-1}$ high). Moreover, the change in the anchoring of expectations imposes an additional constraint ($\varphi_{6,t}$), which is more strongly binding if the gain responds strongly to inflation ($\mathbf{g}_{\pi}(t)$).
%One can simplify this three-equation-system to:
%\begin{align}
%\varphi_{6,t} & = -c fe_t x_{t+1} + \bigg(1+ \frac{fe_t}{fe_{t+1}}(1-k_{t+1}^{-1}) -fe_t \mathbf{g}_{\bar{\pi}}(t) \bigg) \varphi_{6,t+1} -\frac{fe_t}{fe_{t+1}}(1-k_{t+1}^{-1})\varphi_{6,t+2}\label{6'} \\
%0 & = 2\pi_t + 2\frac{\lambda}{\kappa}x_t   - \bigg( \frac{k_t^{-1}}{fe_t} + \mathbf{g}_{\pi}(t)\bigg)\varphi_{6,t} + \frac{k_t^{-1}}{fe_t}\varphi_{6,t+1}\label{1'}
%\end{align}
%Unfortunately, I haven't been able to solve (\ref{6'}) for $\varphi_{6,t}$ and therefore I can't express the target criterion so nicely as before. The only thing I can say is to direct the targeting rule-following central bank to compute $\varphi_{6,t}$ as the solution to (\ref{1'}), and then evaluate (\ref{6'}) as a target criterion. The solution to (\ref{1'}) is given by:
%\begin{equation}
%\varphi_{6,t} = -2\E_t\sum_{i=0}^{\infty}(\pi_{t+i}+\frac{\lambda_x}{\kappa}x_{t+i})\prod_{j=0}^{i-1}\frac{\frac{k_{t+j}^{-1}}{fe_{t+j}}}{\frac{k_{t+j}^{-1}}{fe_{t+j}} + \mathbf{g}_{\pi}(t+j)} \label{sol1'}
%\end{equation}
%Interpretation: the anchoring constraint is not binding ($\varphi_{6,t}=0$) if the CB always hits the target (
%$\pi_{t+i}+\frac{\lambda_x}{\kappa}x_{t+i} = 0 \quad \forall i$); or expectations are always anchored ($k_{t+j}^{-1}=0 \quad \forall j$). 



%%%%%%%%%%     IRFS                %%%%%%%%%%%%%%%%
\clearpage
\section{Impulse responses to iid monpol shocks across a wide range of learning models}
$T=400, N=100, n_{drop}=5,$ shock imposed at $t=25$, calibration as above, Taylor rule assumed to be known, PLM = learn constant only, of inflation only.

\begin{figure}[h!]
\subfigure[Decreasing gain learning]{\includegraphics[scale=\myTinyFigScale]{\myFigPath command_IRFs_approx_pretty_RIR_LH_monpol_dgain_constant_only_pi_only_2020_07_07}}
\hfill
\subfigure[Mean gain]{\includegraphics[scale=\myTinyFigScale]{\myFigPath command_IRFs_approx_pretty_invgain_dgain_constant_only_pi_only_params_psi_pi_1_5_psi_x_0_gbar_0_145_thetbar_16_thettilde_2_5_kap_0_8_lamx_0_05_lami_0_2020_07_07}}
\subfigure[Constant gain learning]{\includegraphics[scale=\myTinyFigScale]{\myFigPath command_IRFs_approx_pretty_RIR_LH_monpol_cgain_constant_only_pi_only_2020_07_07}}
\hfill
\subfigure[Mean gain]{\includegraphics[scale=\myTinyFigScale]{\myFigPath command_IRFs_approx_pretty_invgain_cgain_constant_only_pi_only_params_psi_pi_1_5_psi_x_0_gbar_0_145_thetbar_16_thettilde_2_5_kap_0_8_lamx_0_05_lami_0_2020_07_07}}
\subfigure[CEMP criterion (vector)]{\includegraphics[scale=\myTinyFigScale]{\myFigPath command_IRFs_approx_pretty_RIR_LH_monpol_again_critCEMP_constant_only_pi_only_2020_07_07}}
\hfill
\subfigure[Mean gain]{\includegraphics[scale=\myTinyFigScale]{\myFigPath command_IRFs_approx_pretty_invgain_again_critCEMP_constant_only_pi_only_params_psi_pi_1_5_psi_x_0_gbar_0_145_thetbar_16_thettilde_2_5_kap_0_8_lamx_0_05_lami_0_2020_07_07}}
\caption{IRFs and gain history (sample means) }
\end{figure}


\begin{figure}[h!]
\subfigure[CUSUM criterion (vector)]{\includegraphics[scale=\myTinyFigScale]{\myFigPath command_IRFs_approx_pretty_RIR_LH_monpol_again_critCUSUM_constant_only_pi_only_2020_07_07}}
\hfill
\subfigure[Mean gain]{\includegraphics[scale=\myTinyFigScale]{\myFigPath command_IRFs_approx_pretty_invgain_again_critCUSUM_constant_only_pi_only_params_psi_pi_1_5_psi_x_0_gbar_0_145_thetbar_16_thettilde_2_5_kap_0_8_lamx_0_05_lami_0_2020_07_07}}
\subfigure[Smooth criterion, approximated, using $\alpha^{true}= (0.05;0.025;0;0.025;0.05)$, on $fe \in (-2,2)$.]{\includegraphics[scale=\myTinyFigScale]{\myFigPath command_IRFs_approx_pretty_RIR_LH_monpol_again_critsmooth_constant_only_pi_only_2020_07_07}}
\hfill
\subfigure[Mean gain]{\includegraphics[scale=\myTinyFigScale]{\myFigPath command_IRFs_approx_pretty_invgain_again_critsmooth_constant_only_pi_only_params_psi_pi_1_5_psi_x_0_gbar_0_145_thetbar_16_thettilde_2_5_kap_0_8_lamx_0_05_lami_0_2020_07_07}}

\caption{IRFs and gain history (sample means), continued }
\end{figure}


\end{document}

%%%%%%%%%%%%%    SUBFIGURE  %%%%%%%%%%%
%\begin{figure}[h!]
%\subfigure[Hodrick-Prescott, $\lambda=1600$]{\includegraphics[scale=\myAdjustableFigScale]{\myFigPath materials22_gain_dhat_HP}}
%\hfill % this is great to intro dpace between subfigures
%\subfigure[Hamilton, 4 lags, $h=8$]{\includegraphics[scale=\myAdjustableFigScale]{\myFigPath materials22_gain_dhat_Hamilton}}
%\subfigure[Baxter-King, $(6,32)$ quarters, truncation at 12 lags]{\includegraphics[scale=\myAdjustableFigScale]{\myFigPath materials22_gain_dhat_BK}}
%\caption{Inverse gain for $\hat{d}$ for the different filters}
%\end{figure}

%%%%%%%%%%%%%    TABLE  %%%%%%%%%%%
%\begin{center}
%\begin{table}[h!]
%\caption{$\hat{d}$}
%\begin{tabular}{ c |c |c }
%  & $W = I$ & $W = \text{diag}(\hat{\sigma}_{ac(0)}, \dots, \hat{\sigma}_{ac(K)})$ \\ 
%  \hline
% HP & 77.7899 & 10 \\  
% \hline
% Hamilton & 32.1649 & 10 \\  
% \hline
% BK & 90.3929 & 10    
%\end{tabular}
%\end{table}
%\end{center}





