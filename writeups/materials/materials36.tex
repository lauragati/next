\documentclass[11pt]{article}
\usepackage{amsmath, amsthm, amssymb,lscape, natbib}
\usepackage{mathtools}
\usepackage{subfigure}
\usepackage[font=footnotesize,labelfont=bf]{caption}
\usepackage{graphicx}
\usepackage{colortbl}
\usepackage{hhline}
\usepackage{multirow}
\usepackage{multicol}
\usepackage{setspace}
\usepackage[final]{pdfpages}
\usepackage[left=2.5cm,top=2.5cm,right=2.5cm, bottom=2.5cm]{geometry}
\usepackage{natbib} 
\usepackage{bibentry} 
\newcommand{\bibverse}[1]{\begin{verse} \bibentry{#1} \end{verse}}
\newcommand{\vs}{\vspace{.3in}}
\renewcommand{\ni}{\noindent}
\usepackage{xr-hyper}
\usepackage[]{hyperref}
\hypersetup{
    colorlinks=true,
    linkcolor=blue,
    filecolor=magenta,      
    urlcolor=cyan,
}
 
\urlstyle{same}
\usepackage[capposition=top]{floatrow}
\usepackage{amssymb}
\usepackage{relsize}
\usepackage[dvipsnames]{xcolor}
\usepackage{fancyhdr}
\usepackage{tikz}
 
\pagestyle{fancy} % customize header and footer
\fancyhf{} % clear initial header and footer
%\rhead{Overleaf}
\lhead{\centering \rightmark} % this adds subsection number and name
\lfoot{\centering \rightmark} 
\rfoot{\thepage} % put page number (the centering command puts it in the middle, don't matter if you put it in right or left footer)

\def \myFigPath {../../figures/} 
% BE CAREFUL WITH FIGNAMES, IN LATEX THEY'RE NOT CASE SENSITIVE!!
\def \myTablePath {../../tables/} 

%\definecolor{mygreen}{RGB}{0, 100, 0}
\definecolor{mygreen}{RGB}{0, 128, 0}

\definecolor{citec}{rgb}{0,0,.5}
\definecolor{linkc}{rgb}{0,0,.6}
\definecolor{bcolor}{rgb}{1,1,1}
\hypersetup{
%hidelinks = true
  colorlinks = true,
  urlcolor=linkc,
  linkcolor=linkc,
  citecolor = citec,
  filecolor = linkc,
  pdfauthor={Laura G\'ati},
}


\geometry{left=.83in,right=.89in,top=1in,
bottom=1in}
\linespread{1.5}
\renewcommand{\[}{\begin{equation}}
\renewcommand{\]}{\end{equation}}

% New Options
\newtheorem{prop}{Proposition}
\newtheorem{definition}{Definition}[section]
\newtheorem*{remark}{Remark}
\newtheorem{lemma}{Lemma}
\newtheorem{corollary}{Corollary}
\newtheorem{conjecture}{Conjecture}

%\newtheorem{theorem}{Theorem}[section] % the third argument specifies that their number will be adopted to the section
%\newtheorem{corollary}{Corollary}[theorem]
%\newtheorem{lemma}[theorem]{Lemma}
%\declaretheorem{proposition}
%\linespread{1.3}
%\raggedbottom
%\font\reali=msbm10 at 12pt

% New Commands
\newcommand{\real}{\hbox{\reali R}}
\newcommand{\realp}{\hbox{\reali R}_{\scriptscriptstyle +}}
\newcommand{\realpp}{\hbox{\reali R}_{\scriptscriptstyle ++}}
\newcommand{\R}{\mathbb{R}}
\DeclareMathOperator{\E}{\mathbb{E}}
\DeclareMathOperator{\argmin}{arg\,min}
\newcommand\w{3.0in}
\newcommand\wnum{3.0}
\def\myFigWidth{5.3in}
\def\mySmallerFigWidth{2.1in}
\def\myEvenBiggerFigScale{0.8}
\def\myPointSixFigScale{0.6}
\def\myBiggerFigScale{0.4}
\def\myFigScale{0.3}
\def\myMediumFigScale{0.25}
\def\mySmallFigScale{0.22}
\def\mySmallerFigScale{0.18}
\def\myTinyFigScale{0.16}
\def\myPointFourteenFigScale{0.14}
\def\myTinierFigScale{0.12}
\def\myAdjustableFigScale{0.16}
\newcommand\numberthis{\addtocounter{equation}{1}\tag{\theequation}} % this defines a command to make align only number this line
\newcommand{\code}[1]{\texttt{#1}} %code %

\renewcommand*\contentsname{Overview}
\setcounter{tocdepth}{2}

% define a command to make a huge question mark (it works in math mode)
\newcommand{\bigqm}[1][1]{\text{\larger[#1]{\textbf{?}}}}

\begin{document}

\linespread{1.0}

\title{Materials 36 - Convince that estimation is robust}
\author{Laura G\'ati} 
\date{\today}
\maketitle

%%%%%%%%%%%%%%%%%%%%             DOCUMENT           %%%%%%%%%%%%%%%%%% 

\tableofcontents

%\listoffigures

\newpage

\section{Calibration issues}
\subsection{$\alpha=Prob(\text{keep same price})=$ ?}
So far I've used 0.5.
\begin{itemize}
\item Nikolay Hristov notes: expected duration of contract = $\frac{1}{1-\alpha}$ periods.
\item Evidence on average duration of prices:
\begin{itemize}
\item Bils \& Klenow (2004): 4.3 months
\item Klenow \& Kryvstov (2008): mean 7-9 months, median 4-7 months
\item Nakamura \& Steinsson (2008): 7-9 months
\item Klenow \& Malin (2010): 6.9 months
\item Eichenbaum, Jaimovitch \& Rebelo (2008, published as 2011): 10.6 months
\end{itemize}
\item[$\rightarrow$] On average this gives us 7.56 months, a little more than two quarters. The implied $\alpha \approx 0.6$. Rotemberg \& Woodford (1997) calibrate $\alpha=0.66$. To err on the flexible price side, I set $\alpha =0.5$.
\end{itemize}
\subsection{The composite parameter $\kappa$ once and for all}
I've used $\kappa = \frac{(1-\alpha\beta)}{\alpha} \zeta$ where I should have used $\kappa = \frac{(1-\alpha)(1-\alpha\beta)}{\alpha} \zeta$. In Preston (2005), and I think this is also Woodford's favorite specification, $\zeta = \frac{\omega + \sigma^{-1}}{1+\omega\theta}$.
Let's define terms:
\begin{itemize}
\item $\alpha$: Prob(keep price unchanged)
\item $\beta$: discount factor
\item $\zeta$: measure of strategic complementarity in price setting. The smaller $\zeta$, the more complementarity. This depends on a bunch of things:
	\begin{itemize}
	\item homogenous vs. specific factor markets ($s_y = 0$ or not)
	\item constant vs. variable desired markup ($\epsilon_{\mu}=0$ or not)
	\item no vs. intermediate inputs ($s_m = 0$ or not)
	\end{itemize}
	In particular (Prop 3.3 in Woodford 2011, equation 1.43, Chapter 3,  p. 171):
	\begin{equation}
	\zeta = \frac{(1-\mu s_m)(s_y + s_Y)}{1+\theta [\epsilon_{\mu}+(1-\mu s_m)s_y]}
	\end{equation}
with 
\begin{itemize}
\item $\theta$: price elasticity of demand
\item $\mu(x)$: markup function
\item $\epsilon_{\mu}$: elasticity of markup function (how much do target markups change at different levels of output)
\item $s(y,Y,\xi)$: real marginal cost function
\item $s_y$: elasticity of real marginal cost function wrt firm i's output, $y_t(i)$
\item $s_Y$: elasticity of real marginal cost function wrt aggregate output, $y_t$
\item $s_m$: elasticity of real marginal cost function wrt intermediate inputs, $m_t(i)$
\end{itemize}
Then expression $\zeta = \frac{\omega + \sigma^{-1}}{1+\omega\theta}$ is obtained by assuming no intermediate inputs, constant desired markups wrt. output levels and specific factor markets, so that
\begin{equation}
\zeta =  \frac{s_y + s_Y}{1+ s_y \theta}
\end{equation}
What is $\omega$? It's the derivative of the MC function wrt own output, but this only coincides with $s_y$ for specific factor markets. Woodford shows that for specific factor markets, $s_y = \omega, s_Y = \sigma^{-1}$, while for common factor markets $s_Y = \omega + \sigma^{-1}$ because for the latter, there is no distinction between own and aggregate output for the purpose of wage setting and thus marginal cost. So, more broadly, $\omega$ is a measure of how marginal cost reacts to some wage-relevant measure of output. Woodford, Chapter 3, (1.16), p. 152:
\begin{equation}
\omega = \underbrace{\omega_w}_{=\eta, \text{ Frisch elasticity of disutility of labor wrt output}} + \underbrace{\omega_p}_{\text{elasticity of MPL wrt output}}
\end{equation}
Denoting the Frisch elasticity as $\eta$, and noting that for Cobb-Douglas, $\partial MPL / \partial y_t = 0$, 
\begin{equation}
\omega = \eta
\end{equation}		
\item Chari, Kehoe \& McGrattan (2000) and Woodford (2011) values: $\theta =10, \sigma =1, \omega = 1.25, \beta = 0.99$
These values are not controversial. I just want to check that the Frisch elasticity is Kosher, because by setting $\omega = 1.25$, we are implicitly setting the Frisch. 
According to Susanto (compare my summaries Part 1, p. 56 Mac and Part 2 p. 46-47 Mac), the inverse Frisch elasticity, $\varepsilon_{H,W}=\eta^{-1}$ needs to be 4 to flatten the labor supply curve (Barro-King comovement puzzle), but estimates suggests it's $<1$. Here, $\omega=1.25=\eta$ implies $\eta^{-1}=4/5 <1$. I guess that's at least in line with estimates. 
\item The value of $\kappa$ \\
Susanto suggests (somewhere in my notes) that for a NK model to display reasonable dynamics, $\kappa$ needs to be below 0.1, but preferably even less than 0.01. \\
Chari, Kehoe \& McGrattan (2000) and Woodford (2011) values with $\alpha = 0.5$ $\rightarrow$ $\kappa = 0.0842$. 
Chari, Kehoe \& McGrattan (2000) and Woodford (2011) values with $\alpha = 0.6$ $\rightarrow$ $\kappa = 0.0451$. 
Chari, Kehoe \& McGrattan (2000) and Woodford (2011) values with $\alpha = 0.66$ $\rightarrow$ $\kappa = 0.0298$. 
Chari, Kehoe \& McGrattan (2000) and Woodford (2011) values with $\alpha = 0.7$ $\rightarrow$ $\kappa = 0.0219$. 
\end{itemize}

\section{Back to estimation}
\subsection{Simulated data with different seed}
\begin{figure}[h!]
\subfigure[$\alpha^{true} = ( 0.1;    0.05;    0.001;    0.00001;    0.02;    0.09)$]{\includegraphics[scale=\myTinyFigScale]{\myFigPath command_acf_sim_data_univariate_alph_true_04_Jul_2020}}
\hfill
\subfigure[Gain for a simulation using $\alpha^{true}$]{\includegraphics[scale=\myTinyFigScale]{\myFigPath command_acf_sim_data_univariate_true_gain_sim_constant_only_pi_only_04_Jul_2020}}
\subfigure[$\alpha^{true}, \alpha_0, \hat{\alpha}$, right now for nonconvergent solution using default configs]{\includegraphics[scale=\myTinyFigScale]{\myFigPath command_GMM_LOMgain_univariate_alphas_04_Jul_2020}}
\hfill
\subfigure[Autocovariogram, right now for nonconvergent solution using default configs]{\includegraphics[scale=\mySmallerFigScale]{\myFigPath command_GMM_LOMgain_univariate_autocovariogram_sim_nfe_6_resnorm_1046_04_Jul_2020}}
\caption{A seed for shocks of \texttt{rng(1)} when true data was generated using \texttt{rng(0)}}
\end{figure}


\begin{figure}[h!]
\subfigure[$\alpha^{true}, \alpha_0, mean(\hat{\alpha}^{top10})$, \colorbox{yellow}{no additional moments}]{\includegraphics[scale=\myTinyFigScale]{\myFigPath command_GMM_LOMgain_univariate_alphas_resnorm_539_08_Jul_2020}}
\hfill
\subfigure[Autocovariogram, no additional moments]{\includegraphics[scale=\mySmallerFigScale]{\myFigPath command_GMM_LOMgain_univariate_autocovariogram_sim_nfe_5_resnorm_539_08_Jul_2020}}
\subfigure[$\alpha^{true}, \alpha_0, mean(\hat{\alpha}^{top10})$, \colorbox{yellow}{convexity moments} imposed with weight 10M]{\includegraphics[scale=\myTinyFigScale]{\myFigPath command_GMM_LOMgain_univariate_alphas_resnorm_563_08_Jul_2020}}
\hfill
\subfigure[Autocovariogram, convexity moments imposed with weight 10M]{\includegraphics[scale=\mySmallerFigScale]{\myFigPath command_GMM_LOMgain_univariate_autocovariogram_sim_nfe_5_resnorm_563_08_Jul_2020}}
\caption{Mean of top 10 estimates for 20 draws of shock histories when true data was generated using \texttt{rng(0)} and $\alpha^{true} = (0.05;    0.025;         0;    0.025;    0.05), fe \in (-2,2)$}
\end{figure}

\begin{figure}[h!]
\subfigure[$\alpha^{true}, \alpha_0, mean(\hat{\alpha}), \colorbox{yellow}{N=1000}$, convexity moments imposed with weight 10M]{\includegraphics[scale=\myTinyFigScale]{\myFigPath command_GMM_LOMgain_univariate_alphas_resnorm_153_09_Jul_2020}}
\hfill
\subfigure[Autocovariogram, convexity moments imposed with weight 10M]{\includegraphics[scale=\mySmallerFigScale]{\myFigPath command_GMM_LOMgain_univariate_autocovariogram_sim_nfe_5_resnorm_153_09_Jul_2020}}
\caption{Mean estimates for increasing $N$, imposing convexity}
\end{figure}

\begin{figure}[h!]
\subfigure[$\alpha^{true}, \alpha_0, mean(\hat{\alpha}), \colorbox{yellow}{N=2000}$, convexity moments imposed with weight 10M]{\includegraphics[scale=\mySmallFigScale]{\myFigPath command_GMM_LOMgain_univariate_alphas_resnorm_153_N_2000_09_Jul_2020}}
\hfill
\subfigure[Autocovariogram, convexity moments imposed with weight 10M]{\includegraphics[scale=\myFigScale]{\myFigPath command_GMM_LOMgain_univariate_autocovariogram_sim_nfe_5_resnorm_153_N_2000_09_Jul_2020}}
\subfigure[$\alpha^{true}, \alpha_0, mean(\hat{\alpha}), \colorbox{yellow}{N=10000} $, convexity moments imposed with weight 10M, $mean(\hat{\alpha})= 0.0463;
    0.0227;
    0.02;
    0.0223;
    0.0469)$]{\includegraphics[scale=\mySmallFigScale]{\myFigPath command_GMM_LOMgain_univariate_alphas_resnorm_110_10_Jul_2020}}
\hfill
\subfigure[Autocovariogram, convexity moments imposed with weight 10M]{\includegraphics[scale=\mySmallFigScale]{\myFigPath command_GMM_LOMgain_univariate_autocovariogram_sim_nfe_5_resnorm_110_10_Jul_2020}}
\caption{Mean estimates for increasing $N$, imposing convexity, continued}
\end{figure}

\clearpage
\subsubsection{The range $fe\in(-1,1)$ seems ill identified}

\begin{figure}[h!]
\subfigure[$\alpha^{true}, \alpha_0, mean(\hat{\alpha}), \colorbox{yellow}{N=10}$]{\includegraphics[scale=\myTinyFigScale]{\myFigPath command_GMM_LOMgain_univariate_alphas_resnorm_1025_13_Jul_2020}}
\hfill
\subfigure[Autocovariogram]{\includegraphics[scale=\mySmallerFigScale]{\myFigPath command_GMM_LOMgain_univariate_autocovariogram_sim_nfe_6_resnorm_1025_13_Jul_2020}}
\subfigure[$\alpha^{true}, \alpha_0, mean(\hat{\alpha}), \colorbox{yellow}{N=100} $, $mean(\hat{\alpha})= ( 0.1048;    0.0295;    0.0167;    0.0155;    0.0261;    0.1206)$]{\includegraphics[scale=\myTinyFigScale]{\myFigPath command_GMM_LOMgain_univariate_alphas_resnorm_643_13_Jul_2020}}
\hfill
\subfigure[Autocovariogram]{\includegraphics[scale=\mySmallerFigScale]{\myFigPath command_GMM_LOMgain_univariate_autocovariogram_sim_nfe_6_resnorm_643_13_Jul_2020}}
\subfigure[$\alpha^{true}, \alpha_0, mean(\hat{\alpha}), \colorbox{yellow}{N=1000} $, $mean(\hat{\alpha})= (0.1049;    0.0307;    0.0163;    0.0157;    0.0311;    0.1087)$]{\includegraphics[scale=\myTinyFigScale]{\myFigPath command_GMM_LOMgain_univariate_alphas_resnorm_149_13_Jul_2020}}
\hfill
\subfigure[Autocovariogram]{\includegraphics[scale=\mySmallerFigScale]{\myFigPath command_GMM_LOMgain_univariate_autocovariogram_sim_nfe_6_resnorm_149_13_Jul_2020}}
\caption{Mean estimates for increasing $N$, imposing convexity with weight 100K, truth with $nfe=6, fe \in(-3.5,3.5)$ }
\end{figure}

\begin{figure}[h!]
\subfigure[$\alpha^{true}, \alpha_0, mean(\hat{\alpha}), \colorbox{yellow}{N=10000}$ , $mean(\hat{\alpha})= (0.1114;    0.0296;    0.0163;    0.0163;    0.0297;    0.1123)$]{\includegraphics[scale=\myTinyFigScale]{\myFigPath command_GMM_LOMgain_univariate_alphas_resnorm_147_13_Jul_2020}}
\hfill
\subfigure[Autocovariogram]{\includegraphics[scale=\mySmallerFigScale]{\myFigPath command_GMM_LOMgain_univariate_autocovariogram_sim_nfe_6_resnorm_147_13_Jul_2020}}
\subfigure[$\alpha^{true}, \alpha_0, mean(\hat{\alpha}), \colorbox{yellow}{N=100000} $, $mean(\hat{\alpha})= ()$]{\includegraphics[scale=\myTinyFigScale]{\myFigPath placeholder}}
\hfill
\subfigure[Autocovariogram]{\includegraphics[scale=\mySmallerFigScale]{\myFigPath placeholder}}
\caption{Mean estimates for increasing $N$, imposing convexity with weight 100K, truth with $nfe=6, fe \in(-3.5,3.5)$, continued }
\end{figure}


\clearpage
\subsection{Autocovariogram for real data}
\begin{figure}[h!]
\subfigure[$mean(\hat{\alpha})=(0.0979;    0.0076;    0.0058;    0.006;    0.0864), \colorbox{yellow}{N=1000} $, \colorbox{yellow}{convexity weight 10M}]{\includegraphics[scale=\myTinyFigScale]{\myFigPath command_GMM_LOMgain_univariate_alph_opt_resnorm_87_10_Jul_2020}}
\hfill
\subfigure[Autocovariogram]{\includegraphics[scale=\mySmallerFigScale]{\myFigPath command_GMM_LOMgain_univariate_autocovariogram_nfe_5_resnorm_87_10_Jul_2020}}

\subfigure[$mean(\hat{\alpha})=(0.0773;    0.0103;    0.0097;    0.0104;    0.0771), \colorbox{yellow}{N=1000} $, \colorbox{yellow}{convexity weight 100K}]{\includegraphics[scale=\myTinyFigScale]{\myFigPath command_GMM_LOMgain_univariate_alph_opt_resnorm_87_11_Jul_2020}}
\hfill
\subfigure[Autocovariogram]{\includegraphics[scale=\mySmallerFigScale]{\myFigPath command_GMM_LOMgain_univariate_autocovariogram_nfe_5_resnorm_87_11_Jul_2020}}
\caption{Mean estimated parameters over the cross-section of size $N$, convexity imposed, mean moment not imposed}
\end{figure}

\clearpage
\section{Impulse responses to iid monpol shocks across a wide range of learning models}
$T=400, N=100, n_{drop}=5,$ shock imposed at $t=25$, calibration as above, Taylor rule assumed to be known, PLM = learn constant only, of inflation only.

\begin{figure}[h!]
\subfigure[Decreasing gain learning]{\includegraphics[scale=\myTinyFigScale]{\myFigPath command_IRFs_approx_pretty_RIR_LH_monpol_dgain_constant_only_pi_only_2020_07_07}}
\hfill
\subfigure[Mean gain]{\includegraphics[scale=\myTinyFigScale]{\myFigPath command_IRFs_approx_pretty_invgain_dgain_constant_only_pi_only_params_psi_pi_1_5_psi_x_0_gbar_0_145_thetbar_16_thettilde_2_5_kap_0_8_lamx_0_05_lami_0_2020_07_07}}
\subfigure[Constant gain learning]{\includegraphics[scale=\myTinyFigScale]{\myFigPath command_IRFs_approx_pretty_RIR_LH_monpol_cgain_constant_only_pi_only_2020_07_07}}
\hfill
\subfigure[Mean gain]{\includegraphics[scale=\myTinyFigScale]{\myFigPath command_IRFs_approx_pretty_invgain_cgain_constant_only_pi_only_params_psi_pi_1_5_psi_x_0_gbar_0_145_thetbar_16_thettilde_2_5_kap_0_8_lamx_0_05_lami_0_2020_07_07}}
\subfigure[CEMP criterion (vector)]{\includegraphics[scale=\myTinyFigScale]{\myFigPath command_IRFs_approx_pretty_RIR_LH_monpol_again_critCEMP_constant_only_pi_only_2020_07_07}}
\hfill
\subfigure[Mean gain]{\includegraphics[scale=\myTinyFigScale]{\myFigPath command_IRFs_approx_pretty_invgain_again_critCEMP_constant_only_pi_only_params_psi_pi_1_5_psi_x_0_gbar_0_145_thetbar_16_thettilde_2_5_kap_0_8_lamx_0_05_lami_0_2020_07_07}}
\caption{IRFs and gain history (sample means) }
\end{figure}


\begin{figure}[h!]
\subfigure[CUSUM criterion (vector)]{\includegraphics[scale=\myTinyFigScale]{\myFigPath command_IRFs_approx_pretty_RIR_LH_monpol_again_critCUSUM_constant_only_pi_only_2020_07_07}}
\hfill
\subfigure[Mean gain]{\includegraphics[scale=\myTinyFigScale]{\myFigPath command_IRFs_approx_pretty_invgain_again_critCUSUM_constant_only_pi_only_params_psi_pi_1_5_psi_x_0_gbar_0_145_thetbar_16_thettilde_2_5_kap_0_8_lamx_0_05_lami_0_2020_07_07}}
\subfigure[Smooth criterion, approximated, using $\alpha^{true}= (0.05;0.025;0;0.025;0.05)$, on $fe \in (-2,2)$.]{\includegraphics[scale=\myTinyFigScale]{\myFigPath command_IRFs_approx_pretty_RIR_LH_monpol_again_critsmooth_constant_only_pi_only_2020_07_07}}
\hfill
\subfigure[Mean gain]{\includegraphics[scale=\myTinyFigScale]{\myFigPath command_IRFs_approx_pretty_invgain_again_critsmooth_constant_only_pi_only_params_psi_pi_1_5_psi_x_0_gbar_0_145_thetbar_16_thettilde_2_5_kap_0_8_lamx_0_05_lami_0_2020_07_07}}

\caption{IRFs and gain history (sample means), continued }
\end{figure}

%\begin{figure}[h!]
%\subfigure[]{\includegraphics[scale=\myTinyFigScale]{\myFigPath placeholder}}
%\hfill
%\subfigure[$\hat{\alpha}$]{\includegraphics[scale=\myTinyFigScale]{\myFigPath placeholder}}
%\subfigure[Residuals]{\includegraphics[scale=\myTinyFigScale]{\myFigPath placeholder}}
%\hfill
%\subfigure[Moments]{\includegraphics[scale=\myTinyFigScale]{\myFigPath placeholder}}
%\caption{caption}
%\end{figure}

%%%%%%%%%%%%%%%%%%%%%%%%%%%%%%%%%%%%%%%%%%%%%%%%%%%%%%%%%%%%%                                              APPENDIX
%%%%%%%%%%%%%%%%%%%%%%%%%%%%%%%%%%%%%%%%%%%%%%%%%%%%%%%%%%%%
    \clearpage
%    \newpage
\appendix
% the following command makes equation numbering include the section first, but just for what follows
\numberwithin{equation}{section}
\section{Model summary}

\vspace{-0.5cm}

\begin{align}
x_t &=  -\sigma i_t +\hat{\E}_t \sum_{T=t}^{\infty} \beta^{T-t }\big( (1-\beta)x_{T+1} - \sigma(\beta i_{T+1} - \pi_{T+1}) +\sigma r_T^n \big)  \label{A1}  \\
\pi_t &= \kappa x_t +\hat{\E}_t \sum_{T=t}^{\infty} (\alpha\beta)^{T-t }\big( \kappa \alpha \beta x_{T+1} + (1-\alpha)\beta \pi_{T+1} + u_T\big) \label{A2}  \\
i_t &= \psi_{\pi}\pi_t + \psi_{x} x_t  + \bar{i}_t \label{TR} \quad \quad (\text{if imposed})
\end{align}

\vspace{-1.2cm}

\begin{align}
\text{PLM:} \quad \quad & \hat{\E}_t z_{t+h}  =  a_{t-1} + bh_x^{h-1}s_t  \quad \forall h\geq 1 \quad \quad b = g_x\; h_x \quad \quad  \label{PLM} \\
\text{Updating:} \quad \quad & a_{t}  =a_{t-1} +k_t^{-1}\big(z_{t} -(a_{t-1}+b s_{t-1}) \big)  \label{A5} \\
\text{Anchoring function:} \quad \quad & k^{-1}_t  = \rho_k k^{-1}_{t-1} + \gamma_k fe_{t-1}^2 \label{A6}\\
\text{Forecast error:} \quad \quad & fe_{t-1}  = z_t - (a_{t-1}+b s_{t-1}) \label{A7} \\
\text{LH expectations:} \quad \quad & f_a(t) = \frac{1}{1-\alpha\beta}a_{t-1}  + b(\mathbb{I}_{nx} - \alpha\beta h)^{-1}s_t \quad \quad  f_b(t) = \frac{1}{1-\beta}a_{t-1}  + b(\mathbb{I}_{nx} - \beta h)^{-1}s_t  \label{A8}
\end{align}

\vspace{-0.5cm}

This notation captures vector learning ($z$ learned) for intercept only. For scalar learning, $a_t= \begin{pmatrix} \bar{\pi}_t & 0 & 0\end{pmatrix}' $ and $b_1$ designates the first row of $b$. The observables $(\pi, x)$ are determined as:
\begin{align}
x_t &=  -\sigma i_t + \begin{bmatrix} \sigma & 1-\beta & -\sigma\beta \end{bmatrix} f_b + \sigma \begin{bmatrix} 1 & 0 & 0 \end{bmatrix} (\mathbb{I}_{nx} - \beta h_x)^{-1} s_t \label{A9} \\
\pi_t &= \kappa x_t  + \begin{bmatrix} (1-\alpha)\beta & \kappa\alpha\beta & 0 \end{bmatrix}  f_a + \begin{bmatrix} 0 & 0 & 1 \end{bmatrix}  (\mathbb{I}_{nx} - \alpha \beta h_x)^{-1}  s_t \label{A10}
\end{align}

\section{Target criterion}\label{target_crit_levels}
The target criterion in the simplified model (scalar learning of inflation intercept only, $k_t^{-1} = \mathbf{g}(fe_{t-1})$):
\begin{align*}
\pi_t  = -\frac{\lambda_x}{\kappa}\bigg\{x_t - \frac{(1-\alpha)\beta}{1-\alpha\beta} \bigg(k_t^{-1}+((\pi_t - \bar{\pi}_{t-1}-b_1 s_{t-1}))\mathbf{g}_{\pi}(t) \bigg) \\
\bigg(\E_t\sum_{i=1}^{\infty}x_{t+i}\prod_{j=0}^{i-1}(1-k_{t+1+j}^{-1} - (\pi_{t+1+j} - \bar{\pi}_{t+j}-b_1 s_{t+j})\mathbf{g_{\bar{\pi}}}(t+j)) \bigg)
\bigg\} \numberthis \label{target}
\end{align*}
where I'm using the notation that $\prod_{j=0}^{0} \equiv 1$. For interpretation purposes, let me rewrite this as follows:
\begin{align*}
\pi_t  = & \; \textcolor{red}{-\frac{\lambda_x}{\kappa} x_t} \textcolor{blue}{ \; + \frac{\lambda_x}{\kappa} \frac{(1-\alpha)\beta}{1-\alpha\beta} \bigg(k_t^{-1}+ fe^{eve}_{t|t-1}\mathbf{g}_{\pi}(t) \bigg)\E_t\sum_{i=1}^{\infty}x_{t+i}}  \\
& \textcolor{mygreen}{- \frac{\lambda_x}{\kappa} \frac{(1-\alpha)\beta}{1-\alpha\beta} \bigg(k_t^{-1}+ fe^{eve}_{t|t-1}\mathbf{g}_{\pi}(t) \bigg) \bigg(\E_t\sum_{i=1}^{\infty}x_{t+i}\prod_{j=0}^{i-1}(k_{t+1+j}^{-1} + fe^{eve}_{t+1+j|t+j})\mathbf{g_{\bar{\pi}}}(t+j) \bigg)}
\numberthis \label{target_interpretation}
\end{align*}
Interpretation: \textcolor{red}{tradeoffs from discretion in RE} + \textcolor{blue}{effect of current level and change of the gain on future tradeoffs} + \textcolor{mygreen}{effect of future expected levels and changes of the gain on future tradeoffs}

%\section{A target criterion system for an anchoring function specified for gain changes}\label{target_crit_changes}
%\begin{equation}
%k_t = k_{t-1} + \mathbf{g}(fe_{t|t-1})
%\end{equation}
%Turns out the $k_{t-1}$ adds one $\varphi_{6,t+1}$ too many which makes the target criterion unwieldy. The FOCs of the Ramsey problem are
%\begin{align}
%& 2\pi_t + 2\frac{\lambda}{\kappa}x_t -k_t^{-1} \varphi_{5,t} - \mathbf{g}_{\pi}(t)\varphi_{6,t}  = 0 \label{gaspar22}\\
%& c x_{t+1} + \varphi_{5,t} -(1-k_t^{-1})\varphi_{5,t+1} +\mathbf{g}_{\bar{\pi}}(t)\varphi_{6,t+1} = 0 \label{gaspar21}\\
%& \varphi_{6,t} \; \textcolor{red}{+\; \varphi_{6,t+1}} = fe_t \varphi_{5,t} \label{constraints}
%\end{align}
%where the red multiplier is the new element vis-a-vis the case where the anchoring function is specified in levels ($k_t^{-1} = \mathbf{g}(fe_{t-1})$, as in App. \ref{target_crit_levels}), and I'm using the shorthand notation
%\begin{align}
%c & = -\frac{2(1-\alpha)\beta}{1-\alpha\beta}\frac{\lambda}{\kappa} \\ 
%fe_t & = \pi_t - \bar{\pi}_{t-1}-b s_{t-1}
%\end{align}
%(\ref{gaspar22}) says that in anchoring, the discretion tradeoff is complemented with tradeoffs coming from learning ($\varphi_{5,t}$), which are more binding when expectations are unanchored ($k_{t}^{-1}$ high). Moreover, the change in the anchoring of expectations imposes an additional constraint ($\varphi_{6,t}$), which is more strongly binding if the gain responds strongly to inflation ($\mathbf{g}_{\pi}(t)$).
%One can simplify this three-equation-system to:
%\begin{align}
%\varphi_{6,t} & = -c fe_t x_{t+1} + \bigg(1+ \frac{fe_t}{fe_{t+1}}(1-k_{t+1}^{-1}) -fe_t \mathbf{g}_{\bar{\pi}}(t) \bigg) \varphi_{6,t+1} -\frac{fe_t}{fe_{t+1}}(1-k_{t+1}^{-1})\varphi_{6,t+2}\label{6'} \\
%0 & = 2\pi_t + 2\frac{\lambda}{\kappa}x_t   - \bigg( \frac{k_t^{-1}}{fe_t} + \mathbf{g}_{\pi}(t)\bigg)\varphi_{6,t} + \frac{k_t^{-1}}{fe_t}\varphi_{6,t+1}\label{1'}
%\end{align}
%Unfortunately, I haven't been able to solve (\ref{6'}) for $\varphi_{6,t}$ and therefore I can't express the target criterion so nicely as before. The only thing I can say is to direct the targeting rule-following central bank to compute $\varphi_{6,t}$ as the solution to (\ref{1'}), and then evaluate (\ref{6'}) as a target criterion. The solution to (\ref{1'}) is given by:
%\begin{equation}
%\varphi_{6,t} = -2\E_t\sum_{i=0}^{\infty}(\pi_{t+i}+\frac{\lambda_x}{\kappa}x_{t+i})\prod_{j=0}^{i-1}\frac{\frac{k_{t+j}^{-1}}{fe_{t+j}}}{\frac{k_{t+j}^{-1}}{fe_{t+j}} + \mathbf{g}_{\pi}(t+j)} \label{sol1'}
%\end{equation}
%Interpretation: the anchoring constraint is not binding ($\varphi_{6,t}=0$) if the CB always hits the target (
%$\pi_{t+i}+\frac{\lambda_x}{\kappa}x_{t+i} = 0 \quad \forall i$); or expectations are always anchored ($k_{t+j}^{-1}=0 \quad \forall j$). 


\end{document}

%%%%%%%%%%%%%    SUBFIGURE  %%%%%%%%%%%
%\begin{figure}[h!]
%\subfigure[Hodrick-Prescott, $\lambda=1600$]{\includegraphics[scale=\myAdjustableFigScale]{\myFigPath materials22_gain_dhat_HP}}
%\hfill % this is great to intro dpace between subfigures
%\subfigure[Hamilton, 4 lags, $h=8$]{\includegraphics[scale=\myAdjustableFigScale]{\myFigPath materials22_gain_dhat_Hamilton}}
%\subfigure[Baxter-King, $(6,32)$ quarters, truncation at 12 lags]{\includegraphics[scale=\myAdjustableFigScale]{\myFigPath materials22_gain_dhat_BK}}
%\caption{Inverse gain for $\hat{d}$ for the different filters}
%\end{figure}

%%%%%%%%%%%%%    TABLE  %%%%%%%%%%%
%\begin{center}
%\begin{table}[h!]
%\caption{$\hat{d}$}
%\begin{tabular}{ c |c |c }
%  & $W = I$ & $W = \text{diag}(\hat{\sigma}_{ac(0)}, \dots, \hat{\sigma}_{ac(K)})$ \\ 
%  \hline
% HP & 77.7899 & 10 \\  
% \hline
% Hamilton & 32.1649 & 10 \\  
% \hline
% BK & 90.3929 & 10    
%\end{tabular}
%\end{table}
%\end{center}





