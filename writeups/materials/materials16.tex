\documentclass[11pt]{article}
\usepackage{amsmath, amsthm, amssymb,lscape, natbib}
\usepackage{mathtools}
\usepackage{subfigure}
\usepackage[font=footnotesize,labelfont=bf]{caption}
\usepackage{graphicx}
\usepackage{colortbl}
\usepackage{hhline}
\usepackage{multirow}
\usepackage{multicol}
\usepackage{setspace}
\usepackage[final]{pdfpages}
\usepackage[left=2.5cm,top=2.5cm,right=2.5cm, bottom=2.5cm]{geometry}
\usepackage{natbib} 
\usepackage{bibentry} 
\newcommand{\bibverse}[1]{\begin{verse} \bibentry{#1} \end{verse}}
\newcommand{\vs}{\vspace{.3in}}
\renewcommand{\ni}{\noindent}
\usepackage{xr-hyper}
\usepackage[]{hyperref}
\hypersetup{
    colorlinks=true,
    linkcolor=blue,
    filecolor=magenta,      
    urlcolor=cyan,
}
 
\urlstyle{same}
\usepackage[capposition=top]{floatrow}
\usepackage{amssymb}
\usepackage{relsize}
\usepackage[dvipsnames]{xcolor}
\usepackage{fancyhdr}
\usepackage{tikz}
 
\pagestyle{fancy} % customize header and footer
\fancyhf{} % clear initial header and footer
%\rhead{Overleaf}
\lhead{\centering \rightmark} % this adds subsection number and name
\lfoot{\centering \rightmark} 
\rfoot{\thepage} % put page number (the centering command puts it in the middle, don't matter if you put it in right or left footer)

\def \myFigPath {../figures/} 
% BE CAREFUL WITH FIGNAMES, IN LATEX THEY'RE NOT CASE SENSITIVE!!
\def \myTablePath {../tables/} 

%\definecolor{mygreen}{RGB}{0, 100, 0}
\definecolor{mygreen}{RGB}{0, 128, 0}

\definecolor{citec}{rgb}{0,0,.5}
\definecolor{linkc}{rgb}{0,0,.6}
\definecolor{bcolor}{rgb}{1,1,1}
\hypersetup{
%hidelinks = true
  colorlinks = true,
  urlcolor=linkc,
  linkcolor=linkc,
  citecolor = citec,
  filecolor = linkc,
  pdfauthor={Laura G\'ati},
}


\geometry{left=.83in,right=.89in,top=1in,
bottom=1in}
\linespread{1.5}
\renewcommand{\[}{\begin{equation}}
\renewcommand{\]}{\end{equation}}

% New Options
\newtheorem{prop}{Proposition}
\newtheorem{definition}{Definition}[section]
\newtheorem*{remark}{Remark}
\newtheorem{lemma}{Lemma}
\newtheorem{corollary}{Corollary}
\newtheorem{conjecture}{Conjecture}

%\newtheorem{theorem}{Theorem}[section] % the third argument specifies that their number will be adopted to the section
%\newtheorem{corollary}{Corollary}[theorem]
%\newtheorem{lemma}[theorem]{Lemma}
%\declaretheorem{proposition}
%\linespread{1.3}
%\raggedbottom
%\font\reali=msbm10 at 12pt

% New Commands
\newcommand{\real}{\hbox{\reali R}}
\newcommand{\realp}{\hbox{\reali R}_{\scriptscriptstyle +}}
\newcommand{\realpp}{\hbox{\reali R}_{\scriptscriptstyle ++}}
\newcommand{\R}{\mathbb{R}}
\DeclareMathOperator{\E}{\mathbb{E}}
\DeclareMathOperator{\argmin}{arg\,min}
\newcommand\w{3.0in}
\newcommand\wnum{3.0}
\def\myFigWidth{5.3in}
\def\mySmallerFigWidth{2.1in}
\def\myEvenBiggerFigScale{0.8}
\def\myPointSixFigScale{0.6}
\def\myBiggerFigScale{0.4}
\def\myFigScale{0.3}
\def\myMediumFigScale{0.25}
\def\mySmallFigScale{0.22}
\def\mySmallerFigScale{0.18}
\def\myTinyFigScale{0.16}
\def\myPointFourteenFigScale{0.14}
\def\myTinierFigScale{0.12}
\def\myAdjustableFigScale{0.14}
\newcommand\numberthis{\addtocounter{equation}{1}\tag{\theequation}} % this defines a command to make align only number this line
\newcommand{\code}[1]{\texttt{#1}} %code %

\renewcommand*\contentsname{Overview}
\setcounter{tocdepth}{2}

% define a command to make a huge question mark (it works in math mode)
\newcommand{\bigqm}[1][1]{\text{\larger[#1]{\textbf{?}}}}

\begin{document}

\linespread{1.0}

\title{Materials 16 - Preparing Clough rough draft, simulation-based results and abstract-ish something}
\author{Laura G\'ati} 
\date{\today}
\maketitle

%%%%%%%%%%%%%%%%%%%%             DOCUMENT           %%%%%%%%%%%%%%%%%% 

\tableofcontents

%\listoffigures

\newpage
\section{Model summary}
\begin{align}
x_t &=  -\sigma i_t +\hat{\E}_t \sum_{T=t}^{\infty} \beta^{T-t }\big( (1-\beta)x_{T+1} - \sigma(\beta i_{T+1} - \pi_{T+1}) +\sigma r_T^n \big)  \label{prestons18}  \\
\pi_t &= \kappa x_t +\hat{\E}_t \sum_{T=t}^{\infty} (\alpha\beta)^{T-t }\big( \kappa \alpha \beta x_{T+1} + (1-\alpha)\beta \pi_{T+1} + u_T\big) \label{prestons19}  \\
i_t &= \psi_{\pi}\pi_t + \psi_{x} x_t  + \bar{i}_t \label{TR}
\end{align}
\begin{equation}
\hat{\E}_t z_{t+h} =  \bar{z}_{t-1} + bh_x^{h-1}s_t  \quad \forall h\geq 1 \quad \quad b = g_x\; h_x \quad \quad \text{PLM} \label{PLM}
\end{equation}
\begin{equation}
\bar{z}_{t} = \bar{z}_{t-1} +k_t^{-1}\underbrace{\big(z_{t} -(\bar{z}_{t-1}+bs_{t-1}) \big)}_{\text{fcst error using (\ref{PLM})} } 
\end{equation}
(Vector learning. For scalar learning, $\bar{z}= \begin{pmatrix} \bar{\pi} & 0 & 0\end{pmatrix}' $. I'm also not writing the case where the slope $b$ is also learned.)
 \begin{align*}
k_t & = \begin{cases} k_{t-1}+1 \quad \text{when} \quad \theta^{CEMP} < \bar{\theta} \quad \text{or}  \quad  \theta_t < \tilde{\theta}  \\ \bar{g}^{-1}  \quad \text{otherwise.}\numberthis
\end{cases} 
\end{align*}

\subsection{The CEMP vs. the CUSUM criterion}

CEMP's criterion  
\begin{equation}
\theta_t^{CEMP} = \max | \Sigma^{-1} ( \phi - \begin{bmatrix} F & G \end{bmatrix}) |
\end{equation}
where $\Sigma$ is the VC matrix of shocks, $\phi$ is the estimated matrix, $[F,G]$ is the ALM.

\noindent CUSUM-criterion
\begin{align}
\omega_t & =  \omega_{t-1} + \kappa k_{t-1}^{-1}(f_t f_t'  -\omega_{t-1})\\
\theta_t^{CUSUM} & =  \theta_{t-1} + \kappa k_{t-1}^{-1}(f_t'\omega_t^{-1}f_t -\theta_{t-1})
\end{align}

where $f$ is the most recent forecast error and $\omega$ is the estimated FEV. 



\newpage
\section{Simulated $\psi_{\pi}^*$ and CB losses, RE against learning, fixing $\psi_x = 0$}

\subsection{RE against CEMP-criterion}
\begin{figure}[h!]
\subfigure[$\bar{\theta}=2$, Loss]{\includegraphics[scale=\myAdjustableFigScale]{\myFigPath plot_sim_loss_loss_again_critCEMP_constant_only_params_psi_pi_1_5_psi_x_0_gbar_0_145_thetbar_2_thettilde_2_5_kap_0_8_alph_CB_0_2020_02_06}}
\subfigure[$\bar{\theta}=2$ Inverse gains for optimal $\psi_{\pi}^*$]{\includegraphics[scale=\myAdjustableFigScale]{\myFigPath command_IRFs_anchoring_loss_again_critCEMP_constant_only_params_psi_pi_1_3_psi_x_0_gbar_0_145_thetbar_2_thettilde_2_5_kap_0_8_alph_CB_0_2020_02_06}}
\subfigure[$\bar{\theta}=4$, Loss]{\includegraphics[scale=\myAdjustableFigScale]{\myFigPath plot_sim_loss_loss_again_critCEMP_constant_only_params_psi_pi_1_5_psi_x_0_gbar_0_145_thetbar_4_thettilde_2_5_kap_0_8_alph_CB_0_2020_02_06}}
\subfigure[$\bar{\theta}=4$ Inverse gains for optimal $\psi_{\pi}^*$]{\includegraphics[scale=\myAdjustableFigScale]{\myFigPath command_IRFs_anchoring_loss_again_critCEMP_constant_only_params_psi_pi_1_4_psi_x_0_gbar_0_145_thetbar_4_thettilde_2_5_kap_0_8_alph_CB_0_2020_02_06}}
\subfigure[$\bar{\theta}=6$, Loss]{\includegraphics[scale=\myAdjustableFigScale]{\myFigPath plot_sim_loss_loss_again_critCEMP_constant_only_params_psi_pi_1_5_psi_x_0_gbar_0_145_thetbar_6_thettilde_2_5_kap_0_8_alph_CB_0_2020_02_06}}
\subfigure[$\bar{\theta}=6$ Inverse gains for optimal $\psi_{\pi}^*$]{\includegraphics[scale=\myAdjustableFigScale]{\myFigPath command_IRFs_anchoring_loss_again_critCEMP_constant_only_params_psi_pi_1_55_psi_x_0_gbar_0_145_thetbar_6_thettilde_2_5_kap_0_8_alph_CB_0_2020_02_06}}
\caption{CB losses, RE against learning with CEMP's criterion}
\end{figure}

\begin{itemize}
\item When $\bar{\theta}=2$, you get unanchored for or $\psi_{\pi} \geq 1.25$
\item When $\bar{\theta}=4$, you get unanchored for or $\psi_{\pi} \geq 1.8$
\item When $\bar{\theta}=6$, you get unanchored for $\psi_{\pi} \geq 2.5$
\end{itemize}
$\rightarrow$ so usually when the choice of aggressiveness on inflation matters for anchoring, mon pol chooses to anchor. But not when this would involve a ``too low'' $\psi_{\pi}$.
\newpage
\subsection{RE against CUSUM-criterion}
\begin{figure}[h!]
\subfigure[$\tilde{\theta}=1$, Loss]{\includegraphics[scale=\myAdjustableFigScale]{\myFigPath plot_sim_loss_loss_again_critCUSUM_constant_only_params_psi_pi_1_5_psi_x_0_gbar_0_145_thetbar_4_thettilde_1_kap_0_8_alph_CB_0_2020_02_06}}
\subfigure[$\tilde{\theta}=1$ Inverse gains for optimal $\psi_{\pi}^*$]{\includegraphics[scale=\myAdjustableFigScale]{\myFigPath command_IRFs_anchoring_loss_again_critCUSUM_constant_only_params_psi_pi_1_3_psi_x_0_gbar_0_145_thetbar_4_thettilde_1_kap_0_8_alph_CB_0_2020_02_06}}

\subfigure[$\tilde{\theta}=2.5$, Loss]{\includegraphics[scale=\myAdjustableFigScale]{\myFigPath plot_sim_loss_loss_again_critCUSUM_constant_only_params_psi_pi_1_5_psi_x_0_gbar_0_145_thetbar_4_thettilde_2_5_kap_0_8_alph_CB_0_2020_02_06}}
\subfigure[$\tilde{\theta}=2.5$ Inverse gains for optimal $\psi_{\pi}^*$]{\includegraphics[scale=\myAdjustableFigScale]{\myFigPath command_IRFs_anchoring_loss_again_critCUSUM_constant_only_params_psi_pi_1_6_psi_x_0_gbar_0_145_thetbar_4_thettilde_2_5_kap_0_8_alph_CB_0_2020_02_06}}

\subfigure[$\tilde{\theta}=4$, Loss]{\includegraphics[scale=\myAdjustableFigScale]{\myFigPath plot_sim_loss_loss_again_critCUSUM_constant_only_params_psi_pi_1_5_psi_x_0_gbar_0_145_thetbar_4_thettilde_4_kap_0_8_alph_CB_0_2020_02_06}}
\subfigure[$\tilde{\theta}=4$ Inverse gains for optimal $\psi_{\pi}^*$]{\includegraphics[scale=\myAdjustableFigScale]{\myFigPath command_IRFs_anchoring_loss_again_critCUSUM_constant_only_params_psi_pi_1_3_psi_x_0_gbar_0_145_thetbar_4_thettilde_4_kap_0_8_alph_CB_0_2020_02_06}}
\caption{CB losses, RE against learning with CUSUM criterion}
\end{figure}

Note:
\begin{itemize}
\item When $\tilde{\theta}=1$, you never get anchoring for any value of $\psi_{\pi}\in(1,2]$
\item When $\tilde{\theta}=2.5$, you're unanchored for low $\psi_{\pi}$, anchored for high
\item When $\tilde{\theta}=4$, you always get anchoring for any value of $\psi_{\pi}\in(1,2.5]$
\end{itemize}
$\rightarrow$ so $\tilde{\theta}=2.5$ is the interesting case because this is where the choice of aggressiveness on inflation matters for anchoring. As we can see, when mon pol can, it chooses to anchor.

\section{``Best-case'' abstract}
\begin{abstract}
This paper analyzes optimal monetary policy in a model where expectation formation is characterized by potential anchoring of expectations. In the spirit of Carvalho et al. 2019, I model anchored expectations using an endogenous gain adaptive learning framework. Expectations are anchored if the private sector's forecast errors are sufficiently small such that agents choose a decreasing gain. Anchored expectations are thus a metric for the public's trust in the central bank's commitment to the long-run target. I embed the anchoring expectation formation in an otherwise standard macro model with nominal rigidities and solve for optimal monetary policy. I find that the central bank trades off the costs and benefits of anchoring expectations. Having anchored expectations reduces the volatility of observables, but anchoring expectations is costly in terms of inducing volatility. Optimal policy is therefore conditioned on the stance of expectations.   
\end{abstract}

\section{Optimal policy}

\begin{enumerate}
\item You seem to have doubts.
\item Endogenous gain framework is not differentiable. 
\end{enumerate}


\end{document}





