\documentclass[11pt]{article}
\usepackage{amsmath, amsthm, amssymb,lscape, natbib}
\usepackage{mathtools}
\usepackage{subfigure}
\usepackage[font=footnotesize,labelfont=bf]{caption}
\usepackage{graphicx}
\usepackage{colortbl}
\usepackage{hhline}
\usepackage{multirow}
\usepackage{multicol}
\usepackage{setspace}
\usepackage[final]{pdfpages}
\usepackage[left=2.5cm,top=2.5cm,right=2.5cm, bottom=2.5cm]{geometry}
\usepackage{natbib} 
\usepackage{bibentry} 
\newcommand{\bibverse}[1]{\begin{verse} \bibentry{#1} \end{verse}}
\newcommand{\vs}{\vspace{.3in}}
\renewcommand{\ni}{\noindent}
\usepackage{xr-hyper}
\usepackage[]{hyperref}
\hypersetup{
    colorlinks=true,
    linkcolor=blue,
    filecolor=magenta,      
    urlcolor=cyan,
}
 
\urlstyle{same}
\usepackage[capposition=top]{floatrow}
\usepackage{amssymb}
\usepackage{relsize}
\usepackage[dvipsnames]{xcolor}
\usepackage{fancyhdr}
\usepackage{tikz}
 
\pagestyle{fancy} % customize header and footer
\fancyhf{} % clear initial header and footer
%\rhead{Overleaf}
\lhead{\centering \rightmark} % this adds subsection number and name
\lfoot{\centering \rightmark} 
\rfoot{\thepage} % put page number (the centering command puts it in the middle, don't matter if you put it in right or left footer)

\def \myFigPath {../figures/} 
% BE CAREFUL WITH FIGNAMES, IN LATEX THEY'RE NOT CASE SENSITIVE!!
\def \myTablePath {../tables/} 

%\definecolor{mygreen}{RGB}{0, 100, 0}
\definecolor{mygreen}{RGB}{0, 128, 0}

\definecolor{citec}{rgb}{0,0,.5}
\definecolor{linkc}{rgb}{0,0,.6}
\definecolor{bcolor}{rgb}{1,1,1}
\hypersetup{
%hidelinks = true
  colorlinks = true,
  urlcolor=linkc,
  linkcolor=linkc,
  citecolor = citec,
  filecolor = linkc,
  pdfauthor={Laura G\'ati},
}


\geometry{left=.83in,right=.89in,top=1in,
bottom=1in}
\linespread{1.5}
\renewcommand{\[}{\begin{equation}}
\renewcommand{\]}{\end{equation}}

% New Options
\newtheorem{prop}{Proposition}
\newtheorem{definition}{Definition}[section]
\newtheorem*{remark}{Remark}
\newtheorem{lemma}{Lemma}
\newtheorem{corollary}{Corollary}
\newtheorem{conjecture}{Conjecture}

%\newtheorem{theorem}{Theorem}[section] % the third argument specifies that their number will be adopted to the section
%\newtheorem{corollary}{Corollary}[theorem]
%\newtheorem{lemma}[theorem]{Lemma}
%\declaretheorem{proposition}
%\linespread{1.3}
%\raggedbottom
%\font\reali=msbm10 at 12pt

% New Commands
\newcommand{\real}{\hbox{\reali R}}
\newcommand{\realp}{\hbox{\reali R}_{\scriptscriptstyle +}}
\newcommand{\realpp}{\hbox{\reali R}_{\scriptscriptstyle ++}}
\newcommand{\R}{\mathbb{R}}
\DeclareMathOperator{\E}{\mathbb{E}}
\DeclareMathOperator{\argmin}{arg\,min}
\newcommand\w{3.0in}
\newcommand\wnum{3.0}
\def\myFigWidth{5.3in}
\def\mySmallerFigWidth{2.1in}
\def\myEvenBiggerFigScale{0.8}
\def\myPointSixFigScale{0.6}
\def\myBiggerFigScale{0.4}
\def\myFigScale{0.3}
\def\myMediumFigScale{0.25}
\def\mySmallFigScale{0.22}
\def\mySmallerFigScale{0.18}
\def\myTinyFigScale{0.16}
\def\myPointFourteenFigScale{0.14}
\def\myTinierFigScale{0.12}
\def\myAdjustableFigScale{0.14}
\newcommand\numberthis{\addtocounter{equation}{1}\tag{\theequation}} % this defines a command to make align only number this line
\newcommand{\code}[1]{\texttt{#1}} %code %

\renewcommand*\contentsname{Overview}
\setcounter{tocdepth}{2}

% define a command to make a huge question mark (it works in math mode)
\newcommand{\bigqm}[1][1]{\text{\larger[#1]{\textbf{?}}}}

\begin{document}

\linespread{1.0}

\title{Materials 15 - More on the CEMP vs. CUSUM criteria and optimal Taylor rule coefficients}
\author{Laura G\'ati} 
\date{\today}
\maketitle

%%%%%%%%%%%%%%%%%%%%             DOCUMENT           %%%%%%%%%%%%%%%%%% 

\tableofcontents

%\listoffigures

\newpage
\section{Model summary}
\begin{align}
x_t &=  -\sigma i_t +\hat{\E}_t \sum_{T=t}^{\infty} \beta^{T-t }\big( (1-\beta)x_{T+1} - \sigma(\beta i_{T+1} - \pi_{T+1}) +\sigma r_T^n \big)  \label{prestons18}  \\
\pi_t &= \kappa x_t +\hat{\E}_t \sum_{T=t}^{\infty} (\alpha\beta)^{T-t }\big( \kappa \alpha \beta x_{T+1} + (1-\alpha)\beta \pi_{T+1} + u_T\big) \label{prestons19}  \\
i_t &= \psi_{\pi}\pi_t + \psi_{x} x_t  + \bar{i}_t \label{TR}
\end{align}
\begin{equation}
\hat{\E}_t z_{t+h} =  \bar{z}_{t-1} + bh_x^{h-1}s_t  \quad \forall h\geq 1 \quad \quad b = g_x\; h_x \quad \quad \text{PLM} \label{PLM}
\end{equation}
\begin{equation}
\bar{z}_{t} = \bar{z}_{t-1} +k_t^{-1}\underbrace{\big(z_{t} -(\bar{z}_{t-1}+bs_{t-1}) \big)}_{\text{fcst error using (\ref{PLM})} } 
\end{equation}
(Vector learning. For scalar learning, $\bar{z}= \begin{pmatrix} \bar{\pi} & 0 & 0\end{pmatrix}' $. I'm also not writing the case where the slope $b$ is also learned.)
 \begin{align*}
k_t & = \begin{cases} k_{t-1}+1 \quad \text{for decreasing gain learning}  \\ \bar{g}^{-1}  \quad \text{for constant gain learning.}\numberthis
\end{cases} 
\end{align*}

\begin{figure}[h!]
\subfigure[Learning constant only]{\includegraphics[scale=\myAdjustableFigScale]{\myFigPath command_IRFs_many_learning_RIR_LH_monpol_cgain_gbar_0_145_default_learning_true_baseline_no_info_ass_constant_only}}
\subfigure[Learning slope and constant]{\includegraphics[scale=\myAdjustableFigScale]{\myFigPath command_IRFs_many_learning_RIR_LH_monpol_cgain_gbar_0_145_default_learning_true_baseline_no_info_ass_slope_and_constant}}
\caption{Reference: baseline model}
\end{figure}

\clearpage


\newpage
\section{The CEMP vs. the CUSUM criterion}

CEMP's criterion  
\begin{align}
\theta_t & = |\hat{\E}_{t-1}\pi_t - \E_{t-1}\pi_t | / (\text{Var(shocks)}) \\
\text{i.e.} \quad &\text{PLM- E[ALM], scaled by shocks}
\end{align}

For my version of CEMP's criterion, I rewrite the ALM
\begin{align}
z_t & = A_a f_a + A_b f_b + A_s s_t \\
\text{as} \quad & z_t =  F  +Gs_t  \\
\Leftrightarrow \quad & z_t = \begin{bmatrix} F & G \end{bmatrix} \begin{bmatrix} 1 \\ s_t\end{bmatrix}
\end{align}
Then, since the PLM is $z_t = \phi  \begin{bmatrix} 1 \\ s_t\end{bmatrix}$, the generalized CEMP criterion becomes
\begin{equation}
\theta_t = \max | \Sigma^{-1} ( \phi - \begin{bmatrix} F & G \end{bmatrix}) |
\end{equation}
where $\Sigma$ is the VC matrix of shocks.
As for the CUSUM criterion, what I did in Materials 5 was
\begin{align}
 \omega_t & =  \omega_{t-1} + \kappa k_{t-1}^{-1}(FE_t^2 -\omega_{t-1})\\
\theta_t & =  \theta_{t-1} + \kappa k_{t-1}^{-1}(FE_t^2/\omega_t -\theta_{t-1})\
\end{align}
where $FE_t$ is the most recent short-run forecast error ($ny\times 1$), and $\omega_t$ is the agents' estimate of the forecast error variance ($ny \times ny$). To take into account that these are now matrices, I now write
\begin{align}
\omega_t & =  \omega_{t-1} + \kappa k_{t-1}^{-1}(f_t f_t'  -\omega_{t-1})\\
\theta_t & =  \theta_{t-1} + \kappa k_{t-1}^{-1}(f_t'\omega_t^{-1}f_t -\theta_{t-1})
\end{align}

(Note: I'm using L\"utkepohl's \emph{Introduction to Multiple Time Series Analysis}, p. 160 to reformulate the CUSUM criterion as a statistic that has a $\chi^2$ distribution.)




\newpage
\section{Investigating the behavior of CEMP and CUSUM criteria}

\subsection{Anchoring as a function of $\psi_{\pi}$, fixing $\psi_x = 0, \bar{\theta}=4, \tilde{\theta}=2.5$}
\begin{figure}[h!]
\subfigure[CEMP's criterion]{\includegraphics[scale=\myAdjustableFigScale]{\myFigPath command_IRFs_anchoring_loss_again_critCEMP_constant_only_params_psi_pi_1_01_psi_x_0_gbar_0_145_thetbar_4_thettilde_0_1_kap_0_8_alph_CB_0_2020_01_28}}
\subfigure[CUSUM criterion]{\includegraphics[scale=\myAdjustableFigScale]{\myFigPath command_IRFs_anchoring_loss_again_critCUSUM_constant_only_params_psi_pi_1_01_psi_x_0_gbar_0_145_thetbar_4_thettilde_2_5_kap_0_8_alph_CB_0_2020_02_04}}
\caption{Inverse gains, $\psi_{\pi} =1.01$}
\end{figure}

\begin{figure}[h!]
\subfigure[CEMP's criterion]{\includegraphics[scale=\myAdjustableFigScale]{\myFigPath command_IRFs_anchoring_loss_again_critCEMP_constant_only_params_psi_pi_1_5_psi_x_0_gbar_0_145_thetbar_4_thettilde_0_1_kap_0_8_alph_CB_0_2020_01_28}}
\subfigure[CUSUM criterion]{\includegraphics[scale=\myAdjustableFigScale]{\myFigPath command_IRFs_anchoring_loss_again_critCUSUM_constant_only_params_psi_pi_1_5_psi_x_0_gbar_0_145_thetbar_4_thettilde_2_5_kap_0_8_alph_CB_0_2020_02_04}}
\caption{Inverse gains, $\psi_{\pi} =1.5$}
\end{figure}

\begin{figure}[h!]
\subfigure[CEMP's criterion]{\includegraphics[scale=\myAdjustableFigScale]{\myFigPath command_IRFs_anchoring_loss_again_critCEMP_constant_only_params_psi_pi_2_psi_x_0_gbar_0_145_thetbar_4_thettilde_0_1_kap_0_8_alph_CB_0_2020_01_28}}
\subfigure[CUSUM criterion]{\includegraphics[scale=\myAdjustableFigScale]{\myFigPath command_IRFs_anchoring_loss_again_critCUSUM_constant_only_params_psi_pi_2_psi_x_0_gbar_0_145_thetbar_4_thettilde_2_5_kap_0_8_alph_CB_0_2020_02_04}}
\caption{Inverse gains, $\psi_{\pi} =2$}
\end{figure}

\subsection{Why do the two criteria behave opposite ways?}

A restatement of the two criteria: you get unanchored expectations if:
\begin{equation}
\theta_t^{CEMP} = \max(\Sigma^{-1}| ( \phi - \begin{bmatrix} F & G \end{bmatrix}) |) > \bar{\theta} \quad \quad \text{vs.} \quad \quad  \theta_t^{CUSUM} = f' \omega^{-1}f > \tilde{\theta}
\end{equation}
where $\phi$ is the agents' estimated matrix, $F,G$ are the ALM matrices, $f$ is the one-period ahead forecast error and $\omega$ is the estimated forecast error variance matrix. 

Sloppily, we can write the two criteria as:
\begin{equation}
\frac{\E(FE)}{\text{Var(noise)}} \quad \quad \text{vs.} \quad \quad \frac{FE^2}{\hat{FEV}}
\end{equation}

\begin{itemize}
\item The big difference in the two is the way they treat noise: CEMP integrates it out, while CUSUM tries to cancel it out by dividing by the estimated FEV.
\item Squaring the numerator in the CEMP criterion doesn't change its behavior. 
\item Trying various ways of not squaring or taking the square root of the CUSUM criterion doesn't change its behavior.
\item Inputting Var(noise) instead of $\hat{FEV}$ in the CUSUM denominator, or simply using the CUSUM numerator only changes its behavior! You obtain the CEMP-criterion behavior: little anchoring for high $\psi_{\pi}$, much anchoring for low $\psi_{\pi}$.
\item[$\rightarrow$] Is the difference due to using the $FEV$ instead of Var(noise) or does it come from using an \emph{estimated} FEV?
\item Decompose forecast errors as $FE = $ part due to learning + noise. CEMP's criterion is able to distinguish between those two and responds only to the part due to learning. The CUSUM criterion cannot. To see this, write the CUSUM criterion as a function of the CEMP criterion:
\begin{align*}
FE & = \phi s_{t-1}-[F,G]s_t  \\ 
&=  \phi s_{t-1}-[F,G]s_{t-1} -[F,G]\varepsilon_{t} \\
& = (\phi-{F,G})s_{t-1} -[F,G]\varepsilon_{t} \quad \quad | \text{ignoring} \; \Sigma^{-1} \; \text{and $s_{t-1}$, this is}\\
& = \theta^{CEMP} - [F,G]\varepsilon_{t}  \numberthis \quad \quad | \text{Nb.: 2nd component conflates learning \& noise} \\
FE^2 & = (\theta^{CEMP})^2 + [F,G]^2\varepsilon_{t}^2 -2 \theta^{CEMP} [F,G]\varepsilon_{t} \numberthis \\
FEV & = \E(FE^2) = (\theta^{CEMP})^2 + [F,G]^2\Sigma
\end{align*}

\begin{align*}
\rightarrow \hat{FEV} & = \E(FE^2) = (\theta^{CEMP})^2 + [F,G]^2\hat{\Sigma} \numberthis \\
\Rightarrow \theta^{CUSUM} & = \frac{(\theta^{CEMP})^2 + [F,G]^2\varepsilon_{t}^2 -2 \theta^{CEMP} [F,G]\varepsilon_{t}}{ (\theta^{CEMP})^2 + [F,G]^2\hat{\Sigma}} \quad \quad | \text{Supposing that $\hat{\Sigma}\approx \varepsilon_t^2$} \\
 \theta^{CUSUM} & = 1- 2\frac{ \theta^{CEMP} [F,G]\varepsilon_{t}}{(\theta^{CEMP})^2 + [F,G]^2\hat{\Sigma}}\numberthis \label{thetcusum}
\end{align*}
\end{itemize}
$\Rightarrow$ Suggests that the difference comes from dividing the CUSUM-criterion by the FEV (doesn't matter if it's estimated or not). The reason this division is not innocent is because the CUSUM-criterion doesn't integrate out shocks - it relies on forecast errors, \emph{not expected forecast errors}, as the CEMP criterion does. Thus the cross-term $-2 \theta^{CEMP} [F,G]\varepsilon_{t}$ is not cancelled out, introducing movement in the opposite direction as the CEMP-criterion.

What's the intuition? \\
In absence of knowing the model, CUSUM-agents aren't able to evaluate the expectations $\E(FE)$ and $\E(FE^2)=FEV$. Thus they are not able to distinguish between forecast errors coming from learning or from noise. In particular, to evaluate the forecast error variance, ideally they'd want to compute it as
\begin{equation}
FEV = \text{variation coming from learning} +  \text{variation coming from noise} - 2\text{cov(learn, noise)}
\end{equation}
But since they are not able to distinguish between these components, the cross-term cov(learn, noise) contaminates their $FE^2$ expression, leading to a discrepancy between $FE^2$ and their estimated $\hat{FEV}$.

In a sense this is like omitted variable bias, which biases their estimate $\theta^{CUSUM}$. % Whenever learning errors go the same way as errors from noise, we get a downward (upward?) bias, while if the errors go in opposite ways we get an upward (downward?) bias. 
In particular, I'd suggest that errors from learning and noise are likely to go in the same direction, which means that $\theta^{CUSUM}$ will move in the opposite direction as $\theta^{CEMP}$.

\vspace{1cm}

\emph{A small side-note:
\begin{itemize}
\item the CUSUM-criterion is very sensitive to initialization: \\
Especially the choice of $\theta_0$ matters a lot for initial dynamics which is why on these graphs you initially have anchoring b/c I initialize $\theta_0$ as zero. If, on the contrary, I initialize it as $\tilde{\theta}$ or higher, you don't get the initial anchored period. The same effect is achieved by setting a burn-in period of 100 or so.
\end{itemize}}

\newpage
\section{Analytical expressions for optimal Taylor rule coefficients}
Following Woodford's \emph{Interest and Prices}, here's a procedure to obtain optimal Taylor rule coefficients. In Woodford's terminology, this consists of solving for the \emph{optimal noninertial plan} (oni) for the endogenous variables,  and then doing coefficient comparison between the Taylor rule and the \emph{oni}.
\subsection{In-a-nutshell algorithm for optimal Taylor rule coefficients}
\begin{enumerate}
\item Postulate conjectures $z_t = \bar{z} +f_ju_t + g_j \hat{r}_t^n, \quad j=\pi,x,i \quad z=(\pi,x,i)'$ for the model consisting of an NKPC and NKIS relation and the AR(1) shocks $u, \hat{r}^n$, where $\hat{r}_t^n = r^n_t - \bar{r}$ (so that the natural rate has a drift, but $\hat{r}^n$ is detrended).
\item Plug the conjectures into the two model equations to derive 2 constraints on the 3 deterministic components $(\bar{\pi}, \bar{x}, \bar{i})$ and 4 constraints on the 6 coefficients on disturbances, $f_j, g_j, \quad j=\pi,x,i$. (Use the known LOMs of shocks to write everything in terms of time $t$ shocks.)
\item Solve 3 sets of optimizations
\begin{enumerate}
\item $(\bar{\pi}, \bar{x}, \bar{i}) = \argmin L^{det} \quad s.t.\quad  \text{the 2 constraints on deterministic components}$
\item $f_j = \argmin L^{stab,u} \quad s.t.\quad  \text{ 2 out of 4 constraints on shock-coefficients}$, $j=\pi,x,i$. 
\item $g_j = \argmin L^{stab,r} \quad s.t.\quad  \text{ the last 2 out of 4 constraints on shock-coefficients}$, $j=\pi,x,i$. 
\end{enumerate}
\item Compare coefficients of Taylor rule to \emph{oni}-solution of $i_t$.
\end{enumerate}

\subsection{Details}
\begin{enumerate}
\item The optimal noninertial plan (oni)\\
A purely forward-looking set of optimal policies that specifies a LOM for each endogenous variable as the sum of a deterministic component (a long-run average, denoted above by ``bar") and a state-contingent component with an optimal response to state $t$ disturbances (the $f_j$ and $g_j$ above). Moreover, the deterministic components are optimal from a timeless perspective (i.e. they minimize $L^{det}$), and the state-contingent components minimize fluctuations coming from shocks ($L^{stab,u}, L^{stab,r}$). 
\item The loss function of the monetary authority and its decomposition into deterministic and shock-contingent parts \\
Following Woodford, I augment my loss function with some concern for interest rate stabilization:
\begin{equation}
L^{CB} =\E_t \sum_{T=t}^{\infty}\{\pi_T^2 +\lambda_x(x_T - x^*)^2 +\lambda_i(i_T - i^*)\}
\end{equation}
Woodford decomposes this as $L^{CB}= L^{det} + L^{stab}$ where the former only depends on the ``deterministic component of the equilibrium paths of the target variables,'' while the latter ``depends only on the equilibrium responses to unexpected shocks'' (p. 509). In particular:
\begin{align}
L^{det} & = \sum_{T=t}^{\infty}\beta^{T-t}\{\E_t{\pi_T^2} +\lambda_x(\E_tx_T -x^*)^2 + \lambda_i(\E_ti_T -i^*)^2 \}\\
L^{stab} & = \sum_{T=t}^{\infty}\beta^{T-t}\{\text{var}_t(\pi_T)+\lambda_x\text{var}_t(x_T) + \lambda_i\text{var}_t(i_T) \}
\end{align}
Woodford then further decomposes $L^{stab}$ into an element conditional on each shock, but this is only for algebraic convenience.
\item The Taylor rule \\
Woodford postulates a Taylor rule of the form
\begin{equation}
i_t = \bar{i} + \phi_{\pi}(\pi_t - \bar{\pi}) + \phi_x(x_t -\bar{x})/4 
\end{equation}
(He divides by 4 to make the output gap quarterly.) Substituting in the conjectured and solved-for \emph{oni}-solutions for the endogenous variables, one obtains:
\begin{align}
i_t & = \bar{i} + \phi_{\pi}(f_{\pi}u_t + g_{\pi}\hat{r}_t^n) + \phi_x(f_{x}u_t + g_{x}\hat{r}_t^n)/4 \\
i_t & = \bar{i} + f_i u_t + g_i \hat{r}^n_t
\end{align}
allowing one to solve for $(\phi_{\pi}^*, \phi_x^*)$ as the solution to
\begin{align}
f_i & = \phi_{\pi}f_{\pi} + \phi_xf_{x} \\
g_i & = \phi_{\pi}g_{\pi} + \phi_xg_{x}
\end{align}
\end{enumerate}

\newpage
\subsection{Optimal Taylor rule coefficients for RE model, with the simplifying assumption $\rho_u = \rho_r = \rho$}
\begin{align}
\phi_{\pi}^* & = \frac{\kappa  \sigma }{\lambda_i(\rho -1) (\beta  \rho -1)-\kappa  \lambda_i \rho  \sigma } \label{opt_phipi_RE}
\\
\phi_{x}^* & =  \frac{\lambda_x\sigma  (1-\beta  \rho )}{\lambda_i (\rho -1) (\beta  \rho -1)-\kappa  \lambda_x \rho  \sigma } \label{opt_phix_RE}
\end{align}
which is - fabulously enough - exactly what Woodford obtains.
\subsection{Optimal Taylor rule coefficients for the learning model, with the simplifying assumption $\rho_u = \rho_r = \rho$}
\begin{align}
\psi_{\pi}^* & =
\frac{\kappa  \sigma  (\beta  (\rho -1)-1) (\alpha  \beta  (\rho -1)-1)}{\kappa  \lambda_i \sigma  (\alpha  \beta  (\rho -1)-1)+\beta  \lambda_i (\rho -1) (\alpha  \beta  (\rho -1)+\beta -1)} \label{opt_psipi_learn} \\
\psi_{x}^* & = 
\frac{\lambda_x\sigma  (\beta  (\rho -1)-1) (\alpha  \beta  (\rho -1)+\beta -1)}{\kappa  \lambda_i \sigma  (\alpha  \beta  (\rho -1)-1)+\beta  \lambda_i (\rho -1) (\alpha  \beta  (\rho -1)+\beta -1)} \label{opt_psix_learn}
\end{align}

For a simple calibration of $\lambda_i =1$ and $\lambda_x=0$ and
$\{\beta \to 0.99,\sigma \to 1,\kappa \to 0.16,\rho \to 0.3,\alpha \to 0.5\}$, I get $\phi_{\pi}^*=0.360279$ and $\psi_{\pi}^*=18.7714$.

\begin{figure}[h!]
\includegraphics[scale=1]{\myFigPath materials15_opt_coeffs}
\caption{Optimal Taylor rule coefficients for $\lambda_i = 1$ as a function of $\rho$}
\end{figure}

Like Woodford observes, equations (\ref{opt_phipi_RE})-(\ref{opt_psix_learn}) impose certain bounds on $\rho$: these bounds are higher for RE than for learning. Another point to note is that decreasing $\lambda_i$ just scales the optimal coefficients as a function of $\rho$ up, but preserves the functional form.

\end{document}





