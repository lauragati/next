\documentclass[11pt]{article}
\usepackage{amsmath, amsthm, amssymb,lscape, natbib}
\usepackage{mathtools}
\usepackage{subfigure}
\usepackage[font=footnotesize,labelfont=bf]{caption}
\usepackage{graphicx}
\usepackage{colortbl}
\usepackage{hhline}
\usepackage{multirow}
\usepackage{multicol}
\usepackage{setspace}
\usepackage[final]{pdfpages}
\usepackage[left=2.5cm,top=2.5cm,right=2.5cm, bottom=2.5cm]{geometry}
\usepackage{natbib} 
\usepackage{bibentry} 
\newcommand{\bibverse}[1]{\begin{verse} \bibentry{#1} \end{verse}}
\newcommand{\vs}{\vspace{.3in}}
\renewcommand{\ni}{\noindent}
\usepackage{xr-hyper}
\usepackage[]{hyperref}
\hypersetup{
    colorlinks=true,
    linkcolor=blue,
    filecolor=magenta,      
    urlcolor=cyan,
}
 
\urlstyle{same}
\usepackage[capposition=top]{floatrow}
\usepackage{amssymb}
\usepackage{relsize}
\usepackage[dvipsnames]{xcolor}
\usepackage{fancyhdr}
\usepackage{tikz}
 
\pagestyle{fancy} % customize header and footer
\fancyhf{} % clear initial header and footer
%\rhead{Overleaf}
\lhead{\centering \rightmark} % this adds subsection number and name
\lfoot{\centering \rightmark} 
\rfoot{\thepage} % put page number (the centering command puts it in the middle, don't matter if you put it in right or left footer)

\def \myFigPath {../figures/} 
% BE CAREFUL WITH FIGNAMES, IN LATEX THEY'RE NOT CASE SENSITIVE!!
\def \myTablePath {../tables/} 

%\definecolor{mygreen}{RGB}{0, 100, 0}
\definecolor{mygreen}{RGB}{0, 128, 0}

\definecolor{citec}{rgb}{0,0,.5}
\definecolor{linkc}{rgb}{0,0,.6}
\definecolor{bcolor}{rgb}{1,1,1}
\hypersetup{
%hidelinks = true
  colorlinks = true,
  urlcolor=linkc,
  linkcolor=linkc,
  citecolor = citec,
  filecolor = linkc,
  pdfauthor={Laura G\'ati},
}


\geometry{left=.83in,right=.89in,top=1in,
bottom=1in}
\linespread{1.5}
\renewcommand{\[}{\begin{equation}}
\renewcommand{\]}{\end{equation}}

% New Options
\newtheorem{prop}{Proposition}
\newtheorem{definition}{Definition}[section]
\newtheorem*{remark}{Remark}
\newtheorem{lemma}{Lemma}
\newtheorem{corollary}{Corollary}
\newtheorem{conjecture}{Conjecture}

%\newtheorem{theorem}{Theorem}[section] % the third argument specifies that their number will be adopted to the section
%\newtheorem{corollary}{Corollary}[theorem]
%\newtheorem{lemma}[theorem]{Lemma}
%\declaretheorem{proposition}
%\linespread{1.3}
%\raggedbottom
%\font\reali=msbm10 at 12pt

% New Commands
\newcommand{\real}{\hbox{\reali R}}
\newcommand{\realp}{\hbox{\reali R}_{\scriptscriptstyle +}}
\newcommand{\realpp}{\hbox{\reali R}_{\scriptscriptstyle ++}}
\newcommand{\R}{\mathbb{R}}
\DeclareMathOperator{\E}{\mathbb{E}}
\DeclareMathOperator{\argmin}{arg\,min}
\newcommand\w{3.0in}
\newcommand\wnum{3.0}
\def\myFigWidth{5.3in}
\def\mySmallerFigWidth{2.1in}
\def\myEvenBiggerFigScale{0.8}
\def\myPointSixFigScale{0.6}
\def\myBiggerFigScale{0.4}
\def\myFigScale{0.3}
\def\myMediumFigScale{0.25}
\def\mySmallFigScale{0.22}
\def\mySmallerFigScale{0.18}
\def\myTinyFigScale{0.16}
\def\myPointFourteenFigScale{0.14}
\def\myTinierFigScale{0.12}
\def\myAdjustableFigScale{0.14}
\newcommand\numberthis{\addtocounter{equation}{1}\tag{\theequation}} % this defines a command to make align only number this line
\newcommand{\code}[1]{\texttt{#1}} %code %

\renewcommand*\contentsname{Overview}
\setcounter{tocdepth}{2}

% define a command to make a huge question mark (it works in math mode)
\newcommand{\bigqm}[1][1]{\text{\larger[#1]{\textbf{?}}}}

\begin{document}

\linespread{1.0}

\title{Favorite result so far - target criterion for anchoring}
\author{Laura G\'ati} 
\date{\today}
\maketitle

%%%%%%%%%%%%%%%%%%%%             DOCUMENT           %%%%%%%%%%%%%%%%%% 

%\tableofcontents

%\listoffigures

My favorite result so far is Equation (\ref{target}), which is the target criterion in the case of the anchoring model. 
\begin{align}
\pi_t  = -\frac{\lambda_x}{\kappa}\bigg\{x_t - \frac{(1-\alpha)\beta}{1-\alpha\beta} \bigg(k_t^{-1}+((\pi_t - \bar{\pi}_{t-1}-b_1 s_{t-1}))\mathbf{g}_{\pi}(t) \bigg) 
\bigg(\E_t\sum_{i=1}^{\infty}x_{t+i}\prod_{j=1}^{i-1}(1-k_{t+j}^{-1}(\pi_{t+1+j} - \bar{\pi}_{t+j}-b_1 s_{t+j})) \bigg)
\bigg\} \label{target}
\end{align}

There are two main reasons why I think this is a neat result. The first is that it's an intuitive description of the relationship between inflation and the output gap that optimal monetary policy in the anchoring model seeks to bring about. It instructs the central bank to trade off inflation stabilization with current output gap stabilization, plus a term that captures future output gap stabilization, taking the current and expected future stance of learning and anchoring into account. 

The second reason is that you can cleanly isolate the effects of learning vis-\`a-vis rational expectations on the one hand, and anchoring versus exogenous gain learning on the other. Shut down both learning and anchoring and you get the discretionary RE target criterion. Add learning but no anchoring and you get the Moln\'ar and Santoro (2014) result, which is a novel intertemporal tradeoff due to learning. Add anchoring and you get another intertemporal tradeoff that is captured by the term $forecast \; error \times \mathbf{g_{\pi}}$, which says plainly that monetary policy needs to consider the effects of its actions on how anchoring changes with realized inflation. Pretty neat!


\end{document}





