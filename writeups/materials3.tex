\documentclass[11pt]{article}
\usepackage{amsmath, amsthm, amssymb,lscape, natbib}
\usepackage{mathtools}
\usepackage{subfigure}
\usepackage[font=footnotesize,labelfont=bf]{caption}
\usepackage{graphicx}
\usepackage{colortbl}
\usepackage{hhline}
\usepackage{multirow}
\usepackage{multicol}
\usepackage{setspace}
\usepackage[final]{pdfpages}
\usepackage[left=2.5cm,top=2.5cm,right=2.5cm, bottom=2.5cm]{geometry}
\usepackage{natbib} 
\usepackage{bibentry} 
\newcommand{\bibverse}[1]{\begin{verse} \bibentry{#1} \end{verse}}
\newcommand{\vs}{\vspace{.3in}}
\renewcommand{\ni}{\noindent}
\usepackage{xr-hyper}
\usepackage[]{hyperref}
\usepackage[capposition=top]{floatrow}
\usepackage{amssymb}


\def \myFigPath {../figures/} 
% BE CAREFUL WITH FIGNAMES, IN LATEX THEY'RE NOT CASE SENSITIVE!!
\def \myTablePath {../tables/} 

\definecolor{citec}{rgb}{0,0,.5}
\definecolor{linkc}{rgb}{0,0,.6}
\definecolor{bcolor}{rgb}{1,1,1}
\hypersetup{
%hidelinks = true
  colorlinks = true,
  urlcolor=linkc,
  linkcolor=linkc,
  citecolor = citec,
  filecolor = linkc,
  pdfauthor={Laura G\'ati},
}


\geometry{left=.83in,right=.89in,top=1in,
bottom=1in}
\linespread{1.5}
\renewcommand{\[}{\begin{equation}}
\renewcommand{\]}{\end{equation}}

% New Options
\newtheorem{prop}{Proposition}
\newtheorem{definition}{Definition}[section]
\newtheorem*{remark}{Remark}
\newtheorem{lemma}{Lemma}
\newtheorem{corollary}{Corollary}
%\newtheorem{theorem}{Theorem}[section] % the third argument specifies that their number will be adopted to the section
%\newtheorem{corollary}{Corollary}[theorem]
%\newtheorem{lemma}[theorem]{Lemma}
%\declaretheorem{proposition}
%\linespread{1.3}
%\raggedbottom
%\font\reali=msbm10 at 12pt

% New Commands
\newcommand{\real}{\hbox{\reali R}}
\newcommand{\realp}{\hbox{\reali R}_{\scriptscriptstyle +}}
\newcommand{\realpp}{\hbox{\reali R}_{\scriptscriptstyle ++}}
\newcommand{\R}{\mathbb{R}}
\DeclareMathOperator{\E}{\mathbb{E}}
\DeclareMathOperator{\argmin}{arg\,min}
\newcommand\w{3.0in}
\newcommand\wnum{3.0}
\def\myFigWidth{5.3in}
\def\mySmallerFigWidth{2.1in}
\def\myEvenBiggerFigScale{0.8}
\def\myPointSixFigScale{0.6}
\def\myBiggerFigScale{0.4}
\def\myFigScale{0.3}
\def\mySmallFigScale{0.22}
\def\mySmallerFigScale{0.18}
\def\myTinyFigScale{0.16}
\def\myPointFourteenFigScale{0.14}
\def\myTinierFigScale{0.12}
\newcommand\numberthis{\addtocounter{equation}{1}\tag{\theequation}} % this defines a command to make align only number this line
\newcommand{\code}[1]{\texttt{#1}} %code %

\renewcommand*\contentsname{Overview}
\setcounter{tocdepth}{2}

\begin{document}

\linespread{1.0}

\title{Materials 3 - Special cases}
\author{Laura G\'ati} 
\date{\today}
\maketitle

%%%%%%%%%%%%%%%%%%%%             DOCUMENT           %%%%%%%%%%%%%%%%%% 

\tableofcontents

%\listoffigures

\newpage
\section{The models to be simulated}
\begin{enumerate}
\item Rational expectations NK model (RE)
\item Euler equation approach learning NK model \`a la Bullard \& Mitra (2002)  (EE)
\item LR expectations learning NK model \`a la Preston (2005)  (LR)
\item (Eventually: LR expectations learning NK model \`a la Preston with anchoring \`a la CEMP)
\end{enumerate}

The difference between these models is 1) in the expectations (rational or not), 2) in the number of horizons of expectations that need to be summed (1 vs. infinite). So what I'm going to do consists of 2 steps: 
\begin{enumerate}
\item Write a learning rule that takes the form of Preston's, but that nests CEMP, and has a decreasing gain.
\item Write out $f_a$ and $f_b$ as truncated sums of $h$-period ahead forecasts. When $h=1$, EE and LR (models (\ref{LOM_EE}) and (\ref{LOM_LR})) should coincide.
\end{enumerate}

\subsection{RE}
\begin{align}
x_t &= \E_t x_{t+1} - \sigma(i_t - \E_t \pi_{t+1}) +\sigma r_t^n \label{NKIS} \\
\pi_t &= \kappa x_t +\beta \E_t \pi_{t+1} + u_t  \label{NKPC} \\
i_t &= \bar{i}_t + \psi_{\pi}\pi_t + \psi_{x} x_t  \label{TR}
\end{align}
\subsection{EE}
\begin{align}
x_t &= \hat{\E}_t x_{t+1} - \sigma(i_t - \hat{\E}_t \pi_{t+1}) +\sigma r_t^n \tag{Preston, eq. (13)} \label{prestons13} \\
\pi_t &= \kappa x_t +\beta \hat{\E}_t \pi_{t+1} + u_t \tag{Preston, eq. (14)} \label{prestons14}  \\
i_t &= \bar{i}_t + \psi_{\pi}\pi_t + \psi_{x} x_t \tag{Preston, eq. (27) } 
\end{align}
\subsection{LR}
\begin{align}
x_t &=  -\sigma i_t +\hat{\E}_t \sum_{T=t}^{\infty} \beta^{T-t }\big( (1-\beta)x_{T+1} - \sigma(\beta i_{T+1} - \pi_{T+1}) +\sigma r_T^n \big) \tag{Preston, eq. (18)} \label{prestons18}  \\
\pi_t &= \kappa x_t +\hat{\E}_t \sum_{T=t}^{\infty} (\alpha\beta)^{T-t }\big( \kappa \alpha \beta x_{T+1} + (1-\alpha)\beta \pi_{T+1} + u_T\big)\tag{Preston, eq. (19)} \label{prestons19}  \\
i_t &= \psi_{\pi}\pi_t + \psi_{x} x_t + \bar{i}_t \tag{Preston, eq. (27)} 
\end{align}
One issue is that if I set $T=t$, I don't think \ref{prestons18} reduces to \ref{prestons13}, nor does \ref{prestons19} reduce to \ref{prestons14}.
 \section{Compact notation}
Innovations are summarized as:
 \begin{align*}
 s_t & = P s_{t-1} + \epsilon_t 
 \quad \text{where} \quad 
 s_t & \equiv \begin{pmatrix} r_t^n \\ \bar{i}_t \\ u_t 
 \end{pmatrix} \quad 
 P  \equiv \begin{pmatrix} \rho_r & 0 & 0 \\ 0& \rho_i & 0 \\ 0&0& \rho_u 
 \end{pmatrix}  \quad 
 \epsilon_t \equiv \begin{pmatrix}\varepsilon_t^{r} \\ \varepsilon_t^{i}  \\ \varepsilon_t^{u} 
 \end{pmatrix}  \quad  \text{and } \quad \Sigma  =  \begin{pmatrix} \sigma_r & 0 & 0 \\ 0& \sigma_i & 0 \\ 0&0& \sigma_u 
 \end{pmatrix} 
 \end{align*}
 Let $z_t$ summarize the endogenous variables as
 \begin{equation}
 z_t \equiv \begin{pmatrix} \pi_t \\ x_t \\ i_t
 \end{pmatrix}
 \end{equation}
 Then I can write the models compactly as
 \begin{align}
z_t & = A_p^{RE} \E_t z_{t+1} + A_s^{RE} s_t \label{LOM_RE} \\
z_t & = A_p^{RE} \hat{\E}_t z_{t+1} + A_s^{RE} s_t \label{LOM_EE} \\
z_t & = A_a^{LR} f_a + A_b^{LR} f_b + A_s^{LR} s_t \label{LOM_LR} \\
s_t & = P s_{t-1} + \epsilon_t \label{exog}
\end{align}
 where $f_a$ and $f_b$ capture discounted sum of expectations at all horizons of the endogenous states $z$ (following Preston, I refer to these objects as ``long-run expectations''):
  \begin{align}
f_a  \equiv  \hat{\E}_t\sum_{T=t}^{\infty} (\alpha\beta)^{T-t } z_{T+1} \quad \quad \quad \quad f_b  \equiv \hat{\E}_t\sum_{T=t}^{\infty} (\beta)^{T-t } z_{T+1} \label{fafb}
\end{align}
and the coefficient matrices are given by:
\begin{align}
A_p^{RE} & = \begin{pmatrix} \beta + \frac{\kappa\sigma}{w} (1-\psi_{\pi}\beta) & \frac{\kappa}{w} & 0\\
 \frac{\sigma}{w} (1-\psi_{\pi}\beta) & \frac{1}{w}& 0\\ 
\psi_{\pi}\big( \beta + \frac{\kappa\sigma}{w} (1-\psi_{\pi}\beta) \big) +\psi_x\frac{\sigma}{w} (1-\psi_{\pi}\beta)&  \psi_x (\frac{1}{w})+ \psi_{\pi} (\frac{\kappa}{w})& 0\end{pmatrix} \quad \\
A_s^{RE} &= \begin{pmatrix}   \frac{\kappa\sigma}{w}  &-\frac{\kappa\sigma}{w}  & 1-\frac{\kappa\sigma\psi_{\pi}}{w}\\
 \frac{ \sigma}{w} &  -\frac{\sigma}{w} & -\frac{\sigma\psi_{\pi}}{w}\\ 
 \psi_x( \frac{\sigma}{w}) + \psi_{\pi}( \frac{\kappa\sigma}{w}) & \psi_x(- \frac{\sigma}{w}) + \psi_{\pi}(- \frac{\kappa\sigma}{w}) +1 &  \psi_x(-\frac{\sigma\psi_{\pi}}{w}) + \psi_{\pi}( 1-\frac{\kappa\sigma\psi_{\pi}}{w})\end{pmatrix}  
\\
A_a^{LR} & = \begin{pmatrix} g_{\pi a} \\ g_{x a} \\ \psi_{\pi}g_{\pi a} + \psi_xg_{x a}
\end{pmatrix}
\quad A_b^{LR} = \begin{pmatrix} g_{\pi b} \\ g_{x b} \\ \psi_{\pi}g_{\pi b} + \psi_xg_{x b}
\end{pmatrix}
 \quad A_s^{LR} = \begin{pmatrix} g_{\pi s} \\ g_{x s} \\ \psi_{\pi}g_{\pi s} + \psi_xg_{x s} + \begin{bmatrix} 0 & 1& 0\end{bmatrix}
\end{pmatrix} \\
g_{\pi a} & =(1-\frac{\kappa\sigma\psi_{\pi}}{w} )  \begin{bmatrix}(1-\alpha)\beta, \kappa\alpha\beta, 0 \end{bmatrix} \\
g_{x a} & =  \frac{-\sigma\psi_{\pi}}{w} \begin{bmatrix}(1-\alpha)\beta, \kappa\alpha\beta, 0 \end{bmatrix}\\
g_{\pi b} & = \frac{\kappa}{w} \begin{bmatrix}\sigma(1-\beta\psi_{\pi}), (1-\beta-\beta\sigma\psi_x, 0 \end{bmatrix}\\
g_{x b} & = \frac{1}{w} \begin{bmatrix}\sigma(1-\beta\psi_{\pi}), (1-\beta-\beta\sigma\psi_x, 0 \end{bmatrix} \\
g_{\pi s} & = (1-\frac{\kappa\sigma\psi_{\pi}}{w} )\begin{bmatrix} 0&0&1 \end{bmatrix} (I_3 - \alpha\beta P)^{-1} -\frac{\kappa\sigma}{w}\begin{bmatrix} -1&1&0 \end{bmatrix} (I_3 -\beta P)^{-1}\\
g_{x s} & =  \frac{-\sigma\psi_{\pi}}{w} \begin{bmatrix} 0&0&1 \end{bmatrix}(I_3 - \alpha\beta P)^{-1}  -\frac{\sigma}{w}\begin{bmatrix} -1&1&0 \end{bmatrix}(I_3 -\beta P)^{-1}\\
w & = 1+\sigma\psi_x +\kappa\sigma\psi_{\pi}
\end{align}
\clearpage

 \section{Learning}

In Preston (2005), agents forecast the endogenous variables using the exogenous ones as
\begin{equation}
z_t = a_{t} + b_{t} s_{t} + \epsilon_t \quad  \tag{Preston, p. 101}
\end{equation}
which I suspect isn't precise about the timing. Therefore, I write a general PLM of the form
\begin{equation}
z_t = a_{t-2} + b_{t-2} s_{t-1} + \epsilon_t \quad  \label{generalPLM}
\end{equation}
and then $\phi_{t-2} = (a_{t-2}, b_{t-2})$, here $3\times4$, so that agents learn both a constant and a slope term. Later, I will simplify here so that agents only learn about the constant, i.e. about CEMP's drift term:
\begin{equation}
z_t = \bar{z}_{t-2} + Ps_{t-1} + \epsilon_t \label{PLM}  
\end{equation}
so that $\phi_{t-2} = (\bar{z}_{t-2}, I_3)$, and $\hat{\E}_t z_{t+1} = \phi_{t-1}\begin{bmatrix} 1 \\ Ps_{t} \end{bmatrix} $. I'm actually quite worried about the assumption that agents only learn about the constant because it seems like a permanent deviation from RE: might it screw up E-stability? I'm also worried about the way $P$ is treated in Preston. I've added it, because I think it makes sense that at time $t-1$, you expect $s_t$ to be $Ps_{t-1}$. Or is it the case that the learning of the slope embodies $P$? I don't think so, because agents know the structure of exogenous states.

Anticipated utility implies that
\begin{equation}
\hat{\E}_{t-1}{\phi_{t+h}} = \hat{\E}_{t-1}{\phi_{t}} \equiv \phi_{t-1} \quad \forall \; h\geq0 
\end{equation}
This is a little tricky. It doesn't only mean that agents today mistakenly believe that they will not update the forecasting rule. It also implies that the belief about $\phi_t$ was formed at $t-1$.
Assuming RE about the exogenous process and anticipated utility, this implies that $h$-horizon forecasts are constructed as:
\begin{equation}
\hat{\E}_t z_{t+h} = \bar{z}_{t-1} + P^{h}s_t  \quad \forall h\geq 1 \label{PLM_fcst}
\end{equation}
and the regression coefficients are updated using (for now) a decreasing gain RLS algorithm:
\begin{align}
\phi_t  & = \phi_{t-1} + t^{-1} \mathbf{R_t^{-1}s_{t-1}}\bigg(z_{t-1} - \phi_{t-1} \begin{bmatrix} 1 \\ s_{t-1} \end{bmatrix} \bigg) \\
R_t &= R_{t-1} +  t^{-1} \bigg( \begin{bmatrix} 1 \\ s_{t-1} \end{bmatrix} \begin{bmatrix} 1 & s_{t-1} \end{bmatrix}  - R_{t-1} \bigg)
\end{align}
$R_t$ is $4\times 4$ and $\phi_t$ is $3 \times 4$. Two questions:
\begin{enumerate}
\item Can this formulation capture the special case that agents only learn about the constant?
\item The bold $R_t^{-1}s_{t-1}$ indicates a difference to CEMP's learning algorithm: these terms are missing in CEMP. 
\end{enumerate}
And a note: CEMP is a special case of this model, with the gain switching between decreasing and constant according to the anchoring mechanism. I'm leaving that out for the time being. 

\section{ALMs}
\subsection{RE}
With some abuse of terminology, call the state-space representation the ALM of the RE model:
\begin{align}
x_{t} & = hx \; x_{t-1} + \eta s_t \label{state_eq}\\
z_t & = gx \; x_t \label{obs_eq}
\end{align}
Then I can write the ``ALM" as
\begin{align}
z_t & = gx \; hx \; x_{t-1} + gx \; \eta s_t  \label{ALM_RE}
\end{align}
Since this ALM implies no constant, I initialize $\bar{z}_0$ as a $3\times1$ zero vector, and thus $\phi_0 = \begin{bmatrix} \bar{z}_0 & P\end{bmatrix} $ (and $P = I_3 \; hx$). Analogously, I initialize $R$ as a $R_0 = \begin{bmatrix} 1 & \mathbf{0} \\ \mathbf{0} & \Sigma_x \end{bmatrix}$, where $\Sigma_x$ is the VC matrix of the states from the RE solution. 
\subsection{EE}
I just need to use (\ref{PLM_fcst}) to evaluate one-period ahead forecasts, and plug those into (\ref{LOM_EE}).

\subsection{LR}
Evaluate analytical ``LR expectations'' (\ref{fafb}) using the PLM (\ref{PLM_fcst}), 
\begin{equation}
f_a = \frac{1}{1-\alpha\beta}\bar{z}_{t-1}  + P(I_3 - \alpha\beta P)^{-1}s_t \quad \quad \quad f_b = \frac{1}{1-\beta}\bar{z}_{t-1}  + P(I_3 - \beta P)^{-1}s_t  \label{fafb_analytical}
\end{equation}
and plug them into (\ref{LOM_LR}). In the general case, where agents learn the slope too, $b_{t-1}$ replaces $P$ in the above expression:
\begin{equation}
f_a = \frac{1}{1-\alpha\beta}a_{t-1}  + b_{t-1}(I_3 - \alpha\beta P)^{-1}s_t \quad \quad \quad f_b = \frac{1}{1-\beta}a_{t-1}  + b_{t-1}(I_3 - \beta P)^{-1}s_t  \label{fafb_analytical_general}
\end{equation}
Alternatively I can evaluate each $h$-period forecast individually using (\ref{PLM_fcst}), and then sum $H$ of these terms, discounting appropriately. Earlier, it seemed that already a $H=100$ is not a bad approximation of $\infty$-horizons, but now that only holds for $f_a$. For $f_b$ to be accurate, I need at least $H=10000$. Why?



\section{Timeline in the learning models}
\begin{enumerate}
\item[] \underline{$t=0$}: Initialize $\phi_{t-1} = \phi_0$ at the RE solution.
\item[] For each $t$:
\item Evaluate expectations $t+s$ (the one-period ahead, ($s=1$) or the full 1 to $\infty$-period ahead $(s=1,\dots, \infty)$) given $\phi_{t-1}$ and states dated $t$
\item Evaluate ALM given expectations: ``today's observables are a function of expectations and today's state''
\item Update learning: $\phi_t = $ RLS of $\phi_{t-1}$ and fcst error between today's data and yesterday's forecast
\end{enumerate}

\section{Special cases towards general case: procedure}
\begin{enumerate}
\item Simulate RE model $\checkmark$
\item Simulate EE model where agents learn both slope and constant $\checkmark$
In this case, expectations take the easier form:
\begin{equation}
\hat{\E}_t z_{t+h} = \phi_{t-1} \begin{bmatrix} 1 \\P^{h-1}s_t \end{bmatrix}  \quad \forall h\geq 1 \label{PLM_fcst_general}
\end{equation}
	\begin{itemize}
	\item Simulate using the ``implicit ALM": rearranging the expectational matrix equation that underlies the solution to the model, you obtain the simulated observables $z_t$ without explicitly writing out the ALM $\checkmark$
	\item Simulate using the ``explicit ALM", equation (\ref{LOM_EE}), plugging in expectations evaluated separately. $\checkmark$
	\item[] The cool thing is: when I do the above two steps, I obtain the same simulated observables, so I know I'm doing it correctly.
 	\end{itemize}
\item Simulate LR model where agents learn both slope and constant, extend horizons from 1 to infinity
\item Simulate EE model where agents learn only the constant
\item Simulate LR model where agents learn only the constant, extend horizons from 1 to infinity
\end{enumerate}

%%\newpage
%\section{Two initial simulations}	
%\begin{figure}[h!]
%\subfigure[Observables 1]{
%\includegraphics[scale = \mySmallerFigScale]{\myFigPath materials2_observables1}}
%\subfigure[Nonlinear states 1]{
%\includegraphics[scale = \mySmallerFigScale]{\myFigPath materials2_nonlin_states1}}
%\subfigure[Observables 2]{
%\includegraphics[scale = \mySmallerFigScale]{\myFigPath materials2_observables2}}
%\subfigure[Nonlinear states 2]{
%\includegraphics[scale = \mySmallerFigScale]{\myFigPath materials2_nonlin_states2}}
%\end{figure}








 
 
\end{document}



