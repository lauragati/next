\documentclass[11pt]{article}
\usepackage{amsmath, amsthm, amssymb,lscape, natbib}
\usepackage{mathtools}
\usepackage{subfigure}
\usepackage[font=footnotesize,labelfont=bf]{caption}
\usepackage{graphicx}
\usepackage{colortbl}
\usepackage{hhline}
\usepackage{multirow}
\usepackage{multicol}
\usepackage{setspace}
\usepackage[final]{pdfpages}
\usepackage[left=2.5cm,top=2.5cm,right=2.5cm, bottom=2.5cm]{geometry}
\usepackage{natbib} 
\usepackage{bibentry} 
\newcommand{\bibverse}[1]{\begin{verse} \bibentry{#1} \end{verse}}
\newcommand{\vs}{\vspace{.3in}}
\renewcommand{\ni}{\noindent}
\usepackage{xr-hyper}
\usepackage[]{hyperref}
\hypersetup{
    colorlinks=true,
    linkcolor=blue,
    filecolor=magenta,      
    urlcolor=cyan,
}
 
\urlstyle{same}
\usepackage[capposition=top]{floatrow}
\usepackage{amssymb}
\usepackage{relsize}
\usepackage[dvipsnames]{xcolor}
\usepackage{fancyhdr}
\usepackage{tikz}
 
\pagestyle{fancy} % customize header and footer
\fancyhf{} % clear initial header and footer
%\rhead{Overleaf}
\lhead{\centering \rightmark} % this adds subsection number and name
\lfoot{\centering \rightmark} 
\rfoot{\thepage} % put page number (the centering command puts it in the middle, don't matter if you put it in right or left footer)

\def \myFigPath {../figures/} 
% BE CAREFUL WITH FIGNAMES, IN LATEX THEY'RE NOT CASE SENSITIVE!!
\def \myTablePath {../tables/} 

%\definecolor{mygreen}{RGB}{0, 100, 0}
\definecolor{mygreen}{RGB}{0, 128, 0}

\definecolor{citec}{rgb}{0,0,.5}
\definecolor{linkc}{rgb}{0,0,.6}
\definecolor{bcolor}{rgb}{1,1,1}
\hypersetup{
%hidelinks = true
  colorlinks = true,
  urlcolor=linkc,
  linkcolor=linkc,
  citecolor = citec,
  filecolor = linkc,
  pdfauthor={Laura G\'ati},
}


\geometry{left=.83in,right=.89in,top=1in,
bottom=1in}
\linespread{1.5}
\renewcommand{\[}{\begin{equation}}
\renewcommand{\]}{\end{equation}}

% New Options
\newtheorem{prop}{Proposition}
\newtheorem{definition}{Definition}[section]
\newtheorem*{remark}{Remark}
\newtheorem{lemma}{Lemma}
\newtheorem{corollary}{Corollary}
\newtheorem{conjecture}{Conjecture}

%\newtheorem{theorem}{Theorem}[section] % the third argument specifies that their number will be adopted to the section
%\newtheorem{corollary}{Corollary}[theorem]
%\newtheorem{lemma}[theorem]{Lemma}
%\declaretheorem{proposition}
%\linespread{1.3}
%\raggedbottom
%\font\reali=msbm10 at 12pt

% New Commands
\newcommand{\real}{\hbox{\reali R}}
\newcommand{\realp}{\hbox{\reali R}_{\scriptscriptstyle +}}
\newcommand{\realpp}{\hbox{\reali R}_{\scriptscriptstyle ++}}
\newcommand{\R}{\mathbb{R}}
\DeclareMathOperator{\E}{\mathbb{E}}
\DeclareMathOperator{\argmin}{arg\,min}
\newcommand\w{3.0in}
\newcommand\wnum{3.0}
\def\myFigWidth{5.3in}
\def\mySmallerFigWidth{2.1in}
\def\myEvenBiggerFigScale{0.8}
\def\myPointSixFigScale{0.6}
\def\myBiggerFigScale{0.4}
\def\myFigScale{0.3}
\def\myMediumFigScale{0.25}
\def\mySmallFigScale{0.22}
\def\mySmallerFigScale{0.18}
\def\myTinyFigScale{0.16}
\def\myPointFourteenFigScale{0.14}
\def\myTinierFigScale{0.12}
\def\myAdjustableFigScale{0.14}
\newcommand\numberthis{\addtocounter{equation}{1}\tag{\theequation}} % this defines a command to make align only number this line
\newcommand{\code}[1]{\texttt{#1}} %code %

\renewcommand*\contentsname{Overview}
\setcounter{tocdepth}{2}

% define a command to make a huge question mark (it works in math mode)
\newcommand{\bigqm}[1][1]{\text{\larger[#1]{\textbf{?}}}}

\begin{document}

\linespread{1.0}

\title{Materials 15 - More on the CEMP vs. CUSUM criteria}
\author{Laura G\'ati} 
\date{\today}
\maketitle

%%%%%%%%%%%%%%%%%%%%             DOCUMENT           %%%%%%%%%%%%%%%%%% 

\tableofcontents

%\listoffigures

\newpage
\section{Model summary}
\begin{align}
x_t &=  -\sigma i_t +\hat{\E}_t \sum_{T=t}^{\infty} \beta^{T-t }\big( (1-\beta)x_{T+1} - \sigma(\beta i_{T+1} - \pi_{T+1}) +\sigma r_T^n \big)  \label{prestons18}  \\
\pi_t &= \kappa x_t +\hat{\E}_t \sum_{T=t}^{\infty} (\alpha\beta)^{T-t }\big( \kappa \alpha \beta x_{T+1} + (1-\alpha)\beta \pi_{T+1} + u_T\big) \label{prestons19}  \\
i_t &= \psi_{\pi}\pi_t + \psi_{x} x_t  + \bar{i}_t \label{TR}
\end{align}
\begin{equation}
\hat{\E}_t z_{t+h} =  \bar{z}_{t-1} + bh_x^{h-1}s_t  \quad \forall h\geq 1 \quad \quad b = g_x\; h_x \quad \quad \text{PLM} \label{PLM}
\end{equation}
\begin{equation}
\bar{z}_{t} = \bar{z}_{t-1} +k_t^{-1}\underbrace{\big(z_{t} -(\bar{z}_{t-1}+bs_{t-1}) \big)}_{\text{fcst error using (\ref{PLM})} } 
\end{equation}
(Vector learning. For scalar learning, $\bar{z}= \begin{pmatrix} \bar{\pi} & 0 & 0\end{pmatrix}' $. I'm also not writing the case where the slope $b$ is also learned.)
 \begin{align*}
k_t & = \begin{cases} k_{t-1}+1 \quad \text{for decreasing gain learning}  \\ \bar{g}^{-1}  \quad \text{for constant gain learning.}\numberthis
\end{cases} 
\end{align*}

\begin{figure}[h!]
\subfigure[Learning constant only]{\includegraphics[scale=\myAdjustableFigScale]{\myFigPath command_IRFs_many_learning_RIR_LH_monpol_cgain_gbar_0_145_default_learning_true_baseline_no_info_ass_constant_only}}
\subfigure[Learning slope and constant]{\includegraphics[scale=\myAdjustableFigScale]{\myFigPath command_IRFs_many_learning_RIR_LH_monpol_cgain_gbar_0_145_default_learning_true_baseline_no_info_ass_slope_and_constant}}
\caption{Reference: baseline model}
\end{figure}

\clearpage


\newpage
\section{The CEMP vs. the CUSUM criterion}

CEMP's criterion  
\begin{align}
\theta_t & = |\hat{\E}_{t-1}\pi_t - \E_{t-1}\pi_t | / (\text{Var(shocks)}) \\
\text{i.e.} \quad &\text{PLM- E[ALM], scaled by shocks}
\end{align}

For my version of CEMP's criterion, I rewrite the ALM
\begin{align}
z_t & = A_a f_a + A_b f_b + A_s s_t \\
\text{as} \quad & z_t =  F  +Gs_t  \\
\Leftrightarrow \quad & z_t = \begin{bmatrix} F & G \end{bmatrix} \begin{bmatrix} 1 \\ s_t\end{bmatrix}
\end{align}
Then, since the PLM is $z_t = \phi  \begin{bmatrix} 1 \\ s_t\end{bmatrix}$, the generalized CEMP criterion becomes
\begin{equation}
\theta_t = \max | \Sigma^{-1} ( \phi - \begin{bmatrix} F & G \end{bmatrix}) |
\end{equation}
where $\Sigma$ is the VC matrix of shocks.
As for the CUSUM criterion, what I did in Materials 5 was
\begin{align}
 \omega_t & =  \omega_{t-1} + \kappa k_{t-1}^{-1}(FE_t^2 -\omega_{t-1})\\
\theta_t & =  \theta_{t-1} + \kappa k_{t-1}^{-1}(FE_t^2/\omega_t -\theta_{t-1})\
\end{align}
where $FE_t$ is the most recent short-run forecast error ($ny\times 1$), and $\omega_t$ is the agents' estimate of the forecast error variance ($ny \times ny$). To take into account that these are now matrices, I now write
\begin{align}
\omega_t & =  \omega_{t-1} + \kappa k_{t-1}^{-1}(FE_t FE_t' -\omega_{t-1})\\
\theta_t & =  \theta_{t-1} + \kappa k_{t-1}^{-1}\text{mean}((\omega_t^{-1}FE_t FE_t' -\theta_{t-1}))
\end{align}




\newpage
\section{Investigating the behavior of CEMP and CUSUM criteria}

\subsection{Anchoring as a function of $\psi_{\pi}$, fixing $\psi_x = 0, \bar{\theta}=4, \tilde{\theta}=0.2$}
\begin{figure}[h!]
\subfigure[CEMP's criterion]{\includegraphics[scale=\myAdjustableFigScale]{\myFigPath command_IRFs_anchoring_loss_again_critCEMP_constant_only_params_psi_pi_1_01_psi_x_0_gbar_0_145_thetbar_4_thettilde_0_1_kap_0_8_alph_CB_0_2020_01_28}}
\subfigure[CUSUM criterion]{\includegraphics[scale=\myAdjustableFigScale]{\myFigPath command_IRFs_anchoring_loss_again_critCUSUM_constant_only_params_psi_pi_1_01_psi_x_0_gbar_0_145_thetbar_1_thettilde_0_2_kap_0_8_alph_CB_0_2020_01_28}}
\caption{Inverse gains, $\psi_{\pi} =1.01$}
\end{figure}

\begin{figure}[h!]
\subfigure[CEMP's criterion]{\includegraphics[scale=\myAdjustableFigScale]{\myFigPath command_IRFs_anchoring_loss_again_critCEMP_constant_only_params_psi_pi_1_5_psi_x_0_gbar_0_145_thetbar_4_thettilde_0_1_kap_0_8_alph_CB_0_2020_01_28}}
\subfigure[CUSUM criterion]{\includegraphics[scale=\myAdjustableFigScale]{\myFigPath command_IRFs_anchoring_loss_again_critCUSUM_constant_only_params_psi_pi_1_5_psi_x_0_gbar_0_145_thetbar_1_thettilde_0_2_kap_0_8_alph_CB_0_2020_01_28}}
\caption{Inverse gains, $\psi_{\pi} =1.5$}
\end{figure}

\begin{figure}[h!]
\subfigure[CEMP's criterion]{\includegraphics[scale=\myAdjustableFigScale]{\myFigPath command_IRFs_anchoring_loss_again_critCEMP_constant_only_params_psi_pi_2_psi_x_0_gbar_0_145_thetbar_4_thettilde_0_1_kap_0_8_alph_CB_0_2020_01_28}}
\subfigure[CUSUM criterion]{\includegraphics[scale=\myAdjustableFigScale]{\myFigPath command_IRFs_anchoring_loss_again_critCUSUM_constant_only_params_psi_pi_1_8_psi_x_0_gbar_0_145_thetbar_1_thettilde_0_2_kap_0_8_alph_CB_0_2020_01_28}}
\caption{Inverse gains, $\psi_{\pi} =2$}
\end{figure}

\subsection{Why do the two criteria behave opposite ways?}

A rough restatement of the two criteria: you get unanchored expectations if:
\begin{equation}
\theta_t^{CEMP} = | ( \phi - \begin{bmatrix} F & G \end{bmatrix}) | > \bar{\theta} \quad \quad \text{vs.} \quad \quad  \theta_t^{CUSUM} = f' \omega^{-1}f > \tilde{\theta}
\end{equation}
where $\phi$ is the agents' estimated matrix, $F,G$ are the ALM matrices that incorporate long-horizon expectations, $f$ is the one-period ahead forecast error and $\omega$ is the estimated forecast error variance matrix. (Note: I'm using L\"utkepohl's \emph{Introduction to Multiple Time Series Analysis}, p. 160 to reformulate the CUSUM criterion as a statistic that has a $\chi^2$ distribution.)

Here's the key difference between the two criteria:
\begin{itemize}
\item $F,G$ incorporate LH expectations. Thus when $\psi_{\pi}$ is large, $F,G$ move a lot, opening up the gap between $\phi$ and itself, leading to unanchored expectations.
\item $f$ doesn't incorporate long-horizon expectations and thus doesn't move as much. In fact, when $\psi_{\pi}$ is large, current inflation responds less, and thus one-period ahead forecast errors are \emph{smaller}; you get more anchoring. 
\end{itemize}


\end{document}





