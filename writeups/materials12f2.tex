\documentclass[11pt]{article}
\usepackage{amsmath, amsthm, amssymb,lscape, natbib}
\usepackage{mathtools}
\usepackage{subfigure}
\usepackage[font=footnotesize,labelfont=bf]{caption}
\usepackage{graphicx}
\usepackage{colortbl}
\usepackage{hhline}
\usepackage{multirow}
\usepackage{multicol}
\usepackage{setspace}
\usepackage[final]{pdfpages}
\usepackage[left=2.5cm,top=2.5cm,right=2.5cm, bottom=2.5cm]{geometry}
\usepackage{natbib} 
\usepackage{bibentry} 
\newcommand{\bibverse}[1]{\begin{verse} \bibentry{#1} \end{verse}}
\newcommand{\vs}{\vspace{.3in}}
\renewcommand{\ni}{\noindent}
\usepackage{xr-hyper}
\usepackage[]{hyperref}
\hypersetup{
    colorlinks=true,
    linkcolor=blue,
    filecolor=magenta,      
    urlcolor=cyan,
}
 
\urlstyle{same}
\usepackage[capposition=top]{floatrow}
\usepackage{amssymb}
\usepackage{relsize}
\usepackage[dvipsnames]{xcolor}
\usepackage{fancyhdr}
\usepackage{tikz}
 
\pagestyle{fancy} % customize header and footer
\fancyhf{} % clear initial header and footer
%\rhead{Overleaf}
\lhead{\centering \rightmark} % this adds subsection number and name
\lfoot{\centering \rightmark} 
\rfoot{\thepage} % put page number (the centering command puts it in the middle, don't matter if you put it in right or left footer)

\def \myFigPath {../figures/} 
% BE CAREFUL WITH FIGNAMES, IN LATEX THEY'RE NOT CASE SENSITIVE!!
\def \myTablePath {../tables/} 

%\definecolor{mygreen}{RGB}{0, 100, 0}
\definecolor{mygreen}{RGB}{0, 128, 0}

\definecolor{citec}{rgb}{0,0,.5}
\definecolor{linkc}{rgb}{0,0,.6}
\definecolor{bcolor}{rgb}{1,1,1}
\hypersetup{
%hidelinks = true
  colorlinks = true,
  urlcolor=linkc,
  linkcolor=linkc,
  citecolor = citec,
  filecolor = linkc,
  pdfauthor={Laura G\'ati},
}


\geometry{left=.83in,right=.89in,top=1in,
bottom=1in}
\linespread{1.5}
\renewcommand{\[}{\begin{equation}}
\renewcommand{\]}{\end{equation}}

% New Options
\newtheorem{prop}{Proposition}
\newtheorem{definition}{Definition}[section]
\newtheorem*{remark}{Remark}
\newtheorem{lemma}{Lemma}
\newtheorem{corollary}{Corollary}
\newtheorem{conjecture}{Conjecture}

%\newtheorem{theorem}{Theorem}[section] % the third argument specifies that their number will be adopted to the section
%\newtheorem{corollary}{Corollary}[theorem]
%\newtheorem{lemma}[theorem]{Lemma}
%\declaretheorem{proposition}
%\linespread{1.3}
%\raggedbottom
%\font\reali=msbm10 at 12pt

% New Commands
\newcommand{\real}{\hbox{\reali R}}
\newcommand{\realp}{\hbox{\reali R}_{\scriptscriptstyle +}}
\newcommand{\realpp}{\hbox{\reali R}_{\scriptscriptstyle ++}}
\newcommand{\R}{\mathbb{R}}
\DeclareMathOperator{\E}{\mathbb{E}}
\DeclareMathOperator{\argmin}{arg\,min}
\newcommand\w{3.0in}
\newcommand\wnum{3.0}
\def\myFigWidth{5.3in}
\def\mySmallerFigWidth{2.1in}
\def\myEvenBiggerFigScale{0.8}
\def\myPointSixFigScale{0.6}
\def\myBiggerFigScale{0.4}
\def\myFigScale{0.3}
\def\myMediumFigScale{0.25}
\def\mySmallFigScale{0.22}
\def\mySmallerFigScale{0.18}
\def\myTinyFigScale{0.16}
\def\myPointFourteenFigScale{0.14}
\def\myTinierFigScale{0.12}
\def\myAdjustableFigScale{0.18}
\newcommand\numberthis{\addtocounter{equation}{1}\tag{\theequation}} % this defines a command to make align only number this line
\newcommand{\code}[1]{\texttt{#1}} %code %

\renewcommand*\contentsname{Overview}
\setcounter{tocdepth}{2}

% define a command to make a huge question mark (it works in math mode)
\newcommand{\bigqm}[1][1]{\text{\larger[#1]{\textbf{?}}}}

\begin{document}

\linespread{1.0}

\title{Materials 12f - ``pil"-extension of baseline model\\
- Lagged inflation in TR
\\
\small{See Notes 7 Jan 2020}}
\author{Laura G\'ati} 
\date{\today}
\maketitle

%%%%%%%%%%%%%%%%%%%%             DOCUMENT           %%%%%%%%%%%%%%%%%% 

%\tableofcontents

%\listoffigures

Compare Mathematica (\texttt{materials12f2.nb}).

\textcolor{red}{Red} stuff are changes compared to the baseline model.
\section{Model equations and goal}
\begin{align}
x_t &=  -\sigma i_t +\hat{\E}_t \sum_{T=t}^{\infty} \beta^{T-t }\big( (1-\beta)x_{T+1} - \sigma(\beta i_{T+1} - \pi_{T+1}) +\sigma r_T^n \big)  \label{prestons18}  \\
\pi_t &= \kappa x_t +\hat{\E}_t \sum_{T=t}^{\infty} (\alpha\beta)^{T-t }\big( \kappa \alpha \beta x_{T+1} + (1-\alpha)\beta \pi_{T+1} + u_T\big) \label{prestons19}  \\
i_t &= \psi_{\pi}\textcolor{red}{\pi_{t-1}}+ \psi_{x} x_t  + \bar{i}_t \label{TR}
\end{align}
Compact notation
\begin{equation}
z_t = \begin{bmatrix} \pi_t \\ x_t \\ i_t \end{bmatrix} = A_a f_{a} + A_b f_{b} + A_s s_t \quad \text{with} \quad  s_t = \begin{bmatrix} r_t^n \\ \bar{i}_t \\ u_t \end{bmatrix}
\end{equation}

\section{MN matrices}
\begin{equation}
\underbrace{\begin{bmatrix} \sigma\psi_{\pi}\textcolor{red}{\beta} &  1+\sigma\psi_x\\ 1& -\kappa \end{bmatrix}}_{\equiv M}\begin{bmatrix} \pi_t \\ x_t \end{bmatrix} = \underbrace{\begin{bmatrix} \begin{bmatrix} \sigma(1-\textcolor{red}{\beta^2}\psi_{\pi}), &1-\beta-\sigma\beta\psi_x ,&0 \end{bmatrix}f_b +d_{x,s} s_t + \textcolor{red}{\begin{bmatrix} 0, &0, &0, & -\sigma\psi_{\pi}\end{bmatrix}s_t}\\ \begin{bmatrix} (1-\alpha)\beta, &\kappa\alpha\beta ,& 0 \end{bmatrix} f_a +d_{\pi,s} s_t \end{bmatrix}}_{\equiv N}
\end{equation}
where 
\begin{align*}
 d_{x,s} & =  -\sigma\begin{bmatrix} -1 &1 & 0 & \textcolor{red}{0}  \end{bmatrix}InxBhx \quad \quad InxBhx \equiv (I_{nx}-\beta h_x)^{-1} \numberthis \\
 d_{\pi,s} & =  \begin{bmatrix} 0&0 & 1 & \textcolor{red}{0}  \end{bmatrix}InxABhx \quad \quad InxABhx \equiv (I_{nx}-\alpha\beta h_x)^{-1} \numberthis \\
  d_{i,s} & =  \begin{bmatrix} 0 &1 & 0 & \textcolor{red}{\psi_{\pi}}  \end{bmatrix} \numberthis
\end{align*}

\section{PQ matrices}
\begin{equation}
\underbrace{\begin{bmatrix} \textcolor{red}{\sigma\beta\psi_{\pi}} & 1&  \sigma \\ 1& -\kappa & 0 \\ \textcolor{red}{0}  & -\psi_{x} & 1 \end{bmatrix}}_{\equiv P}\begin{bmatrix} \pi_t \\ x_t \\i_t \end{bmatrix} = \underbrace{\begin{bmatrix} \begin{bmatrix}\sigma, &1-\beta, &\beta  (-\sigma )\end{bmatrix}f_b +c_{x,s} s_t \\ \begin{bmatrix}(1-\alpha ) \beta ,&\alpha  \beta  \kappa ,&0\end{bmatrix}f_a +c_{\pi,s} s_t \\ c_{i,s} s_t \end{bmatrix}}_{\equiv Q}
\end{equation}
where 
\begin{align}
c_{x,s} & = \sigma  \left(
\begin{array}{ccc}
 1 & 0 & 0 \\
\end{array}
\right).\text{InxBhx};\\
c_{\pi,s} & = \left(
\begin{array}{ccc}
 0 & 0 & 1 \\
\end{array}
\right).\text{InxABhx}\\
c_{i,s} & = \left(
\begin{array}{cccc}
 0 & 1 & 0 &\textcolor{red}{\psi_{\pi}} \\
\end{array}
\right) = d_{i,s}
\end{align}
where $\text{InxABhx}$ and $\text{InxBhx}$ are the same as before.
The (*)-relation is
\begin{equation}
f_b(3) = \textcolor{red}{\beta}\psi_{\pi}f_b(1) + \psi_{x}f_b(2) + \frac{1}{\beta}\{ \begin{bmatrix} 0 &1 & 0 & \textcolor{red}{0}  \end{bmatrix}(I_{nx}-\beta h_x)^{-1} s_t - \begin{bmatrix} 0&1 & 0 & \textcolor{red}{0} \end{bmatrix}s_t  \} \tag{*}
\end{equation}
where in (*) there was a $\psi_{\pi}\pi_t$ term that I moved to the LHS of the PQ equation ($P(1,1)$).

The Matlab code that uses this is \texttt{matrices\_A\_12f2.m}.


\end{document}





