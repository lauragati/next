\documentclass[11pt]{article}
\usepackage{amsmath, amsthm, amssymb,lscape, natbib}
\usepackage{mathtools}
\usepackage{subfigure}
\usepackage[font=footnotesize,labelfont=bf]{caption}
\usepackage{graphicx}
\usepackage{colortbl}
\usepackage{hhline}
\usepackage{multirow}
\usepackage{multicol}
\usepackage{setspace}
\usepackage[final]{pdfpages}
\usepackage[left=2.5cm,top=2.5cm,right=2.5cm, bottom=2.5cm]{geometry}
\usepackage{natbib} 
\usepackage{bibentry} 
\newcommand{\bibverse}[1]{\begin{verse} \bibentry{#1} \end{verse}}
\newcommand{\vs}{\vspace{.3in}}
\renewcommand{\ni}{\noindent}
\usepackage{xr-hyper}
\usepackage[]{hyperref}
\usepackage[capposition=top]{floatrow}
\usepackage{amssymb}
\usepackage{relsize}
\usepackage[dvipsnames]{xcolor}
\usepackage{fancyhdr}
 
\pagestyle{fancy} % customize header and footer
\fancyhf{} % clear initial header and footer
%\rhead{Overleaf}
\lhead{\centering \rightmark} % this adds subsection number and name
\rfoot{\centering \thepage} % put page number (the centering command puts it in the middle, don't matter if you put it in right or left footer)

\def \myFigPath {../figures/} 
% BE CAREFUL WITH FIGNAMES, IN LATEX THEY'RE NOT CASE SENSITIVE!!
\def \myTablePath {../tables/} 

%\definecolor{mygreen}{RGB}{0, 100, 0}
\definecolor{mygreen}{RGB}{0, 128, 0}

\definecolor{citec}{rgb}{0,0,.5}
\definecolor{linkc}{rgb}{0,0,.6}
\definecolor{bcolor}{rgb}{1,1,1}
\hypersetup{
%hidelinks = true
  colorlinks = true,
  urlcolor=linkc,
  linkcolor=linkc,
  citecolor = citec,
  filecolor = linkc,
  pdfauthor={Laura G\'ati},
}


\geometry{left=.83in,right=.89in,top=1in,
bottom=1in}
\linespread{1.5}
\renewcommand{\[}{\begin{equation}}
\renewcommand{\]}{\end{equation}}

% New Options
\newtheorem{prop}{Proposition}
\newtheorem{definition}{Definition}[section]
\newtheorem*{remark}{Remark}
\newtheorem{lemma}{Lemma}
\newtheorem{corollary}{Corollary}
%\newtheorem{theorem}{Theorem}[section] % the third argument specifies that their number will be adopted to the section
%\newtheorem{corollary}{Corollary}[theorem]
%\newtheorem{lemma}[theorem]{Lemma}
%\declaretheorem{proposition}
%\linespread{1.3}
%\raggedbottom
%\font\reali=msbm10 at 12pt

% New Commands
\newcommand{\real}{\hbox{\reali R}}
\newcommand{\realp}{\hbox{\reali R}_{\scriptscriptstyle +}}
\newcommand{\realpp}{\hbox{\reali R}_{\scriptscriptstyle ++}}
\newcommand{\R}{\mathbb{R}}
\DeclareMathOperator{\E}{\mathbb{E}}
\DeclareMathOperator{\argmin}{arg\,min}
\newcommand\w{3.0in}
\newcommand\wnum{3.0}
\def\myFigWidth{5.3in}
\def\mySmallerFigWidth{2.1in}
\def\myEvenBiggerFigScale{0.8}
\def\myPointSixFigScale{0.6}
\def\myBiggerFigScale{0.4}
\def\myFigScale{0.3}
\def\mySmallFigScale{0.22}
\def\mySmallerFigScale{0.18}
\def\myTinyFigScale{0.16}
\def\myPointFourteenFigScale{0.14}
\def\myTinierFigScale{0.12}
\newcommand\numberthis{\addtocounter{equation}{1}\tag{\theequation}} % this defines a command to make align only number this line
\newcommand{\code}[1]{\texttt{#1}} %code %

\renewcommand*\contentsname{Overview}
\setcounter{tocdepth}{2}

% define a command to make a huge question mark (it works in math mode)
\newcommand{\bigqm}[1][1]{\text{\larger[#1]{\textbf{?}}}}

\begin{document}

\linespread{1.0}

\title{Materials 6 - More on IRFs}
\author{Laura G\'ati} 
\date{\today}
\maketitle

%%%%%%%%%%%%%%%%%%%%             DOCUMENT           %%%%%%%%%%%%%%%%%% 

\tableofcontents

%\listoffigures

%\newpage


\newpage
\section{Model summary, adding $\rho i_{t-1}$}
\begin{align}
x_t &=  -\sigma i_t +\hat{\E}_t \sum_{T=t}^{\infty} \beta^{T-t }\big( (1-\beta)x_{T+1} - \sigma(\beta i_{T+1} - \pi_{T+1}) +\sigma r_T^n \big)  \label{prestons18}  \\
\pi_t &= \kappa x_t +\hat{\E}_t \sum_{T=t}^{\infty} (\alpha\beta)^{T-t }\big( \kappa \alpha \beta x_{T+1} + (1-\alpha)\beta \pi_{T+1} + u_T\big) \label{prestons19}  \\
i_t &= \psi_{\pi}\pi_t + \psi_{x} x_t  + \textcolor{blue}{\rho i_{t-1}} + \bar{i}_t \label{TR}
\end{align}
\begin{equation}
\hat{\E}_t z_{t+h} =  \begin{bmatrix}\bar{\pi}_{t-1} \\ 0 \\ 0 \end{bmatrix}+ bP^{h-1}s_t  \quad \forall h\geq 1 \quad \quad b = gx \; hx \quad \quad \text{PLM} \label{PLM}
\end{equation}
\begin{equation}
\bar{\pi}_{t} = \bar{\pi}_{t-1} +k_t^{-1}\underbrace{\big(\pi_{t} -(\bar{\pi}_{t-1}+b_1s_{t-1}) \big)}_{\text{fcst error using (\ref{PLM})} } \quad \quad  \text{($b_1$ is the first row of $b$)}
\end{equation}
 \begin{align*}
k_t & = \mathbb{I}\times(k_{t-1}+1) + (1-\mathbb{I}) \times \bar{g}^{-1}  \label{gain} \numberthis\\
\mathbb{I} & = \begin{cases} 1 \quad \text{if} \; \theta_t \leq \bar{\theta}  \\ 0 \quad \text{otherwise.}\numberthis
\end{cases} \\
\theta_t & = |\hat{\E}_{t-1}\pi_t - \E_{t-1}\pi_t| / \sigma_s \quad \quad \text{CEMP criterion for the gain}\label{criterion}\numberthis
\end{align*}

The alternative criterion for the choice of gain is a recursive variant of the CUSUM-test (Brown, Durbin, Evans 1975):
\begin{enumerate}
\item Let $FE_t$ denote the short-run forecast error, and $\omega_t$ firms' estimate of the FE variance.
\item Let $\kappa \in (0,1)$ and $\tilde{\theta}$ be the new threshold value for the criterion.
\item Then for initial $(\omega_0, \theta_0)$, firms in every period estimate the criterion and the FEV as:
\begin{align}
 \omega_t & =  \omega_{t-1} + \kappa k_{t-1}^{-1}(FE_t^2 -\omega_{t-1})\\
\theta_t & =  \theta_{t-1} + \kappa k_{t-1}^{-1}(FE_t^2/\omega_t -\theta_{t-1})\\
k_t & = \mathbb{I}\times(k_{t-1}+1) + (1-\mathbb{I}) \times \bar{g}^{-1} \\
\mathbb{I} & = 1 \quad \text{if} \quad  \theta_t \leq \tilde{\theta}
\end{align}
\end{enumerate}

\newpage
\section{Compact notation - with lagged interest rate term in TR}
 \begin{align}
z_t & = A_p^{RE} \E_t z_{t+1} + A_s^{RE} s_t \label{LOM_RE} \\
z_t & = A_a^{LH} f_a(t) + A_b^{LH} f_b(t) + A_s^{LH} s_t \label{LOM_LH} \\
s_t & = P s_{t-1} + \epsilon_t \label{exog} \\
s'_t & = hx \; s'_{t-1} + \epsilon'_t \label{exog} \\
 \quad \text{where} \quad 
 s'_t & \equiv \begin{pmatrix} r_t^n \\ \bar{i}_t \\ u_t \\ \textcolor{blue}{i_{t-1}}
 \end{pmatrix} \quad 
 hx  \equiv \begin{pmatrix} \rho_r & 0 & 0 & \textcolor{blue}{0} \\ 0& \rho_i & 0 & \textcolor{blue}{0} \\ 0&0& \rho_u & \textcolor{blue}{0}  \\ 
 \textcolor{blue}{gx_{3,1}}&\textcolor{blue}{gx_{3,1}}& \textcolor{blue}{gx_{3,3}} & \textcolor{blue}{gx_{3,4}}
 \end{pmatrix}  \quad 
 \epsilon'_t \equiv \begin{pmatrix}\varepsilon_t^{r} \\ \varepsilon_t^{i}  \\ \varepsilon_t^{u} \\ \textcolor{blue}{0} 
 \end{pmatrix}  \quad  \text{and } \quad \Sigma'  =  \begin{pmatrix} \sigma_r & 0 & 0 & \textcolor{blue}{0} \\ 0& \sigma_i & 0 & \textcolor{blue}{0}  \\ 0&0& \sigma_u & \textcolor{blue}{0}  \\ \textcolor{blue}{0}  & \textcolor{blue}{0} & \textcolor{mygreen}{0} & \textcolor{blue}{0} &
 \end{pmatrix} 
\end{align}

$i_{t-1}$ is an endogenous state and breaks the link that previously had $P = hx$; now this is no longer true. In particular, using Matlabby notation, $P = hx(1:3,1:3)$. What I don't get though is why $gx_{3,4} \neq \rho$? $\textcolor{mygreen}{\bigqm[5]}$ 

Adding $i_{t-1}$ to the state vector fortunately doesn't change $A^{RE}_p, A^{LH}_a$ or $A^{LH}_b$, but it does change $A^{RE}_s$ and $A^{LH}_s$. The latter two get an additional column to account for the new state variable. With $g_{i, j} \; i=\pi,x, \; j=a,b$ unchanged from Materials 4, the new coefficient matrices are given by (new elements highlighted in \textcolor{blue}{blue}):
\begin{align}
A_s^{RE} &= \begin{pmatrix}   \frac{\kappa\sigma}{w}  &-\frac{\kappa\sigma}{w}  & 1-\frac{\kappa\sigma\psi_{\pi}}{w} & \textcolor{blue}{0}\\
 \frac{ \sigma}{w} &  -\frac{\sigma}{w} & -\frac{\sigma\psi_{\pi}}{w} & \textcolor{blue}{0}\\ 
 \psi_x( \frac{\sigma}{w}) + \psi_{\pi}( \frac{\kappa\sigma}{w}) & \psi_x(- \frac{\sigma}{w}) + \psi_{\pi}(- \frac{\kappa\sigma}{w}) +1 &  \psi_x(-\frac{\sigma\psi_{\pi}}{w}) + \psi_{\pi}( 1-\frac{\kappa\sigma\psi_{\pi}}{w}) & \textcolor{blue}{\rho}\end{pmatrix}  
\\
 A_s^{LR} & = \begin{pmatrix} g_{\pi s} \\ g_{x s} \\ \psi_{\pi}g_{\pi s} + \psi_xg_{x s} + \begin{bmatrix} 0 & 1& 0 & \textcolor{blue}{\rho}\end{bmatrix}
\end{pmatrix} \\
g_{\pi s} & = (1-\frac{\kappa\sigma\psi_{\pi}}{w} )\begin{bmatrix} 0&0&1 & \textcolor{blue}{0}\end{bmatrix} (I_{\textcolor{blue}{4}} - \alpha\beta \textcolor{blue}{hx})^{-1} -\frac{\kappa\sigma}{w}\begin{bmatrix} -1&1&0 & \textcolor{blue}{\rho} \end{bmatrix} (I_{\textcolor{blue}{4}} -\beta \textcolor{blue}{hx})^{-1}\\
g_{x s} & =  \frac{-\sigma\psi_{\pi}}{w} \begin{bmatrix} 0&0&1& \textcolor{blue}{0} \end{bmatrix}(I_{\textcolor{blue}{4}} - \alpha\beta \textcolor{blue}{hx})^{-1}  -\frac{\sigma}{w}\begin{bmatrix} -1&1&0 & \textcolor{blue}{\rho}\end{bmatrix}(I_{\textcolor{blue}{4}} -\beta \textcolor{blue}{hx})^{-1}
\end{align}

%%%%%%%%%%%  IRFs
\newpage
\section{Small deviations in $\pi$, large ones in $x$, overshooting - IRFs}

\subsection{Baseline figures, $\psi_x = 0, \psi_{\pi} = 1.5, \rho = 0$}
\begin{figure}[h!]
\subfigure[The observables for the specific shock sequence]{\includegraphics[scale = \mySmallFigScale]{\myFigPath materials6_observables_intrate_smoothing_rho0}} 
\subfigure[Inverse gain and drift for the specific shock sequence, CEMP and CUSUM criterion]{\includegraphics[scale = \mySmallFigScale]{\myFigPath materials6_gain_drift_cusum_intrate_smoothing_rho0} }
\caption{A baseline shock sequence,$\psi_x = 0, \psi_{\pi} = 1.5, \rho = 0$}
\end{figure}		
						


\newpage
\vspace{-0.5cm}
\subsection{IRFs when changing $\rho$}
\vspace{-0.5cm}
\begin{figure}[h!]
\subfigure[\colorbox{yellow}{$\rho$ = 0}]{\includegraphics[scale = \myPointFourteenFigScale]{\myFigPath materials6_IRFs_intrate_smoothing_natrate_rho0}} 
\subfigure[\colorbox{yellow}{$\rho$ = 0.3}]{\includegraphics[scale = \myPointFourteenFigScale]{\myFigPath materials6_IRFs_intrate_smoothing_natrate_rho0_3} }
\subfigure[\colorbox{yellow}{$\rho$ = 0.6}]{\includegraphics[scale = \myPointFourteenFigScale]{\myFigPath materials6_IRFs_intrate_smoothing_natrate_rho0_6} }
\subfigure[\colorbox{yellow}{$\rho$ = 0.9}]{\includegraphics[scale = \myPointFourteenFigScale]{\myFigPath materials6_IRFs_intrate_smoothing_natrate_rho0_9} }
\caption{IRFs to a natural rate shock ($r^n$)}
\end{figure}

\vspace{-0.5cm}
\begin{figure}[h!]
\subfigure[\colorbox{yellow}{$\rho$ = 0}]{\includegraphics[scale = \myPointFourteenFigScale]{\myFigPath materials6_IRFs_intrate_smoothing_monpol_rho0}} 
\subfigure[\colorbox{yellow}{$\rho$ = 0.3}]{\includegraphics[scale = \myPointFourteenFigScale]{\myFigPath materials6_IRFs_intrate_smoothing_monpol_rho0_3} }
\subfigure[\colorbox{yellow}{$\rho$ = 0.6}]{\includegraphics[scale = \myPointFourteenFigScale]{\myFigPath materials6_IRFs_intrate_smoothing_monpol_rho0_6} }
\subfigure[\colorbox{yellow}{$\rho$ = 0.9}]{\includegraphics[scale = \myPointFourteenFigScale]{\myFigPath materials6_IRFs_intrate_smoothing_monpol_rho0_9} }
\caption{IRFs to a monetary policy shock ($\bar{i}$)}
\end{figure}
\vspace{-0.5cm}
\begin{figure}[h!]
\subfigure[\colorbox{yellow}{$\rho$ = 0}]{\includegraphics[scale = \myPointFourteenFigScale]{\myFigPath materials6_IRFs_intrate_smoothing_costpush_rho0}} 
\subfigure[\colorbox{yellow}{$\rho$ = 0.3}]{\includegraphics[scale = \myPointFourteenFigScale]{\myFigPath materials6_IRFs_intrate_smoothing_costpush_rho0_3} }
\subfigure[\colorbox{yellow}{$\rho$ = 0.6}]{\includegraphics[scale = \myPointFourteenFigScale]{\myFigPath materials6_IRFs_intrate_smoothing_costpush_rho0_6} }
\subfigure[\colorbox{yellow}{$\rho$ = 0.9}]{\includegraphics[scale = \myPointFourteenFigScale]{\myFigPath materials6_IRFs_intrate_smoothing_costpush_rho0_9} }
\caption{IRFs to a cost-push shock ($u$)}
\end{figure}

\clearpage	
\subsection{Inverse gain and drift when changing $\rho$, no shocks}

\vspace{-0.5cm}
\begin{figure}[h!]
\subfigure[\colorbox{yellow}{$\rho$ = 0}]{\includegraphics[scale = \myTinierFigScale]{\myFigPath materials6_gain_drift_cusum_intrate_smoothing_rho0}} 
\subfigure[\colorbox{yellow}{$\rho$ = 0.3}]{\includegraphics[scale = \myTinierFigScale]{\myFigPath materials6_gain_drift_cusum_intrate_smoothing_rho0_3} }
\subfigure[\colorbox{yellow}{$\rho$ = 0.6}]{\includegraphics[scale = \myTinierFigScale]{\myFigPath materials6_gain_drift_cusum_intrate_smoothing_rho0_6} }
\subfigure[\colorbox{yellow}{$\rho$ = 0.9}]{\includegraphics[scale = \myTinierFigScale]{\myFigPath materials6_gain_drift_cusum_intrate_smoothing_rho0_9} }
%\caption{Inverse gain and drift when increasing $\rho$}
\end{figure}
\vspace{-0.5cm}

\subsection{Gain and drift conditional on shocks when changing $\rho$}
\begin{figure}[h!]
\subfigure[\colorbox{yellow}{$\rho$ = 0}]{\includegraphics[scale = \myPointFourteenFigScale]{\myFigPath materials6_gain_drift_natrate_rho0}} 
\subfigure[\colorbox{yellow}{$\rho$ = 0.3}]{\includegraphics[scale = \myPointFourteenFigScale]{\myFigPath materials6_gain_drift_natrate_rho0_3} }
\subfigure[\colorbox{yellow}{$\rho$ = 0.6}]{\includegraphics[scale = \myPointFourteenFigScale]{\myFigPath materials6_gain_drift_natrate_rho0_6} }
\subfigure[\colorbox{yellow}{$\rho$ = 0.9}]{\includegraphics[scale = \myPointFourteenFigScale]{\myFigPath materials6_gain_drift_natrate_rho0_9} }
\caption{Mean gain and drift after a natural rate shock ($r^n$)}
\end{figure}
\begin{figure}[h!]
\subfigure[\colorbox{yellow}{$\rho$ = 0}]{\includegraphics[scale = \myPointFourteenFigScale]{\myFigPath materials6_gain_drift_monpol_rho0}} 
\subfigure[\colorbox{yellow}{$\rho$ = 0.3}]{\includegraphics[scale = \myPointFourteenFigScale]{\myFigPath materials6_gain_drift_monpol_rho0_3} }
\subfigure[\colorbox{yellow}{$\rho$ = 0.6}]{\includegraphics[scale = \myPointFourteenFigScale]{\myFigPath materials6_gain_drift_monpol_rho0_6} }
\subfigure[\colorbox{yellow}{$\rho$ = 0.9}]{\includegraphics[scale = \myPointFourteenFigScale]{\myFigPath materials6_gain_drift_monpol_rho0_9} }
\caption{Mean gain and drift after a monetary policy shock ($\bar{i}$)}
\end{figure}
\begin{figure}[h!]
\subfigure[\colorbox{yellow}{$\rho$ = 0}]{\includegraphics[scale = \myPointFourteenFigScale]{\myFigPath materials6_gain_drift_costpush_rho0}} 
\subfigure[\colorbox{yellow}{$\rho$ = 0.3}]{\includegraphics[scale = \myPointFourteenFigScale]{\myFigPath materials6_gain_drift_costpush_rho0_3} }
\subfigure[\colorbox{yellow}{$\rho$ = 0.6}]{\includegraphics[scale = \myPointFourteenFigScale]{\myFigPath materials6_gain_drift_costpush_rho0_6} }
\subfigure[\colorbox{yellow}{$\rho$ = 0.9}]{\includegraphics[scale = \myPointFourteenFigScale]{\myFigPath materials6_gain_drift_costpush_rho0_9} }
\caption{Mean gain and drift after a cost-push shock ($u$)}
\end{figure}

\clearpage
\subsection{IRFs when changing $\psi_{\pi} (\rho =0)$}
\begin{figure}[h!]
\subfigure[\colorbox{yellow}{$\psi_{\pi} $ = 1.1}]{\includegraphics[scale = \myPointFourteenFigScale]{\myFigPath materials6_IRFs_intrate_smoothing_natrate_rho0_psi_pi_1_1}} 
\subfigure[\colorbox{yellow}{$\psi_{\pi} $ = 1.5}]{\includegraphics[scale = \myPointFourteenFigScale]{\myFigPath materials6_IRFs_intrate_smoothing_natrate_rho0_psi_pi_1_5} }
\subfigure[\colorbox{yellow}{$\psi_{\pi} $ = 2}]{\includegraphics[scale = \myPointFourteenFigScale]{\myFigPath materials6_IRFs_intrate_smoothing_natrate_rho0_psi_pi_2} }
\caption{IRFs to a natural rate shock ($r^n$)}
\end{figure}

\begin{figure}[h!]
\subfigure[\colorbox{yellow}{$\psi_{\pi} $ = 1.1}]{\includegraphics[scale = \myPointFourteenFigScale]{\myFigPath materials6_IRFs_intrate_smoothing_monpol_rho0_psi_pi_1_1}} 
\subfigure[\colorbox{yellow}{$\psi_{\pi} $ = 1.5}]{\includegraphics[scale = \myPointFourteenFigScale]{\myFigPath materials6_IRFs_intrate_smoothing_monpol_rho0_psi_pi_1_5} }
\subfigure[\colorbox{yellow}{$\psi_{\pi} $ = 2}]{\includegraphics[scale = \myPointFourteenFigScale]{\myFigPath materials6_IRFs_intrate_smoothing_monpol_rho0_psi_pi_2} }
\caption{IRFs to a monetary policy shock ($\bar{i}$)}
\end{figure}

\begin{figure}[h!]
\subfigure[\colorbox{yellow}{$\psi_{\pi} $ = 1.1}]{\includegraphics[scale = \myPointFourteenFigScale]{\myFigPath materials6_IRFs_intrate_smoothing_costpush_rho0_psi_pi_1_1}} 
\subfigure[\colorbox{yellow}{$\psi_{\pi} $ = 1.5}]{\includegraphics[scale = \myPointFourteenFigScale]{\myFigPath materials6_IRFs_intrate_smoothing_costpush_rho0_psi_pi_1_5} }
\subfigure[\colorbox{yellow}{$\psi_{\pi} $ = 2}]{\includegraphics[scale = \myPointFourteenFigScale]{\myFigPath materials6_IRFs_intrate_smoothing_costpush_rho0_psi_pi_2} }
\caption{IRFs to a cost-push shock ($u$)}
\end{figure}

\clearpage
\subsection{Gain and drift conditional on shocks when changing $\psi_{\pi}$}
\begin{figure}[h!]
\subfigure[\colorbox{yellow}{$\psi_{\pi} $ = 1.1}]{\includegraphics[scale = \myPointFourteenFigScale]{\myFigPath materials6_gain_drift_natrate_rho0_psi_pi_1_1}} 
\subfigure[\colorbox{yellow}{$\psi_{\pi} $ = 1.5}]{\includegraphics[scale = \myPointFourteenFigScale]{\myFigPath materials6_gain_drift_natrate_rho0_psi_pi_1_5} }
\subfigure[\colorbox{yellow}{$\psi_{\pi} $ = 2}]{\includegraphics[scale = \myPointFourteenFigScale]{\myFigPath materials6_gain_drift_natrate_rho0_psi_pi_2} }
\caption{Mean gain and drift after a natural rate shock ($r^n$)}
\end{figure}
\begin{figure}[h!]
\subfigure[\colorbox{yellow}{$\psi_{\pi} $ = 1.1}]{\includegraphics[scale = \myPointFourteenFigScale]{\myFigPath materials6_gain_drift_monpol_rho0_psi_pi_1_1}} 
\subfigure[\colorbox{yellow}{$\psi_{\pi} $ = 1.5}]{\includegraphics[scale = \myPointFourteenFigScale]{\myFigPath materials6_gain_drift_monpol_rho0_psi_pi_1_5} }
\subfigure[\colorbox{yellow}{$\psi_{\pi} $ = 2}]{\includegraphics[scale = \myPointFourteenFigScale]{\myFigPath materials6_gain_drift_monpol_rho0_psi_pi_2} }
\caption{Mean gain and drift after a monetary policy shock ($\bar{i}$)}
\end{figure}
\begin{figure}[h!]
\subfigure[\colorbox{yellow}{$\psi_{\pi} $ = 1.1}]{\includegraphics[scale = \myPointFourteenFigScale]{\myFigPath materials6_gain_drift_costpush_rho0_psi_pi_1_1}} 
\subfigure[\colorbox{yellow}{$\psi_{\pi} $ = 1.5}]{\includegraphics[scale = \myPointFourteenFigScale]{\myFigPath materials6_gain_drift_costpush_rho0_psi_pi_1_5} }
\subfigure[\colorbox{yellow}{$\psi_{\pi} $ = 2}]{\includegraphics[scale = \myPointFourteenFigScale]{\myFigPath materials6_gain_drift_costpush_rho0_psi_pi_2} }
\caption{Mean gain and drift after a cost-push shock ($u$)}
\end{figure}

\clearpage
	\begin{prop} The degree to which inflation responds to expectational deviations from rational expectations (RE) depends on $\kappa$, the slope of the Phillips curve. A lower $\kappa$ means higher price rigidity and translates to current inflation responding less to expectation gaps.
	\end{prop}
	\begin{corollary}Whether expectation gaps between subjective and rational expectations show up in inflation or output depends on the value of $\kappa$. When nominal rigidities are high ($\kappa$ is low), money is strongly nonneutral, and thus output is the margin of adjustment. When prices are flexible, money is neutral and inflation is the margin of adjustment. 
	\end{corollary}
	\begin{prop} The fact that firms and households choose between decreasing and constant gains endogenously (anchoring mechanism) introduces a novel tradeoff for monetary policy relative to exogenous gain learning (decreasing or constant). With exogenous gain learning, too strong monetary responsiveness to shocks leads to overshooting due to increased persistence of inflation. With endogenous gain learning, however, monetary policy has to hit a lower bound on responsiveness in order to keep inflation expectations anchored. With endogenous gain learning, monetary policy faces a tradeoff between overshooting and anchoring. 
	\end{prop}






\end{document}





