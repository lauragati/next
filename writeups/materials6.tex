\documentclass[11pt]{article}
\usepackage{amsmath, amsthm, amssymb,lscape, natbib}
\usepackage{mathtools}
\usepackage{subfigure}
\usepackage[font=footnotesize,labelfont=bf]{caption}
\usepackage{graphicx}
\usepackage{colortbl}
\usepackage{hhline}
\usepackage{multirow}
\usepackage{multicol}
\usepackage{setspace}
\usepackage[final]{pdfpages}
\usepackage[left=2.5cm,top=2.5cm,right=2.5cm, bottom=2.5cm]{geometry}
\usepackage{natbib} 
\usepackage{bibentry} 
\newcommand{\bibverse}[1]{\begin{verse} \bibentry{#1} \end{verse}}
\newcommand{\vs}{\vspace{.3in}}
\renewcommand{\ni}{\noindent}
\usepackage{xr-hyper}
\usepackage[]{hyperref}
\usepackage[capposition=top]{floatrow}
\usepackage{amssymb}

\def \myFigPath {../figures/} 
% BE CAREFUL WITH FIGNAMES, IN LATEX THEY'RE NOT CASE SENSITIVE!!
\def \myTablePath {../tables/} 

\definecolor{citec}{rgb}{0,0,.5}
\definecolor{linkc}{rgb}{0,0,.6}
\definecolor{bcolor}{rgb}{1,1,1}
\hypersetup{
%hidelinks = true
  colorlinks = true,
  urlcolor=linkc,
  linkcolor=linkc,
  citecolor = citec,
  filecolor = linkc,
  pdfauthor={Laura G\'ati},
}


\geometry{left=.83in,right=.89in,top=1in,
bottom=1in}
\linespread{1.5}
\renewcommand{\[}{\begin{equation}}
\renewcommand{\]}{\end{equation}}

% New Options
\newtheorem{prop}{Proposition}
\newtheorem{definition}{Definition}[section]
\newtheorem*{remark}{Remark}
\newtheorem{lemma}{Lemma}
\newtheorem{corollary}{Corollary}
%\newtheorem{theorem}{Theorem}[section] % the third argument specifies that their number will be adopted to the section
%\newtheorem{corollary}{Corollary}[theorem]
%\newtheorem{lemma}[theorem]{Lemma}
%\declaretheorem{proposition}
%\linespread{1.3}
%\raggedbottom
%\font\reali=msbm10 at 12pt

% New Commands
\newcommand{\real}{\hbox{\reali R}}
\newcommand{\realp}{\hbox{\reali R}_{\scriptscriptstyle +}}
\newcommand{\realpp}{\hbox{\reali R}_{\scriptscriptstyle ++}}
\newcommand{\R}{\mathbb{R}}
\DeclareMathOperator{\E}{\mathbb{E}}
\DeclareMathOperator{\argmin}{arg\,min}
\newcommand\w{3.0in}
\newcommand\wnum{3.0}
\def\myFigWidth{5.3in}
\def\mySmallerFigWidth{2.1in}
\def\myEvenBiggerFigScale{0.8}
\def\myPointSixFigScale{0.6}
\def\myBiggerFigScale{0.4}
\def\myFigScale{0.3}
\def\mySmallFigScale{0.22}
\def\mySmallerFigScale{0.18}
\def\myTinyFigScale{0.16}
\def\myPointFourteenFigScale{0.14}
\def\myTinierFigScale{0.12}
\newcommand\numberthis{\addtocounter{equation}{1}\tag{\theequation}} % this defines a command to make align only number this line
\newcommand{\code}[1]{\texttt{#1}} %code %

\renewcommand*\contentsname{Overview}
\setcounter{tocdepth}{2}

\begin{document}

\linespread{1.0}

\title{Materials 6 - More on IRFs}
\author{Laura G\'ati} 
\date{\today}
\maketitle

%%%%%%%%%%%%%%%%%%%%             DOCUMENT           %%%%%%%%%%%%%%%%%% 

\tableofcontents

%\listoffigures

%\newpage

\newpage
\section{Model summary, adding $\rho i_{t-1}$}
\begin{align}
x_t &=  -\sigma i_t +\hat{\E}_t \sum_{T=t}^{\infty} \beta^{T-t }\big( (1-\beta)x_{T+1} - \sigma(\beta i_{T+1} - \pi_{T+1}) +\sigma r_T^n \big)  \label{prestons18}  \\
\pi_t &= \kappa x_t +\hat{\E}_t \sum_{T=t}^{\infty} (\alpha\beta)^{T-t }\big( \kappa \alpha \beta x_{T+1} + (1-\alpha)\beta \pi_{T+1} + u_T\big) \label{prestons19}  \\
i_t &= \psi_{\pi}\pi_t + \psi_{x} x_t  + \textcolor{blue}{\rho i_{t-1}} + \bar{i}_t \label{TR}
\end{align}
\begin{equation}
\hat{\E}_t z_{t+h} =  \begin{bmatrix}\bar{\pi}_{t-1} \\ 0 \\ 0 \end{bmatrix}+ bP^{h-1}s_t  \quad \forall h\geq 1 \quad \quad b = gx \; hx \quad \quad \text{PLM} \label{PLM}
\end{equation}
\begin{equation}
\bar{\pi}_{t} = \bar{\pi}_{t-1} +k_t^{-1}\underbrace{\big(\pi_{t} -(\bar{\pi}_{t-1}+b_1s_{t-1}) \big)}_{\text{fcst error using (\ref{PLM})} } \quad \quad  \text{($b_1$ is the first row of $b$)}
\end{equation}
 \begin{align*}
k_t & = \mathbb{I}\times(k_{t-1}+1) + (1-\mathbb{I}) \times \bar{g}^{-1}  \label{gain} \numberthis\\
\mathbb{I} & = \begin{cases} 1 \quad \text{if} \; \theta_t \leq \bar{\theta}  \\ 0 \quad \text{otherwise.}\numberthis
\end{cases} \\
\theta_t & = |\hat{\E}_{t-1}\pi_t - \E_{t-1}\pi_t| / \sigma_s \quad \quad \text{CEMP criterion for the gain}\label{criterion}\numberthis
\end{align*}

The alternative criterion for the choice of gain is a recursive variant of the CUSUM-test (Brown, Durbin, Evans 1975):
\begin{enumerate}
\item Let $FE_t$ denote the short-run forecast error, and $\omega_t$ firms' estimate of the FE variance.
\item Let $\kappa \in (0,1)$ and $\tilde{\theta}$ be the new threshold value for the criterion.
\item Then for initial $(\omega_0, \theta_0)$, firms in every period estimate the criterion and the FEV as:
\begin{align}
 \omega_t & =  \omega_{t-1} + \kappa k_{t-1}^{-1}(FE_t^2 -\omega_{t-1})\\
\theta_t & =  \theta_{t-1} + \kappa k_{t-1}^{-1}(FE_t^2/\omega_t -\theta_{t-1})\\
k_t & = \mathbb{I}\times(k_{t-1}+1) + (1-\mathbb{I}) \times \bar{g}^{-1} \\
\mathbb{I} & = 1 \quad \text{if} \quad  \theta_t \leq \tilde{\theta}
\end{align}
\end{enumerate}

\newpage
\section{Compact notation - with lagged interest rate term in TR}
 \begin{align}
z_t & = A_p^{RE} \E_t z_{t+1} + A_s^{RE} s_t \label{LOM_RE} \\
z_t & = A_a^{LH} f_a(t) + A_b^{LH} f_b(t) + A_s^{LH} s_t \label{LOM_LH} \\
s_t & = P s_{t-1} + \epsilon_t \label{exog} \\
 \quad \text{where} \quad 
 s_t & \equiv \begin{pmatrix} r_t^n \\ \bar{i}_t \\ u_t \\ i_{t-1}
 \end{pmatrix} \quad 
 P  \equiv \begin{pmatrix} \rho_r & 0 & 0 & 0\\ 0& \rho_i & 0 & 0\\ 0&0& \rho_u & 0 \\ 0&0& 0 & \rho
 \end{pmatrix}  \quad 
 \epsilon_t \equiv \begin{pmatrix}\varepsilon_t^{r} \\ \varepsilon_t^{i}  \\ \varepsilon_t^{u} \\ 0
 \end{pmatrix}  \quad  \text{and } \quad \Sigma  =  \begin{pmatrix} \sigma_r & 0 & 0 & 0 \\ 0& \sigma_i & 0 & 0 \\ 0&0& \sigma_u & 0 \\ 0 & 0& 0& 0&
 \end{pmatrix} 
\end{align}

Adding $i_{t-1}$ to the state vector fortunately doesn't change $A^{RE}_p, A^{LH}_a$ or $A^{LH}_b$, but it does change $A^{RE}_s$ and $A^{LH}_s$. The latter two get an additional column to account for the new state variable. With $g_{i, j} \; i=\pi,x, \; j=a,b$ unchanged from Materials 4, the new coefficient matrices are given by (new elements highlighted in \textcolor{blue}{blue}):
\begin{align}
A_s^{RE} &= \begin{pmatrix}   \frac{\kappa\sigma}{w}  &-\frac{\kappa\sigma}{w}  & 1-\frac{\kappa\sigma\psi_{\pi}}{w} & \textcolor{blue}{0}\\
 \frac{ \sigma}{w} &  -\frac{\sigma}{w} & -\frac{\sigma\psi_{\pi}}{w} & \textcolor{blue}{0}\\ 
 \psi_x( \frac{\sigma}{w}) + \psi_{\pi}( \frac{\kappa\sigma}{w}) & \psi_x(- \frac{\sigma}{w}) + \psi_{\pi}(- \frac{\kappa\sigma}{w}) +1 &  \psi_x(-\frac{\sigma\psi_{\pi}}{w}) + \psi_{\pi}( 1-\frac{\kappa\sigma\psi_{\pi}}{w}) & \textcolor{blue}{\rho}\end{pmatrix}  
\\
 A_s^{LR} & = \begin{pmatrix} g_{\pi s} \\ g_{x s} \\ \psi_{\pi}g_{\pi s} + \psi_xg_{x s} + \begin{bmatrix} 0 & 1& 0 & \textcolor{blue}{\rho}\end{bmatrix}
\end{pmatrix} \\
g_{\pi s} & = (1-\frac{\kappa\sigma\psi_{\pi}}{w} )\begin{bmatrix} 0&0&1 & \textcolor{blue}{0}\end{bmatrix} (I_3 - \alpha\beta P)^{-1} -\frac{\kappa\sigma}{w}\begin{bmatrix} -1&1&0 & \textcolor{blue}{\rho} \end{bmatrix} (I_3 -\beta P)^{-1}\\
g_{x s} & =  \frac{-\sigma\psi_{\pi}}{w} \begin{bmatrix} 0&0&1& \textcolor{blue}{0} \end{bmatrix}(I_3 - \alpha\beta P)^{-1}  -\frac{\sigma}{w}\begin{bmatrix} -1&1&0 & \textcolor{blue}{\rho}\end{bmatrix}(I_3 -\beta P)^{-1}
\end{align}

%%%%%%%%%%%  IRFs

\section{Small deviations in $\pi$, large ones in $x$, overshooting - IRFs to a natural real rate shock}

\begin{figure}[h!]
\subfigure[The observables for the specific shock sequence]{\includegraphics[scale = \myPointFourteenFigScale]{\myFigPath materials5_observables1}} 
\subfigure[Inverse gain and drift for the specific shock sequence]{\includegraphics[scale = \myPointFourteenFigScale]{\myFigPath materials5_gain_drift} }
\caption{A baseline shock sequence}

\end{figure}		
			
		
\begin{figure}[h!]
\includegraphics[scale = \mySmallFigScale]{\myFigPath materials5_IRFs_natrate} 
\caption{IRFs to a natural rate shock ($r^n$)}
\end{figure}		


\clearpage	

	\begin{prop} The degree to which inflation responds to expectational deviations from rational expectations (RE) depends on $\kappa$, the slope of the Phillips curve. A lower $\kappa$ means higher price rigidity and translates to current inflation responding less to expectation gaps.
	\end{prop}
	\begin{corollary}Whether expectation gaps between subjective and rational expectations show up in inflation or output depends on the value of $\kappa$. When nominal rigidities are high ($\kappa$ is low), money is strongly nonneutral, and thus output is the margin of adjustment. When prices are flexible, money is neutral and inflation is the margin of adjustment. 
	\end{corollary}
	\begin{prop} The fact that firms and households choose between decreasing and constant gains endogenously (anchoring mechanism) introduces a novel tradeoff for monetary policy relative to exogenous gain learning (decreasing or constant). With exogenous gain learning, too strong monetary responsiveness to shocks leads to overshooting due to increased persistence of inflation. With endogenous gain learning, however, monetary policy has to hit a lower bound on responsiveness in order to keep inflation expectations anchored. With endogenous gain learning, monetary policy faces a tradeoff between overshooting and anchoring. 
	\end{prop}






\end{document}





