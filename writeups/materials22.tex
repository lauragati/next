\documentclass[11pt]{article}
\usepackage{amsmath, amsthm, amssymb,lscape, natbib}
\usepackage{mathtools}
\usepackage{subfigure}
\usepackage[font=footnotesize,labelfont=bf]{caption}
\usepackage{graphicx}
\usepackage{colortbl}
\usepackage{hhline}
\usepackage{multirow}
\usepackage{multicol}
\usepackage{setspace}
\usepackage[final]{pdfpages}
\usepackage[left=2.5cm,top=2.5cm,right=2.5cm, bottom=2.5cm]{geometry}
\usepackage{natbib} 
\usepackage{bibentry} 
\newcommand{\bibverse}[1]{\begin{verse} \bibentry{#1} \end{verse}}
\newcommand{\vs}{\vspace{.3in}}
\renewcommand{\ni}{\noindent}
\usepackage{xr-hyper}
\usepackage[]{hyperref}
\hypersetup{
    colorlinks=true,
    linkcolor=blue,
    filecolor=magenta,      
    urlcolor=cyan,
}
 
\urlstyle{same}
\usepackage[capposition=top]{floatrow}
\usepackage{amssymb}
\usepackage{relsize}
\usepackage[dvipsnames]{xcolor}
\usepackage{fancyhdr}
\usepackage{tikz}
 
\pagestyle{fancy} % customize header and footer
\fancyhf{} % clear initial header and footer
%\rhead{Overleaf}
\lhead{\centering \rightmark} % this adds subsection number and name
\lfoot{\centering \rightmark} 
\rfoot{\thepage} % put page number (the centering command puts it in the middle, don't matter if you put it in right or left footer)

\def \myFigPath {../figures/} 
% BE CAREFUL WITH FIGNAMES, IN LATEX THEY'RE NOT CASE SENSITIVE!!
\def \myTablePath {../tables/} 

%\definecolor{mygreen}{RGB}{0, 100, 0}
\definecolor{mygreen}{RGB}{0, 128, 0}

\definecolor{citec}{rgb}{0,0,.5}
\definecolor{linkc}{rgb}{0,0,.6}
\definecolor{bcolor}{rgb}{1,1,1}
\hypersetup{
%hidelinks = true
  colorlinks = true,
  urlcolor=linkc,
  linkcolor=linkc,
  citecolor = citec,
  filecolor = linkc,
  pdfauthor={Laura G\'ati},
}


\geometry{left=.83in,right=.89in,top=1in,
bottom=1in}
\linespread{1.5}
\renewcommand{\[}{\begin{equation}}
\renewcommand{\]}{\end{equation}}

% New Options
\newtheorem{prop}{Proposition}
\newtheorem{definition}{Definition}[section]
\newtheorem*{remark}{Remark}
\newtheorem{lemma}{Lemma}
\newtheorem{corollary}{Corollary}
\newtheorem{conjecture}{Conjecture}

%\newtheorem{theorem}{Theorem}[section] % the third argument specifies that their number will be adopted to the section
%\newtheorem{corollary}{Corollary}[theorem]
%\newtheorem{lemma}[theorem]{Lemma}
%\declaretheorem{proposition}
%\linespread{1.3}
%\raggedbottom
%\font\reali=msbm10 at 12pt

% New Commands
\newcommand{\real}{\hbox{\reali R}}
\newcommand{\realp}{\hbox{\reali R}_{\scriptscriptstyle +}}
\newcommand{\realpp}{\hbox{\reali R}_{\scriptscriptstyle ++}}
\newcommand{\R}{\mathbb{R}}
\DeclareMathOperator{\E}{\mathbb{E}}
\DeclareMathOperator{\argmin}{arg\,min}
\newcommand\w{3.0in}
\newcommand\wnum{3.0}
\def\myFigWidth{5.3in}
\def\mySmallerFigWidth{2.1in}
\def\myEvenBiggerFigScale{0.8}
\def\myPointSixFigScale{0.6}
\def\myBiggerFigScale{0.4}
\def\myFigScale{0.3}
\def\myMediumFigScale{0.25}
\def\mySmallFigScale{0.22}
\def\mySmallerFigScale{0.18}
\def\myTinyFigScale{0.16}
\def\myPointFourteenFigScale{0.14}
\def\myTinierFigScale{0.12}
\def\myAdjustableFigScale{0.13}
\newcommand\numberthis{\addtocounter{equation}{1}\tag{\theequation}} % this defines a command to make align only number this line
\newcommand{\code}[1]{\texttt{#1}} %code %

\renewcommand*\contentsname{Overview}
\setcounter{tocdepth}{2}

% define a command to make a huge question mark (it works in math mode)
\newcommand{\bigqm}[1][1]{\text{\larger[#1]{\textbf{?}}}}

\begin{document}

\linespread{1.0}

\title{Materials 22 - GMM of simple anchoring function}
\author{Laura G\'ati} 
\date{\today}
\maketitle

%%%%%%%%%%%%%%%%%%%%             DOCUMENT           %%%%%%%%%%%%%%%%%% 

\tableofcontents

%\listoffigures

\section{Specifications of anchoring function and estimation}

\begin{itemize}
\item Anchoring function
\begin{equation}
k_t = k_{t+1} + \frac{1}{(d \; fe)^2}
\end{equation}
Agents update their PLM using the inverse gain $k_t^{-1}$. Thus the bigger $ \frac{1}{(d \; fe)^2}$,  the more the gain is \emph{decreasing}. Higher forecast errors $fe$ or a higher $d$ means closer to constant gains. I tried the inverse formulation with $h_t \equiv k_t^{-1}$ and
\begin{equation}
h_t = h_{t-1} + (d \; fe_{t-1})^2
\end{equation}
but it always led to explosive simulations.
\item Target: I gather the time series of inflation, output gap and federal funds rate, filter them, and compute empirical autocovariances:
\begin{equation}
ac^{data}(h) \equiv \text{cov}(y_t, y_{t-h})
\end{equation}
for $h=0,\dots,K$, selecting $K=4$. I gather these autocovariances for the three variables in the matrix $AC$. The target then is $ac^{data} \equiv$ vec($AC$) (a $n_y  (K+1) \times 1$ vector, i.e. $15\times 1$). Thus the objective function can be written as:
\begin{equation}
J \equiv (ac^{data}-ac^{model})' W^{-1} (ac^{data}-ac^{model})
\end{equation}

\item Initial $d_0 = 10$.
\end{itemize}

\section{Estimation issues}
\begin{itemize}
\item $W$:
Ideally I'd want to use a weighting matrix with the estimated variances of the target moments on the diagonal:
\begin{equation}
W = \begin{pmatrix} \hat{\sigma}^2_{ac(\pi,0)} & 0 &\dots &   & 0 \\
 0 & \hat{\sigma}^2_{ac(x,0)} & 0 & \dots & 0 \\
 \vdots & & \ddots & & \vdots \\
 0 & \dots &  & 0 & \hat{\sigma}^2_{ac(i,K)}
\end{pmatrix}
\end{equation}
Since I don't fit the data to a time series process, I create bootstrapped samples from the original (filtered) data. This however results in tiny bootstrapped variances, so $W^{-1}$ is huge. 

\end{itemize}


\section{Robustness to different filters}
\begin{figure}[h!]
\subfigure[Hodrick-Prescott, $\lambda=1600$]{\includegraphics[scale=\myAdjustableFigScale]{\myFigPath materials22_HP}}
\subfigure[Hamilton, 4 lags, $h=8$]{\includegraphics[scale=\myAdjustableFigScale]{\myFigPath materials22_Hamilton}}
\subfigure[Baxter-King, $(6,32)$ quarters, truncation at 12 lags]{\includegraphics[scale=\myAdjustableFigScale]{\myFigPath materials22_BK}}
\caption{Cyclical component of inflation filtered using different methods}
\end{figure}

\newpage
\section{Estimates}

\begin{figure}[h!]
\subfigure[Hodrick-Prescott, $\lambda=1600$]{\includegraphics[scale=\myAdjustableFigScale]{\myFigPath materials22_gain_dhat_HP}}
\subfigure[Hamilton, 4 lags, $h=8$]{\includegraphics[scale=\myAdjustableFigScale]{\myFigPath materials22_gain_dhat_Hamilton}}
\subfigure[Baxter-King, $(6,32)$ quarters, truncation at 12 lags]{\includegraphics[scale=\myAdjustableFigScale]{\myFigPath materials22_gain_dhat_BK}}
\caption{Inverse gain for $\hat{d}$ for the different filters}
\end{figure}

\begin{center}
\begin{table}[h!]
\caption{$\hat{d}$}
\begin{tabular}{ c |c |c }
  & $W = I$ & $W = \text{diag}(\hat{\sigma}_{ac(0)}, \dots, \hat{\sigma}_{ac(K)})$ \\ 
  \hline
 HP & 77.7899 & 10 \\  
 \hline
 Hamilton & 32.1649 & 10 \\  
 \hline
 BK & 90.3929 & 10    
\end{tabular}
\end{table}
\end{center}

\section{The other thing: numerical implementation of target criterion}
The target criterion in the simplified model:
\begin{align*}
\pi_t  = -\frac{\lambda_x}{\kappa}\bigg\{x_t - \frac{(1-\alpha)\beta}{1-\alpha\beta} \bigg(k_t^{-1}+((\pi_t - \bar{\pi}_{t-1}-b_1 s_{t-1}))\mathbf{g}_{\pi}(t) \bigg) \\
\bigg(\E_t\sum_{i=1}^{\infty}x_{t+i}\prod_{j=1}^{i-1}(1-k_{t+j}^{-1}(\pi_{t+1+j} - \bar{\pi}_{t+j}-b_1 s_{t+j})) \bigg)
\bigg\} \numberthis \label{target}
\end{align*}

\begin{itemize}
\item I think this is the highest priority.
\end{itemize}


\end{document}





