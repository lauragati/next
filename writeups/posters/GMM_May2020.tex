\documentclass[20pt, a1paper, landscape]{tikzposter}
\usepackage[utf8]{inputenc}

\title{Monetary Policy \& Anchored Expectations - An Endogenous Gain Learning Model}
\author{Laura G\'ati}
\date{\today}
\institute{Boston College}

\usepackage{blindtext}
\usepackage{comment}
\usepackage{amssymb}
\usepackage{amsmath}
\usepackage{booktabs}
\usepackage{verbatim}
\usepackage{caption}
\usepackage{float}
\usepackage{csquotes}
\usepackage{sansmathaccent}
\usepackage{subfigure}
\usepackage{multicol}
\pdfmapfile{+sansmathaccent.map}
\usepackage{pgfplots,tikz}
\usetikzlibrary{tikzmark,calc}
\usepackage{overpic}
\usepackage{color,soul}
\usepackage{stackengine}
\usepackage{caption}

% general appearance
\usetheme{Autumn}
\usecolorstyle{Britain}
\usecolorstyle[colorOne=white,colorTwo=gray]{Britain}

% this big chunk of code redefines the "Slide" style for blocks so that the backgound color of the titles aren't shaded (or graded)
\defineblockstyle{Slide}{
    titlewidthscale=1, bodywidthscale=1, titleleft,
    titleoffsetx=0pt, titleoffsety=0pt, bodyoffsetx=0pt, bodyoffsety=0pt,
    bodyverticalshift=0pt, roundedcorners=0, linewidth=0pt, titleinnersep=1cm,
    bodyinnersep=1cm
}{
    \ifBlockHasTitle%
        % changed "right color=..,left color=.." to "fill=blocktitlebgcolor"
        \draw[draw=none, fill=blocktitlebgcolor] 
           (blocktitle.south west) rectangle (blocktitle.north east);
    \fi%
    \draw[draw=none, fill=blockbodybgcolor] %
        (blockbody.north west) [rounded corners=30] -- (blockbody.south west) --
        (blockbody.south east) [rounded corners=0]-- (blockbody.north east) -- cycle;
}

% appearance of the blocks
\useblockstyle{Slide}
% using the below I can change the color of the block title
%\colorlet{blocktitlebgcolor}{red}


\usepackage{bookman}


% remove annoying bottom right logo
\tikzposterlatexaffectionproofoff

% My own definitions
\def \myFigPath {../../figures/} 
\def \myTablePath {../../tables/} 

\newcommand{\real}{\hbox{\reali R}}
\newcommand{\realp}{\hbox{\reali R}_{\scriptscriptstyle +}}
\newcommand{\realpp}{\hbox{\reali R}_{\scriptscriptstyle ++}}
\newcommand{\R}{\mathbb{R}}
\DeclareMathOperator{\E}{\mathbb{E}}
\DeclareMathOperator{\argmin}{arg\,min}
\newcommand\w{3.0in}
\newcommand\wnum{3.0}
\def\myFigWidth{5.3in}
\def\mySmallerFigWidth{2.1in}
\def\myEvenBiggerFigScale{0.8}
\def\myPointSixFigScale{0.6}
\def\myBiggerFigScale{0.4}
\def\myFigScale{0.3}
\def\mySmallFigScale{0.25}
\def\mySmallerFigScale{0.18}
\def\myTinyFigScale{0.16}
\def\myPointFourteenFigScale{0.14}
\def\myTinierFigScale{0.12}
\def\myEvenTinierFigScale{0.10}
\def\myCrazyTinyFigScale{0.09}

\newtheorem{prop}{Proposition}
%\newtheorem*{remark}{Remark}
\newtheorem{result}{Result}

\newcommand\numberthis{\addtocounter{equation}{1}\tag{\theequation}} % this defines a command to make align only number this line
% creating colored lines for legends
% green solid
\newcommand{\greenline}{\raisebox{2pt}{\tikz{\draw[-,black!40!green,solid,line width = 1.5pt](0,0) -- (5mm,0);}}}
% red solid, dashed, dotted
\newcommand{\redline}{\raisebox{2pt}{\tikz{\draw[-,red,solid,line width = 1.5pt](0,0) -- (5mm,0);}}}
\newcommand{\reddashedline}{\raisebox{2pt}{\tikz{\draw[-,red,dashed,line width = 1.5pt](0,0) -- (5mm,0);}}}
\newcommand{\reddottedline}{\raisebox{2pt}{\tikz{\draw[-,red,densely dotted,line width = 1.5pt](0,0) -- (5mm,0);}}}

% blue solid and dashed
\newcommand{\blueline}{\raisebox{2pt}{\tikz{\draw[-,blue,solid,line width = 1.5pt](0,0) -- (5mm,0);}}}
\newcommand{\bluedashedline}{\raisebox{2pt}{\tikz{\draw[-,blue,dashed,line width = 1.5pt](0,0) -- (5mm,0);}}}
% black solid and dashed
\newcommand{\blackline}{\raisebox{2pt}{\tikz{\draw[-,black,solid,line width = 1.5pt](0,0) -- (5mm,0);}}}
\newcommand{\blackdashedline}{\raisebox{2pt}{\tikz{\draw[-,black,dashed,line width = 1.5pt](0,0) -- (5mm,0);}}}

\definecolor{mygreen}{RGB}{0, 128, 0} % this is Laura's green


%%%%%%%%%%%%%%%%%%%
\begin{document}

\maketitle

\begin{columns}
\column{0.5}
\block{}{
\section{Puzzling Fed behavior fall 2019}
% Note: to have minipages side by side, you MUST NOT have space between the minipages!
\noindent
\begin{minipage}{0.15\textwidth}
\centering
\includegraphics[scale=0.27]{\myFigPath urate_2020_02_09}
    \captionof{figure}{Unemployment rate \\ (\%)}
\end{minipage}
\begin{minipage}{0.15\textwidth}
\centering
\includegraphics[scale=0.27]{\myFigPath frr_2020_02_09}
    \captionof{figure}{Fed funds rate target, \\ upper limit (\%)}
\end{minipage}
\begin{minipage}{0.15\textwidth}
\centering
\includegraphics[scale=0.27]{\myFigPath epi10_2020_02_09}
    \captionof{figure}{Market-based inflation expectations, 10 year, average (\%)}
\end{minipage}

\section{Model with anchoring expectation formation}
Macro model with Calvo nominal friction: standard up to expectation formation $(\hat{\E})$ 



\subsection{Expectation formation}

\begin{itemize}
\item Model solution under rational expectations (RE)
 \begin{align}
 s_t & = h s_{t-1} + \epsilon_t \quad \quad \quad \epsilon_t \sim \mathcal{N}(\mathbf{0},\Sigma) \label{state} \\
 y_t & = g s_t \label{obs_RE}
 \end{align}
\item Here: private sector does not know $g$ $\rightarrow$ estimate using (\ref{state}) \& observed states
\item Households and firms don't know they are identical
\item Special case: private sector doesn't know long-run mean of inflation:
\begin{equation}
\hat{\E}_t \pi_{t+1} = \textcolor{blue}{\bar{\pi}_{t-1}} + g_1h_1s_t  \label{PLM_fcst_general}
\end{equation}
\item Updates estimate of mean inflation using recursive least squares
\begin{equation}
\bar{\pi}_{t}  =\bar{\pi}_{t-1} +k_t \underbrace{\big(\pi_{t} -(\bar{\pi}_{t-1}+b_1 s_{t-1}) \big)}_{\equiv \; fe_{t|t-1} \text{, forecast error} } 
\end{equation}
\end{itemize}

\subsection{Anchoring mechanism}
Endogenous gain as anchoring mechanism:
\begin{equation}
k_t = k_{t-1} + \mathbf{g}(fe_{t|t-1}) \label{gain}
\end{equation}

}
%%%%%%%%%%%%%%% COLUMN 2 %%%%%%%%%%%%%%%%%

\column{0.5}
\block{}{



\subsection{Aggregate laws of motion}
IS- and Phillips curve
  \begin{align}
x_t &=  -\sigma i_t +\hat{\E}_t \sum_{T=t}^{\infty} \beta^{T-t }\big( (1-\beta)x_{T+1} - \sigma(\beta i_{T+1} - \pi_{T+1}) +\sigma r_T^n \big)  \label{NKIS}  \\
\pi_t &= \kappa x_t +\hat{\E}_t \sum_{T=t}^{\infty} (\alpha\beta)^{T-t }\big( \kappa \alpha \beta x_{T+1} + (1-\alpha)\beta \pi_{T+1} + u_T\big) \label{NKPC} 
\end{align}



\section{Ramsey policy under anchoring expectation formation}

\begin{result} Target criterion under anchoring
\begin{align*}
\pi_t  = -\frac{\lambda_x}{\kappa}\bigg\{x_t - \frac{(1-\alpha)\beta}{1-\alpha\beta} \bigg(k_t+((\pi_t - \bar{\pi}_{t-1}-b_1 s_{t-1}))\mathbf{g}_{\pi,t}\bigg) \\
\\
\bigg(\E_t\sum_{i=1}^{\infty}x_{t+i}\prod_{j=0}^{i-1}(1-k_{t+1+j} - (\pi_{t+1+j} - \bar{\pi}_{t+j}-b_1 s_{t+j})\mathbf{g_{\bar{\pi}, t+j}}) \bigg)
\bigg\}  \label{target}
\end{align*}
\end{result}

$\rightarrow$ Two layers of novel intertemporal tradeoffs: can postpone intratemporal tradeoff

\begin{result} For any adaptive learning scheme, the discretion and commitment solutions of the Ramsey problem coincide.  The solution qualitatively resembles discretion and is thus not subject to the time inconsistency problem.
\end{result}

\section*{Implementation?}
\begin{itemize}
\item Need for feedback rules satisfying 
\item Form of feedback rule? Model suggests 
\begin{equation*}
 i_t = \mathbf{f}(\pi_t, k_{t}, \bar{\pi}_{t-1}; t)\quad  \text{nonlinear}
 \end{equation*}
\item[$\rightarrow$] Explains deviations from Taylor rule
\item[$\rightarrow$] Interesting to assess Taylor rule as approximation to optimal rule
\item[] $\hookrightarrow$ Might do better than under RE since commitment plan not feasible here
\item Optimal Taylor rule less aggressive on inflation than under RE
\end{itemize}


}
\end{columns}

%%%%%%%%%%%%%%%%% OLD STUFF %%%%%%%%%%%%%%%%%%%
%\begin{columns}
%\column{0.5}
%\block{Puzzling Fed behavior fall 2019}
%{
%% Note: to have minipages side by side, you MUST NOT have space between the minipages!
%\noindent
%\begin{minipage}{0.15\textwidth}
%\centering
%\includegraphics[scale=0.22]{\myFigPath urate_2020_02_09}
%    \captionof{figure}{Unemployment rate \\ (\%)}
%\end{minipage}
%\begin{minipage}{0.15\textwidth}
%\centering
%\includegraphics[scale=0.22]{\myFigPath frr_2020_02_09}
%    \captionof{figure}{Fed funds rate target, \\ upper limit (\%)}
%\end{minipage}
%\begin{minipage}{0.15\textwidth}
%\centering
%\includegraphics[scale=0.22]{\myFigPath epi10_2020_02_09}
%    \captionof{figure}{Market-based inflation expectations, 10 year, average (\%)}
%\end{minipage}
%}
%\column{0.5}
%\block{Model: IS- and Phillips curve, modified for nonrational expectations}
%{ Macro model with nominal frictions: standard up to expectation formation $(\hat{\E})$ \\
%
%IS- and Phillips curve
%  \begin{align}
%x_t &=  -\sigma i_t +\hat{\E}_t \sum_{T=t}^{\infty} \beta^{T-t }\big( (1-\beta)x_{T+1} - \sigma(\beta i_{T+1} - \pi_{T+1}) +\sigma r_T^n \big)  \label{NKIS}  \\
%\pi_t &= \kappa x_t +\hat{\E}_t \sum_{T=t}^{\infty} (\alpha\beta)^{T-t }\big( \kappa \alpha \beta x_{T+1} + (1-\alpha)\beta \pi_{T+1} + u_T\big) \label{NKPC} 
%\end{align}
%
%
%
%}
%\end{columns}
%
%\begin{columns}
%\column{0.5}
%\block{Anchoring expectation formation}
%{ 
%
%\begin{itemize}
%\item Model solution under rational expectations
% \begin{align}
% s_t & = h s_{t-1} + \epsilon_t \quad \quad \quad \epsilon_t \sim \mathcal{N}(\mathbf{0},\Sigma) \label{state} \\
% y_t & = g s_t \label{obs_RE}
% \end{align}
%\item Here: private sector does not know $g$ $\rightarrow$ estimate using (\ref{state}) \& observed states
%\item Households and firms don't know they are identical
%\item Special case: private sector doesn't know long-run mean of inflation:
%\begin{equation}
%\hat{\E}_t \pi_{t+1} = \textcolor{blue}{\bar{\pi}_{t-1}} + g_1h_1s_t  \label{PLM_fcst_general}
%\end{equation}
%\item Updates estimate of mean inflation using recursive least squares
%\begin{equation}
%\bar{\pi}_{t}  =\bar{\pi}_{t-1} +k_t \underbrace{\big(\pi_{t} -(\bar{\pi}_{t-1}+b_1 s_{t-1}) \big)}_{\equiv \; fe_{t|t-1} \text{, forecast error} } 
%\end{equation}
%\item Endogenous gain as anchoring mechanism:
%\begin{equation}
%k_t = k_{t-1} + \mathbf{g}(fe_{t|t-1}) \label{gain}
%\end{equation}
%\end{itemize}
%}
%
%\column{0.5}
%\block{}
%{\section{Ramsey policy under anchoring expectation formation}
%
%\begin{result} Target criterion under anchoring
%\begin{align*}
%\pi_t  = -\frac{\lambda_x}{\kappa}\bigg\{x_t - \frac{(1-\alpha)\beta}{1-\alpha\beta} \bigg(k_t+((\pi_t - \bar{\pi}_{t-1}-b_1 s_{t-1}))\mathbf{g}_{\pi,t}\bigg) \\
%\\
%\bigg(\E_t\sum_{i=1}^{\infty}x_{t+i}\prod_{j=0}^{i-1}(1-k_{t+1+j} - (\pi_{t+1+j} - \bar{\pi}_{t+j}-b_1 s_{t+j})\mathbf{g_{\bar{\pi}, t+j}}) \bigg)
%\bigg\}  \label{target}
%\end{align*}
%\end{result}
%
%\begin{result} For any adaptive learning scheme, the discretion and commitment solutions of the Ramsey problem coincide. Thus the time consistency problem does not emerge.
%\end{result}
%
%\section*{Implementation?}
%\begin{itemize}
%\item Implementation?
%\item Form of feedback rule? Model suggests $\quad i_t = \mathbf{f}(\pi_t, k_{t}, \bar{\pi}_{t-1}; t)$ nonlinear
%\item[$\rightarrow$] Explains deviations from Taylor rule
%\end{itemize}
%     
%}
%\end{columns}


\end{document}
