\documentclass[11pt]{article}
\usepackage{amsmath, amsthm, amssymb,lscape, natbib}
\usepackage{mathtools}
\usepackage{subfigure}
\usepackage[font=footnotesize,labelfont=bf]{caption}
\usepackage{graphicx}
\usepackage{colortbl}
\usepackage{hhline}
\usepackage{multirow}
\usepackage{multicol}
\usepackage{setspace}
\usepackage[final]{pdfpages}
\usepackage[left=2.5cm,top=2.5cm,right=2.5cm, bottom=2.5cm]{geometry}
\usepackage{natbib} 
\usepackage{bibentry} 
\newcommand{\bibverse}[1]{\begin{verse} \bibentry{#1} \end{verse}}
\newcommand{\vs}{\vspace{.3in}}
\renewcommand{\ni}{\noindent}
\usepackage{xr-hyper}
\usepackage[]{hyperref}
\hypersetup{
    colorlinks=true,
    linkcolor=blue,
    filecolor=magenta,      
    urlcolor=cyan,
}
 
\urlstyle{same}
\usepackage[capposition=top]{floatrow}
\usepackage{amssymb}
\usepackage{relsize}
\usepackage[dvipsnames]{xcolor}
\usepackage{fancyhdr}
\usepackage{tikz}
 
\pagestyle{fancy} % customize header and footer
\fancyhf{} % clear initial header and footer
%\rhead{Overleaf}
\lhead{\centering \rightmark} % this adds subsection number and name
\lfoot{\centering \rightmark} 
\rfoot{\thepage} % put page number (the centering command puts it in the middle, don't matter if you put it in right or left footer)

\def \myFigPath {../figures/} 
% BE CAREFUL WITH FIGNAMES, IN LATEX THEY'RE NOT CASE SENSITIVE!!
\def \myTablePath {../tables/} 

%\definecolor{mygreen}{RGB}{0, 100, 0}
\definecolor{mygreen}{RGB}{0, 128, 0}

\definecolor{citec}{rgb}{0,0,.5}
\definecolor{linkc}{rgb}{0,0,.6}
\definecolor{bcolor}{rgb}{1,1,1}
\hypersetup{
%hidelinks = true
  colorlinks = true,
  urlcolor=linkc,
  linkcolor=linkc,
  citecolor = citec,
  filecolor = linkc,
  pdfauthor={Laura G\'ati},
}


\geometry{left=.83in,right=.89in,top=1in,
bottom=1in}
\linespread{1.5}
\renewcommand{\[}{\begin{equation}}
\renewcommand{\]}{\end{equation}}

% New Options
\newtheorem{prop}{Proposition}
\newtheorem{definition}{Definition}[section]
\newtheorem*{remark}{Remark}
\newtheorem{lemma}{Lemma}
\newtheorem{corollary}{Corollary}
\newtheorem{conjecture}{Conjecture}

%\newtheorem{theorem}{Theorem}[section] % the third argument specifies that their number will be adopted to the section
%\newtheorem{corollary}{Corollary}[theorem]
%\newtheorem{lemma}[theorem]{Lemma}
%\declaretheorem{proposition}
%\linespread{1.3}
%\raggedbottom
%\font\reali=msbm10 at 12pt

% New Commands
\newcommand{\real}{\hbox{\reali R}}
\newcommand{\realp}{\hbox{\reali R}_{\scriptscriptstyle +}}
\newcommand{\realpp}{\hbox{\reali R}_{\scriptscriptstyle ++}}
\newcommand{\R}{\mathbb{R}}
\DeclareMathOperator{\E}{\mathbb{E}}
\DeclareMathOperator{\argmin}{arg\,min}
\newcommand\w{3.0in}
\newcommand\wnum{3.0}
\def\myFigWidth{5.3in}
\def\mySmallerFigWidth{2.1in}
\def\myEvenBiggerFigScale{0.8}
\def\myPointSixFigScale{0.6}
\def\myBiggerFigScale{0.4}
\def\myFigScale{0.3}
\def\myMediumFigScale{0.25}
\def\mySmallFigScale{0.22}
\def\mySmallerFigScale{0.18}
\def\myTinyFigScale{0.16}
\def\myPointFourteenFigScale{0.14}
\def\myTinierFigScale{0.12}
\def\myAdjustableFigScale{0.18}
\newcommand\numberthis{\addtocounter{equation}{1}\tag{\theequation}} % this defines a command to make align only number this line
\newcommand{\code}[1]{\texttt{#1}} %code %

\renewcommand*\contentsname{Overview}
\setcounter{tocdepth}{2}

% define a command to make a huge question mark (it works in math mode)
\newcommand{\bigqm}[1][1]{\text{\larger[#1]{\textbf{?}}}}

\begin{document}

\linespread{1.0}

\title{Materials 12d - Two general solution methods for the learning model, consolidated \\
- 
\\
\small{See Notes 31 Dec 2019}}
\author{Laura G\'ati} 
\date{\today}
\maketitle

%%%%%%%%%%%%%%%%%%%%             DOCUMENT           %%%%%%%%%%%%%%%%%% 

%\tableofcontents

%\listoffigures

\section{Model equations and goal}
\begin{align}
x_t &=  -\sigma i_t +\hat{\E}_t \sum_{T=t}^{\infty} \beta^{T-t }\big( (1-\beta)x_{T+1} - \sigma(\beta i_{T+1} - \pi_{T+1}) +\sigma r_T^n \big)  \label{prestons18}  \\
\pi_t &= \kappa x_t +\hat{\E}_t \sum_{T=t}^{\infty} (\alpha\beta)^{T-t }\big( \kappa \alpha \beta x_{T+1} + (1-\alpha)\beta \pi_{T+1} + u_T\big) \label{prestons19}  \\
i_t &= \psi_{\pi}\pi_t + \psi_{x} x_t  + \rho i_{t-1} + \bar{i}_t \label{TR}
\end{align}
Goal: obtain endogenous stuff as a function of expectations and states:
\begin{equation}
z_t = \begin{bmatrix} \pi_t \\ x_t \\ i_t \end{bmatrix} = A_a f_{a} + A_b f_{b} + A_s s_t
\end{equation}
where I already have expectations $f_a, f_b$ and the state vector can vary by model, but in this default case with $\rho=0$ it is \begin{equation}
s_t = \begin{bmatrix} r_t^n \\ \bar{i}_t \\ u_t \end{bmatrix}
\end{equation}
That is,  we want the matrices $A_a, A_b$ and $A_s$.
\section{Method 1 - M-N method (old method)}
\begin{equation}
\underbrace{\begin{bmatrix} \sigma\psi_{\pi}&  1+\sigma\psi_x\\ 1& -\kappa \end{bmatrix}}_{\equiv M}\begin{bmatrix} \pi_t \\ x_t \end{bmatrix} = \underbrace{\begin{bmatrix} c_{x,b}f_b +c_{x,s} s_t \\ c_{\pi,a}f_a +c_{\pi,s} s_t \end{bmatrix}}_{\equiv N}
\end{equation}
where $M$ and the $c$ are model-specific. In the baseline model they are given by
\begin{align*}
 c_{x,b} & =  \begin{bmatrix} \sigma(1-\beta\psi_{\pi}), &1-\beta-\sigma\beta\psi_x ,&0 \end{bmatrix} \numberthis\\
 c_{x,s} & =  -\sigma\begin{bmatrix} -1 &1 & 0 & \rho \end{bmatrix}(I_{nx}-\beta h_x)^{-1} \numberthis \\
 c_{\pi,a} & =  \begin{bmatrix} (1-\alpha)\beta, &\kappa\alpha\beta ,& 0 \end{bmatrix} \numberthis \\
 c_{\pi,s} & =  \begin{bmatrix} 0&0 & 1 & 0 \end{bmatrix}(I_{nx}-\alpha\beta h_x)^{-1} \numberthis \\
  c_{i,s} & =  \begin{bmatrix} 0 &1 & 0 & \rho \end{bmatrix} \numberthis
\end{align*}
In Mathematica (\texttt{materials12d.nb}), I write the $N$ matrix as: 
\begin{equation}
\begin{bmatrix}  \begin{bmatrix} \sigma(1-\beta\psi_{\pi}), &1-\beta-\sigma\beta\psi_x ,&0 \end{bmatrix} f_b + d_1 \\ \begin{bmatrix} (1-\alpha)\beta, &\kappa\alpha\beta ,& 0 \end{bmatrix}  f_a + d_2 \end{bmatrix}
\end{equation}


Then Mathematica solves for $x$ and $\pi$ as \begin{equation}
\begin{bmatrix} \pi_t^* \\ x_t^* \end{bmatrix} = M^{-1}N \label{MN}
\end{equation}
Then the solution for the interest rate will just be
\begin{equation}
i_t = \psi_{\pi}\pi_t^* + \psi_{x} x_t^*  +\underbrace{c_{is} s_t}_{\equiv d_6}
\end{equation}
The last step is to gather the matrices $g_{i,j}$, the coefficients of $i$ on $j$, $i=x,\pi,i, j=f_a,f_b,s$. Mathematica will output these $g$-matrices and stack them appropriately to give the $A$-matrices:
\begin{equation}
\underbrace{A_a}_{ny \times ny} = \begin{pmatrix} g_{\pi,a} \\ g_{x,a} \\ g_{i,a}\end{pmatrix} \quad 
\underbrace{A_b}_{ny \times ny} = \begin{pmatrix} g_{\pi,b} \\ g_{x,b} \\ g_{i,b}\end{pmatrix} \quad 
\underbrace{A_s}_{ny \times nx} = \begin{pmatrix} g_{\pi,s} \\ g_{x,s} \\ g_{i,s}\end{pmatrix}
\end{equation}
Now you can be copy this directly into Matlab (for the default model \texttt{matrices\_A2.m}), specifying $d1, d2, d6$ in Matlab.

\section{Method 2 - P-Q method (new method)}
Instead of subbing out the interest rate in the original equations, write the system as:
\begin{equation}
\underbrace{\begin{bmatrix} 0& 1&  \sigma \\ 1& -\kappa & 0 \\ -\psi_{\pi} & -\psi_{x} & 1 \end{bmatrix}}_{\equiv P}\begin{bmatrix} \pi_t \\ x_t \\i_t \end{bmatrix} = \underbrace{\begin{bmatrix} c_{x,b}f_b +c_{x,s} s_t \\ c_{\pi,a}f_a +c_{\pi,s} s_t \\ c_{i,s} s_t \end{bmatrix}}_{\equiv Q}
\end{equation}
Then (\ref{MN}) is replaced by 
\begin{equation}
\begin{bmatrix} \pi_t^* \\ x_t^* \\ i_t^* \end{bmatrix} = P^{-1}Q \label{PQ}
\end{equation}
where you also have to impose the relation
\begin{equation}
f_b(3) = \psi_{\pi}f_b(1) + \psi_{x}f_b(2) + \frac{1}{\beta}\{ \begin{bmatrix} -1 &1 & 0  \end{bmatrix}(I_{nx}-\beta h_x)^{-1} s_t - \begin{bmatrix} -1 &1 & 0  \end{bmatrix}s_t  \} \tag{*}
\end{equation}
so that Mathematica recognizes that the interest rate expectations are just a function of those of $\pi$ and $x$. Then the rest works mechanically the same way as in the M-N method. 

The problem with the PQ-method is that since there are a bunch of steps Mathematica has a hard time with (I haven't even figured out how to stop Mathematica from multiplying out the matrices!), the PQ method becomes less robust than the MN method, because in the latter you're more in control of how expectations are treated.

\end{document}





